\documentclass[11pt]{article}

    \usepackage[breakable]{tcolorbox}
    \usepackage{parskip} % Stop auto-indenting (to mimic markdown behaviour)
    
    \usepackage{iftex}
    \ifPDFTeX
    	\usepackage[T1]{fontenc}
    	\usepackage{mathpazo}
    \else
    	\usepackage{fontspec}
    \fi

    % Basic figure setup, for now with no caption control since it's done
    % automatically by Pandoc (which extracts ![](path) syntax from Markdown).
    \usepackage{graphicx}
    % Maintain compatibility with old templates. Remove in nbconvert 6.0
    \let\Oldincludegraphics\includegraphics
    % Ensure that by default, figures have no caption (until we provide a
    % proper Figure object with a Caption API and a way to capture that
    % in the conversion process - todo).
    \usepackage{caption}
    \DeclareCaptionFormat{nocaption}{}
    \captionsetup{format=nocaption,aboveskip=0pt,belowskip=0pt}

    \usepackage[Export]{adjustbox} % Used to constrain images to a maximum size
    \adjustboxset{max size={0.9\linewidth}{0.9\paperheight}}
    \usepackage{float}
    \floatplacement{figure}{H} % forces figures to be placed at the correct location
    \usepackage{xcolor} % Allow colors to be defined
    \usepackage{enumerate} % Needed for markdown enumerations to work
    \usepackage{geometry} % Used to adjust the document margins
    \usepackage{amsmath} % Equations
    \usepackage{amssymb} % Equations
    \usepackage{textcomp} % defines textquotesingle
    % Hack from http://tex.stackexchange.com/a/47451/13684:
    \AtBeginDocument{%
        \def\PYZsq{\textquotesingle}% Upright quotes in Pygmentized code
    }
    \usepackage{upquote} % Upright quotes for verbatim code
    \usepackage{eurosym} % defines \euro
    \usepackage[mathletters]{ucs} % Extended unicode (utf-8) support
    \usepackage{fancyvrb} % verbatim replacement that allows latex
    \usepackage{grffile} % extends the file name processing of package graphics 
                         % to support a larger range
    \makeatletter % fix for grffile with XeLaTeX
    \def\Gread@@xetex#1{%
      \IfFileExists{"\Gin@base".bb}%
      {\Gread@eps{\Gin@base.bb}}%
      {\Gread@@xetex@aux#1}%
    }
    \makeatother

    % The hyperref package gives us a pdf with properly built
    % internal navigation ('pdf bookmarks' for the table of contents,
    % internal cross-reference links, web links for URLs, etc.)
    \usepackage{hyperref}
    % The default LaTeX title has an obnoxious amount of whitespace. By default,
    % titling removes some of it. It also provides customization options.
    \usepackage{titling}
    \usepackage{longtable} % longtable support required by pandoc >1.10
    \usepackage{booktabs}  % table support for pandoc > 1.12.2
    \usepackage[inline]{enumitem} % IRkernel/repr support (it uses the enumerate* environment)
    \usepackage[normalem]{ulem} % ulem is needed to support strikethroughs (\sout)
                                % normalem makes italics be italics, not underlines
    \usepackage{mathrsfs}
    

    
    % Colors for the hyperref package
    \definecolor{urlcolor}{rgb}{0,.145,.698}
    \definecolor{linkcolor}{rgb}{.71,0.21,0.01}
    \definecolor{citecolor}{rgb}{.12,.54,.11}

    % ANSI colors
    \definecolor{ansi-black}{HTML}{3E424D}
    \definecolor{ansi-black-intense}{HTML}{282C36}
    \definecolor{ansi-red}{HTML}{E75C58}
    \definecolor{ansi-red-intense}{HTML}{B22B31}
    \definecolor{ansi-green}{HTML}{00A250}
    \definecolor{ansi-green-intense}{HTML}{007427}
    \definecolor{ansi-yellow}{HTML}{DDB62B}
    \definecolor{ansi-yellow-intense}{HTML}{B27D12}
    \definecolor{ansi-blue}{HTML}{208FFB}
    \definecolor{ansi-blue-intense}{HTML}{0065CA}
    \definecolor{ansi-magenta}{HTML}{D160C4}
    \definecolor{ansi-magenta-intense}{HTML}{A03196}
    \definecolor{ansi-cyan}{HTML}{60C6C8}
    \definecolor{ansi-cyan-intense}{HTML}{258F8F}
    \definecolor{ansi-white}{HTML}{C5C1B4}
    \definecolor{ansi-white-intense}{HTML}{A1A6B2}
    \definecolor{ansi-default-inverse-fg}{HTML}{FFFFFF}
    \definecolor{ansi-default-inverse-bg}{HTML}{000000}

    % commands and environments needed by pandoc snippets
    % extracted from the output of `pandoc -s`
    \providecommand{\tightlist}{%
      \setlength{\itemsep}{0pt}\setlength{\parskip}{0pt}}
    \DefineVerbatimEnvironment{Highlighting}{Verbatim}{commandchars=\\\{\}}
    % Add ',fontsize=\small' for more characters per line
    \newenvironment{Shaded}{}{}
    \newcommand{\KeywordTok}[1]{\textcolor[rgb]{0.00,0.44,0.13}{\textbf{{#1}}}}
    \newcommand{\DataTypeTok}[1]{\textcolor[rgb]{0.56,0.13,0.00}{{#1}}}
    \newcommand{\DecValTok}[1]{\textcolor[rgb]{0.25,0.63,0.44}{{#1}}}
    \newcommand{\BaseNTok}[1]{\textcolor[rgb]{0.25,0.63,0.44}{{#1}}}
    \newcommand{\FloatTok}[1]{\textcolor[rgb]{0.25,0.63,0.44}{{#1}}}
    \newcommand{\CharTok}[1]{\textcolor[rgb]{0.25,0.44,0.63}{{#1}}}
    \newcommand{\StringTok}[1]{\textcolor[rgb]{0.25,0.44,0.63}{{#1}}}
    \newcommand{\CommentTok}[1]{\textcolor[rgb]{0.38,0.63,0.69}{\textit{{#1}}}}
    \newcommand{\OtherTok}[1]{\textcolor[rgb]{0.00,0.44,0.13}{{#1}}}
    \newcommand{\AlertTok}[1]{\textcolor[rgb]{1.00,0.00,0.00}{\textbf{{#1}}}}
    \newcommand{\FunctionTok}[1]{\textcolor[rgb]{0.02,0.16,0.49}{{#1}}}
    \newcommand{\RegionMarkerTok}[1]{{#1}}
    \newcommand{\ErrorTok}[1]{\textcolor[rgb]{1.00,0.00,0.00}{\textbf{{#1}}}}
    \newcommand{\NormalTok}[1]{{#1}}
    
    % Additional commands for more recent versions of Pandoc
    \newcommand{\ConstantTok}[1]{\textcolor[rgb]{0.53,0.00,0.00}{{#1}}}
    \newcommand{\SpecialCharTok}[1]{\textcolor[rgb]{0.25,0.44,0.63}{{#1}}}
    \newcommand{\VerbatimStringTok}[1]{\textcolor[rgb]{0.25,0.44,0.63}{{#1}}}
    \newcommand{\SpecialStringTok}[1]{\textcolor[rgb]{0.73,0.40,0.53}{{#1}}}
    \newcommand{\ImportTok}[1]{{#1}}
    \newcommand{\DocumentationTok}[1]{\textcolor[rgb]{0.73,0.13,0.13}{\textit{{#1}}}}
    \newcommand{\AnnotationTok}[1]{\textcolor[rgb]{0.38,0.63,0.69}{\textbf{\textit{{#1}}}}}
    \newcommand{\CommentVarTok}[1]{\textcolor[rgb]{0.38,0.63,0.69}{\textbf{\textit{{#1}}}}}
    \newcommand{\VariableTok}[1]{\textcolor[rgb]{0.10,0.09,0.49}{{#1}}}
    \newcommand{\ControlFlowTok}[1]{\textcolor[rgb]{0.00,0.44,0.13}{\textbf{{#1}}}}
    \newcommand{\OperatorTok}[1]{\textcolor[rgb]{0.40,0.40,0.40}{{#1}}}
    \newcommand{\BuiltInTok}[1]{{#1}}
    \newcommand{\ExtensionTok}[1]{{#1}}
    \newcommand{\PreprocessorTok}[1]{\textcolor[rgb]{0.74,0.48,0.00}{{#1}}}
    \newcommand{\AttributeTok}[1]{\textcolor[rgb]{0.49,0.56,0.16}{{#1}}}
    \newcommand{\InformationTok}[1]{\textcolor[rgb]{0.38,0.63,0.69}{\textbf{\textit{{#1}}}}}
    \newcommand{\WarningTok}[1]{\textcolor[rgb]{0.38,0.63,0.69}{\textbf{\textit{{#1}}}}}
    
    
    % Define a nice break command that doesn't care if a line doesn't already
    % exist.
    \def\br{\hspace*{\fill} \\* }
    % Math Jax compatibility definitions
    \def\gt{>}
    \def\lt{<}
    \let\Oldtex\TeX
    \let\Oldlatex\LaTeX
    \renewcommand{\TeX}{\textrm{\Oldtex}}
    \renewcommand{\LaTeX}{\textrm{\Oldlatex}}
    % Document parameters
    % Document title
    \title{Lecture-0-Scientific-Computing-with-Python}
    
    
    
    
    
% Pygments definitions
\makeatletter
\def\PY@reset{\let\PY@it=\relax \let\PY@bf=\relax%
    \let\PY@ul=\relax \let\PY@tc=\relax%
    \let\PY@bc=\relax \let\PY@ff=\relax}
\def\PY@tok#1{\csname PY@tok@#1\endcsname}
\def\PY@toks#1+{\ifx\relax#1\empty\else%
    \PY@tok{#1}\expandafter\PY@toks\fi}
\def\PY@do#1{\PY@bc{\PY@tc{\PY@ul{%
    \PY@it{\PY@bf{\PY@ff{#1}}}}}}}
\def\PY#1#2{\PY@reset\PY@toks#1+\relax+\PY@do{#2}}

\expandafter\def\csname PY@tok@gd\endcsname{\def\PY@tc##1{\textcolor[rgb]{0.63,0.00,0.00}{##1}}}
\expandafter\def\csname PY@tok@gu\endcsname{\let\PY@bf=\textbf\def\PY@tc##1{\textcolor[rgb]{0.50,0.00,0.50}{##1}}}
\expandafter\def\csname PY@tok@gt\endcsname{\def\PY@tc##1{\textcolor[rgb]{0.00,0.27,0.87}{##1}}}
\expandafter\def\csname PY@tok@gs\endcsname{\let\PY@bf=\textbf}
\expandafter\def\csname PY@tok@gr\endcsname{\def\PY@tc##1{\textcolor[rgb]{1.00,0.00,0.00}{##1}}}
\expandafter\def\csname PY@tok@cm\endcsname{\let\PY@it=\textit\def\PY@tc##1{\textcolor[rgb]{0.25,0.50,0.50}{##1}}}
\expandafter\def\csname PY@tok@vg\endcsname{\def\PY@tc##1{\textcolor[rgb]{0.10,0.09,0.49}{##1}}}
\expandafter\def\csname PY@tok@vi\endcsname{\def\PY@tc##1{\textcolor[rgb]{0.10,0.09,0.49}{##1}}}
\expandafter\def\csname PY@tok@vm\endcsname{\def\PY@tc##1{\textcolor[rgb]{0.10,0.09,0.49}{##1}}}
\expandafter\def\csname PY@tok@mh\endcsname{\def\PY@tc##1{\textcolor[rgb]{0.40,0.40,0.40}{##1}}}
\expandafter\def\csname PY@tok@cs\endcsname{\let\PY@it=\textit\def\PY@tc##1{\textcolor[rgb]{0.25,0.50,0.50}{##1}}}
\expandafter\def\csname PY@tok@ge\endcsname{\let\PY@it=\textit}
\expandafter\def\csname PY@tok@vc\endcsname{\def\PY@tc##1{\textcolor[rgb]{0.10,0.09,0.49}{##1}}}
\expandafter\def\csname PY@tok@il\endcsname{\def\PY@tc##1{\textcolor[rgb]{0.40,0.40,0.40}{##1}}}
\expandafter\def\csname PY@tok@go\endcsname{\def\PY@tc##1{\textcolor[rgb]{0.53,0.53,0.53}{##1}}}
\expandafter\def\csname PY@tok@cp\endcsname{\def\PY@tc##1{\textcolor[rgb]{0.74,0.48,0.00}{##1}}}
\expandafter\def\csname PY@tok@gi\endcsname{\def\PY@tc##1{\textcolor[rgb]{0.00,0.63,0.00}{##1}}}
\expandafter\def\csname PY@tok@gh\endcsname{\let\PY@bf=\textbf\def\PY@tc##1{\textcolor[rgb]{0.00,0.00,0.50}{##1}}}
\expandafter\def\csname PY@tok@ni\endcsname{\let\PY@bf=\textbf\def\PY@tc##1{\textcolor[rgb]{0.60,0.60,0.60}{##1}}}
\expandafter\def\csname PY@tok@nl\endcsname{\def\PY@tc##1{\textcolor[rgb]{0.63,0.63,0.00}{##1}}}
\expandafter\def\csname PY@tok@nn\endcsname{\let\PY@bf=\textbf\def\PY@tc##1{\textcolor[rgb]{0.00,0.00,1.00}{##1}}}
\expandafter\def\csname PY@tok@no\endcsname{\def\PY@tc##1{\textcolor[rgb]{0.53,0.00,0.00}{##1}}}
\expandafter\def\csname PY@tok@na\endcsname{\def\PY@tc##1{\textcolor[rgb]{0.49,0.56,0.16}{##1}}}
\expandafter\def\csname PY@tok@nb\endcsname{\def\PY@tc##1{\textcolor[rgb]{0.00,0.50,0.00}{##1}}}
\expandafter\def\csname PY@tok@nc\endcsname{\let\PY@bf=\textbf\def\PY@tc##1{\textcolor[rgb]{0.00,0.00,1.00}{##1}}}
\expandafter\def\csname PY@tok@nd\endcsname{\def\PY@tc##1{\textcolor[rgb]{0.67,0.13,1.00}{##1}}}
\expandafter\def\csname PY@tok@ne\endcsname{\let\PY@bf=\textbf\def\PY@tc##1{\textcolor[rgb]{0.82,0.25,0.23}{##1}}}
\expandafter\def\csname PY@tok@nf\endcsname{\def\PY@tc##1{\textcolor[rgb]{0.00,0.00,1.00}{##1}}}
\expandafter\def\csname PY@tok@si\endcsname{\let\PY@bf=\textbf\def\PY@tc##1{\textcolor[rgb]{0.73,0.40,0.53}{##1}}}
\expandafter\def\csname PY@tok@s2\endcsname{\def\PY@tc##1{\textcolor[rgb]{0.73,0.13,0.13}{##1}}}
\expandafter\def\csname PY@tok@nt\endcsname{\let\PY@bf=\textbf\def\PY@tc##1{\textcolor[rgb]{0.00,0.50,0.00}{##1}}}
\expandafter\def\csname PY@tok@nv\endcsname{\def\PY@tc##1{\textcolor[rgb]{0.10,0.09,0.49}{##1}}}
\expandafter\def\csname PY@tok@s1\endcsname{\def\PY@tc##1{\textcolor[rgb]{0.73,0.13,0.13}{##1}}}
\expandafter\def\csname PY@tok@dl\endcsname{\def\PY@tc##1{\textcolor[rgb]{0.73,0.13,0.13}{##1}}}
\expandafter\def\csname PY@tok@ch\endcsname{\let\PY@it=\textit\def\PY@tc##1{\textcolor[rgb]{0.25,0.50,0.50}{##1}}}
\expandafter\def\csname PY@tok@m\endcsname{\def\PY@tc##1{\textcolor[rgb]{0.40,0.40,0.40}{##1}}}
\expandafter\def\csname PY@tok@gp\endcsname{\let\PY@bf=\textbf\def\PY@tc##1{\textcolor[rgb]{0.00,0.00,0.50}{##1}}}
\expandafter\def\csname PY@tok@sh\endcsname{\def\PY@tc##1{\textcolor[rgb]{0.73,0.13,0.13}{##1}}}
\expandafter\def\csname PY@tok@ow\endcsname{\let\PY@bf=\textbf\def\PY@tc##1{\textcolor[rgb]{0.67,0.13,1.00}{##1}}}
\expandafter\def\csname PY@tok@sx\endcsname{\def\PY@tc##1{\textcolor[rgb]{0.00,0.50,0.00}{##1}}}
\expandafter\def\csname PY@tok@bp\endcsname{\def\PY@tc##1{\textcolor[rgb]{0.00,0.50,0.00}{##1}}}
\expandafter\def\csname PY@tok@c1\endcsname{\let\PY@it=\textit\def\PY@tc##1{\textcolor[rgb]{0.25,0.50,0.50}{##1}}}
\expandafter\def\csname PY@tok@fm\endcsname{\def\PY@tc##1{\textcolor[rgb]{0.00,0.00,1.00}{##1}}}
\expandafter\def\csname PY@tok@o\endcsname{\def\PY@tc##1{\textcolor[rgb]{0.40,0.40,0.40}{##1}}}
\expandafter\def\csname PY@tok@kc\endcsname{\let\PY@bf=\textbf\def\PY@tc##1{\textcolor[rgb]{0.00,0.50,0.00}{##1}}}
\expandafter\def\csname PY@tok@c\endcsname{\let\PY@it=\textit\def\PY@tc##1{\textcolor[rgb]{0.25,0.50,0.50}{##1}}}
\expandafter\def\csname PY@tok@mf\endcsname{\def\PY@tc##1{\textcolor[rgb]{0.40,0.40,0.40}{##1}}}
\expandafter\def\csname PY@tok@err\endcsname{\def\PY@bc##1{\setlength{\fboxsep}{0pt}\fcolorbox[rgb]{1.00,0.00,0.00}{1,1,1}{\strut ##1}}}
\expandafter\def\csname PY@tok@mb\endcsname{\def\PY@tc##1{\textcolor[rgb]{0.40,0.40,0.40}{##1}}}
\expandafter\def\csname PY@tok@ss\endcsname{\def\PY@tc##1{\textcolor[rgb]{0.10,0.09,0.49}{##1}}}
\expandafter\def\csname PY@tok@sr\endcsname{\def\PY@tc##1{\textcolor[rgb]{0.73,0.40,0.53}{##1}}}
\expandafter\def\csname PY@tok@mo\endcsname{\def\PY@tc##1{\textcolor[rgb]{0.40,0.40,0.40}{##1}}}
\expandafter\def\csname PY@tok@kd\endcsname{\let\PY@bf=\textbf\def\PY@tc##1{\textcolor[rgb]{0.00,0.50,0.00}{##1}}}
\expandafter\def\csname PY@tok@mi\endcsname{\def\PY@tc##1{\textcolor[rgb]{0.40,0.40,0.40}{##1}}}
\expandafter\def\csname PY@tok@kn\endcsname{\let\PY@bf=\textbf\def\PY@tc##1{\textcolor[rgb]{0.00,0.50,0.00}{##1}}}
\expandafter\def\csname PY@tok@cpf\endcsname{\let\PY@it=\textit\def\PY@tc##1{\textcolor[rgb]{0.25,0.50,0.50}{##1}}}
\expandafter\def\csname PY@tok@kr\endcsname{\let\PY@bf=\textbf\def\PY@tc##1{\textcolor[rgb]{0.00,0.50,0.00}{##1}}}
\expandafter\def\csname PY@tok@s\endcsname{\def\PY@tc##1{\textcolor[rgb]{0.73,0.13,0.13}{##1}}}
\expandafter\def\csname PY@tok@kp\endcsname{\def\PY@tc##1{\textcolor[rgb]{0.00,0.50,0.00}{##1}}}
\expandafter\def\csname PY@tok@w\endcsname{\def\PY@tc##1{\textcolor[rgb]{0.73,0.73,0.73}{##1}}}
\expandafter\def\csname PY@tok@kt\endcsname{\def\PY@tc##1{\textcolor[rgb]{0.69,0.00,0.25}{##1}}}
\expandafter\def\csname PY@tok@sc\endcsname{\def\PY@tc##1{\textcolor[rgb]{0.73,0.13,0.13}{##1}}}
\expandafter\def\csname PY@tok@sb\endcsname{\def\PY@tc##1{\textcolor[rgb]{0.73,0.13,0.13}{##1}}}
\expandafter\def\csname PY@tok@sa\endcsname{\def\PY@tc##1{\textcolor[rgb]{0.73,0.13,0.13}{##1}}}
\expandafter\def\csname PY@tok@k\endcsname{\let\PY@bf=\textbf\def\PY@tc##1{\textcolor[rgb]{0.00,0.50,0.00}{##1}}}
\expandafter\def\csname PY@tok@se\endcsname{\let\PY@bf=\textbf\def\PY@tc##1{\textcolor[rgb]{0.73,0.40,0.13}{##1}}}
\expandafter\def\csname PY@tok@sd\endcsname{\let\PY@it=\textit\def\PY@tc##1{\textcolor[rgb]{0.73,0.13,0.13}{##1}}}

\def\PYZbs{\char`\\}
\def\PYZus{\char`\_}
\def\PYZob{\char`\{}
\def\PYZcb{\char`\}}
\def\PYZca{\char`\^}
\def\PYZam{\char`\&}
\def\PYZlt{\char`\<}
\def\PYZgt{\char`\>}
\def\PYZsh{\char`\#}
\def\PYZpc{\char`\%}
\def\PYZdl{\char`\$}
\def\PYZhy{\char`\-}
\def\PYZsq{\char`\'}
\def\PYZdq{\char`\"}
\def\PYZti{\char`\~}
% for compatibility with earlier versions
\def\PYZat{@}
\def\PYZlb{[}
\def\PYZrb{]}
\makeatother


    % For linebreaks inside Verbatim environment from package fancyvrb. 
    \makeatletter
        \newbox\Wrappedcontinuationbox 
        \newbox\Wrappedvisiblespacebox 
        \newcommand*\Wrappedvisiblespace {\textcolor{red}{\textvisiblespace}} 
        \newcommand*\Wrappedcontinuationsymbol {\textcolor{red}{\llap{\tiny$\m@th\hookrightarrow$}}} 
        \newcommand*\Wrappedcontinuationindent {3ex } 
        \newcommand*\Wrappedafterbreak {\kern\Wrappedcontinuationindent\copy\Wrappedcontinuationbox} 
        % Take advantage of the already applied Pygments mark-up to insert 
        % potential linebreaks for TeX processing. 
        %        {, <, #, %, $, ' and ": go to next line. 
        %        _, }, ^, &, >, - and ~: stay at end of broken line. 
        % Use of \textquotesingle for straight quote. 
        \newcommand*\Wrappedbreaksatspecials {% 
            \def\PYGZus{\discretionary{\char`\_}{\Wrappedafterbreak}{\char`\_}}% 
            \def\PYGZob{\discretionary{}{\Wrappedafterbreak\char`\{}{\char`\{}}% 
            \def\PYGZcb{\discretionary{\char`\}}{\Wrappedafterbreak}{\char`\}}}% 
            \def\PYGZca{\discretionary{\char`\^}{\Wrappedafterbreak}{\char`\^}}% 
            \def\PYGZam{\discretionary{\char`\&}{\Wrappedafterbreak}{\char`\&}}% 
            \def\PYGZlt{\discretionary{}{\Wrappedafterbreak\char`\<}{\char`\<}}% 
            \def\PYGZgt{\discretionary{\char`\>}{\Wrappedafterbreak}{\char`\>}}% 
            \def\PYGZsh{\discretionary{}{\Wrappedafterbreak\char`\#}{\char`\#}}% 
            \def\PYGZpc{\discretionary{}{\Wrappedafterbreak\char`\%}{\char`\%}}% 
            \def\PYGZdl{\discretionary{}{\Wrappedafterbreak\char`\$}{\char`\$}}% 
            \def\PYGZhy{\discretionary{\char`\-}{\Wrappedafterbreak}{\char`\-}}% 
            \def\PYGZsq{\discretionary{}{\Wrappedafterbreak\textquotesingle}{\textquotesingle}}% 
            \def\PYGZdq{\discretionary{}{\Wrappedafterbreak\char`\"}{\char`\"}}% 
            \def\PYGZti{\discretionary{\char`\~}{\Wrappedafterbreak}{\char`\~}}% 
        } 
        % Some characters . , ; ? ! / are not pygmentized. 
        % This macro makes them "active" and they will insert potential linebreaks 
        \newcommand*\Wrappedbreaksatpunct {% 
            \lccode`\~`\.\lowercase{\def~}{\discretionary{\hbox{\char`\.}}{\Wrappedafterbreak}{\hbox{\char`\.}}}% 
            \lccode`\~`\,\lowercase{\def~}{\discretionary{\hbox{\char`\,}}{\Wrappedafterbreak}{\hbox{\char`\,}}}% 
            \lccode`\~`\;\lowercase{\def~}{\discretionary{\hbox{\char`\;}}{\Wrappedafterbreak}{\hbox{\char`\;}}}% 
            \lccode`\~`\:\lowercase{\def~}{\discretionary{\hbox{\char`\:}}{\Wrappedafterbreak}{\hbox{\char`\:}}}% 
            \lccode`\~`\?\lowercase{\def~}{\discretionary{\hbox{\char`\?}}{\Wrappedafterbreak}{\hbox{\char`\?}}}% 
            \lccode`\~`\!\lowercase{\def~}{\discretionary{\hbox{\char`\!}}{\Wrappedafterbreak}{\hbox{\char`\!}}}% 
            \lccode`\~`\/\lowercase{\def~}{\discretionary{\hbox{\char`\/}}{\Wrappedafterbreak}{\hbox{\char`\/}}}% 
            \catcode`\.\active
            \catcode`\,\active 
            \catcode`\;\active
            \catcode`\:\active
            \catcode`\?\active
            \catcode`\!\active
            \catcode`\/\active 
            \lccode`\~`\~ 	
        }
    \makeatother

    \let\OriginalVerbatim=\Verbatim
    \makeatletter
    \renewcommand{\Verbatim}[1][1]{%
        %\parskip\z@skip
        \sbox\Wrappedcontinuationbox {\Wrappedcontinuationsymbol}%
        \sbox\Wrappedvisiblespacebox {\FV@SetupFont\Wrappedvisiblespace}%
        \def\FancyVerbFormatLine ##1{\hsize\linewidth
            \vtop{\raggedright\hyphenpenalty\z@\exhyphenpenalty\z@
                \doublehyphendemerits\z@\finalhyphendemerits\z@
                \strut ##1\strut}%
        }%
        % If the linebreak is at a space, the latter will be displayed as visible
        % space at end of first line, and a continuation symbol starts next line.
        % Stretch/shrink are however usually zero for typewriter font.
        \def\FV@Space {%
            \nobreak\hskip\z@ plus\fontdimen3\font minus\fontdimen4\font
            \discretionary{\copy\Wrappedvisiblespacebox}{\Wrappedafterbreak}
            {\kern\fontdimen2\font}%
        }%
        
        % Allow breaks at special characters using \PYG... macros.
        \Wrappedbreaksatspecials
        % Breaks at punctuation characters . , ; ? ! and / need catcode=\active 	
        \OriginalVerbatim[#1,codes*=\Wrappedbreaksatpunct]%
    }
    \makeatother

    % Exact colors from NB
    \definecolor{incolor}{HTML}{303F9F}
    \definecolor{outcolor}{HTML}{D84315}
    \definecolor{cellborder}{HTML}{CFCFCF}
    \definecolor{cellbackground}{HTML}{F7F7F7}
    
    % prompt
    \makeatletter
    \newcommand{\boxspacing}{\kern\kvtcb@left@rule\kern\kvtcb@boxsep}
    \makeatother
    \newcommand{\prompt}[4]{
        \ttfamily\llap{{\color{#2}[#3]:\hspace{3pt}#4}}\vspace{-\baselineskip}
    }
    

    
    % Prevent overflowing lines due to hard-to-break entities
    \sloppy 
    % Setup hyperref package
    \hypersetup{
      breaklinks=true,  % so long urls are correctly broken across lines
      colorlinks=true,
      urlcolor=urlcolor,
      linkcolor=linkcolor,
      citecolor=citecolor,
      }
    % Slightly bigger margins than the latex defaults
    
    \geometry{verbose,tmargin=1in,bmargin=1in,lmargin=1in,rmargin=1in}
    
    

\begin{document}
    
    \maketitle
    
    

    
    \hypertarget{introduction-to-scientific-computing-with-python}{%
\section{Introduction to scientific computing with
Python}\label{introduction-to-scientific-computing-with-python}}

    J.R. Johansson (jrjohansson at gmail.com)

The latest version of this
\href{http://ipython.org/notebook.html}{IPython notebook} lecture is
available at
\url{http://github.com/jrjohansson/scientific-python-lectures}.

The other notebooks in this lecture series are indexed at
\url{http://jrjohansson.github.io}.

    \hypertarget{the-role-of-computing-in-science}{%
\subsection{The role of computing in
science}\label{the-role-of-computing-in-science}}

    Science has traditionally been divided into experimental and theoretical
disciplines, but during the last several decades computing has emerged
as a very important part of science. Scientific computing is often
closely related to theory, but it also has many characteristics in
common with experimental work. It is therefore often viewed as a new
third branch of science. In most fields of science, computational work
is an important complement to both experiments and theory, and nowadays
a vast majority of both experimental and theoretical papers involve some
numerical calculations, simulations or computer modeling.

In experimental and theoretical sciences there are well established
codes of conducts for how results and methods are published and made
available to other scientists. For example, in theoretical sciences,
derivations, proofs and other results are published in full detail, or
made available upon request. Likewise, in experimental sciences, the
methods used and the results are published, and all experimental data
should be available upon request. It is considered unscientific to
withhold crucial details in a theoretical proof or experimental method,
that would hinder other scientists from replicating and reproducing the
results.

In computational sciences there are not yet any well established
guidelines for how source code and generated data should be handled. For
example, it is relatively rare that source code used in simulations for
published papers are provided to readers, in contrast to the open nature
of experimental and theoretical work. And it is not uncommon that source
code for simulation software is withheld and considered a competitive
advantage (or unnecessary to publish).

However, this issue has recently started to attract increasing
attention, and a number of editorials in high-profile journals have
called for increased openness in computational sciences. Some
prestigious journals, including Science, have even started to demand of
authors to provide the source code for simulation software used in
publications to readers upon request.

Discussions are also ongoing on how to facilitate distribution of
scientific software, for example as supplementary materials to
scientific papers.

    \hypertarget{references}{%
\subsubsection{References}\label{references}}

    \begin{itemize}
\item
  \href{http://dx.doi.org/10.1126/science.1213847}{Reproducible Research
  in Computational Science}, Roger D. Peng, Science 334, 1226 (2011).
\item
  \href{http://dx.doi.org/10.1126/science.1218263}{Shining Light into
  Black Boxes}, A. Morin et al., Science 336, 159-160 (2012).
\item
  \href{http://dx.doi.org/doi:10.1038/nature10836}{The case for open
  computer programs}, D.C. Ince, Nature 482, 485 (2012).
\end{itemize}

    \hypertarget{requirements-on-scientific-computing}{%
\subsection{Requirements on scientific
computing}\label{requirements-on-scientific-computing}}

    \textbf{Replication} and \textbf{reproducibility} are two of the
cornerstones in the scientific method. With respect to numerical work,
complying with these concepts have the following practical implications:

\begin{itemize}
\item
  Replication: An author of a scientific paper that involves numerical
  calculations should be able to rerun the simulations and replicate the
  results upon request. Other scientist should also be able to perform
  the same calculations and obtain the same results, given the
  information about the methods used in a publication.
\item
  Reproducibility: The results obtained from numerical simulations
  should be reproducible with an independent implementation of the
  method, or using a different method altogether.
\end{itemize}

In summary: A sound scientific result should be reproducible, and a
sound scientific study should be replicable.

To achieve these goals, we need to:

\begin{itemize}
\item
  Keep and take note of \emph{exactly} which source code and version
  that was used to produce data and figures in published papers.
\item
  Record information of which version of external software that was
  used. Keep access to the environment that was used.
\item
  Make sure that old codes and notes are backed up and kept for future
  reference.
\item
  Be ready to give additional information about the methods used, and
  perhaps also the simulation codes, to an interested reader who
  requests it (even years after the paper was published!).
\item
  Ideally codes should be published online, to make it easier for other
  scientists interested in the codes to access it.
\end{itemize}

    \hypertarget{tools-for-managing-source-code}{%
\subsubsection{Tools for managing source
code}\label{tools-for-managing-source-code}}

    Ensuring replicability and reprodicibility of scientific simulations is
a \emph{complicated problem}, but there are good tools to help with
this:

\begin{itemize}
\tightlist
\item
  Revision Control System (RCS) software.

  \begin{itemize}
  \tightlist
  \item
    Good choices include:

    \begin{itemize}
    \tightlist
    \item
      git - http://git-scm.com
    \item
      mercurial - http://mercurial.selenic.com. Also known as
      \texttt{hg}.
    \item
      subversion - http://subversion.apache.org. Also known as
      \texttt{svn}.
    \end{itemize}
  \end{itemize}
\item
  Online repositories for source code. Available as both private and
  public repositories.

  \begin{itemize}
  \tightlist
  \item
    Some good alternatives are

    \begin{itemize}
    \tightlist
    \item
      Github - http://www.github.com
    \item
      Bitbucket - http://www.bitbucket.com
    \item
      Privately hosted repositories on the university's or department's
      servers.
    \end{itemize}
  \end{itemize}
\end{itemize}

\hypertarget{note}{%
\paragraph{Note}\label{note}}

Repositories are also excellent for version controlling manuscripts,
figures, thesis files, data files, lab logs, etc. Basically for any
digital content that must be preserved and is frequently updated. Again,
both public and private repositories are readily available. They are
also excellent collaboration tools!

    \hypertarget{what-is-python}{%
\subsection{What is Python?}\label{what-is-python}}

    \href{http://www.python.org/}{Python} is a modern, general-purpose,
object-oriented, high-level programming language.

General characteristics of Python:

\begin{itemize}
\tightlist
\item
  \textbf{clean and simple language:} Easy-to-read and intuitive code,
  easy-to-learn minimalistic syntax, maintainability scales well with
  size of projects.
\item
  \textbf{expressive language:} Fewer lines of code, fewer bugs, easier
  to maintain.
\end{itemize}

Technical details:

\begin{itemize}
\tightlist
\item
  \textbf{dynamically typed:} No need to define the type of variables,
  function arguments or return types.
\item
  \textbf{automatic memory management:} No need to explicitly allocate
  and deallocate memory for variables and data arrays. No memory leak
  bugs.
\item
  \textbf{interpreted:} No need to compile the code. The Python
  interpreter reads and executes the python code directly.
\end{itemize}

Advantages:

\begin{itemize}
\tightlist
\item
  The main advantage is ease of programming, minimizing the time
  required to develop, debug and maintain the code.
\item
  Well designed language that encourage many good programming practices:
\item
  Modular and object-oriented programming, good system for packaging and
  re-use of code. This often results in more transparent, maintainable
  and bug-free code.
\item
  Documentation tightly integrated with the code.
\item
  A large standard library, and a large collection of add-on packages.
\end{itemize}

Disadvantages:

\begin{itemize}
\tightlist
\item
  Since Python is an interpreted and dynamically typed programming
  language, the execution of python code can be slow compared to
  compiled statically typed programming languages, such as C and
  Fortran.
\item
  Somewhat decentralized, with different environment, packages and
  documentation spread out at different places. Can make it harder to
  get started.
\end{itemize}

    \hypertarget{what-makes-python-suitable-for-scientific-computing}{%
\subsection{What makes python suitable for scientific
computing?}\label{what-makes-python-suitable-for-scientific-computing}}

    \begin{itemize}
\item
  Python has a strong position in scientific computing:

  \begin{itemize}
  \tightlist
  \item
    Large community of users, easy to find help and documentation.
  \end{itemize}
\item
  Extensive ecosystem of scientific libraries and environments

  \begin{itemize}
  \tightlist
  \item
    numpy: http://numpy.scipy.org - Numerical Python
  \item
    scipy: http://www.scipy.org - Scientific Python
  \item
    matplotlib: http://www.matplotlib.org - graphics library
  \end{itemize}
\item
  Great performance due to close integration with time-tested and highly
  optimized codes written in C and Fortran:

  \begin{itemize}
  \tightlist
  \item
    blas, atlas blas, lapack, arpack, Intel MKL, \ldots{}
  \end{itemize}
\item
  Good support for

  \begin{itemize}
  \tightlist
  \item
    Parallel processing with processes and threads
  \item
    Interprocess communication (MPI)
  \item
    GPU computing (OpenCL and CUDA)
  \end{itemize}
\item
  Readily available and suitable for use on high-performance computing
  clusters.
\item
  No license costs, no unnecessary use of research budget.
\end{itemize}

    \hypertarget{the-scientific-python-software-stack}{%
\subsubsection{The scientific python software
stack}\label{the-scientific-python-software-stack}}

    

    \hypertarget{python-environments}{%
\subsubsection{Python environments}\label{python-environments}}

    Python is not only a programming language, but often also refers to the
standard implementation of the interpreter (technically referred to as
\href{http://en.wikipedia.org/wiki/CPython}{CPython}) that actually runs
the python code on a computer.

There are also many different environments through which the python
interpreter can be used. Each environment has different advantages and
is suitable for different workflows. One strength of python is that it
is versatile and can be used in complementary ways, but it can be
confusing for beginners so we will start with a brief survey of python
environments that are useful for scientific computing.

    \hypertarget{python-interpreter}{%
\subsubsection{Python interpreter}\label{python-interpreter}}

    The standard way to use the Python programming language is to use the
Python interpreter to run python code. The python interpreter is a
program that reads and execute the python code in files passed to it as
arguments. At the command prompt, the command \texttt{python} is used to
invoke the Python interpreter.

For example, to run a file \texttt{my-program.py} that contains python
code from the command prompt, use::

\begin{verbatim}
$ python my-program.py
\end{verbatim}

We can also start the interpreter by simply typing \texttt{python} at
the command line, and interactively type python code into the
interpreter.

This is often how we want to work when developing scientific
applications, or when doing small calculations. But the standard python
interpreter is not very convenient for this kind of work, due to a
number of limitations.

    \hypertarget{ipython}{%
\subsubsection{IPython}\label{ipython}}

    IPython is an interactive shell that addresses the limitation of the
standard python interpreter, and it is a work-horse for scientific use
of python. It provides an interactive prompt to the python interpreter
with a greatly improved user-friendliness.

Some of the many useful features of IPython includes:

\begin{itemize}
\tightlist
\item
  Command history, which can be browsed with the up and down arrows on
  the keyboard.
\item
  Tab auto-completion.
\item
  In-line editing of code.
\item
  Object introspection, and automatic extract of documentation strings
  from python objects like classes and functions.
\item
  Good interaction with operating system shell.
\item
  Support for multiple parallel back-end processes, that can run on
  computing clusters or cloud services like Amazon EC2.
\end{itemize}

    \hypertarget{ipython-notebook}{%
\subsubsection{IPython notebook}\label{ipython-notebook}}

    \href{http://ipython.org/notebook.html}{IPython notebook} is an
HTML-based notebook environment for Python, similar to Mathematica or
Maple. It is based on the IPython shell, but provides a cell-based
environment with great interactivity, where calculations can be
organized and documented in a structured way.

Although using a web browser as graphical interface, IPython notebooks
are usually run locally, from the same computer that run the browser. To
start a new IPython notebook session, run the following command:

\begin{verbatim}
$ ipython notebook
\end{verbatim}

from a directory where you want the notebooks to be stored. This will
open a new browser window (or a new tab in an existing window) with an
index page where existing notebooks are shown and from which new
notebooks can be created.

    \hypertarget{spyder}{%
\subsubsection{Spyder}\label{spyder}}

    \href{http://code.google.com/p/spyderlib/}{Spyder} is a MATLAB-like IDE
for scientific computing with python. It has the many advantages of a
traditional IDE environment, for example that everything from code
editing, execution and debugging is carried out in a single environment,
and work on different calculations can be organized as projects in the
IDE environment.

Some advantages of Spyder:

\begin{itemize}
\tightlist
\item
  Powerful code editor, with syntax high-lighting, dynamic code
  introspection and integration with the python debugger.
\item
  Variable explorer, IPython command prompt.
\item
  Integrated documentation and help.
\end{itemize}

    \hypertarget{versions-of-python}{%
\subsection{Versions of Python}\label{versions-of-python}}

    There are currently two versions of python: Python 2 and Python 3.
Python 3 will eventually supercede Python 2, but it is not
backward-compatible with Python 2. A lot of existing python code and
packages has been written for Python 2, and it is still the most
wide-spread version. For these lectures either version will be fine, but
it is probably easier to stick with Python 2 for now, because it is more
readily available via prebuilt packages and binary installers.

To see which version of Python you have, run

\begin{verbatim}
$ python --version
Python 2.7.3
$ python3.2 --version
Python 3.2.3
\end{verbatim}

Several versions of Python can be installed in parallel, as shown above.

    \hypertarget{installation}{%
\subsection{Installation}\label{installation}}

    \hypertarget{conda}{%
\subsubsection{Conda}\label{conda}}

    The best way set-up an scientific Python environment is to use the
cross-platform package manager \texttt{conda} from Continuum Analytics.
First download and install miniconda
http://conda.pydata.org/miniconda.html or Anaconda (see below). Next, to
install the required libraries for these notebooks, simply run:

\begin{verbatim}
$ conda install ipython ipython-notebook spyder numpy scipy sympy matplotlib cython
\end{verbatim}

This should be sufficient to get a working environment on any platform
supported by \texttt{conda}.

    \hypertarget{linux}{%
\subsubsection{Linux}\label{linux}}

    In Ubuntu Linux, to installing python and all the requirements run:

\begin{verbatim}
$ sudo apt-get install python ipython ipython-notebook
\end{verbatim}

\$ sudo apt-get install python-numpy python-scipy python-matplotlib
python-sympy \$ sudo apt-get install spyder

    \hypertarget{macos-x}{%
\subsubsection{MacOS X}\label{macos-x}}

    \emph{Macports}

Python is included by default in Mac OS X, but for our purposes it will
be useful to install a new python environment using
\href{http://www.macports.org/}{Macports}, because it makes it much
easier to install all the required additional packages. Using Macports,
we can install what we need with:

\begin{verbatim}
$ sudo port install py27-ipython +pyside+notebook+parallel+scientific
$ sudo port install py27-scipy py27-matplotlib py27-sympy
$ sudo port install py27-spyder
\end{verbatim}

These will associate the commands \texttt{python} and \texttt{ipython}
with the versions installed via macports (instead of the one that is
shipped with Mac OS X), run the following commands:

\begin{verbatim}
$ sudo port select python python27
$ sudo port select ipython ipython27
\end{verbatim}

\emph{Fink}

Or, alternatively, you can use the
\href{http://www.finkproject.org/}{Fink} package manager. After
installing Fink, use the following command to install python and the
packages that we need:

\begin{verbatim}
$ sudo fink install python27 ipython-py27 numpy-py27 matplotlib-py27 scipy-py27 sympy-py27
$ sudo fink install spyder-mac-py27
\end{verbatim}

    \hypertarget{windows}{%
\subsubsection{Windows}\label{windows}}

    Windows lacks a good packaging system, so the easiest way to setup a
Python environment is to install a pre-packaged distribution. Some good
alternatives are:

\begin{itemize}
\tightlist
\item
  \href{http://www.enthought.com/products/epd.php}{Enthought Python
  Distribution}. EPD is a commercial product but is available free for
  academic use.
\item
  \href{http://continuum.io/downloads.html}{Anaconda}. The Anaconda
  Python distribution comes with many scientific computing and data
  science packages and is free, including for commercial use and
  redistribution. It also has add-on products such as Accelerate, IOPro,
  and MKL Optimizations, which have free trials and are free for
  academic use.
\item
  \href{http://code.google.com/p/pythonxy/}{Python(x,y)}. Fully open
  source.
\end{itemize}

\hypertarget{note}{%
\paragraph{Note}\label{note}}

EPD and Anaconda are also available for Linux and Max OS X.

    \hypertarget{further-reading}{%
\subsection{Further reading}\label{further-reading}}

    \begin{itemize}
\tightlist
\item
  \href{http://www.python.org}{Python}. The official Python web site.
\item
  \href{http://docs.python.org/2/tutorial}{Python tutorials}. The
  official Python tutorials.
\item
  \href{http://www.greenteapress.com/thinkpython}{Think Python}. A free
  book on Python.
\end{itemize}

    \hypertarget{python-and-module-versions}{%
\subsection{Python and module
versions}\label{python-and-module-versions}}

    Since there are several different versions of Python and each Python
package has its own release cycle and version number (for example scipy,
numpy, matplotlib, etc., which we installed above and will discuss in
detail in the following lectures), it is important for the
reproducibility of an IPython notebook to record the versions of all
these different software packages. If this is done properly it will be
easy to reproduce the environment that was used to run a notebook, but
if not it can be hard to know what was used to produce the results in a
notebook.

To encourage the practice of recording Python and module versions in
notebooks, I've created a simple IPython extension that produces a table
with versions numbers of selected software components. I believe that it
is a good practice to include this kind of table in every notebook you
create.

To install this IPython extension, use
\texttt{pip\ install\ version\_information}:

    \begin{tcolorbox}[breakable, size=fbox, boxrule=1pt, pad at break*=1mm,colback=cellbackground, colframe=cellborder]
\prompt{In}{incolor}{5}{\boxspacing}
\begin{Verbatim}[commandchars=\\\{\}]
\PY{c+c1}{\PYZsh{} you only need to do this once}
\PY{o}{!}pip3 install \PYZhy{}\PYZhy{}upgrade version\PYZus{}information
\end{Verbatim}
\end{tcolorbox}

    \begin{Verbatim}[commandchars=\\\{\}]
Defaulting to user installation because normal site-packages is not writeable
Requirement already up-to-date: version\_information in c:\textbackslash{}program
files\textbackslash{}python38\textbackslash{}lib\textbackslash{}site-packages (1.0.3)
    \end{Verbatim}

    or alternatively run (deprecated method):
\# you only need to do this once
%install_ext http://raw.github.com/jrjohansson/version_information/master/version_information.py
    Now, to load the extension and produce the version table

    \begin{tcolorbox}[breakable, size=fbox, boxrule=1pt, pad at break*=1mm,colback=cellbackground, colframe=cellborder]
\prompt{In}{incolor}{6}{\boxspacing}
\begin{Verbatim}[commandchars=\\\{\}]
\PY{o}{\PYZpc{}}\PY{k}{load\PYZus{}ext} version\PYZus{}information

\PY{o}{\PYZpc{}}\PY{k}{version\PYZus{}information} numpy, scipy, matplotlib, sympy, version\PYZus{}information
\end{Verbatim}
\end{tcolorbox}

    \begin{Verbatim}[commandchars=\\\{\}]
The version\_information extension is already loaded. To reload it, use:
  \%reload\_ext version\_information
    \end{Verbatim}
 
            
\prompt{Out}{outcolor}{6}{}
    
    \begin{tabular}{|l|l|}\hline
{\bf Software} & {\bf Version} \\ \hline\hline
Python & 3.8.2 64bit [MSC v.1916 64 bit (AMD64)] \\ \hline
IPython & 7.15.0 \\ \hline
OS & Windows 10 10.0.19041 SP0 \\ \hline
numpy & 1.18.5 \\ \hline
scipy & 1.4.1 \\ \hline
matplotlib & 3.2.1 \\ \hline
sympy & 1.5.1 \\ \hline
version\_information & 1.0.3 \\ \hline
\hline \multicolumn{2}{|l|}{Fri Jun 19 10:36:26 2020 India Standard Time} \\ \hline
\end{tabular}


    


    % Add a bibliography block to the postdoc
    
    
    
\end{document}
