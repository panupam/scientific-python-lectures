\documentclass[11pt]{article}

    \usepackage[breakable]{tcolorbox}
    \usepackage{parskip} % Stop auto-indenting (to mimic markdown behaviour)
    
    \usepackage{iftex}
    \ifPDFTeX
    	\usepackage[T1]{fontenc}
    	\usepackage{mathpazo}
    \else
    	\usepackage{fontspec}
    \fi

    % Basic figure setup, for now with no caption control since it's done
    % automatically by Pandoc (which extracts ![](path) syntax from Markdown).
    \usepackage{graphicx}
    % Maintain compatibility with old templates. Remove in nbconvert 6.0
    \let\Oldincludegraphics\includegraphics
    % Ensure that by default, figures have no caption (until we provide a
    % proper Figure object with a Caption API and a way to capture that
    % in the conversion process - todo).
    \usepackage{caption}
    \DeclareCaptionFormat{nocaption}{}
    \captionsetup{format=nocaption,aboveskip=0pt,belowskip=0pt}

    \usepackage[Export]{adjustbox} % Used to constrain images to a maximum size
    \adjustboxset{max size={0.9\linewidth}{0.9\paperheight}}
    \usepackage{float}
    \floatplacement{figure}{H} % forces figures to be placed at the correct location
    \usepackage{xcolor} % Allow colors to be defined
    \usepackage{enumerate} % Needed for markdown enumerations to work
    \usepackage{geometry} % Used to adjust the document margins
    \usepackage{amsmath} % Equations
    \usepackage{amssymb} % Equations
    \usepackage{textcomp} % defines textquotesingle
    % Hack from http://tex.stackexchange.com/a/47451/13684:
    \AtBeginDocument{%
        \def\PYZsq{\textquotesingle}% Upright quotes in Pygmentized code
    }
    \usepackage{upquote} % Upright quotes for verbatim code
    \usepackage{eurosym} % defines \euro
    \usepackage[mathletters]{ucs} % Extended unicode (utf-8) support
    \usepackage{fancyvrb} % verbatim replacement that allows latex
    \usepackage{grffile} % extends the file name processing of package graphics 
                         % to support a larger range
    \makeatletter % fix for grffile with XeLaTeX
    \def\Gread@@xetex#1{%
      \IfFileExists{"\Gin@base".bb}%
      {\Gread@eps{\Gin@base.bb}}%
      {\Gread@@xetex@aux#1}%
    }
    \makeatother

    % The hyperref package gives us a pdf with properly built
    % internal navigation ('pdf bookmarks' for the table of contents,
    % internal cross-reference links, web links for URLs, etc.)
    \usepackage{hyperref}
    % The default LaTeX title has an obnoxious amount of whitespace. By default,
    % titling removes some of it. It also provides customization options.
    \usepackage{titling}
    \usepackage{longtable} % longtable support required by pandoc >1.10
    \usepackage{booktabs}  % table support for pandoc > 1.12.2
    \usepackage[inline]{enumitem} % IRkernel/repr support (it uses the enumerate* environment)
    \usepackage[normalem]{ulem} % ulem is needed to support strikethroughs (\sout)
                                % normalem makes italics be italics, not underlines
    \usepackage{mathrsfs}
    

    
    % Colors for the hyperref package
    \definecolor{urlcolor}{rgb}{0,.145,.698}
    \definecolor{linkcolor}{rgb}{.71,0.21,0.01}
    \definecolor{citecolor}{rgb}{.12,.54,.11}

    % ANSI colors
    \definecolor{ansi-black}{HTML}{3E424D}
    \definecolor{ansi-black-intense}{HTML}{282C36}
    \definecolor{ansi-red}{HTML}{E75C58}
    \definecolor{ansi-red-intense}{HTML}{B22B31}
    \definecolor{ansi-green}{HTML}{00A250}
    \definecolor{ansi-green-intense}{HTML}{007427}
    \definecolor{ansi-yellow}{HTML}{DDB62B}
    \definecolor{ansi-yellow-intense}{HTML}{B27D12}
    \definecolor{ansi-blue}{HTML}{208FFB}
    \definecolor{ansi-blue-intense}{HTML}{0065CA}
    \definecolor{ansi-magenta}{HTML}{D160C4}
    \definecolor{ansi-magenta-intense}{HTML}{A03196}
    \definecolor{ansi-cyan}{HTML}{60C6C8}
    \definecolor{ansi-cyan-intense}{HTML}{258F8F}
    \definecolor{ansi-white}{HTML}{C5C1B4}
    \definecolor{ansi-white-intense}{HTML}{A1A6B2}
    \definecolor{ansi-default-inverse-fg}{HTML}{FFFFFF}
    \definecolor{ansi-default-inverse-bg}{HTML}{000000}

    % commands and environments needed by pandoc snippets
    % extracted from the output of `pandoc -s`
    \providecommand{\tightlist}{%
      \setlength{\itemsep}{0pt}\setlength{\parskip}{0pt}}
    \DefineVerbatimEnvironment{Highlighting}{Verbatim}{commandchars=\\\{\}}
    % Add ',fontsize=\small' for more characters per line
    \newenvironment{Shaded}{}{}
    \newcommand{\KeywordTok}[1]{\textcolor[rgb]{0.00,0.44,0.13}{\textbf{{#1}}}}
    \newcommand{\DataTypeTok}[1]{\textcolor[rgb]{0.56,0.13,0.00}{{#1}}}
    \newcommand{\DecValTok}[1]{\textcolor[rgb]{0.25,0.63,0.44}{{#1}}}
    \newcommand{\BaseNTok}[1]{\textcolor[rgb]{0.25,0.63,0.44}{{#1}}}
    \newcommand{\FloatTok}[1]{\textcolor[rgb]{0.25,0.63,0.44}{{#1}}}
    \newcommand{\CharTok}[1]{\textcolor[rgb]{0.25,0.44,0.63}{{#1}}}
    \newcommand{\StringTok}[1]{\textcolor[rgb]{0.25,0.44,0.63}{{#1}}}
    \newcommand{\CommentTok}[1]{\textcolor[rgb]{0.38,0.63,0.69}{\textit{{#1}}}}
    \newcommand{\OtherTok}[1]{\textcolor[rgb]{0.00,0.44,0.13}{{#1}}}
    \newcommand{\AlertTok}[1]{\textcolor[rgb]{1.00,0.00,0.00}{\textbf{{#1}}}}
    \newcommand{\FunctionTok}[1]{\textcolor[rgb]{0.02,0.16,0.49}{{#1}}}
    \newcommand{\RegionMarkerTok}[1]{{#1}}
    \newcommand{\ErrorTok}[1]{\textcolor[rgb]{1.00,0.00,0.00}{\textbf{{#1}}}}
    \newcommand{\NormalTok}[1]{{#1}}
    
    % Additional commands for more recent versions of Pandoc
    \newcommand{\ConstantTok}[1]{\textcolor[rgb]{0.53,0.00,0.00}{{#1}}}
    \newcommand{\SpecialCharTok}[1]{\textcolor[rgb]{0.25,0.44,0.63}{{#1}}}
    \newcommand{\VerbatimStringTok}[1]{\textcolor[rgb]{0.25,0.44,0.63}{{#1}}}
    \newcommand{\SpecialStringTok}[1]{\textcolor[rgb]{0.73,0.40,0.53}{{#1}}}
    \newcommand{\ImportTok}[1]{{#1}}
    \newcommand{\DocumentationTok}[1]{\textcolor[rgb]{0.73,0.13,0.13}{\textit{{#1}}}}
    \newcommand{\AnnotationTok}[1]{\textcolor[rgb]{0.38,0.63,0.69}{\textbf{\textit{{#1}}}}}
    \newcommand{\CommentVarTok}[1]{\textcolor[rgb]{0.38,0.63,0.69}{\textbf{\textit{{#1}}}}}
    \newcommand{\VariableTok}[1]{\textcolor[rgb]{0.10,0.09,0.49}{{#1}}}
    \newcommand{\ControlFlowTok}[1]{\textcolor[rgb]{0.00,0.44,0.13}{\textbf{{#1}}}}
    \newcommand{\OperatorTok}[1]{\textcolor[rgb]{0.40,0.40,0.40}{{#1}}}
    \newcommand{\BuiltInTok}[1]{{#1}}
    \newcommand{\ExtensionTok}[1]{{#1}}
    \newcommand{\PreprocessorTok}[1]{\textcolor[rgb]{0.74,0.48,0.00}{{#1}}}
    \newcommand{\AttributeTok}[1]{\textcolor[rgb]{0.49,0.56,0.16}{{#1}}}
    \newcommand{\InformationTok}[1]{\textcolor[rgb]{0.38,0.63,0.69}{\textbf{\textit{{#1}}}}}
    \newcommand{\WarningTok}[1]{\textcolor[rgb]{0.38,0.63,0.69}{\textbf{\textit{{#1}}}}}
    
    
    % Define a nice break command that doesn't care if a line doesn't already
    % exist.
    \def\br{\hspace*{\fill} \\* }
    % Math Jax compatibility definitions
    \def\gt{>}
    \def\lt{<}
    \let\Oldtex\TeX
    \let\Oldlatex\LaTeX
    \renewcommand{\TeX}{\textrm{\Oldtex}}
    \renewcommand{\LaTeX}{\textrm{\Oldlatex}}
    % Document parameters
    % Document title
    \title{Lecture-3-Scipy}
    
    
    
    
    
% Pygments definitions
\makeatletter
\def\PY@reset{\let\PY@it=\relax \let\PY@bf=\relax%
    \let\PY@ul=\relax \let\PY@tc=\relax%
    \let\PY@bc=\relax \let\PY@ff=\relax}
\def\PY@tok#1{\csname PY@tok@#1\endcsname}
\def\PY@toks#1+{\ifx\relax#1\empty\else%
    \PY@tok{#1}\expandafter\PY@toks\fi}
\def\PY@do#1{\PY@bc{\PY@tc{\PY@ul{%
    \PY@it{\PY@bf{\PY@ff{#1}}}}}}}
\def\PY#1#2{\PY@reset\PY@toks#1+\relax+\PY@do{#2}}

\expandafter\def\csname PY@tok@w\endcsname{\def\PY@tc##1{\textcolor[rgb]{0.73,0.73,0.73}{##1}}}
\expandafter\def\csname PY@tok@c\endcsname{\let\PY@it=\textit\def\PY@tc##1{\textcolor[rgb]{0.25,0.50,0.50}{##1}}}
\expandafter\def\csname PY@tok@cp\endcsname{\def\PY@tc##1{\textcolor[rgb]{0.74,0.48,0.00}{##1}}}
\expandafter\def\csname PY@tok@k\endcsname{\let\PY@bf=\textbf\def\PY@tc##1{\textcolor[rgb]{0.00,0.50,0.00}{##1}}}
\expandafter\def\csname PY@tok@kp\endcsname{\def\PY@tc##1{\textcolor[rgb]{0.00,0.50,0.00}{##1}}}
\expandafter\def\csname PY@tok@kt\endcsname{\def\PY@tc##1{\textcolor[rgb]{0.69,0.00,0.25}{##1}}}
\expandafter\def\csname PY@tok@o\endcsname{\def\PY@tc##1{\textcolor[rgb]{0.40,0.40,0.40}{##1}}}
\expandafter\def\csname PY@tok@ow\endcsname{\let\PY@bf=\textbf\def\PY@tc##1{\textcolor[rgb]{0.67,0.13,1.00}{##1}}}
\expandafter\def\csname PY@tok@nb\endcsname{\def\PY@tc##1{\textcolor[rgb]{0.00,0.50,0.00}{##1}}}
\expandafter\def\csname PY@tok@nf\endcsname{\def\PY@tc##1{\textcolor[rgb]{0.00,0.00,1.00}{##1}}}
\expandafter\def\csname PY@tok@nc\endcsname{\let\PY@bf=\textbf\def\PY@tc##1{\textcolor[rgb]{0.00,0.00,1.00}{##1}}}
\expandafter\def\csname PY@tok@nn\endcsname{\let\PY@bf=\textbf\def\PY@tc##1{\textcolor[rgb]{0.00,0.00,1.00}{##1}}}
\expandafter\def\csname PY@tok@ne\endcsname{\let\PY@bf=\textbf\def\PY@tc##1{\textcolor[rgb]{0.82,0.25,0.23}{##1}}}
\expandafter\def\csname PY@tok@nv\endcsname{\def\PY@tc##1{\textcolor[rgb]{0.10,0.09,0.49}{##1}}}
\expandafter\def\csname PY@tok@no\endcsname{\def\PY@tc##1{\textcolor[rgb]{0.53,0.00,0.00}{##1}}}
\expandafter\def\csname PY@tok@nl\endcsname{\def\PY@tc##1{\textcolor[rgb]{0.63,0.63,0.00}{##1}}}
\expandafter\def\csname PY@tok@ni\endcsname{\let\PY@bf=\textbf\def\PY@tc##1{\textcolor[rgb]{0.60,0.60,0.60}{##1}}}
\expandafter\def\csname PY@tok@na\endcsname{\def\PY@tc##1{\textcolor[rgb]{0.49,0.56,0.16}{##1}}}
\expandafter\def\csname PY@tok@nt\endcsname{\let\PY@bf=\textbf\def\PY@tc##1{\textcolor[rgb]{0.00,0.50,0.00}{##1}}}
\expandafter\def\csname PY@tok@nd\endcsname{\def\PY@tc##1{\textcolor[rgb]{0.67,0.13,1.00}{##1}}}
\expandafter\def\csname PY@tok@s\endcsname{\def\PY@tc##1{\textcolor[rgb]{0.73,0.13,0.13}{##1}}}
\expandafter\def\csname PY@tok@sd\endcsname{\let\PY@it=\textit\def\PY@tc##1{\textcolor[rgb]{0.73,0.13,0.13}{##1}}}
\expandafter\def\csname PY@tok@si\endcsname{\let\PY@bf=\textbf\def\PY@tc##1{\textcolor[rgb]{0.73,0.40,0.53}{##1}}}
\expandafter\def\csname PY@tok@se\endcsname{\let\PY@bf=\textbf\def\PY@tc##1{\textcolor[rgb]{0.73,0.40,0.13}{##1}}}
\expandafter\def\csname PY@tok@sr\endcsname{\def\PY@tc##1{\textcolor[rgb]{0.73,0.40,0.53}{##1}}}
\expandafter\def\csname PY@tok@ss\endcsname{\def\PY@tc##1{\textcolor[rgb]{0.10,0.09,0.49}{##1}}}
\expandafter\def\csname PY@tok@sx\endcsname{\def\PY@tc##1{\textcolor[rgb]{0.00,0.50,0.00}{##1}}}
\expandafter\def\csname PY@tok@m\endcsname{\def\PY@tc##1{\textcolor[rgb]{0.40,0.40,0.40}{##1}}}
\expandafter\def\csname PY@tok@gh\endcsname{\let\PY@bf=\textbf\def\PY@tc##1{\textcolor[rgb]{0.00,0.00,0.50}{##1}}}
\expandafter\def\csname PY@tok@gu\endcsname{\let\PY@bf=\textbf\def\PY@tc##1{\textcolor[rgb]{0.50,0.00,0.50}{##1}}}
\expandafter\def\csname PY@tok@gd\endcsname{\def\PY@tc##1{\textcolor[rgb]{0.63,0.00,0.00}{##1}}}
\expandafter\def\csname PY@tok@gi\endcsname{\def\PY@tc##1{\textcolor[rgb]{0.00,0.63,0.00}{##1}}}
\expandafter\def\csname PY@tok@gr\endcsname{\def\PY@tc##1{\textcolor[rgb]{1.00,0.00,0.00}{##1}}}
\expandafter\def\csname PY@tok@ge\endcsname{\let\PY@it=\textit}
\expandafter\def\csname PY@tok@gs\endcsname{\let\PY@bf=\textbf}
\expandafter\def\csname PY@tok@gp\endcsname{\let\PY@bf=\textbf\def\PY@tc##1{\textcolor[rgb]{0.00,0.00,0.50}{##1}}}
\expandafter\def\csname PY@tok@go\endcsname{\def\PY@tc##1{\textcolor[rgb]{0.53,0.53,0.53}{##1}}}
\expandafter\def\csname PY@tok@gt\endcsname{\def\PY@tc##1{\textcolor[rgb]{0.00,0.27,0.87}{##1}}}
\expandafter\def\csname PY@tok@err\endcsname{\def\PY@bc##1{\setlength{\fboxsep}{0pt}\fcolorbox[rgb]{1.00,0.00,0.00}{1,1,1}{\strut ##1}}}
\expandafter\def\csname PY@tok@kc\endcsname{\let\PY@bf=\textbf\def\PY@tc##1{\textcolor[rgb]{0.00,0.50,0.00}{##1}}}
\expandafter\def\csname PY@tok@kd\endcsname{\let\PY@bf=\textbf\def\PY@tc##1{\textcolor[rgb]{0.00,0.50,0.00}{##1}}}
\expandafter\def\csname PY@tok@kn\endcsname{\let\PY@bf=\textbf\def\PY@tc##1{\textcolor[rgb]{0.00,0.50,0.00}{##1}}}
\expandafter\def\csname PY@tok@kr\endcsname{\let\PY@bf=\textbf\def\PY@tc##1{\textcolor[rgb]{0.00,0.50,0.00}{##1}}}
\expandafter\def\csname PY@tok@bp\endcsname{\def\PY@tc##1{\textcolor[rgb]{0.00,0.50,0.00}{##1}}}
\expandafter\def\csname PY@tok@fm\endcsname{\def\PY@tc##1{\textcolor[rgb]{0.00,0.00,1.00}{##1}}}
\expandafter\def\csname PY@tok@vc\endcsname{\def\PY@tc##1{\textcolor[rgb]{0.10,0.09,0.49}{##1}}}
\expandafter\def\csname PY@tok@vg\endcsname{\def\PY@tc##1{\textcolor[rgb]{0.10,0.09,0.49}{##1}}}
\expandafter\def\csname PY@tok@vi\endcsname{\def\PY@tc##1{\textcolor[rgb]{0.10,0.09,0.49}{##1}}}
\expandafter\def\csname PY@tok@vm\endcsname{\def\PY@tc##1{\textcolor[rgb]{0.10,0.09,0.49}{##1}}}
\expandafter\def\csname PY@tok@sa\endcsname{\def\PY@tc##1{\textcolor[rgb]{0.73,0.13,0.13}{##1}}}
\expandafter\def\csname PY@tok@sb\endcsname{\def\PY@tc##1{\textcolor[rgb]{0.73,0.13,0.13}{##1}}}
\expandafter\def\csname PY@tok@sc\endcsname{\def\PY@tc##1{\textcolor[rgb]{0.73,0.13,0.13}{##1}}}
\expandafter\def\csname PY@tok@dl\endcsname{\def\PY@tc##1{\textcolor[rgb]{0.73,0.13,0.13}{##1}}}
\expandafter\def\csname PY@tok@s2\endcsname{\def\PY@tc##1{\textcolor[rgb]{0.73,0.13,0.13}{##1}}}
\expandafter\def\csname PY@tok@sh\endcsname{\def\PY@tc##1{\textcolor[rgb]{0.73,0.13,0.13}{##1}}}
\expandafter\def\csname PY@tok@s1\endcsname{\def\PY@tc##1{\textcolor[rgb]{0.73,0.13,0.13}{##1}}}
\expandafter\def\csname PY@tok@mb\endcsname{\def\PY@tc##1{\textcolor[rgb]{0.40,0.40,0.40}{##1}}}
\expandafter\def\csname PY@tok@mf\endcsname{\def\PY@tc##1{\textcolor[rgb]{0.40,0.40,0.40}{##1}}}
\expandafter\def\csname PY@tok@mh\endcsname{\def\PY@tc##1{\textcolor[rgb]{0.40,0.40,0.40}{##1}}}
\expandafter\def\csname PY@tok@mi\endcsname{\def\PY@tc##1{\textcolor[rgb]{0.40,0.40,0.40}{##1}}}
\expandafter\def\csname PY@tok@il\endcsname{\def\PY@tc##1{\textcolor[rgb]{0.40,0.40,0.40}{##1}}}
\expandafter\def\csname PY@tok@mo\endcsname{\def\PY@tc##1{\textcolor[rgb]{0.40,0.40,0.40}{##1}}}
\expandafter\def\csname PY@tok@ch\endcsname{\let\PY@it=\textit\def\PY@tc##1{\textcolor[rgb]{0.25,0.50,0.50}{##1}}}
\expandafter\def\csname PY@tok@cm\endcsname{\let\PY@it=\textit\def\PY@tc##1{\textcolor[rgb]{0.25,0.50,0.50}{##1}}}
\expandafter\def\csname PY@tok@cpf\endcsname{\let\PY@it=\textit\def\PY@tc##1{\textcolor[rgb]{0.25,0.50,0.50}{##1}}}
\expandafter\def\csname PY@tok@c1\endcsname{\let\PY@it=\textit\def\PY@tc##1{\textcolor[rgb]{0.25,0.50,0.50}{##1}}}
\expandafter\def\csname PY@tok@cs\endcsname{\let\PY@it=\textit\def\PY@tc##1{\textcolor[rgb]{0.25,0.50,0.50}{##1}}}

\def\PYZbs{\char`\\}
\def\PYZus{\char`\_}
\def\PYZob{\char`\{}
\def\PYZcb{\char`\}}
\def\PYZca{\char`\^}
\def\PYZam{\char`\&}
\def\PYZlt{\char`\<}
\def\PYZgt{\char`\>}
\def\PYZsh{\char`\#}
\def\PYZpc{\char`\%}
\def\PYZdl{\char`\$}
\def\PYZhy{\char`\-}
\def\PYZsq{\char`\'}
\def\PYZdq{\char`\"}
\def\PYZti{\char`\~}
% for compatibility with earlier versions
\def\PYZat{@}
\def\PYZlb{[}
\def\PYZrb{]}
\makeatother


    % For linebreaks inside Verbatim environment from package fancyvrb. 
    \makeatletter
        \newbox\Wrappedcontinuationbox 
        \newbox\Wrappedvisiblespacebox 
        \newcommand*\Wrappedvisiblespace {\textcolor{red}{\textvisiblespace}} 
        \newcommand*\Wrappedcontinuationsymbol {\textcolor{red}{\llap{\tiny$\m@th\hookrightarrow$}}} 
        \newcommand*\Wrappedcontinuationindent {3ex } 
        \newcommand*\Wrappedafterbreak {\kern\Wrappedcontinuationindent\copy\Wrappedcontinuationbox} 
        % Take advantage of the already applied Pygments mark-up to insert 
        % potential linebreaks for TeX processing. 
        %        {, <, #, %, $, ' and ": go to next line. 
        %        _, }, ^, &, >, - and ~: stay at end of broken line. 
        % Use of \textquotesingle for straight quote. 
        \newcommand*\Wrappedbreaksatspecials {% 
            \def\PYGZus{\discretionary{\char`\_}{\Wrappedafterbreak}{\char`\_}}% 
            \def\PYGZob{\discretionary{}{\Wrappedafterbreak\char`\{}{\char`\{}}% 
            \def\PYGZcb{\discretionary{\char`\}}{\Wrappedafterbreak}{\char`\}}}% 
            \def\PYGZca{\discretionary{\char`\^}{\Wrappedafterbreak}{\char`\^}}% 
            \def\PYGZam{\discretionary{\char`\&}{\Wrappedafterbreak}{\char`\&}}% 
            \def\PYGZlt{\discretionary{}{\Wrappedafterbreak\char`\<}{\char`\<}}% 
            \def\PYGZgt{\discretionary{\char`\>}{\Wrappedafterbreak}{\char`\>}}% 
            \def\PYGZsh{\discretionary{}{\Wrappedafterbreak\char`\#}{\char`\#}}% 
            \def\PYGZpc{\discretionary{}{\Wrappedafterbreak\char`\%}{\char`\%}}% 
            \def\PYGZdl{\discretionary{}{\Wrappedafterbreak\char`\$}{\char`\$}}% 
            \def\PYGZhy{\discretionary{\char`\-}{\Wrappedafterbreak}{\char`\-}}% 
            \def\PYGZsq{\discretionary{}{\Wrappedafterbreak\textquotesingle}{\textquotesingle}}% 
            \def\PYGZdq{\discretionary{}{\Wrappedafterbreak\char`\"}{\char`\"}}% 
            \def\PYGZti{\discretionary{\char`\~}{\Wrappedafterbreak}{\char`\~}}% 
        } 
        % Some characters . , ; ? ! / are not pygmentized. 
        % This macro makes them "active" and they will insert potential linebreaks 
        \newcommand*\Wrappedbreaksatpunct {% 
            \lccode`\~`\.\lowercase{\def~}{\discretionary{\hbox{\char`\.}}{\Wrappedafterbreak}{\hbox{\char`\.}}}% 
            \lccode`\~`\,\lowercase{\def~}{\discretionary{\hbox{\char`\,}}{\Wrappedafterbreak}{\hbox{\char`\,}}}% 
            \lccode`\~`\;\lowercase{\def~}{\discretionary{\hbox{\char`\;}}{\Wrappedafterbreak}{\hbox{\char`\;}}}% 
            \lccode`\~`\:\lowercase{\def~}{\discretionary{\hbox{\char`\:}}{\Wrappedafterbreak}{\hbox{\char`\:}}}% 
            \lccode`\~`\?\lowercase{\def~}{\discretionary{\hbox{\char`\?}}{\Wrappedafterbreak}{\hbox{\char`\?}}}% 
            \lccode`\~`\!\lowercase{\def~}{\discretionary{\hbox{\char`\!}}{\Wrappedafterbreak}{\hbox{\char`\!}}}% 
            \lccode`\~`\/\lowercase{\def~}{\discretionary{\hbox{\char`\/}}{\Wrappedafterbreak}{\hbox{\char`\/}}}% 
            \catcode`\.\active
            \catcode`\,\active 
            \catcode`\;\active
            \catcode`\:\active
            \catcode`\?\active
            \catcode`\!\active
            \catcode`\/\active 
            \lccode`\~`\~ 	
        }
    \makeatother

    \let\OriginalVerbatim=\Verbatim
    \makeatletter
    \renewcommand{\Verbatim}[1][1]{%
        %\parskip\z@skip
        \sbox\Wrappedcontinuationbox {\Wrappedcontinuationsymbol}%
        \sbox\Wrappedvisiblespacebox {\FV@SetupFont\Wrappedvisiblespace}%
        \def\FancyVerbFormatLine ##1{\hsize\linewidth
            \vtop{\raggedright\hyphenpenalty\z@\exhyphenpenalty\z@
                \doublehyphendemerits\z@\finalhyphendemerits\z@
                \strut ##1\strut}%
        }%
        % If the linebreak is at a space, the latter will be displayed as visible
        % space at end of first line, and a continuation symbol starts next line.
        % Stretch/shrink are however usually zero for typewriter font.
        \def\FV@Space {%
            \nobreak\hskip\z@ plus\fontdimen3\font minus\fontdimen4\font
            \discretionary{\copy\Wrappedvisiblespacebox}{\Wrappedafterbreak}
            {\kern\fontdimen2\font}%
        }%
        
        % Allow breaks at special characters using \PYG... macros.
        \Wrappedbreaksatspecials
        % Breaks at punctuation characters . , ; ? ! and / need catcode=\active 	
        \OriginalVerbatim[#1,codes*=\Wrappedbreaksatpunct]%
    }
    \makeatother

    % Exact colors from NB
    \definecolor{incolor}{HTML}{303F9F}
    \definecolor{outcolor}{HTML}{D84315}
    \definecolor{cellborder}{HTML}{CFCFCF}
    \definecolor{cellbackground}{HTML}{F7F7F7}
    
    % prompt
    \makeatletter
    \newcommand{\boxspacing}{\kern\kvtcb@left@rule\kern\kvtcb@boxsep}
    \makeatother
    \newcommand{\prompt}[4]{
        \ttfamily\llap{{\color{#2}[#3]:\hspace{3pt}#4}}\vspace{-\baselineskip}
    }
    

    
    % Prevent overflowing lines due to hard-to-break entities
    \sloppy 
    % Setup hyperref package
    \hypersetup{
      breaklinks=true,  % so long urls are correctly broken across lines
      colorlinks=true,
      urlcolor=urlcolor,
      linkcolor=linkcolor,
      citecolor=citecolor,
      }
    % Slightly bigger margins than the latex defaults
    
    \geometry{verbose,tmargin=1in,bmargin=1in,lmargin=1in,rmargin=1in}
    
    

\begin{document}
    
    \maketitle
    
    

    
    \hypertarget{scipy---library-of-scientific-algorithms-for-python}{%
\section{SciPy - Library of scientific algorithms for
Python}\label{scipy---library-of-scientific-algorithms-for-python}}

    J.R. Johansson (jrjohansson at gmail.com)

The latest version of this
\href{http://ipython.org/notebook.html}{IPython notebook} lecture is
available at
\url{http://github.com/jrjohansson/scientific-python-lectures}.

The other notebooks in this lecture series are indexed at
\url{http://jrjohansson.github.io}.

    \begin{tcolorbox}[breakable, size=fbox, boxrule=1pt, pad at break*=1mm,colback=cellbackground, colframe=cellborder]
\prompt{In}{incolor}{1}{\boxspacing}
\begin{Verbatim}[commandchars=\\\{\}]
\PY{c+c1}{\PYZsh{} what is this line all about? Answer in lecture 4}
\PY{o}{\PYZpc{}}\PY{k}{matplotlib} inline
\PY{k+kn}{import} \PY{n+nn}{matplotlib}\PY{n+nn}{.}\PY{n+nn}{pyplot} \PY{k}{as} \PY{n+nn}{plt}
\PY{k+kn}{from} \PY{n+nn}{IPython}\PY{n+nn}{.}\PY{n+nn}{display} \PY{k+kn}{import} \PY{n}{Image}
\end{Verbatim}
\end{tcolorbox}

    \hypertarget{introduction}{%
\subsection{Introduction}\label{introduction}}

    The SciPy framework builds on top of the low-level NumPy framework for
multidimensional arrays, and provides a large number of higher-level
scientific algorithms. Some of the topics that SciPy covers are:

\begin{itemize}
\tightlist
\item
  Special functions
  (\href{http://docs.scipy.org/doc/scipy/reference/special.html}{scipy.special})
\item
  Integration
  (\href{http://docs.scipy.org/doc/scipy/reference/integrate.html}{scipy.integrate})
\item
  Optimization
  (\href{http://docs.scipy.org/doc/scipy/reference/optimize.html}{scipy.optimize})
\item
  Interpolation
  (\href{http://docs.scipy.org/doc/scipy/reference/interpolate.html}{scipy.interpolate})
\item
  Fourier Transforms
  (\href{http://docs.scipy.org/doc/scipy/reference/fftpack.html}{scipy.fftpack})
\item
  Signal Processing
  (\href{http://docs.scipy.org/doc/scipy/reference/signal.html}{scipy.signal})
\item
  Linear Algebra
  (\href{http://docs.scipy.org/doc/scipy/reference/linalg.html}{scipy.linalg})
\item
  Sparse Eigenvalue Problems
  (\href{http://docs.scipy.org/doc/scipy/reference/sparse.html}{scipy.sparse})
\item
  Statistics
  (\href{http://docs.scipy.org/doc/scipy/reference/stats.html}{scipy.stats})
\item
  Multi-dimensional image processing
  (\href{http://docs.scipy.org/doc/scipy/reference/ndimage.html}{scipy.ndimage})
\item
  File IO
  (\href{http://docs.scipy.org/doc/scipy/reference/io.html}{scipy.io})
\end{itemize}

Each of these submodules provides a number of functions and classes that
can be used to solve problems in their respective topics.

In this lecture we will look at how to use some of these subpackages.

To access the SciPy package in a Python program, we start by importing
everything from the \texttt{scipy} module.

    \begin{tcolorbox}[breakable, size=fbox, boxrule=1pt, pad at break*=1mm,colback=cellbackground, colframe=cellborder]
\prompt{In}{incolor}{2}{\boxspacing}
\begin{Verbatim}[commandchars=\\\{\}]
\PY{k+kn}{from} \PY{n+nn}{scipy} \PY{k+kn}{import} \PY{o}{*}
\end{Verbatim}
\end{tcolorbox}

    If we only need to use part of the SciPy framework we can selectively
include only those modules we are interested in. For example, to include
the linear algebra package under the name \texttt{la}, we can do:

    \begin{tcolorbox}[breakable, size=fbox, boxrule=1pt, pad at break*=1mm,colback=cellbackground, colframe=cellborder]
\prompt{In}{incolor}{3}{\boxspacing}
\begin{Verbatim}[commandchars=\\\{\}]
\PY{k+kn}{import} \PY{n+nn}{scipy}\PY{n+nn}{.}\PY{n+nn}{linalg} \PY{k}{as} \PY{n+nn}{la}
\end{Verbatim}
\end{tcolorbox}

    \hypertarget{special-functions}{%
\subsection{Special functions}\label{special-functions}}

    A large number of mathematical special functions are important for many
computional physics problems. SciPy provides implementations of a very
extensive set of special functions. For details, see the list of
functions in the reference documention at
http://docs.scipy.org/doc/scipy/reference/special.html\#module-scipy.special.

To demonstrate the typical usage of special functions we will look in
more detail at the Bessel functions:

    \begin{tcolorbox}[breakable, size=fbox, boxrule=1pt, pad at break*=1mm,colback=cellbackground, colframe=cellborder]
\prompt{In}{incolor}{4}{\boxspacing}
\begin{Verbatim}[commandchars=\\\{\}]
\PY{c+c1}{\PYZsh{}}
\PY{c+c1}{\PYZsh{} The scipy.special module includes a large number of Bessel\PYZhy{}functions}
\PY{c+c1}{\PYZsh{} Here we will use the functions jn and yn, which are the Bessel functions }
\PY{c+c1}{\PYZsh{} of the first and second kind and real\PYZhy{}valued order. We also include the }
\PY{c+c1}{\PYZsh{} function jn\PYZus{}zeros and yn\PYZus{}zeros that gives the zeroes of the functions jn}
\PY{c+c1}{\PYZsh{} and yn.}
\PY{c+c1}{\PYZsh{}}
\PY{k+kn}{from} \PY{n+nn}{scipy}\PY{n+nn}{.}\PY{n+nn}{special} \PY{k+kn}{import} \PY{n}{jn}\PY{p}{,} \PY{n}{yn}\PY{p}{,} \PY{n}{jn\PYZus{}zeros}\PY{p}{,} \PY{n}{yn\PYZus{}zeros}
\end{Verbatim}
\end{tcolorbox}

    \begin{tcolorbox}[breakable, size=fbox, boxrule=1pt, pad at break*=1mm,colback=cellbackground, colframe=cellborder]
\prompt{In}{incolor}{5}{\boxspacing}
\begin{Verbatim}[commandchars=\\\{\}]
\PY{n}{n} \PY{o}{=} \PY{l+m+mi}{0}    \PY{c+c1}{\PYZsh{} order}
\PY{n}{x} \PY{o}{=} \PY{l+m+mf}{0.0}

\PY{c+c1}{\PYZsh{} Bessel function of first kind}
\PY{n+nb}{print} \PY{l+s+s2}{\PYZdq{}}\PY{l+s+s2}{J\PYZus{}}\PY{l+s+si}{\PYZpc{}d}\PY{l+s+s2}{(}\PY{l+s+si}{\PYZpc{}f}\PY{l+s+s2}{) = }\PY{l+s+si}{\PYZpc{}f}\PY{l+s+s2}{\PYZdq{}} \PY{o}{\PYZpc{}} \PY{p}{(}\PY{n}{n}\PY{p}{,} \PY{n}{x}\PY{p}{,} \PY{n}{jn}\PY{p}{(}\PY{n}{n}\PY{p}{,} \PY{n}{x}\PY{p}{)}\PY{p}{)}

\PY{n}{x} \PY{o}{=} \PY{l+m+mf}{1.0}
\PY{c+c1}{\PYZsh{} Bessel function of second kind}
\PY{n+nb}{print} \PY{l+s+s2}{\PYZdq{}}\PY{l+s+s2}{Y\PYZus{}}\PY{l+s+si}{\PYZpc{}d}\PY{l+s+s2}{(}\PY{l+s+si}{\PYZpc{}f}\PY{l+s+s2}{) = }\PY{l+s+si}{\PYZpc{}f}\PY{l+s+s2}{\PYZdq{}} \PY{o}{\PYZpc{}} \PY{p}{(}\PY{n}{n}\PY{p}{,} \PY{n}{x}\PY{p}{,} \PY{n}{yn}\PY{p}{(}\PY{n}{n}\PY{p}{,} \PY{n}{x}\PY{p}{)}\PY{p}{)}
\end{Verbatim}
\end{tcolorbox}

    \begin{Verbatim}[commandchars=\\\{\}]

          File "<ipython-input-5-4a9e29886bb1>", line 5
        print "J\_\%d(\%f) = \%f" \% (n, x, jn(n, x))
              \^{}
    SyntaxError: invalid syntax
    

    \end{Verbatim}

    \begin{tcolorbox}[breakable, size=fbox, boxrule=1pt, pad at break*=1mm,colback=cellbackground, colframe=cellborder]
\prompt{In}{incolor}{6}{\boxspacing}
\begin{Verbatim}[commandchars=\\\{\}]
\PY{n}{x} \PY{o}{=} \PY{n}{linspace}\PY{p}{(}\PY{l+m+mi}{0}\PY{p}{,} \PY{l+m+mi}{10}\PY{p}{,} \PY{l+m+mi}{100}\PY{p}{)}

\PY{n}{fig}\PY{p}{,} \PY{n}{ax} \PY{o}{=} \PY{n}{plt}\PY{o}{.}\PY{n}{subplots}\PY{p}{(}\PY{p}{)}
\PY{k}{for} \PY{n}{n} \PY{o+ow}{in} \PY{n+nb}{range}\PY{p}{(}\PY{l+m+mi}{4}\PY{p}{)}\PY{p}{:}
    \PY{n}{ax}\PY{o}{.}\PY{n}{plot}\PY{p}{(}\PY{n}{x}\PY{p}{,} \PY{n}{jn}\PY{p}{(}\PY{n}{n}\PY{p}{,} \PY{n}{x}\PY{p}{)}\PY{p}{,} \PY{n}{label}\PY{o}{=}\PY{l+s+sa}{r}\PY{l+s+s2}{\PYZdq{}}\PY{l+s+s2}{\PYZdl{}J\PYZus{}}\PY{l+s+si}{\PYZpc{}d}\PY{l+s+s2}{(x)\PYZdl{}}\PY{l+s+s2}{\PYZdq{}} \PY{o}{\PYZpc{}} \PY{n}{n}\PY{p}{)}
\PY{n}{ax}\PY{o}{.}\PY{n}{legend}\PY{p}{(}\PY{p}{)}\PY{p}{;}
\end{Verbatim}
\end{tcolorbox}

    \begin{Verbatim}[commandchars=\\\{\}]
<ipython-input-6-2df7696e85df>:1: DeprecationWarning: scipy.linspace is
deprecated and will be removed in SciPy 2.0.0, use numpy.linspace instead
  x = linspace(0, 10, 100)
    \end{Verbatim}

    \begin{center}
    \adjustimage{max size={0.9\linewidth}{0.9\paperheight}}{Lecture-3-Scipy_files/Lecture-3-Scipy_12_1.png}
    \end{center}
    { \hspace*{\fill} \\}
    
    \begin{tcolorbox}[breakable, size=fbox, boxrule=1pt, pad at break*=1mm,colback=cellbackground, colframe=cellborder]
\prompt{In}{incolor}{7}{\boxspacing}
\begin{Verbatim}[commandchars=\\\{\}]
\PY{c+c1}{\PYZsh{} zeros of Bessel functions}
\PY{n}{n} \PY{o}{=} \PY{l+m+mi}{0} \PY{c+c1}{\PYZsh{} order}
\PY{n}{m} \PY{o}{=} \PY{l+m+mi}{4} \PY{c+c1}{\PYZsh{} number of roots to compute}
\PY{n}{jn\PYZus{}zeros}\PY{p}{(}\PY{n}{n}\PY{p}{,} \PY{n}{m}\PY{p}{)}
\end{Verbatim}
\end{tcolorbox}

            \begin{tcolorbox}[breakable, size=fbox, boxrule=.5pt, pad at break*=1mm, opacityfill=0]
\prompt{Out}{outcolor}{7}{\boxspacing}
\begin{Verbatim}[commandchars=\\\{\}]
array([ 2.40482556,  5.52007811,  8.65372791, 11.79153444])
\end{Verbatim}
\end{tcolorbox}
        
    \hypertarget{integration}{%
\subsection{Integration}\label{integration}}

    \hypertarget{numerical-integration-quadrature}{%
\subsubsection{Numerical integration:
quadrature}\label{numerical-integration-quadrature}}

    Numerical evaluation of a function of the type

\(\displaystyle \int_a^b f(x) dx\)

is called \emph{numerical quadrature}, or simply \emph{quadature}. SciPy
provides a series of functions for different kind of quadrature, for
example the \texttt{quad}, \texttt{dblquad} and \texttt{tplquad} for
single, double and triple integrals, respectively.

    \begin{tcolorbox}[breakable, size=fbox, boxrule=1pt, pad at break*=1mm,colback=cellbackground, colframe=cellborder]
\prompt{In}{incolor}{8}{\boxspacing}
\begin{Verbatim}[commandchars=\\\{\}]
\PY{k+kn}{from} \PY{n+nn}{scipy}\PY{n+nn}{.}\PY{n+nn}{integrate} \PY{k+kn}{import} \PY{n}{quad}\PY{p}{,} \PY{n}{dblquad}\PY{p}{,} \PY{n}{tplquad}
\end{Verbatim}
\end{tcolorbox}

    The \texttt{quad} function takes a large number of optional arguments,
which can be used to fine-tune the behaviour of the function (try
\texttt{help(quad)} for details).

The basic usage is as follows:

    \begin{tcolorbox}[breakable, size=fbox, boxrule=1pt, pad at break*=1mm,colback=cellbackground, colframe=cellborder]
\prompt{In}{incolor}{9}{\boxspacing}
\begin{Verbatim}[commandchars=\\\{\}]
\PY{c+c1}{\PYZsh{} define a simple function for the integrand}
\PY{k}{def} \PY{n+nf}{f}\PY{p}{(}\PY{n}{x}\PY{p}{)}\PY{p}{:}
    \PY{k}{return} \PY{n}{x}
\end{Verbatim}
\end{tcolorbox}

    \begin{tcolorbox}[breakable, size=fbox, boxrule=1pt, pad at break*=1mm,colback=cellbackground, colframe=cellborder]
\prompt{In}{incolor}{10}{\boxspacing}
\begin{Verbatim}[commandchars=\\\{\}]
\PY{n}{x\PYZus{}lower} \PY{o}{=} \PY{l+m+mi}{0} \PY{c+c1}{\PYZsh{} the lower limit of x}
\PY{n}{x\PYZus{}upper} \PY{o}{=} \PY{l+m+mi}{1} \PY{c+c1}{\PYZsh{} the upper limit of x}

\PY{n}{val}\PY{p}{,} \PY{n}{abserr} \PY{o}{=} \PY{n}{quad}\PY{p}{(}\PY{n}{f}\PY{p}{,} \PY{n}{x\PYZus{}lower}\PY{p}{,} \PY{n}{x\PYZus{}upper}\PY{p}{)}

\PY{n+nb}{print} \PY{l+s+s2}{\PYZdq{}}\PY{l+s+s2}{integral value =}\PY{l+s+s2}{\PYZdq{}}\PY{p}{,} \PY{n}{val}\PY{p}{,} \PY{l+s+s2}{\PYZdq{}}\PY{l+s+s2}{, absolute error =}\PY{l+s+s2}{\PYZdq{}}\PY{p}{,} \PY{n}{abserr} 
\end{Verbatim}
\end{tcolorbox}

    \begin{Verbatim}[commandchars=\\\{\}]

          File "<ipython-input-10-ce8a12fe5a6f>", line 6
        print "integral value =", val, ", absolute error =", abserr
              \^{}
    SyntaxError: Missing parentheses in call to 'print'. Did you mean print("integral value =", val, ", absolute error =", abserr)?
    

    \end{Verbatim}

    If we need to pass extra arguments to integrand function we can use the
\texttt{args} keyword argument:

    \begin{tcolorbox}[breakable, size=fbox, boxrule=1pt, pad at break*=1mm,colback=cellbackground, colframe=cellborder]
\prompt{In}{incolor}{11}{\boxspacing}
\begin{Verbatim}[commandchars=\\\{\}]
\PY{k}{def} \PY{n+nf}{integrand}\PY{p}{(}\PY{n}{x}\PY{p}{,} \PY{n}{n}\PY{p}{)}\PY{p}{:}
    \PY{l+s+sd}{\PYZdq{}\PYZdq{}\PYZdq{}}
\PY{l+s+sd}{    Bessel function of first kind and order n. }
\PY{l+s+sd}{    \PYZdq{}\PYZdq{}\PYZdq{}}
    \PY{k}{return} \PY{n}{jn}\PY{p}{(}\PY{n}{n}\PY{p}{,} \PY{n}{x}\PY{p}{)}


\PY{n}{x\PYZus{}lower} \PY{o}{=} \PY{l+m+mi}{0}  \PY{c+c1}{\PYZsh{} the lower limit of x}
\PY{n}{x\PYZus{}upper} \PY{o}{=} \PY{l+m+mi}{10} \PY{c+c1}{\PYZsh{} the upper limit of x}

\PY{n}{val}\PY{p}{,} \PY{n}{abserr} \PY{o}{=} \PY{n}{quad}\PY{p}{(}\PY{n}{integrand}\PY{p}{,} \PY{n}{x\PYZus{}lower}\PY{p}{,} \PY{n}{x\PYZus{}upper}\PY{p}{,} \PY{n}{args}\PY{o}{=}\PY{p}{(}\PY{l+m+mi}{3}\PY{p}{,}\PY{p}{)}\PY{p}{)}

\PY{n+nb}{print} \PY{n}{val}\PY{p}{,} \PY{n}{abserr} 
\end{Verbatim}
\end{tcolorbox}

    \begin{Verbatim}[commandchars=\\\{\}]

          File "<ipython-input-11-ec663057c1c9>", line 13
        print val, abserr
              \^{}
    SyntaxError: Missing parentheses in call to 'print'. Did you mean print(val, abserr)?
    

    \end{Verbatim}

    For simple functions we can use a lambda function (name-less function)
instead of explicitly defining a function for the integrand:

    \begin{tcolorbox}[breakable, size=fbox, boxrule=1pt, pad at break*=1mm,colback=cellbackground, colframe=cellborder]
\prompt{In}{incolor}{12}{\boxspacing}
\begin{Verbatim}[commandchars=\\\{\}]
\PY{n}{val}\PY{p}{,} \PY{n}{abserr} \PY{o}{=} \PY{n}{quad}\PY{p}{(}\PY{k}{lambda} \PY{n}{x}\PY{p}{:} \PY{n}{exp}\PY{p}{(}\PY{o}{\PYZhy{}}\PY{n}{x} \PY{o}{*}\PY{o}{*} \PY{l+m+mi}{2}\PY{p}{)}\PY{p}{,} \PY{o}{\PYZhy{}}\PY{n}{Inf}\PY{p}{,} \PY{n}{Inf}\PY{p}{)}

\PY{n+nb}{print} \PY{l+s+s2}{\PYZdq{}}\PY{l+s+s2}{numerical  =}\PY{l+s+s2}{\PYZdq{}}\PY{p}{,} \PY{n}{val}\PY{p}{,} \PY{n}{abserr}

\PY{n}{analytical} \PY{o}{=} \PY{n}{sqrt}\PY{p}{(}\PY{n}{pi}\PY{p}{)}
\PY{n+nb}{print} \PY{l+s+s2}{\PYZdq{}}\PY{l+s+s2}{analytical =}\PY{l+s+s2}{\PYZdq{}}\PY{p}{,} \PY{n}{analytical}
\end{Verbatim}
\end{tcolorbox}

    \begin{Verbatim}[commandchars=\\\{\}]

          File "<ipython-input-12-701c84ebcf41>", line 3
        print "numerical  =", val, abserr
              \^{}
    SyntaxError: Missing parentheses in call to 'print'. Did you mean print("numerical  =", val, abserr)?
    

    \end{Verbatim}

    As show in the example above, we can also use `Inf' or `-Inf' as
integral limits.

Higher-dimensional integration works in the same way:

    \begin{tcolorbox}[breakable, size=fbox, boxrule=1pt, pad at break*=1mm,colback=cellbackground, colframe=cellborder]
\prompt{In}{incolor}{13}{\boxspacing}
\begin{Verbatim}[commandchars=\\\{\}]
\PY{k}{def} \PY{n+nf}{integrand}\PY{p}{(}\PY{n}{x}\PY{p}{,} \PY{n}{y}\PY{p}{)}\PY{p}{:}
    \PY{k}{return} \PY{n}{exp}\PY{p}{(}\PY{o}{\PYZhy{}}\PY{n}{x}\PY{o}{*}\PY{o}{*}\PY{l+m+mi}{2}\PY{o}{\PYZhy{}}\PY{n}{y}\PY{o}{*}\PY{o}{*}\PY{l+m+mi}{2}\PY{p}{)}

\PY{n}{x\PYZus{}lower} \PY{o}{=} \PY{l+m+mi}{0}  
\PY{n}{x\PYZus{}upper} \PY{o}{=} \PY{l+m+mi}{10}
\PY{n}{y\PYZus{}lower} \PY{o}{=} \PY{l+m+mi}{0}
\PY{n}{y\PYZus{}upper} \PY{o}{=} \PY{l+m+mi}{10}

\PY{n}{val}\PY{p}{,} \PY{n}{abserr} \PY{o}{=} \PY{n}{dblquad}\PY{p}{(}\PY{n}{integrand}\PY{p}{,} \PY{n}{x\PYZus{}lower}\PY{p}{,} \PY{n}{x\PYZus{}upper}\PY{p}{,} \PY{k}{lambda} \PY{n}{x} \PY{p}{:} \PY{n}{y\PYZus{}lower}\PY{p}{,} \PY{k}{lambda} \PY{n}{x}\PY{p}{:} \PY{n}{y\PYZus{}upper}\PY{p}{)}

\PY{n+nb}{print} \PY{n}{val}\PY{p}{,} \PY{n}{abserr} 
\end{Verbatim}
\end{tcolorbox}

    \begin{Verbatim}[commandchars=\\\{\}]

          File "<ipython-input-13-0aba04f9ec4f>", line 11
        print val, abserr
              \^{}
    SyntaxError: Missing parentheses in call to 'print'. Did you mean print(val, abserr)?
    

    \end{Verbatim}

    Note how we had to pass lambda functions for the limits for the y
integration, since these in general can be functions of x.

    \hypertarget{ordinary-differential-equations-odes}{%
\subsection{Ordinary differential equations
(ODEs)}\label{ordinary-differential-equations-odes}}

    SciPy provides two different ways to solve ODEs: An API based on the
function \texttt{odeint}, and object-oriented API based on the class
\texttt{ode}. Usually \texttt{odeint} is easier to get started with, but
the \texttt{ode} class offers some finer level of control.

Here we will use the \texttt{odeint} functions. For more information
about the class \texttt{ode}, try \texttt{help(ode)}. It does pretty
much the same thing as \texttt{odeint}, but in an object-oriented
fashion.

To use \texttt{odeint}, first import it from the
\texttt{scipy.integrate} module

    \begin{tcolorbox}[breakable, size=fbox, boxrule=1pt, pad at break*=1mm,colback=cellbackground, colframe=cellborder]
\prompt{In}{incolor}{14}{\boxspacing}
\begin{Verbatim}[commandchars=\\\{\}]
\PY{k+kn}{from} \PY{n+nn}{scipy}\PY{n+nn}{.}\PY{n+nn}{integrate} \PY{k+kn}{import} \PY{n}{odeint}\PY{p}{,} \PY{n}{ode}
\end{Verbatim}
\end{tcolorbox}

    A system of ODEs are usually formulated on standard form before it is
attacked numerically. The standard form is:

\(y' = f(y, t)\)

where

\(y = [y_1(t), y_2(t), ..., y_n(t)]\)

and \(f\) is some function that gives the derivatives of the function
\(y_i(t)\). To solve an ODE we need to know the function \(f\) and an
initial condition, \(y(0)\).

Note that higher-order ODEs can always be written in this form by
introducing new variables for the intermediate derivatives.

Once we have defined the Python function \texttt{f} and array
\texttt{y\_0} (that is \(f\) and \(y(0)\) in the mathematical
formulation), we can use the \texttt{odeint} function as:

\begin{verbatim}
y_t = odeint(f, y_0, t)
\end{verbatim}

where \texttt{t} is and array with time-coordinates for which to solve
the ODE problem. \texttt{y\_t} is an array with one row for each point
in time in \texttt{t}, where each column corresponds to a solution
\texttt{y\_i(t)} at that point in time.

We will see how we can implement \texttt{f} and \texttt{y\_0} in Python
code in the examples below.

    \hypertarget{example-double-pendulum}{%
\paragraph{Example: double pendulum}\label{example-double-pendulum}}

    Let's consider a physical example: The double compound pendulum,
described in some detail here:
http://en.wikipedia.org/wiki/Double\_pendulum

    \begin{tcolorbox}[breakable, size=fbox, boxrule=1pt, pad at break*=1mm,colback=cellbackground, colframe=cellborder]
\prompt{In}{incolor}{15}{\boxspacing}
\begin{Verbatim}[commandchars=\\\{\}]
\PY{n}{Image}\PY{p}{(}\PY{n}{url}\PY{o}{=}\PY{l+s+s1}{\PYZsq{}}\PY{l+s+s1}{http://upload.wikimedia.org/wikipedia/commons/c/c9/Double\PYZhy{}compound\PYZhy{}pendulum\PYZhy{}dimensioned.svg}\PY{l+s+s1}{\PYZsq{}}\PY{p}{)}
\end{Verbatim}
\end{tcolorbox}

            \begin{tcolorbox}[breakable, size=fbox, boxrule=.5pt, pad at break*=1mm, opacityfill=0]
\prompt{Out}{outcolor}{15}{\boxspacing}
\begin{Verbatim}[commandchars=\\\{\}]
<IPython.core.display.Image object>
\end{Verbatim}
\end{tcolorbox}
        
    The equations of motion of the pendulum are given on the wiki page:

\({\dot \theta_1} = \frac{6}{m\ell^2} \frac{ 2 p_{\theta_1} - 3 \cos(\theta_1-\theta_2) p_{\theta_2}}{16 - 9 \cos^2(\theta_1-\theta_2)}\)

\({\dot \theta_2} = \frac{6}{m\ell^2} \frac{ 8 p_{\theta_2} - 3 \cos(\theta_1-\theta_2) p_{\theta_1}}{16 - 9 \cos^2(\theta_1-\theta_2)}.\)

\({\dot p_{\theta_1}} = -\frac{1}{2} m \ell^2 \left [ {\dot \theta_1} {\dot \theta_2} \sin (\theta_1-\theta_2) + 3 \frac{g}{\ell} \sin \theta_1 \right ]\)

\({\dot p_{\theta_2}} = -\frac{1}{2} m \ell^2 \left [ -{\dot \theta_1} {\dot \theta_2} \sin (\theta_1-\theta_2) + \frac{g}{\ell} \sin \theta_2 \right]\)

To make the Python code simpler to follow, let's introduce new variable
names and the vector notation:
\(x = [\theta_1, \theta_2, p_{\theta_1}, p_{\theta_2}]\)

\({\dot x_1} = \frac{6}{m\ell^2} \frac{ 2 x_3 - 3 \cos(x_1-x_2) x_4}{16 - 9 \cos^2(x_1-x_2)}\)

\({\dot x_2} = \frac{6}{m\ell^2} \frac{ 8 x_4 - 3 \cos(x_1-x_2) x_3}{16 - 9 \cos^2(x_1-x_2)}\)

\({\dot x_3} = -\frac{1}{2} m \ell^2 \left [ {\dot x_1} {\dot x_2} \sin (x_1-x_2) + 3 \frac{g}{\ell} \sin x_1 \right ]\)

\({\dot x_4} = -\frac{1}{2} m \ell^2 \left [ -{\dot x_1} {\dot x_2} \sin (x_1-x_2) + \frac{g}{\ell} \sin x_2 \right]\)

    \begin{tcolorbox}[breakable, size=fbox, boxrule=1pt, pad at break*=1mm,colback=cellbackground, colframe=cellborder]
\prompt{In}{incolor}{16}{\boxspacing}
\begin{Verbatim}[commandchars=\\\{\}]
\PY{n}{g} \PY{o}{=} \PY{l+m+mf}{9.82}
\PY{n}{L} \PY{o}{=} \PY{l+m+mf}{0.5}
\PY{n}{m} \PY{o}{=} \PY{l+m+mf}{0.1}

\PY{k}{def} \PY{n+nf}{dx}\PY{p}{(}\PY{n}{x}\PY{p}{,} \PY{n}{t}\PY{p}{)}\PY{p}{:}
    \PY{l+s+sd}{\PYZdq{}\PYZdq{}\PYZdq{}}
\PY{l+s+sd}{    The right\PYZhy{}hand side of the pendulum ODE}
\PY{l+s+sd}{    \PYZdq{}\PYZdq{}\PYZdq{}}
    \PY{n}{x1}\PY{p}{,} \PY{n}{x2}\PY{p}{,} \PY{n}{x3}\PY{p}{,} \PY{n}{x4} \PY{o}{=} \PY{n}{x}\PY{p}{[}\PY{l+m+mi}{0}\PY{p}{]}\PY{p}{,} \PY{n}{x}\PY{p}{[}\PY{l+m+mi}{1}\PY{p}{]}\PY{p}{,} \PY{n}{x}\PY{p}{[}\PY{l+m+mi}{2}\PY{p}{]}\PY{p}{,} \PY{n}{x}\PY{p}{[}\PY{l+m+mi}{3}\PY{p}{]}
    
    \PY{n}{dx1} \PY{o}{=} \PY{l+m+mf}{6.0}\PY{o}{/}\PY{p}{(}\PY{n}{m}\PY{o}{*}\PY{n}{L}\PY{o}{*}\PY{o}{*}\PY{l+m+mi}{2}\PY{p}{)} \PY{o}{*} \PY{p}{(}\PY{l+m+mi}{2} \PY{o}{*} \PY{n}{x3} \PY{o}{\PYZhy{}} \PY{l+m+mi}{3} \PY{o}{*} \PY{n}{cos}\PY{p}{(}\PY{n}{x1}\PY{o}{\PYZhy{}}\PY{n}{x2}\PY{p}{)} \PY{o}{*} \PY{n}{x4}\PY{p}{)}\PY{o}{/}\PY{p}{(}\PY{l+m+mi}{16} \PY{o}{\PYZhy{}} \PY{l+m+mi}{9} \PY{o}{*} \PY{n}{cos}\PY{p}{(}\PY{n}{x1}\PY{o}{\PYZhy{}}\PY{n}{x2}\PY{p}{)}\PY{o}{*}\PY{o}{*}\PY{l+m+mi}{2}\PY{p}{)}
    \PY{n}{dx2} \PY{o}{=} \PY{l+m+mf}{6.0}\PY{o}{/}\PY{p}{(}\PY{n}{m}\PY{o}{*}\PY{n}{L}\PY{o}{*}\PY{o}{*}\PY{l+m+mi}{2}\PY{p}{)} \PY{o}{*} \PY{p}{(}\PY{l+m+mi}{8} \PY{o}{*} \PY{n}{x4} \PY{o}{\PYZhy{}} \PY{l+m+mi}{3} \PY{o}{*} \PY{n}{cos}\PY{p}{(}\PY{n}{x1}\PY{o}{\PYZhy{}}\PY{n}{x2}\PY{p}{)} \PY{o}{*} \PY{n}{x3}\PY{p}{)}\PY{o}{/}\PY{p}{(}\PY{l+m+mi}{16} \PY{o}{\PYZhy{}} \PY{l+m+mi}{9} \PY{o}{*} \PY{n}{cos}\PY{p}{(}\PY{n}{x1}\PY{o}{\PYZhy{}}\PY{n}{x2}\PY{p}{)}\PY{o}{*}\PY{o}{*}\PY{l+m+mi}{2}\PY{p}{)}
    \PY{n}{dx3} \PY{o}{=} \PY{o}{\PYZhy{}}\PY{l+m+mf}{0.5} \PY{o}{*} \PY{n}{m} \PY{o}{*} \PY{n}{L}\PY{o}{*}\PY{o}{*}\PY{l+m+mi}{2} \PY{o}{*} \PY{p}{(} \PY{n}{dx1} \PY{o}{*} \PY{n}{dx2} \PY{o}{*} \PY{n}{sin}\PY{p}{(}\PY{n}{x1}\PY{o}{\PYZhy{}}\PY{n}{x2}\PY{p}{)} \PY{o}{+} \PY{l+m+mi}{3} \PY{o}{*} \PY{p}{(}\PY{n}{g}\PY{o}{/}\PY{n}{L}\PY{p}{)} \PY{o}{*} \PY{n}{sin}\PY{p}{(}\PY{n}{x1}\PY{p}{)}\PY{p}{)}
    \PY{n}{dx4} \PY{o}{=} \PY{o}{\PYZhy{}}\PY{l+m+mf}{0.5} \PY{o}{*} \PY{n}{m} \PY{o}{*} \PY{n}{L}\PY{o}{*}\PY{o}{*}\PY{l+m+mi}{2} \PY{o}{*} \PY{p}{(}\PY{o}{\PYZhy{}}\PY{n}{dx1} \PY{o}{*} \PY{n}{dx2} \PY{o}{*} \PY{n}{sin}\PY{p}{(}\PY{n}{x1}\PY{o}{\PYZhy{}}\PY{n}{x2}\PY{p}{)} \PY{o}{+} \PY{p}{(}\PY{n}{g}\PY{o}{/}\PY{n}{L}\PY{p}{)} \PY{o}{*} \PY{n}{sin}\PY{p}{(}\PY{n}{x2}\PY{p}{)}\PY{p}{)}
    
    \PY{k}{return} \PY{p}{[}\PY{n}{dx1}\PY{p}{,} \PY{n}{dx2}\PY{p}{,} \PY{n}{dx3}\PY{p}{,} \PY{n}{dx4}\PY{p}{]}
\end{Verbatim}
\end{tcolorbox}

    \begin{tcolorbox}[breakable, size=fbox, boxrule=1pt, pad at break*=1mm,colback=cellbackground, colframe=cellborder]
\prompt{In}{incolor}{17}{\boxspacing}
\begin{Verbatim}[commandchars=\\\{\}]
\PY{c+c1}{\PYZsh{} choose an initial state}
\PY{n}{x0} \PY{o}{=} \PY{p}{[}\PY{n}{pi}\PY{o}{/}\PY{l+m+mi}{4}\PY{p}{,} \PY{n}{pi}\PY{o}{/}\PY{l+m+mi}{2}\PY{p}{,} \PY{l+m+mi}{0}\PY{p}{,} \PY{l+m+mi}{0}\PY{p}{]}
\end{Verbatim}
\end{tcolorbox}

    \begin{tcolorbox}[breakable, size=fbox, boxrule=1pt, pad at break*=1mm,colback=cellbackground, colframe=cellborder]
\prompt{In}{incolor}{18}{\boxspacing}
\begin{Verbatim}[commandchars=\\\{\}]
\PY{c+c1}{\PYZsh{} time coodinate to solve the ODE for: from 0 to 10 seconds}
\PY{n}{t} \PY{o}{=} \PY{n}{linspace}\PY{p}{(}\PY{l+m+mi}{0}\PY{p}{,} \PY{l+m+mi}{10}\PY{p}{,} \PY{l+m+mi}{250}\PY{p}{)}
\end{Verbatim}
\end{tcolorbox}

    \begin{Verbatim}[commandchars=\\\{\}]
<ipython-input-18-02dd0324cced>:2: DeprecationWarning: scipy.linspace is
deprecated and will be removed in SciPy 2.0.0, use numpy.linspace instead
  t = linspace(0, 10, 250)
    \end{Verbatim}

    \begin{tcolorbox}[breakable, size=fbox, boxrule=1pt, pad at break*=1mm,colback=cellbackground, colframe=cellborder]
\prompt{In}{incolor}{19}{\boxspacing}
\begin{Verbatim}[commandchars=\\\{\}]
\PY{c+c1}{\PYZsh{} solve the ODE problem}
\PY{n}{x} \PY{o}{=} \PY{n}{odeint}\PY{p}{(}\PY{n}{dx}\PY{p}{,} \PY{n}{x0}\PY{p}{,} \PY{n}{t}\PY{p}{)}
\end{Verbatim}
\end{tcolorbox}

    \begin{Verbatim}[commandchars=\\\{\}]
<ipython-input-16-82fff5157982>:11: DeprecationWarning: scipy.cos is deprecated
and will be removed in SciPy 2.0.0, use numpy.cos instead
  dx1 = 6.0/(m*L**2) * (2 * x3 - 3 * cos(x1-x2) * x4)/(16 - 9 * cos(x1-x2)**2)
<ipython-input-16-82fff5157982>:12: DeprecationWarning: scipy.cos is deprecated
and will be removed in SciPy 2.0.0, use numpy.cos instead
  dx2 = 6.0/(m*L**2) * (8 * x4 - 3 * cos(x1-x2) * x3)/(16 - 9 * cos(x1-x2)**2)
<ipython-input-16-82fff5157982>:13: DeprecationWarning: scipy.sin is deprecated
and will be removed in SciPy 2.0.0, use numpy.sin instead
  dx3 = -0.5 * m * L**2 * ( dx1 * dx2 * sin(x1-x2) + 3 * (g/L) * sin(x1))
<ipython-input-16-82fff5157982>:14: DeprecationWarning: scipy.sin is deprecated
and will be removed in SciPy 2.0.0, use numpy.sin instead
  dx4 = -0.5 * m * L**2 * (-dx1 * dx2 * sin(x1-x2) + (g/L) * sin(x2))
    \end{Verbatim}

    \begin{tcolorbox}[breakable, size=fbox, boxrule=1pt, pad at break*=1mm,colback=cellbackground, colframe=cellborder]
\prompt{In}{incolor}{20}{\boxspacing}
\begin{Verbatim}[commandchars=\\\{\}]
\PY{c+c1}{\PYZsh{} plot the angles as a function of time}

\PY{n}{fig}\PY{p}{,} \PY{n}{axes} \PY{o}{=} \PY{n}{plt}\PY{o}{.}\PY{n}{subplots}\PY{p}{(}\PY{l+m+mi}{1}\PY{p}{,}\PY{l+m+mi}{2}\PY{p}{,} \PY{n}{figsize}\PY{o}{=}\PY{p}{(}\PY{l+m+mi}{12}\PY{p}{,}\PY{l+m+mi}{4}\PY{p}{)}\PY{p}{)}
\PY{n}{axes}\PY{p}{[}\PY{l+m+mi}{0}\PY{p}{]}\PY{o}{.}\PY{n}{plot}\PY{p}{(}\PY{n}{t}\PY{p}{,} \PY{n}{x}\PY{p}{[}\PY{p}{:}\PY{p}{,} \PY{l+m+mi}{0}\PY{p}{]}\PY{p}{,} \PY{l+s+s1}{\PYZsq{}}\PY{l+s+s1}{r}\PY{l+s+s1}{\PYZsq{}}\PY{p}{,} \PY{n}{label}\PY{o}{=}\PY{l+s+s2}{\PYZdq{}}\PY{l+s+s2}{theta1}\PY{l+s+s2}{\PYZdq{}}\PY{p}{)}
\PY{n}{axes}\PY{p}{[}\PY{l+m+mi}{0}\PY{p}{]}\PY{o}{.}\PY{n}{plot}\PY{p}{(}\PY{n}{t}\PY{p}{,} \PY{n}{x}\PY{p}{[}\PY{p}{:}\PY{p}{,} \PY{l+m+mi}{1}\PY{p}{]}\PY{p}{,} \PY{l+s+s1}{\PYZsq{}}\PY{l+s+s1}{b}\PY{l+s+s1}{\PYZsq{}}\PY{p}{,} \PY{n}{label}\PY{o}{=}\PY{l+s+s2}{\PYZdq{}}\PY{l+s+s2}{theta2}\PY{l+s+s2}{\PYZdq{}}\PY{p}{)}


\PY{n}{x1} \PY{o}{=} \PY{o}{+} \PY{n}{L} \PY{o}{*} \PY{n}{sin}\PY{p}{(}\PY{n}{x}\PY{p}{[}\PY{p}{:}\PY{p}{,} \PY{l+m+mi}{0}\PY{p}{]}\PY{p}{)}
\PY{n}{y1} \PY{o}{=} \PY{o}{\PYZhy{}} \PY{n}{L} \PY{o}{*} \PY{n}{cos}\PY{p}{(}\PY{n}{x}\PY{p}{[}\PY{p}{:}\PY{p}{,} \PY{l+m+mi}{0}\PY{p}{]}\PY{p}{)}

\PY{n}{x2} \PY{o}{=} \PY{n}{x1} \PY{o}{+} \PY{n}{L} \PY{o}{*} \PY{n}{sin}\PY{p}{(}\PY{n}{x}\PY{p}{[}\PY{p}{:}\PY{p}{,} \PY{l+m+mi}{1}\PY{p}{]}\PY{p}{)}
\PY{n}{y2} \PY{o}{=} \PY{n}{y1} \PY{o}{\PYZhy{}} \PY{n}{L} \PY{o}{*} \PY{n}{cos}\PY{p}{(}\PY{n}{x}\PY{p}{[}\PY{p}{:}\PY{p}{,} \PY{l+m+mi}{1}\PY{p}{]}\PY{p}{)}
    
\PY{n}{axes}\PY{p}{[}\PY{l+m+mi}{1}\PY{p}{]}\PY{o}{.}\PY{n}{plot}\PY{p}{(}\PY{n}{x1}\PY{p}{,} \PY{n}{y1}\PY{p}{,} \PY{l+s+s1}{\PYZsq{}}\PY{l+s+s1}{r}\PY{l+s+s1}{\PYZsq{}}\PY{p}{,} \PY{n}{label}\PY{o}{=}\PY{l+s+s2}{\PYZdq{}}\PY{l+s+s2}{pendulum1}\PY{l+s+s2}{\PYZdq{}}\PY{p}{)}
\PY{n}{axes}\PY{p}{[}\PY{l+m+mi}{1}\PY{p}{]}\PY{o}{.}\PY{n}{plot}\PY{p}{(}\PY{n}{x2}\PY{p}{,} \PY{n}{y2}\PY{p}{,} \PY{l+s+s1}{\PYZsq{}}\PY{l+s+s1}{b}\PY{l+s+s1}{\PYZsq{}}\PY{p}{,} \PY{n}{label}\PY{o}{=}\PY{l+s+s2}{\PYZdq{}}\PY{l+s+s2}{pendulum2}\PY{l+s+s2}{\PYZdq{}}\PY{p}{)}
\PY{n}{axes}\PY{p}{[}\PY{l+m+mi}{1}\PY{p}{]}\PY{o}{.}\PY{n}{set\PYZus{}ylim}\PY{p}{(}\PY{p}{[}\PY{o}{\PYZhy{}}\PY{l+m+mi}{1}\PY{p}{,} \PY{l+m+mi}{0}\PY{p}{]}\PY{p}{)}
\PY{n}{axes}\PY{p}{[}\PY{l+m+mi}{1}\PY{p}{]}\PY{o}{.}\PY{n}{set\PYZus{}xlim}\PY{p}{(}\PY{p}{[}\PY{l+m+mi}{1}\PY{p}{,} \PY{o}{\PYZhy{}}\PY{l+m+mi}{1}\PY{p}{]}\PY{p}{)}\PY{p}{;}
\end{Verbatim}
\end{tcolorbox}

    \begin{Verbatim}[commandchars=\\\{\}]
<ipython-input-20-c258a0156c14>:8: DeprecationWarning: scipy.sin is deprecated
and will be removed in SciPy 2.0.0, use numpy.sin instead
  x1 = + L * sin(x[:, 0])
<ipython-input-20-c258a0156c14>:9: DeprecationWarning: scipy.cos is deprecated
and will be removed in SciPy 2.0.0, use numpy.cos instead
  y1 = - L * cos(x[:, 0])
<ipython-input-20-c258a0156c14>:11: DeprecationWarning: scipy.sin is deprecated
and will be removed in SciPy 2.0.0, use numpy.sin instead
  x2 = x1 + L * sin(x[:, 1])
<ipython-input-20-c258a0156c14>:12: DeprecationWarning: scipy.cos is deprecated
and will be removed in SciPy 2.0.0, use numpy.cos instead
  y2 = y1 - L * cos(x[:, 1])
    \end{Verbatim}

    \begin{center}
    \adjustimage{max size={0.9\linewidth}{0.9\paperheight}}{Lecture-3-Scipy_files/Lecture-3-Scipy_40_1.png}
    \end{center}
    { \hspace*{\fill} \\}
    
    Simple annimation of the pendulum motion. We will see how to make better
animation in Lecture 4.

    \begin{tcolorbox}[breakable, size=fbox, boxrule=1pt, pad at break*=1mm,colback=cellbackground, colframe=cellborder]
\prompt{In}{incolor}{21}{\boxspacing}
\begin{Verbatim}[commandchars=\\\{\}]
\PY{k+kn}{from} \PY{n+nn}{IPython}\PY{n+nn}{.}\PY{n+nn}{display} \PY{k+kn}{import} \PY{n}{display}\PY{p}{,} \PY{n}{clear\PYZus{}output}
\PY{k+kn}{import} \PY{n+nn}{time}
\end{Verbatim}
\end{tcolorbox}

    \begin{tcolorbox}[breakable, size=fbox, boxrule=1pt, pad at break*=1mm,colback=cellbackground, colframe=cellborder]
\prompt{In}{incolor}{22}{\boxspacing}
\begin{Verbatim}[commandchars=\\\{\}]
\PY{n}{fig}\PY{p}{,} \PY{n}{ax} \PY{o}{=} \PY{n}{plt}\PY{o}{.}\PY{n}{subplots}\PY{p}{(}\PY{n}{figsize}\PY{o}{=}\PY{p}{(}\PY{l+m+mi}{4}\PY{p}{,}\PY{l+m+mi}{4}\PY{p}{)}\PY{p}{)}

\PY{k}{for} \PY{n}{t\PYZus{}idx}\PY{p}{,} \PY{n}{tt} \PY{o+ow}{in} \PY{n+nb}{enumerate}\PY{p}{(}\PY{n}{t}\PY{p}{[}\PY{p}{:}\PY{l+m+mi}{200}\PY{p}{]}\PY{p}{)}\PY{p}{:}

    \PY{n}{x1} \PY{o}{=} \PY{o}{+} \PY{n}{L} \PY{o}{*} \PY{n}{sin}\PY{p}{(}\PY{n}{x}\PY{p}{[}\PY{n}{t\PYZus{}idx}\PY{p}{,} \PY{l+m+mi}{0}\PY{p}{]}\PY{p}{)}
    \PY{n}{y1} \PY{o}{=} \PY{o}{\PYZhy{}} \PY{n}{L} \PY{o}{*} \PY{n}{cos}\PY{p}{(}\PY{n}{x}\PY{p}{[}\PY{n}{t\PYZus{}idx}\PY{p}{,} \PY{l+m+mi}{0}\PY{p}{]}\PY{p}{)}

    \PY{n}{x2} \PY{o}{=} \PY{n}{x1} \PY{o}{+} \PY{n}{L} \PY{o}{*} \PY{n}{sin}\PY{p}{(}\PY{n}{x}\PY{p}{[}\PY{n}{t\PYZus{}idx}\PY{p}{,} \PY{l+m+mi}{1}\PY{p}{]}\PY{p}{)}
    \PY{n}{y2} \PY{o}{=} \PY{n}{y1} \PY{o}{\PYZhy{}} \PY{n}{L} \PY{o}{*} \PY{n}{cos}\PY{p}{(}\PY{n}{x}\PY{p}{[}\PY{n}{t\PYZus{}idx}\PY{p}{,} \PY{l+m+mi}{1}\PY{p}{]}\PY{p}{)}
    
    \PY{n}{ax}\PY{o}{.}\PY{n}{cla}\PY{p}{(}\PY{p}{)}    
    \PY{n}{ax}\PY{o}{.}\PY{n}{plot}\PY{p}{(}\PY{p}{[}\PY{l+m+mi}{0}\PY{p}{,} \PY{n}{x1}\PY{p}{]}\PY{p}{,} \PY{p}{[}\PY{l+m+mi}{0}\PY{p}{,} \PY{n}{y1}\PY{p}{]}\PY{p}{,} \PY{l+s+s1}{\PYZsq{}}\PY{l+s+s1}{r.\PYZhy{}}\PY{l+s+s1}{\PYZsq{}}\PY{p}{)}
    \PY{n}{ax}\PY{o}{.}\PY{n}{plot}\PY{p}{(}\PY{p}{[}\PY{n}{x1}\PY{p}{,} \PY{n}{x2}\PY{p}{]}\PY{p}{,} \PY{p}{[}\PY{n}{y1}\PY{p}{,} \PY{n}{y2}\PY{p}{]}\PY{p}{,} \PY{l+s+s1}{\PYZsq{}}\PY{l+s+s1}{b.\PYZhy{}}\PY{l+s+s1}{\PYZsq{}}\PY{p}{)}
    \PY{n}{ax}\PY{o}{.}\PY{n}{set\PYZus{}ylim}\PY{p}{(}\PY{p}{[}\PY{o}{\PYZhy{}}\PY{l+m+mf}{1.5}\PY{p}{,} \PY{l+m+mf}{0.5}\PY{p}{]}\PY{p}{)}
    \PY{n}{ax}\PY{o}{.}\PY{n}{set\PYZus{}xlim}\PY{p}{(}\PY{p}{[}\PY{l+m+mi}{1}\PY{p}{,} \PY{o}{\PYZhy{}}\PY{l+m+mi}{1}\PY{p}{]}\PY{p}{)}

    \PY{n}{clear\PYZus{}output}\PY{p}{(}\PY{p}{)} 
    \PY{n}{display}\PY{p}{(}\PY{n}{fig}\PY{p}{)}

    \PY{n}{time}\PY{o}{.}\PY{n}{sleep}\PY{p}{(}\PY{l+m+mf}{0.1}\PY{p}{)}
\end{Verbatim}
\end{tcolorbox}

    \begin{center}
    \adjustimage{max size={0.9\linewidth}{0.9\paperheight}}{Lecture-3-Scipy_files/Lecture-3-Scipy_43_0.png}
    \end{center}
    { \hspace*{\fill} \\}
    
    \begin{center}
    \adjustimage{max size={0.9\linewidth}{0.9\paperheight}}{Lecture-3-Scipy_files/Lecture-3-Scipy_43_1.png}
    \end{center}
    { \hspace*{\fill} \\}
    
    \hypertarget{example-damped-harmonic-oscillator}{%
\paragraph{Example: Damped harmonic
oscillator}\label{example-damped-harmonic-oscillator}}

    ODE problems are important in computational physics, so we will look at
one more example: the damped harmonic oscillation. This problem is well
described on the wiki page: http://en.wikipedia.org/wiki/Damping

The equation of motion for the damped oscillator is:

\(\displaystyle \frac{\mathrm{d}^2x}{\mathrm{d}t^2} + 2\zeta\omega_0\frac{\mathrm{d}x}{\mathrm{d}t} + \omega^2_0 x = 0\)

where \(x\) is the position of the oscillator, \(\omega_0\) is the
frequency, and \(\zeta\) is the damping ratio. To write this
second-order ODE on standard form we introduce
\(p = \frac{\mathrm{d}x}{\mathrm{d}t}\):

\(\displaystyle \frac{\mathrm{d}p}{\mathrm{d}t} = - 2\zeta\omega_0 p - \omega^2_0 x\)

\(\displaystyle \frac{\mathrm{d}x}{\mathrm{d}t} = p\)

In the implementation of this example we will add extra arguments to the
RHS function for the ODE, rather than using global variables as we did
in the previous example. As a consequence of the extra arguments to the
RHS, we need to pass an keyword argument \texttt{args} to the
\texttt{odeint} function:

    \begin{tcolorbox}[breakable, size=fbox, boxrule=1pt, pad at break*=1mm,colback=cellbackground, colframe=cellborder]
\prompt{In}{incolor}{23}{\boxspacing}
\begin{Verbatim}[commandchars=\\\{\}]
\PY{k}{def} \PY{n+nf}{dy}\PY{p}{(}\PY{n}{y}\PY{p}{,} \PY{n}{t}\PY{p}{,} \PY{n}{zeta}\PY{p}{,} \PY{n}{w0}\PY{p}{)}\PY{p}{:}
    \PY{l+s+sd}{\PYZdq{}\PYZdq{}\PYZdq{}}
\PY{l+s+sd}{    The right\PYZhy{}hand side of the damped oscillator ODE}
\PY{l+s+sd}{    \PYZdq{}\PYZdq{}\PYZdq{}}
    \PY{n}{x}\PY{p}{,} \PY{n}{p} \PY{o}{=} \PY{n}{y}\PY{p}{[}\PY{l+m+mi}{0}\PY{p}{]}\PY{p}{,} \PY{n}{y}\PY{p}{[}\PY{l+m+mi}{1}\PY{p}{]}
    
    \PY{n}{dx} \PY{o}{=} \PY{n}{p}
    \PY{n}{dp} \PY{o}{=} \PY{o}{\PYZhy{}}\PY{l+m+mi}{2} \PY{o}{*} \PY{n}{zeta} \PY{o}{*} \PY{n}{w0} \PY{o}{*} \PY{n}{p} \PY{o}{\PYZhy{}} \PY{n}{w0}\PY{o}{*}\PY{o}{*}\PY{l+m+mi}{2} \PY{o}{*} \PY{n}{x}

    \PY{k}{return} \PY{p}{[}\PY{n}{dx}\PY{p}{,} \PY{n}{dp}\PY{p}{]}
\end{Verbatim}
\end{tcolorbox}

    \begin{tcolorbox}[breakable, size=fbox, boxrule=1pt, pad at break*=1mm,colback=cellbackground, colframe=cellborder]
\prompt{In}{incolor}{24}{\boxspacing}
\begin{Verbatim}[commandchars=\\\{\}]
\PY{c+c1}{\PYZsh{} initial state: }
\PY{n}{y0} \PY{o}{=} \PY{p}{[}\PY{l+m+mf}{1.0}\PY{p}{,} \PY{l+m+mf}{0.0}\PY{p}{]}
\end{Verbatim}
\end{tcolorbox}

    \begin{tcolorbox}[breakable, size=fbox, boxrule=1pt, pad at break*=1mm,colback=cellbackground, colframe=cellborder]
\prompt{In}{incolor}{25}{\boxspacing}
\begin{Verbatim}[commandchars=\\\{\}]
\PY{c+c1}{\PYZsh{} time coodinate to solve the ODE for}
\PY{n}{t} \PY{o}{=} \PY{n}{linspace}\PY{p}{(}\PY{l+m+mi}{0}\PY{p}{,} \PY{l+m+mi}{10}\PY{p}{,} \PY{l+m+mi}{1000}\PY{p}{)}
\PY{n}{w0} \PY{o}{=} \PY{l+m+mi}{2}\PY{o}{*}\PY{n}{pi}\PY{o}{*}\PY{l+m+mf}{1.0}
\end{Verbatim}
\end{tcolorbox}

    \begin{Verbatim}[commandchars=\\\{\}]
<ipython-input-25-c9ca301b6aa9>:2: DeprecationWarning: scipy.linspace is
deprecated and will be removed in SciPy 2.0.0, use numpy.linspace instead
  t = linspace(0, 10, 1000)
    \end{Verbatim}

    \begin{tcolorbox}[breakable, size=fbox, boxrule=1pt, pad at break*=1mm,colback=cellbackground, colframe=cellborder]
\prompt{In}{incolor}{26}{\boxspacing}
\begin{Verbatim}[commandchars=\\\{\}]
\PY{c+c1}{\PYZsh{} solve the ODE problem for three different values of the damping ratio}

\PY{n}{y1} \PY{o}{=} \PY{n}{odeint}\PY{p}{(}\PY{n}{dy}\PY{p}{,} \PY{n}{y0}\PY{p}{,} \PY{n}{t}\PY{p}{,} \PY{n}{args}\PY{o}{=}\PY{p}{(}\PY{l+m+mf}{0.0}\PY{p}{,} \PY{n}{w0}\PY{p}{)}\PY{p}{)} \PY{c+c1}{\PYZsh{} undamped}
\PY{n}{y2} \PY{o}{=} \PY{n}{odeint}\PY{p}{(}\PY{n}{dy}\PY{p}{,} \PY{n}{y0}\PY{p}{,} \PY{n}{t}\PY{p}{,} \PY{n}{args}\PY{o}{=}\PY{p}{(}\PY{l+m+mf}{0.2}\PY{p}{,} \PY{n}{w0}\PY{p}{)}\PY{p}{)} \PY{c+c1}{\PYZsh{} under damped}
\PY{n}{y3} \PY{o}{=} \PY{n}{odeint}\PY{p}{(}\PY{n}{dy}\PY{p}{,} \PY{n}{y0}\PY{p}{,} \PY{n}{t}\PY{p}{,} \PY{n}{args}\PY{o}{=}\PY{p}{(}\PY{l+m+mf}{1.0}\PY{p}{,} \PY{n}{w0}\PY{p}{)}\PY{p}{)} \PY{c+c1}{\PYZsh{} critial damping}
\PY{n}{y4} \PY{o}{=} \PY{n}{odeint}\PY{p}{(}\PY{n}{dy}\PY{p}{,} \PY{n}{y0}\PY{p}{,} \PY{n}{t}\PY{p}{,} \PY{n}{args}\PY{o}{=}\PY{p}{(}\PY{l+m+mf}{5.0}\PY{p}{,} \PY{n}{w0}\PY{p}{)}\PY{p}{)} \PY{c+c1}{\PYZsh{} over damped}
\end{Verbatim}
\end{tcolorbox}

    \begin{tcolorbox}[breakable, size=fbox, boxrule=1pt, pad at break*=1mm,colback=cellbackground, colframe=cellborder]
\prompt{In}{incolor}{27}{\boxspacing}
\begin{Verbatim}[commandchars=\\\{\}]
\PY{n}{fig}\PY{p}{,} \PY{n}{ax} \PY{o}{=} \PY{n}{plt}\PY{o}{.}\PY{n}{subplots}\PY{p}{(}\PY{p}{)}
\PY{n}{ax}\PY{o}{.}\PY{n}{plot}\PY{p}{(}\PY{n}{t}\PY{p}{,} \PY{n}{y1}\PY{p}{[}\PY{p}{:}\PY{p}{,}\PY{l+m+mi}{0}\PY{p}{]}\PY{p}{,} \PY{l+s+s1}{\PYZsq{}}\PY{l+s+s1}{k}\PY{l+s+s1}{\PYZsq{}}\PY{p}{,} \PY{n}{label}\PY{o}{=}\PY{l+s+s2}{\PYZdq{}}\PY{l+s+s2}{undamped}\PY{l+s+s2}{\PYZdq{}}\PY{p}{,} \PY{n}{linewidth}\PY{o}{=}\PY{l+m+mf}{0.25}\PY{p}{)}
\PY{n}{ax}\PY{o}{.}\PY{n}{plot}\PY{p}{(}\PY{n}{t}\PY{p}{,} \PY{n}{y2}\PY{p}{[}\PY{p}{:}\PY{p}{,}\PY{l+m+mi}{0}\PY{p}{]}\PY{p}{,} \PY{l+s+s1}{\PYZsq{}}\PY{l+s+s1}{r}\PY{l+s+s1}{\PYZsq{}}\PY{p}{,} \PY{n}{label}\PY{o}{=}\PY{l+s+s2}{\PYZdq{}}\PY{l+s+s2}{under damped}\PY{l+s+s2}{\PYZdq{}}\PY{p}{)}
\PY{n}{ax}\PY{o}{.}\PY{n}{plot}\PY{p}{(}\PY{n}{t}\PY{p}{,} \PY{n}{y3}\PY{p}{[}\PY{p}{:}\PY{p}{,}\PY{l+m+mi}{0}\PY{p}{]}\PY{p}{,} \PY{l+s+s1}{\PYZsq{}}\PY{l+s+s1}{b}\PY{l+s+s1}{\PYZsq{}}\PY{p}{,} \PY{n}{label}\PY{o}{=}\PY{l+s+sa}{r}\PY{l+s+s2}{\PYZdq{}}\PY{l+s+s2}{critical damping}\PY{l+s+s2}{\PYZdq{}}\PY{p}{)}
\PY{n}{ax}\PY{o}{.}\PY{n}{plot}\PY{p}{(}\PY{n}{t}\PY{p}{,} \PY{n}{y4}\PY{p}{[}\PY{p}{:}\PY{p}{,}\PY{l+m+mi}{0}\PY{p}{]}\PY{p}{,} \PY{l+s+s1}{\PYZsq{}}\PY{l+s+s1}{g}\PY{l+s+s1}{\PYZsq{}}\PY{p}{,} \PY{n}{label}\PY{o}{=}\PY{l+s+s2}{\PYZdq{}}\PY{l+s+s2}{over damped}\PY{l+s+s2}{\PYZdq{}}\PY{p}{)}
\PY{n}{ax}\PY{o}{.}\PY{n}{legend}\PY{p}{(}\PY{p}{)}\PY{p}{;}
\end{Verbatim}
\end{tcolorbox}

    \begin{center}
    \adjustimage{max size={0.9\linewidth}{0.9\paperheight}}{Lecture-3-Scipy_files/Lecture-3-Scipy_50_0.png}
    \end{center}
    { \hspace*{\fill} \\}
    
    \hypertarget{fourier-transform}{%
\subsection{Fourier transform}\label{fourier-transform}}

    Fourier transforms are one of the universal tools in computational
physics, which appear over and over again in different contexts. SciPy
provides functions for accessing the classic
\href{http://www.netlib.org/fftpack/}{FFTPACK} library from NetLib,
which is an efficient and well tested FFT library written in FORTRAN.
The SciPy API has a few additional convenience functions, but overall
the API is closely related to the original FORTRAN library.

To use the \texttt{fftpack} module in a python program, include it
using:

    \begin{tcolorbox}[breakable, size=fbox, boxrule=1pt, pad at break*=1mm,colback=cellbackground, colframe=cellborder]
\prompt{In}{incolor}{28}{\boxspacing}
\begin{Verbatim}[commandchars=\\\{\}]
\PY{k+kn}{from} \PY{n+nn}{numpy}\PY{n+nn}{.}\PY{n+nn}{fft} \PY{k+kn}{import} \PY{n}{fftfreq}
\PY{k+kn}{from} \PY{n+nn}{scipy}\PY{n+nn}{.}\PY{n+nn}{fftpack} \PY{k+kn}{import} \PY{o}{*}
\end{Verbatim}
\end{tcolorbox}

    To demonstrate how to do a fast Fourier transform with SciPy, let's look
at the FFT of the solution to the damped oscillator from the previous
section:

    \begin{tcolorbox}[breakable, size=fbox, boxrule=1pt, pad at break*=1mm,colback=cellbackground, colframe=cellborder]
\prompt{In}{incolor}{29}{\boxspacing}
\begin{Verbatim}[commandchars=\\\{\}]
\PY{n}{N} \PY{o}{=} \PY{n+nb}{len}\PY{p}{(}\PY{n}{t}\PY{p}{)}
\PY{n}{dt} \PY{o}{=} \PY{n}{t}\PY{p}{[}\PY{l+m+mi}{1}\PY{p}{]}\PY{o}{\PYZhy{}}\PY{n}{t}\PY{p}{[}\PY{l+m+mi}{0}\PY{p}{]}

\PY{c+c1}{\PYZsh{} calculate the fast fourier transform}
\PY{c+c1}{\PYZsh{} y2 is the solution to the under\PYZhy{}damped oscillator from the previous section}
\PY{n}{F} \PY{o}{=} \PY{n}{fft}\PY{p}{(}\PY{n}{y2}\PY{p}{[}\PY{p}{:}\PY{p}{,}\PY{l+m+mi}{0}\PY{p}{]}\PY{p}{)} 

\PY{c+c1}{\PYZsh{} calculate the frequencies for the components in F}
\PY{n}{w} \PY{o}{=} \PY{n}{fftfreq}\PY{p}{(}\PY{n}{N}\PY{p}{,} \PY{n}{dt}\PY{p}{)}
\end{Verbatim}
\end{tcolorbox}

    \begin{tcolorbox}[breakable, size=fbox, boxrule=1pt, pad at break*=1mm,colback=cellbackground, colframe=cellborder]
\prompt{In}{incolor}{30}{\boxspacing}
\begin{Verbatim}[commandchars=\\\{\}]
\PY{n}{fig}\PY{p}{,} \PY{n}{ax} \PY{o}{=} \PY{n}{plt}\PY{o}{.}\PY{n}{subplots}\PY{p}{(}\PY{n}{figsize}\PY{o}{=}\PY{p}{(}\PY{l+m+mi}{9}\PY{p}{,}\PY{l+m+mi}{3}\PY{p}{)}\PY{p}{)}
\PY{n}{ax}\PY{o}{.}\PY{n}{plot}\PY{p}{(}\PY{n}{w}\PY{p}{,} \PY{n+nb}{abs}\PY{p}{(}\PY{n}{F}\PY{p}{)}\PY{p}{)}\PY{p}{;}
\end{Verbatim}
\end{tcolorbox}

    \begin{center}
    \adjustimage{max size={0.9\linewidth}{0.9\paperheight}}{Lecture-3-Scipy_files/Lecture-3-Scipy_56_0.png}
    \end{center}
    { \hspace*{\fill} \\}
    
    Since the signal is real, the spectrum is symmetric. We therefore only
need to plot the part that corresponds to the postive frequencies. To
extract that part of the \texttt{w} and \texttt{F} we can use some of
the indexing tricks for NumPy arrays that we saw in Lecture 2:

    \begin{tcolorbox}[breakable, size=fbox, boxrule=1pt, pad at break*=1mm,colback=cellbackground, colframe=cellborder]
\prompt{In}{incolor}{31}{\boxspacing}
\begin{Verbatim}[commandchars=\\\{\}]
\PY{n}{indices} \PY{o}{=} \PY{n}{where}\PY{p}{(}\PY{n}{w} \PY{o}{\PYZgt{}} \PY{l+m+mi}{0}\PY{p}{)} \PY{c+c1}{\PYZsh{} select only indices for elements that corresponds to positive frequencies}
\PY{n}{w\PYZus{}pos} \PY{o}{=} \PY{n}{w}\PY{p}{[}\PY{n}{indices}\PY{p}{]}
\PY{n}{F\PYZus{}pos} \PY{o}{=} \PY{n}{F}\PY{p}{[}\PY{n}{indices}\PY{p}{]}
\end{Verbatim}
\end{tcolorbox}

    \begin{Verbatim}[commandchars=\\\{\}]
<ipython-input-31-9b01025eb9eb>:1: DeprecationWarning: scipy.where is deprecated
and will be removed in SciPy 2.0.0, use numpy.where instead
  indices = where(w > 0) \# select only indices for elements that corresponds to
positive frequencies
    \end{Verbatim}

    \begin{tcolorbox}[breakable, size=fbox, boxrule=1pt, pad at break*=1mm,colback=cellbackground, colframe=cellborder]
\prompt{In}{incolor}{32}{\boxspacing}
\begin{Verbatim}[commandchars=\\\{\}]
\PY{n}{fig}\PY{p}{,} \PY{n}{ax} \PY{o}{=} \PY{n}{plt}\PY{o}{.}\PY{n}{subplots}\PY{p}{(}\PY{n}{figsize}\PY{o}{=}\PY{p}{(}\PY{l+m+mi}{9}\PY{p}{,}\PY{l+m+mi}{3}\PY{p}{)}\PY{p}{)}
\PY{n}{ax}\PY{o}{.}\PY{n}{plot}\PY{p}{(}\PY{n}{w\PYZus{}pos}\PY{p}{,} \PY{n+nb}{abs}\PY{p}{(}\PY{n}{F\PYZus{}pos}\PY{p}{)}\PY{p}{)}
\PY{n}{ax}\PY{o}{.}\PY{n}{set\PYZus{}xlim}\PY{p}{(}\PY{l+m+mi}{0}\PY{p}{,} \PY{l+m+mi}{5}\PY{p}{)}\PY{p}{;}
\end{Verbatim}
\end{tcolorbox}

    \begin{center}
    \adjustimage{max size={0.9\linewidth}{0.9\paperheight}}{Lecture-3-Scipy_files/Lecture-3-Scipy_59_0.png}
    \end{center}
    { \hspace*{\fill} \\}
    
    As expected, we now see a peak in the spectrum that is centered around
1, which is the frequency we used in the damped oscillator example.

    \hypertarget{linear-algebra}{%
\subsection{Linear algebra}\label{linear-algebra}}

    The linear algebra module contains a lot of matrix related functions,
including linear equation solving, eigenvalue solvers, matrix functions
(for example matrix-exponentiation), a number of different
decompositions (SVD, LU, cholesky), etc.

Detailed documetation is available at:
http://docs.scipy.org/doc/scipy/reference/linalg.html

Here we will look at how to use some of these functions:

    \hypertarget{linear-equation-systems}{%
\subsubsection{Linear equation systems}\label{linear-equation-systems}}

    Linear equation systems on the matrix form

\(A x = b\)

where \(A\) is a matrix and \(x,b\) are vectors can be solved like:

    \begin{tcolorbox}[breakable, size=fbox, boxrule=1pt, pad at break*=1mm,colback=cellbackground, colframe=cellborder]
\prompt{In}{incolor}{33}{\boxspacing}
\begin{Verbatim}[commandchars=\\\{\}]
\PY{k+kn}{from} \PY{n+nn}{scipy}\PY{n+nn}{.}\PY{n+nn}{linalg} \PY{k+kn}{import} \PY{o}{*}
\end{Verbatim}
\end{tcolorbox}

    \begin{tcolorbox}[breakable, size=fbox, boxrule=1pt, pad at break*=1mm,colback=cellbackground, colframe=cellborder]
\prompt{In}{incolor}{34}{\boxspacing}
\begin{Verbatim}[commandchars=\\\{\}]
\PY{n}{A} \PY{o}{=} \PY{n}{array}\PY{p}{(}\PY{p}{[}\PY{p}{[}\PY{l+m+mi}{1}\PY{p}{,}\PY{l+m+mi}{2}\PY{p}{,}\PY{l+m+mi}{3}\PY{p}{]}\PY{p}{,} \PY{p}{[}\PY{l+m+mi}{4}\PY{p}{,}\PY{l+m+mi}{5}\PY{p}{,}\PY{l+m+mi}{6}\PY{p}{]}\PY{p}{,} \PY{p}{[}\PY{l+m+mi}{7}\PY{p}{,}\PY{l+m+mi}{8}\PY{p}{,}\PY{l+m+mi}{9}\PY{p}{]}\PY{p}{]}\PY{p}{)}
\PY{n}{b} \PY{o}{=} \PY{n}{array}\PY{p}{(}\PY{p}{[}\PY{l+m+mi}{1}\PY{p}{,}\PY{l+m+mi}{2}\PY{p}{,}\PY{l+m+mi}{3}\PY{p}{]}\PY{p}{)}
\end{Verbatim}
\end{tcolorbox}

    \begin{Verbatim}[commandchars=\\\{\}]
<ipython-input-34-d72a989a316d>:1: DeprecationWarning: scipy.array is deprecated
and will be removed in SciPy 2.0.0, use numpy.array instead
  A = array([[1,2,3], [4,5,6], [7,8,9]])
<ipython-input-34-d72a989a316d>:2: DeprecationWarning: scipy.array is deprecated
and will be removed in SciPy 2.0.0, use numpy.array instead
  b = array([1,2,3])
    \end{Verbatim}

    \begin{tcolorbox}[breakable, size=fbox, boxrule=1pt, pad at break*=1mm,colback=cellbackground, colframe=cellborder]
\prompt{In}{incolor}{35}{\boxspacing}
\begin{Verbatim}[commandchars=\\\{\}]
\PY{n}{x} \PY{o}{=} \PY{n}{solve}\PY{p}{(}\PY{n}{A}\PY{p}{,} \PY{n}{b}\PY{p}{)}

\PY{n}{x}
\end{Verbatim}
\end{tcolorbox}

    \begin{Verbatim}[commandchars=\\\{\}]

        ---------------------------------------------------------------------------

        LinAlgError                               Traceback (most recent call last)

        <ipython-input-35-1be648719d5f> in <module>
    ----> 1 x = solve(A, b)
          2 
          3 x
    

        c:\textbackslash{}program files\textbackslash{}python38\textbackslash{}lib\textbackslash{}site-packages\textbackslash{}scipy\textbackslash{}linalg\textbackslash{}basic.py in solve(a, b, sym\_pos, lower, overwrite\_a, overwrite\_b, debug, check\_finite, assume\_a, transposed)
        214                                                (a1, b1))
        215         lu, ipvt, info = getrf(a1, overwrite\_a=overwrite\_a)
    --> 216         \_solve\_check(n, info)
        217         x, info = getrs(lu, ipvt, b1,
        218                         trans=trans, overwrite\_b=overwrite\_b)
    

        c:\textbackslash{}program files\textbackslash{}python38\textbackslash{}lib\textbackslash{}site-packages\textbackslash{}scipy\textbackslash{}linalg\textbackslash{}basic.py in \_solve\_check(n, info, lamch, rcond)
         29                          '.'.format(-info))
         30     elif 0 < info:
    ---> 31         raise LinAlgError('Matrix is singular.')
         32 
         33     if lamch is None:
    

        LinAlgError: Matrix is singular.

    \end{Verbatim}

    \begin{tcolorbox}[breakable, size=fbox, boxrule=1pt, pad at break*=1mm,colback=cellbackground, colframe=cellborder]
\prompt{In}{incolor}{36}{\boxspacing}
\begin{Verbatim}[commandchars=\\\{\}]
\PY{c+c1}{\PYZsh{} check}
\PY{n}{dot}\PY{p}{(}\PY{n}{A}\PY{p}{,} \PY{n}{x}\PY{p}{)} \PY{o}{\PYZhy{}} \PY{n}{b}
\end{Verbatim}
\end{tcolorbox}

    \begin{Verbatim}[commandchars=\\\{\}]
<ipython-input-36-65dbf7dd8f1d>:2: DeprecationWarning: scipy.dot is deprecated
and will be removed in SciPy 2.0.0, use numpy.dot instead
  dot(A, x) - b
    \end{Verbatim}

    \begin{Verbatim}[commandchars=\\\{\}]

        ---------------------------------------------------------------------------

        ValueError                                Traceback (most recent call last)

        <ipython-input-36-65dbf7dd8f1d> in <module>
          1 \# check
    ----> 2 dot(A, x) - b
    

        c:\textbackslash{}program files\textbackslash{}python38\textbackslash{}lib\textbackslash{}site-packages\textbackslash{}scipy\textbackslash{}\_lib\textbackslash{}deprecation.py in call(*args, **kwargs)
         18             warnings.warn(msg, category=DeprecationWarning,
         19                           stacklevel=stacklevel)
    ---> 20             return fun(*args, **kwargs)
         21         call.\_\_doc\_\_ = msg
         22         return call
    

        <\_\_array\_function\_\_ internals> in dot(*args, **kwargs)
    

        ValueError: shapes (3,3) and (250,4) not aligned: 3 (dim 1) != 250 (dim 0)

    \end{Verbatim}

    We can also do the same with

\(A X = B\)

where \(A, B, X\) are matrices:

    \begin{tcolorbox}[breakable, size=fbox, boxrule=1pt, pad at break*=1mm,colback=cellbackground, colframe=cellborder]
\prompt{In}{incolor}{37}{\boxspacing}
\begin{Verbatim}[commandchars=\\\{\}]
\PY{n}{A} \PY{o}{=} \PY{n}{rand}\PY{p}{(}\PY{l+m+mi}{3}\PY{p}{,}\PY{l+m+mi}{3}\PY{p}{)}
\PY{n}{B} \PY{o}{=} \PY{n}{rand}\PY{p}{(}\PY{l+m+mi}{3}\PY{p}{,}\PY{l+m+mi}{3}\PY{p}{)}
\end{Verbatim}
\end{tcolorbox}

    \begin{Verbatim}[commandchars=\\\{\}]
<ipython-input-37-fb78b91fd340>:1: DeprecationWarning: scipy.rand is deprecated
and will be removed in SciPy 2.0.0, use numpy.random.rand instead
  A = rand(3,3)
<ipython-input-37-fb78b91fd340>:2: DeprecationWarning: scipy.rand is deprecated
and will be removed in SciPy 2.0.0, use numpy.random.rand instead
  B = rand(3,3)
    \end{Verbatim}

    \begin{tcolorbox}[breakable, size=fbox, boxrule=1pt, pad at break*=1mm,colback=cellbackground, colframe=cellborder]
\prompt{In}{incolor}{38}{\boxspacing}
\begin{Verbatim}[commandchars=\\\{\}]
\PY{n}{X} \PY{o}{=} \PY{n}{solve}\PY{p}{(}\PY{n}{A}\PY{p}{,} \PY{n}{B}\PY{p}{)}
\end{Verbatim}
\end{tcolorbox}

    \begin{tcolorbox}[breakable, size=fbox, boxrule=1pt, pad at break*=1mm,colback=cellbackground, colframe=cellborder]
\prompt{In}{incolor}{39}{\boxspacing}
\begin{Verbatim}[commandchars=\\\{\}]
\PY{n}{X}
\end{Verbatim}
\end{tcolorbox}

            \begin{tcolorbox}[breakable, size=fbox, boxrule=.5pt, pad at break*=1mm, opacityfill=0]
\prompt{Out}{outcolor}{39}{\boxspacing}
\begin{Verbatim}[commandchars=\\\{\}]
array([[ 0.09848085,  0.27559363,  0.79789233],
       [-0.03249301,  0.75583669,  0.07135543],
       [ 0.17460686,  0.44178777, -0.08412888]])
\end{Verbatim}
\end{tcolorbox}
        
    \begin{tcolorbox}[breakable, size=fbox, boxrule=1pt, pad at break*=1mm,colback=cellbackground, colframe=cellborder]
\prompt{In}{incolor}{40}{\boxspacing}
\begin{Verbatim}[commandchars=\\\{\}]
\PY{c+c1}{\PYZsh{} check}
\PY{n}{norm}\PY{p}{(}\PY{n}{dot}\PY{p}{(}\PY{n}{A}\PY{p}{,} \PY{n}{X}\PY{p}{)} \PY{o}{\PYZhy{}} \PY{n}{B}\PY{p}{)}
\end{Verbatim}
\end{tcolorbox}

    \begin{Verbatim}[commandchars=\\\{\}]
<ipython-input-40-03be49e89a6e>:2: DeprecationWarning: scipy.dot is deprecated
and will be removed in SciPy 2.0.0, use numpy.dot instead
  norm(dot(A, X) - B)
    \end{Verbatim}

            \begin{tcolorbox}[breakable, size=fbox, boxrule=.5pt, pad at break*=1mm, opacityfill=0]
\prompt{Out}{outcolor}{40}{\boxspacing}
\begin{Verbatim}[commandchars=\\\{\}]
2.797157557069881e-17
\end{Verbatim}
\end{tcolorbox}
        
    \hypertarget{eigenvalues-and-eigenvectors}{%
\subsubsection{Eigenvalues and
eigenvectors}\label{eigenvalues-and-eigenvectors}}

    The eigenvalue problem for a matrix \(A\):

\(\displaystyle A v_n = \lambda_n v_n\)

where \(v_n\) is the \(n\)th eigenvector and \(\lambda_n\) is the
\(n\)th eigenvalue.

To calculate eigenvalues of a matrix, use the \texttt{eigvals} and for
calculating both eigenvalues and eigenvectors, use the function
\texttt{eig}:

    \begin{tcolorbox}[breakable, size=fbox, boxrule=1pt, pad at break*=1mm,colback=cellbackground, colframe=cellborder]
\prompt{In}{incolor}{41}{\boxspacing}
\begin{Verbatim}[commandchars=\\\{\}]
\PY{n}{evals} \PY{o}{=} \PY{n}{eigvals}\PY{p}{(}\PY{n}{A}\PY{p}{)}
\end{Verbatim}
\end{tcolorbox}

    \begin{tcolorbox}[breakable, size=fbox, boxrule=1pt, pad at break*=1mm,colback=cellbackground, colframe=cellborder]
\prompt{In}{incolor}{42}{\boxspacing}
\begin{Verbatim}[commandchars=\\\{\}]
\PY{n}{evals}
\end{Verbatim}
\end{tcolorbox}

            \begin{tcolorbox}[breakable, size=fbox, boxrule=.5pt, pad at break*=1mm, opacityfill=0]
\prompt{Out}{outcolor}{42}{\boxspacing}
\begin{Verbatim}[commandchars=\\\{\}]
array([ 1.48259965+0.j        , -0.26360912+0.54975957j,
       -0.26360912-0.54975957j])
\end{Verbatim}
\end{tcolorbox}
        
    \begin{tcolorbox}[breakable, size=fbox, boxrule=1pt, pad at break*=1mm,colback=cellbackground, colframe=cellborder]
\prompt{In}{incolor}{43}{\boxspacing}
\begin{Verbatim}[commandchars=\\\{\}]
\PY{n}{evals}\PY{p}{,} \PY{n}{evecs} \PY{o}{=} \PY{n}{eig}\PY{p}{(}\PY{n}{A}\PY{p}{)}
\end{Verbatim}
\end{tcolorbox}

    \begin{tcolorbox}[breakable, size=fbox, boxrule=1pt, pad at break*=1mm,colback=cellbackground, colframe=cellborder]
\prompt{In}{incolor}{44}{\boxspacing}
\begin{Verbatim}[commandchars=\\\{\}]
\PY{n}{evals}
\end{Verbatim}
\end{tcolorbox}

            \begin{tcolorbox}[breakable, size=fbox, boxrule=.5pt, pad at break*=1mm, opacityfill=0]
\prompt{Out}{outcolor}{44}{\boxspacing}
\begin{Verbatim}[commandchars=\\\{\}]
array([ 1.48259965+0.j        , -0.26360912+0.54975957j,
       -0.26360912-0.54975957j])
\end{Verbatim}
\end{tcolorbox}
        
    \begin{tcolorbox}[breakable, size=fbox, boxrule=1pt, pad at break*=1mm,colback=cellbackground, colframe=cellborder]
\prompt{In}{incolor}{45}{\boxspacing}
\begin{Verbatim}[commandchars=\\\{\}]
\PY{n}{evecs}
\end{Verbatim}
\end{tcolorbox}

            \begin{tcolorbox}[breakable, size=fbox, boxrule=.5pt, pad at break*=1mm, opacityfill=0]
\prompt{Out}{outcolor}{45}{\boxspacing}
\begin{Verbatim}[commandchars=\\\{\}]
array([[-0.59578591+0.j        ,  0.42773997+0.41602475j,
         0.42773997-0.41602475j],
       [-0.61777776+0.j        ,  0.32500586-0.39447121j,
         0.32500586+0.39447121j],
       [-0.51321515+0.j        , -0.6186482 +0.j        ,
        -0.6186482 -0.j        ]])
\end{Verbatim}
\end{tcolorbox}
        
    The eigenvectors corresponding to the \(n\)th eigenvalue (stored in
\texttt{evals{[}n{]}}) is the \(n\)th \emph{column} in \texttt{evecs},
i.e., \texttt{evecs{[}:,n{]}}. To verify this, let's try mutiplying
eigenvectors with the matrix and compare to the product of the
eigenvector and the eigenvalue:

    \begin{tcolorbox}[breakable, size=fbox, boxrule=1pt, pad at break*=1mm,colback=cellbackground, colframe=cellborder]
\prompt{In}{incolor}{46}{\boxspacing}
\begin{Verbatim}[commandchars=\\\{\}]
\PY{n}{n} \PY{o}{=} \PY{l+m+mi}{1}

\PY{n}{norm}\PY{p}{(}\PY{n}{dot}\PY{p}{(}\PY{n}{A}\PY{p}{,} \PY{n}{evecs}\PY{p}{[}\PY{p}{:}\PY{p}{,}\PY{n}{n}\PY{p}{]}\PY{p}{)} \PY{o}{\PYZhy{}} \PY{n}{evals}\PY{p}{[}\PY{n}{n}\PY{p}{]} \PY{o}{*} \PY{n}{evecs}\PY{p}{[}\PY{p}{:}\PY{p}{,}\PY{n}{n}\PY{p}{]}\PY{p}{)}
\end{Verbatim}
\end{tcolorbox}

    \begin{Verbatim}[commandchars=\\\{\}]
<ipython-input-46-289ff0b508b5>:3: DeprecationWarning: scipy.dot is deprecated
and will be removed in SciPy 2.0.0, use numpy.dot instead
  norm(dot(A, evecs[:,n]) - evals[n] * evecs[:,n])
    \end{Verbatim}

            \begin{tcolorbox}[breakable, size=fbox, boxrule=.5pt, pad at break*=1mm, opacityfill=0]
\prompt{Out}{outcolor}{46}{\boxspacing}
\begin{Verbatim}[commandchars=\\\{\}]
2.435541875787129e-16
\end{Verbatim}
\end{tcolorbox}
        
    There are also more specialized eigensolvers, like the \texttt{eigh} for
Hermitian matrices.

    \hypertarget{matrix-operations}{%
\subsubsection{Matrix operations}\label{matrix-operations}}

    \begin{tcolorbox}[breakable, size=fbox, boxrule=1pt, pad at break*=1mm,colback=cellbackground, colframe=cellborder]
\prompt{In}{incolor}{47}{\boxspacing}
\begin{Verbatim}[commandchars=\\\{\}]
\PY{c+c1}{\PYZsh{} the matrix inverse}
\PY{n}{inv}\PY{p}{(}\PY{n}{A}\PY{p}{)}
\end{Verbatim}
\end{tcolorbox}

            \begin{tcolorbox}[breakable, size=fbox, boxrule=.5pt, pad at break*=1mm, opacityfill=0]
\prompt{Out}{outcolor}{47}{\boxspacing}
\begin{Verbatim}[commandchars=\\\{\}]
array([[-0.75596532,  1.55429748, -0.21036934],
       [-0.37661276,  0.11216194,  1.11410373],
       [ 1.46744288, -0.7717956 , -0.10000618]])
\end{Verbatim}
\end{tcolorbox}
        
    \begin{tcolorbox}[breakable, size=fbox, boxrule=1pt, pad at break*=1mm,colback=cellbackground, colframe=cellborder]
\prompt{In}{incolor}{48}{\boxspacing}
\begin{Verbatim}[commandchars=\\\{\}]
\PY{c+c1}{\PYZsh{} determinant}
\PY{n}{det}\PY{p}{(}\PY{n}{A}\PY{p}{)}
\end{Verbatim}
\end{tcolorbox}

            \begin{tcolorbox}[breakable, size=fbox, boxrule=.5pt, pad at break*=1mm, opacityfill=0]
\prompt{Out}{outcolor}{48}{\boxspacing}
\begin{Verbatim}[commandchars=\\\{\}]
0.5511198761464301
\end{Verbatim}
\end{tcolorbox}
        
    \begin{tcolorbox}[breakable, size=fbox, boxrule=1pt, pad at break*=1mm,colback=cellbackground, colframe=cellborder]
\prompt{In}{incolor}{49}{\boxspacing}
\begin{Verbatim}[commandchars=\\\{\}]
\PY{c+c1}{\PYZsh{} norms of various orders}
\PY{n}{norm}\PY{p}{(}\PY{n}{A}\PY{p}{,} \PY{n+nb}{ord}\PY{o}{=}\PY{l+m+mi}{2}\PY{p}{)}\PY{p}{,} \PY{n}{norm}\PY{p}{(}\PY{n}{A}\PY{p}{,} \PY{n+nb}{ord}\PY{o}{=}\PY{n}{Inf}\PY{p}{)}
\end{Verbatim}
\end{tcolorbox}

            \begin{tcolorbox}[breakable, size=fbox, boxrule=.5pt, pad at break*=1mm, opacityfill=0]
\prompt{Out}{outcolor}{49}{\boxspacing}
\begin{Verbatim}[commandchars=\\\{\}]
(1.5304704006299419, 1.6102008947137665)
\end{Verbatim}
\end{tcolorbox}
        
    \hypertarget{sparse-matrices}{%
\subsubsection{Sparse matrices}\label{sparse-matrices}}

    Sparse matrices are often useful in numerical simulations dealing with
large systems, if the problem can be described in matrix form where the
matrices or vectors mostly contains zeros. Scipy has a good support for
sparse matrices, with basic linear algebra operations (such as equation
solving, eigenvalue calculations, etc).

There are many possible strategies for storing sparse matrices in an
efficient way. Some of the most common are the so-called coordinate form
(COO), list of list (LIL) form, and compressed-sparse column CSC (and
row, CSR). Each format has some advantanges and disadvantages. Most
computational algorithms (equation solving, matrix-matrix
multiplication, etc) can be efficiently implemented using CSR or CSC
formats, but they are not so intuitive and not so easy to initialize. So
often a sparse matrix is initially created in COO or LIL format (where
we can efficiently add elements to the sparse matrix data), and then
converted to CSC or CSR before used in real calcalations.

For more information about these sparse formats, see
e.g.~http://en.wikipedia.org/wiki/Sparse\_matrix

When we create a sparse matrix we have to choose which format it should
be stored in. For example,

    \begin{tcolorbox}[breakable, size=fbox, boxrule=1pt, pad at break*=1mm,colback=cellbackground, colframe=cellborder]
\prompt{In}{incolor}{50}{\boxspacing}
\begin{Verbatim}[commandchars=\\\{\}]
\PY{k+kn}{from} \PY{n+nn}{scipy}\PY{n+nn}{.}\PY{n+nn}{sparse} \PY{k+kn}{import} \PY{o}{*}
\end{Verbatim}
\end{tcolorbox}

    \begin{tcolorbox}[breakable, size=fbox, boxrule=1pt, pad at break*=1mm,colback=cellbackground, colframe=cellborder]
\prompt{In}{incolor}{51}{\boxspacing}
\begin{Verbatim}[commandchars=\\\{\}]
\PY{c+c1}{\PYZsh{} dense matrix}
\PY{n}{M} \PY{o}{=} \PY{n}{array}\PY{p}{(}\PY{p}{[}\PY{p}{[}\PY{l+m+mi}{1}\PY{p}{,}\PY{l+m+mi}{0}\PY{p}{,}\PY{l+m+mi}{0}\PY{p}{,}\PY{l+m+mi}{0}\PY{p}{]}\PY{p}{,} \PY{p}{[}\PY{l+m+mi}{0}\PY{p}{,}\PY{l+m+mi}{3}\PY{p}{,}\PY{l+m+mi}{0}\PY{p}{,}\PY{l+m+mi}{0}\PY{p}{]}\PY{p}{,} \PY{p}{[}\PY{l+m+mi}{0}\PY{p}{,}\PY{l+m+mi}{1}\PY{p}{,}\PY{l+m+mi}{1}\PY{p}{,}\PY{l+m+mi}{0}\PY{p}{]}\PY{p}{,} \PY{p}{[}\PY{l+m+mi}{1}\PY{p}{,}\PY{l+m+mi}{0}\PY{p}{,}\PY{l+m+mi}{0}\PY{p}{,}\PY{l+m+mi}{1}\PY{p}{]}\PY{p}{]}\PY{p}{)}\PY{p}{;} \PY{n}{M}
\end{Verbatim}
\end{tcolorbox}

    \begin{Verbatim}[commandchars=\\\{\}]
<ipython-input-51-156055dd1015>:2: DeprecationWarning: scipy.array is deprecated
and will be removed in SciPy 2.0.0, use numpy.array instead
  M = array([[1,0,0,0], [0,3,0,0], [0,1,1,0], [1,0,0,1]]); M
    \end{Verbatim}

            \begin{tcolorbox}[breakable, size=fbox, boxrule=.5pt, pad at break*=1mm, opacityfill=0]
\prompt{Out}{outcolor}{51}{\boxspacing}
\begin{Verbatim}[commandchars=\\\{\}]
array([[1, 0, 0, 0],
       [0, 3, 0, 0],
       [0, 1, 1, 0],
       [1, 0, 0, 1]])
\end{Verbatim}
\end{tcolorbox}
        
    \begin{tcolorbox}[breakable, size=fbox, boxrule=1pt, pad at break*=1mm,colback=cellbackground, colframe=cellborder]
\prompt{In}{incolor}{52}{\boxspacing}
\begin{Verbatim}[commandchars=\\\{\}]
\PY{c+c1}{\PYZsh{} convert from dense to sparse}
\PY{n}{A} \PY{o}{=} \PY{n}{csr\PYZus{}matrix}\PY{p}{(}\PY{n}{M}\PY{p}{)}\PY{p}{;} \PY{n}{A}
\end{Verbatim}
\end{tcolorbox}

            \begin{tcolorbox}[breakable, size=fbox, boxrule=.5pt, pad at break*=1mm, opacityfill=0]
\prompt{Out}{outcolor}{52}{\boxspacing}
\begin{Verbatim}[commandchars=\\\{\}]
<4x4 sparse matrix of type '<class 'numpy.intc'>'
        with 6 stored elements in Compressed Sparse Row format>
\end{Verbatim}
\end{tcolorbox}
        
    \begin{tcolorbox}[breakable, size=fbox, boxrule=1pt, pad at break*=1mm,colback=cellbackground, colframe=cellborder]
\prompt{In}{incolor}{53}{\boxspacing}
\begin{Verbatim}[commandchars=\\\{\}]
\PY{c+c1}{\PYZsh{} convert from sparse to dense}
\PY{n}{A}\PY{o}{.}\PY{n}{todense}\PY{p}{(}\PY{p}{)}
\end{Verbatim}
\end{tcolorbox}

            \begin{tcolorbox}[breakable, size=fbox, boxrule=.5pt, pad at break*=1mm, opacityfill=0]
\prompt{Out}{outcolor}{53}{\boxspacing}
\begin{Verbatim}[commandchars=\\\{\}]
matrix([[1, 0, 0, 0],
        [0, 3, 0, 0],
        [0, 1, 1, 0],
        [1, 0, 0, 1]], dtype=int32)
\end{Verbatim}
\end{tcolorbox}
        
    More efficient way to create sparse matrices: create an empty matrix and
populate with using matrix indexing (avoids creating a potentially large
dense matrix)

    \begin{tcolorbox}[breakable, size=fbox, boxrule=1pt, pad at break*=1mm,colback=cellbackground, colframe=cellborder]
\prompt{In}{incolor}{54}{\boxspacing}
\begin{Verbatim}[commandchars=\\\{\}]
\PY{n}{A} \PY{o}{=} \PY{n}{lil\PYZus{}matrix}\PY{p}{(}\PY{p}{(}\PY{l+m+mi}{4}\PY{p}{,}\PY{l+m+mi}{4}\PY{p}{)}\PY{p}{)} \PY{c+c1}{\PYZsh{} empty 4x4 sparse matrix}
\PY{n}{A}\PY{p}{[}\PY{l+m+mi}{0}\PY{p}{,}\PY{l+m+mi}{0}\PY{p}{]} \PY{o}{=} \PY{l+m+mi}{1}
\PY{n}{A}\PY{p}{[}\PY{l+m+mi}{1}\PY{p}{,}\PY{l+m+mi}{1}\PY{p}{]} \PY{o}{=} \PY{l+m+mi}{3}
\PY{n}{A}\PY{p}{[}\PY{l+m+mi}{2}\PY{p}{,}\PY{l+m+mi}{2}\PY{p}{]} \PY{o}{=} \PY{n}{A}\PY{p}{[}\PY{l+m+mi}{2}\PY{p}{,}\PY{l+m+mi}{1}\PY{p}{]} \PY{o}{=} \PY{l+m+mi}{1}
\PY{n}{A}\PY{p}{[}\PY{l+m+mi}{3}\PY{p}{,}\PY{l+m+mi}{3}\PY{p}{]} \PY{o}{=} \PY{n}{A}\PY{p}{[}\PY{l+m+mi}{3}\PY{p}{,}\PY{l+m+mi}{0}\PY{p}{]} \PY{o}{=} \PY{l+m+mi}{1}
\PY{n}{A}
\end{Verbatim}
\end{tcolorbox}

            \begin{tcolorbox}[breakable, size=fbox, boxrule=.5pt, pad at break*=1mm, opacityfill=0]
\prompt{Out}{outcolor}{54}{\boxspacing}
\begin{Verbatim}[commandchars=\\\{\}]
<4x4 sparse matrix of type '<class 'numpy.float64'>'
        with 6 stored elements in List of Lists format>
\end{Verbatim}
\end{tcolorbox}
        
    \begin{tcolorbox}[breakable, size=fbox, boxrule=1pt, pad at break*=1mm,colback=cellbackground, colframe=cellborder]
\prompt{In}{incolor}{55}{\boxspacing}
\begin{Verbatim}[commandchars=\\\{\}]
\PY{n}{A}\PY{o}{.}\PY{n}{todense}\PY{p}{(}\PY{p}{)}
\end{Verbatim}
\end{tcolorbox}

            \begin{tcolorbox}[breakable, size=fbox, boxrule=.5pt, pad at break*=1mm, opacityfill=0]
\prompt{Out}{outcolor}{55}{\boxspacing}
\begin{Verbatim}[commandchars=\\\{\}]
matrix([[1., 0., 0., 0.],
        [0., 3., 0., 0.],
        [0., 1., 1., 0.],
        [1., 0., 0., 1.]])
\end{Verbatim}
\end{tcolorbox}
        
    Converting between different sparse matrix formats:

    \begin{tcolorbox}[breakable, size=fbox, boxrule=1pt, pad at break*=1mm,colback=cellbackground, colframe=cellborder]
\prompt{In}{incolor}{56}{\boxspacing}
\begin{Verbatim}[commandchars=\\\{\}]
\PY{n}{A}
\end{Verbatim}
\end{tcolorbox}

            \begin{tcolorbox}[breakable, size=fbox, boxrule=.5pt, pad at break*=1mm, opacityfill=0]
\prompt{Out}{outcolor}{56}{\boxspacing}
\begin{Verbatim}[commandchars=\\\{\}]
<4x4 sparse matrix of type '<class 'numpy.float64'>'
        with 6 stored elements in List of Lists format>
\end{Verbatim}
\end{tcolorbox}
        
    \begin{tcolorbox}[breakable, size=fbox, boxrule=1pt, pad at break*=1mm,colback=cellbackground, colframe=cellborder]
\prompt{In}{incolor}{57}{\boxspacing}
\begin{Verbatim}[commandchars=\\\{\}]
\PY{n}{A} \PY{o}{=} \PY{n}{csr\PYZus{}matrix}\PY{p}{(}\PY{n}{A}\PY{p}{)}\PY{p}{;} \PY{n}{A}
\end{Verbatim}
\end{tcolorbox}

            \begin{tcolorbox}[breakable, size=fbox, boxrule=.5pt, pad at break*=1mm, opacityfill=0]
\prompt{Out}{outcolor}{57}{\boxspacing}
\begin{Verbatim}[commandchars=\\\{\}]
<4x4 sparse matrix of type '<class 'numpy.float64'>'
        with 6 stored elements in Compressed Sparse Row format>
\end{Verbatim}
\end{tcolorbox}
        
    \begin{tcolorbox}[breakable, size=fbox, boxrule=1pt, pad at break*=1mm,colback=cellbackground, colframe=cellborder]
\prompt{In}{incolor}{58}{\boxspacing}
\begin{Verbatim}[commandchars=\\\{\}]
\PY{n}{A} \PY{o}{=} \PY{n}{csc\PYZus{}matrix}\PY{p}{(}\PY{n}{A}\PY{p}{)}\PY{p}{;} \PY{n}{A}
\end{Verbatim}
\end{tcolorbox}

            \begin{tcolorbox}[breakable, size=fbox, boxrule=.5pt, pad at break*=1mm, opacityfill=0]
\prompt{Out}{outcolor}{58}{\boxspacing}
\begin{Verbatim}[commandchars=\\\{\}]
<4x4 sparse matrix of type '<class 'numpy.float64'>'
        with 6 stored elements in Compressed Sparse Column format>
\end{Verbatim}
\end{tcolorbox}
        
    We can compute with sparse matrices like with dense matrices:

    \begin{tcolorbox}[breakable, size=fbox, boxrule=1pt, pad at break*=1mm,colback=cellbackground, colframe=cellborder]
\prompt{In}{incolor}{59}{\boxspacing}
\begin{Verbatim}[commandchars=\\\{\}]
\PY{n}{A}\PY{o}{.}\PY{n}{todense}\PY{p}{(}\PY{p}{)}
\end{Verbatim}
\end{tcolorbox}

            \begin{tcolorbox}[breakable, size=fbox, boxrule=.5pt, pad at break*=1mm, opacityfill=0]
\prompt{Out}{outcolor}{59}{\boxspacing}
\begin{Verbatim}[commandchars=\\\{\}]
matrix([[1., 0., 0., 0.],
        [0., 3., 0., 0.],
        [0., 1., 1., 0.],
        [1., 0., 0., 1.]])
\end{Verbatim}
\end{tcolorbox}
        
    \begin{tcolorbox}[breakable, size=fbox, boxrule=1pt, pad at break*=1mm,colback=cellbackground, colframe=cellborder]
\prompt{In}{incolor}{60}{\boxspacing}
\begin{Verbatim}[commandchars=\\\{\}]
\PY{p}{(}\PY{n}{A} \PY{o}{*} \PY{n}{A}\PY{p}{)}\PY{o}{.}\PY{n}{todense}\PY{p}{(}\PY{p}{)}
\end{Verbatim}
\end{tcolorbox}

            \begin{tcolorbox}[breakable, size=fbox, boxrule=.5pt, pad at break*=1mm, opacityfill=0]
\prompt{Out}{outcolor}{60}{\boxspacing}
\begin{Verbatim}[commandchars=\\\{\}]
matrix([[1., 0., 0., 0.],
        [0., 9., 0., 0.],
        [0., 4., 1., 0.],
        [2., 0., 0., 1.]])
\end{Verbatim}
\end{tcolorbox}
        
    \begin{tcolorbox}[breakable, size=fbox, boxrule=1pt, pad at break*=1mm,colback=cellbackground, colframe=cellborder]
\prompt{In}{incolor}{61}{\boxspacing}
\begin{Verbatim}[commandchars=\\\{\}]
\PY{n}{A}\PY{o}{.}\PY{n}{todense}\PY{p}{(}\PY{p}{)}
\end{Verbatim}
\end{tcolorbox}

            \begin{tcolorbox}[breakable, size=fbox, boxrule=.5pt, pad at break*=1mm, opacityfill=0]
\prompt{Out}{outcolor}{61}{\boxspacing}
\begin{Verbatim}[commandchars=\\\{\}]
matrix([[1., 0., 0., 0.],
        [0., 3., 0., 0.],
        [0., 1., 1., 0.],
        [1., 0., 0., 1.]])
\end{Verbatim}
\end{tcolorbox}
        
    \begin{tcolorbox}[breakable, size=fbox, boxrule=1pt, pad at break*=1mm,colback=cellbackground, colframe=cellborder]
\prompt{In}{incolor}{62}{\boxspacing}
\begin{Verbatim}[commandchars=\\\{\}]
\PY{n}{A}\PY{o}{.}\PY{n}{dot}\PY{p}{(}\PY{n}{A}\PY{p}{)}\PY{o}{.}\PY{n}{todense}\PY{p}{(}\PY{p}{)}
\end{Verbatim}
\end{tcolorbox}

            \begin{tcolorbox}[breakable, size=fbox, boxrule=.5pt, pad at break*=1mm, opacityfill=0]
\prompt{Out}{outcolor}{62}{\boxspacing}
\begin{Verbatim}[commandchars=\\\{\}]
matrix([[1., 0., 0., 0.],
        [0., 9., 0., 0.],
        [0., 4., 1., 0.],
        [2., 0., 0., 1.]])
\end{Verbatim}
\end{tcolorbox}
        
    \begin{tcolorbox}[breakable, size=fbox, boxrule=1pt, pad at break*=1mm,colback=cellbackground, colframe=cellborder]
\prompt{In}{incolor}{63}{\boxspacing}
\begin{Verbatim}[commandchars=\\\{\}]
\PY{n}{v} \PY{o}{=} \PY{n}{array}\PY{p}{(}\PY{p}{[}\PY{l+m+mi}{1}\PY{p}{,}\PY{l+m+mi}{2}\PY{p}{,}\PY{l+m+mi}{3}\PY{p}{,}\PY{l+m+mi}{4}\PY{p}{]}\PY{p}{)}\PY{p}{[}\PY{p}{:}\PY{p}{,}\PY{n}{newaxis}\PY{p}{]}\PY{p}{;} \PY{n}{v}
\end{Verbatim}
\end{tcolorbox}

    \begin{Verbatim}[commandchars=\\\{\}]
<ipython-input-63-c06991884813>:1: DeprecationWarning: scipy.array is deprecated
and will be removed in SciPy 2.0.0, use numpy.array instead
  v = array([1,2,3,4])[:,newaxis]; v
    \end{Verbatim}

            \begin{tcolorbox}[breakable, size=fbox, boxrule=.5pt, pad at break*=1mm, opacityfill=0]
\prompt{Out}{outcolor}{63}{\boxspacing}
\begin{Verbatim}[commandchars=\\\{\}]
array([[1],
       [2],
       [3],
       [4]])
\end{Verbatim}
\end{tcolorbox}
        
    \begin{tcolorbox}[breakable, size=fbox, boxrule=1pt, pad at break*=1mm,colback=cellbackground, colframe=cellborder]
\prompt{In}{incolor}{64}{\boxspacing}
\begin{Verbatim}[commandchars=\\\{\}]
\PY{c+c1}{\PYZsh{} sparse matrix \PYZhy{} dense vector multiplication}
\PY{n}{A} \PY{o}{*} \PY{n}{v}
\end{Verbatim}
\end{tcolorbox}

            \begin{tcolorbox}[breakable, size=fbox, boxrule=.5pt, pad at break*=1mm, opacityfill=0]
\prompt{Out}{outcolor}{64}{\boxspacing}
\begin{Verbatim}[commandchars=\\\{\}]
array([[1.],
       [6.],
       [5.],
       [5.]])
\end{Verbatim}
\end{tcolorbox}
        
    \begin{tcolorbox}[breakable, size=fbox, boxrule=1pt, pad at break*=1mm,colback=cellbackground, colframe=cellborder]
\prompt{In}{incolor}{65}{\boxspacing}
\begin{Verbatim}[commandchars=\\\{\}]
\PY{c+c1}{\PYZsh{} same result with dense matrix \PYZhy{} dense vector multiplcation}
\PY{n}{A}\PY{o}{.}\PY{n}{todense}\PY{p}{(}\PY{p}{)} \PY{o}{*} \PY{n}{v}
\end{Verbatim}
\end{tcolorbox}

            \begin{tcolorbox}[breakable, size=fbox, boxrule=.5pt, pad at break*=1mm, opacityfill=0]
\prompt{Out}{outcolor}{65}{\boxspacing}
\begin{Verbatim}[commandchars=\\\{\}]
matrix([[1.],
        [6.],
        [5.],
        [5.]])
\end{Verbatim}
\end{tcolorbox}
        
    \hypertarget{optimization}{%
\subsection{Optimization}\label{optimization}}

    Optimization (finding minima or maxima of a function) is a large field
in mathematics, and optimization of complicated functions or in many
variables can be rather involved. Here we will only look at a few very
simple cases. For a more detailed introduction to optimization with
SciPy see:
http://scipy-lectures.github.com/advanced/mathematical\_optimization/index.html

To use the optimization module in scipy first include the
\texttt{optimize} module:

    \begin{tcolorbox}[breakable, size=fbox, boxrule=1pt, pad at break*=1mm,colback=cellbackground, colframe=cellborder]
\prompt{In}{incolor}{66}{\boxspacing}
\begin{Verbatim}[commandchars=\\\{\}]
\PY{k+kn}{from} \PY{n+nn}{scipy} \PY{k+kn}{import} \PY{n}{optimize}
\end{Verbatim}
\end{tcolorbox}

    \hypertarget{finding-a-minima}{%
\subsubsection{Finding a minima}\label{finding-a-minima}}

    Let's first look at how to find the minima of a simple function of a
single variable:

    \begin{tcolorbox}[breakable, size=fbox, boxrule=1pt, pad at break*=1mm,colback=cellbackground, colframe=cellborder]
\prompt{In}{incolor}{67}{\boxspacing}
\begin{Verbatim}[commandchars=\\\{\}]
\PY{k}{def} \PY{n+nf}{f}\PY{p}{(}\PY{n}{x}\PY{p}{)}\PY{p}{:}
    \PY{k}{return} \PY{l+m+mi}{4}\PY{o}{*}\PY{n}{x}\PY{o}{*}\PY{o}{*}\PY{l+m+mi}{3} \PY{o}{+} \PY{p}{(}\PY{n}{x}\PY{o}{\PYZhy{}}\PY{l+m+mi}{2}\PY{p}{)}\PY{o}{*}\PY{o}{*}\PY{l+m+mi}{2} \PY{o}{+} \PY{n}{x}\PY{o}{*}\PY{o}{*}\PY{l+m+mi}{4}
\end{Verbatim}
\end{tcolorbox}

    \begin{tcolorbox}[breakable, size=fbox, boxrule=1pt, pad at break*=1mm,colback=cellbackground, colframe=cellborder]
\prompt{In}{incolor}{68}{\boxspacing}
\begin{Verbatim}[commandchars=\\\{\}]
\PY{n}{fig}\PY{p}{,} \PY{n}{ax}  \PY{o}{=} \PY{n}{plt}\PY{o}{.}\PY{n}{subplots}\PY{p}{(}\PY{p}{)}
\PY{n}{x} \PY{o}{=} \PY{n}{linspace}\PY{p}{(}\PY{o}{\PYZhy{}}\PY{l+m+mi}{5}\PY{p}{,} \PY{l+m+mi}{3}\PY{p}{,} \PY{l+m+mi}{100}\PY{p}{)}
\PY{n}{ax}\PY{o}{.}\PY{n}{plot}\PY{p}{(}\PY{n}{x}\PY{p}{,} \PY{n}{f}\PY{p}{(}\PY{n}{x}\PY{p}{)}\PY{p}{)}\PY{p}{;}
\end{Verbatim}
\end{tcolorbox}

    \begin{Verbatim}[commandchars=\\\{\}]
<ipython-input-68-b4dec7d143f8>:2: DeprecationWarning: scipy.linspace is
deprecated and will be removed in SciPy 2.0.0, use numpy.linspace instead
  x = linspace(-5, 3, 100)
    \end{Verbatim}

    \begin{center}
    \adjustimage{max size={0.9\linewidth}{0.9\paperheight}}{Lecture-3-Scipy_files/Lecture-3-Scipy_115_1.png}
    \end{center}
    { \hspace*{\fill} \\}
    
    We can use the \texttt{fmin\_bfgs} function to find the minima of a
function:

    \begin{tcolorbox}[breakable, size=fbox, boxrule=1pt, pad at break*=1mm,colback=cellbackground, colframe=cellborder]
\prompt{In}{incolor}{69}{\boxspacing}
\begin{Verbatim}[commandchars=\\\{\}]
\PY{n}{x\PYZus{}min} \PY{o}{=} \PY{n}{optimize}\PY{o}{.}\PY{n}{fmin\PYZus{}bfgs}\PY{p}{(}\PY{n}{f}\PY{p}{,} \PY{o}{\PYZhy{}}\PY{l+m+mi}{2}\PY{p}{)}
\PY{n}{x\PYZus{}min} 
\end{Verbatim}
\end{tcolorbox}

    \begin{Verbatim}[commandchars=\\\{\}]
Optimization terminated successfully.
         Current function value: -3.506641
         Iterations: 5
         Function evaluations: 24
         Gradient evaluations: 8
    \end{Verbatim}

            \begin{tcolorbox}[breakable, size=fbox, boxrule=.5pt, pad at break*=1mm, opacityfill=0]
\prompt{Out}{outcolor}{69}{\boxspacing}
\begin{Verbatim}[commandchars=\\\{\}]
array([-2.67298155])
\end{Verbatim}
\end{tcolorbox}
        
    \begin{tcolorbox}[breakable, size=fbox, boxrule=1pt, pad at break*=1mm,colback=cellbackground, colframe=cellborder]
\prompt{In}{incolor}{70}{\boxspacing}
\begin{Verbatim}[commandchars=\\\{\}]
\PY{n}{optimize}\PY{o}{.}\PY{n}{fmin\PYZus{}bfgs}\PY{p}{(}\PY{n}{f}\PY{p}{,} \PY{l+m+mf}{0.5}\PY{p}{)} 
\end{Verbatim}
\end{tcolorbox}

    \begin{Verbatim}[commandchars=\\\{\}]
Optimization terminated successfully.
         Current function value: 2.804988
         Iterations: 3
         Function evaluations: 15
         Gradient evaluations: 5
    \end{Verbatim}

            \begin{tcolorbox}[breakable, size=fbox, boxrule=.5pt, pad at break*=1mm, opacityfill=0]
\prompt{Out}{outcolor}{70}{\boxspacing}
\begin{Verbatim}[commandchars=\\\{\}]
array([0.46961745])
\end{Verbatim}
\end{tcolorbox}
        
    We can also use the \texttt{brent} or \texttt{fminbound} functions. They
have a bit different syntax and use different algorithms.

    \begin{tcolorbox}[breakable, size=fbox, boxrule=1pt, pad at break*=1mm,colback=cellbackground, colframe=cellborder]
\prompt{In}{incolor}{71}{\boxspacing}
\begin{Verbatim}[commandchars=\\\{\}]
\PY{n}{optimize}\PY{o}{.}\PY{n}{brent}\PY{p}{(}\PY{n}{f}\PY{p}{)}
\end{Verbatim}
\end{tcolorbox}

            \begin{tcolorbox}[breakable, size=fbox, boxrule=.5pt, pad at break*=1mm, opacityfill=0]
\prompt{Out}{outcolor}{71}{\boxspacing}
\begin{Verbatim}[commandchars=\\\{\}]
0.46961743402759754
\end{Verbatim}
\end{tcolorbox}
        
    \begin{tcolorbox}[breakable, size=fbox, boxrule=1pt, pad at break*=1mm,colback=cellbackground, colframe=cellborder]
\prompt{In}{incolor}{72}{\boxspacing}
\begin{Verbatim}[commandchars=\\\{\}]
\PY{n}{optimize}\PY{o}{.}\PY{n}{fminbound}\PY{p}{(}\PY{n}{f}\PY{p}{,} \PY{o}{\PYZhy{}}\PY{l+m+mi}{4}\PY{p}{,} \PY{l+m+mi}{2}\PY{p}{)}
\end{Verbatim}
\end{tcolorbox}

            \begin{tcolorbox}[breakable, size=fbox, boxrule=.5pt, pad at break*=1mm, opacityfill=0]
\prompt{Out}{outcolor}{72}{\boxspacing}
\begin{Verbatim}[commandchars=\\\{\}]
-2.6729822917513886
\end{Verbatim}
\end{tcolorbox}
        
    \hypertarget{finding-a-solution-to-a-function}{%
\subsubsection{Finding a solution to a
function}\label{finding-a-solution-to-a-function}}

    To find the root for a function of the form \(f(x) = 0\) we can use the
\texttt{fsolve} function. It requires an initial guess:

    \begin{tcolorbox}[breakable, size=fbox, boxrule=1pt, pad at break*=1mm,colback=cellbackground, colframe=cellborder]
\prompt{In}{incolor}{73}{\boxspacing}
\begin{Verbatim}[commandchars=\\\{\}]
\PY{n}{omega\PYZus{}c} \PY{o}{=} \PY{l+m+mf}{3.0}
\PY{k}{def} \PY{n+nf}{f}\PY{p}{(}\PY{n}{omega}\PY{p}{)}\PY{p}{:}
    \PY{c+c1}{\PYZsh{} a transcendental equation: resonance frequencies of a low\PYZhy{}Q SQUID terminated microwave resonator}
    \PY{k}{return} \PY{n}{tan}\PY{p}{(}\PY{l+m+mi}{2}\PY{o}{*}\PY{n}{pi}\PY{o}{*}\PY{n}{omega}\PY{p}{)} \PY{o}{\PYZhy{}} \PY{n}{omega\PYZus{}c}\PY{o}{/}\PY{n}{omega}
\end{Verbatim}
\end{tcolorbox}

    \begin{tcolorbox}[breakable, size=fbox, boxrule=1pt, pad at break*=1mm,colback=cellbackground, colframe=cellborder]
\prompt{In}{incolor}{74}{\boxspacing}
\begin{Verbatim}[commandchars=\\\{\}]
\PY{n}{fig}\PY{p}{,} \PY{n}{ax}  \PY{o}{=} \PY{n}{plt}\PY{o}{.}\PY{n}{subplots}\PY{p}{(}\PY{n}{figsize}\PY{o}{=}\PY{p}{(}\PY{l+m+mi}{10}\PY{p}{,}\PY{l+m+mi}{4}\PY{p}{)}\PY{p}{)}
\PY{n}{x} \PY{o}{=} \PY{n}{linspace}\PY{p}{(}\PY{l+m+mi}{0}\PY{p}{,} \PY{l+m+mi}{3}\PY{p}{,} \PY{l+m+mi}{1000}\PY{p}{)}
\PY{n}{y} \PY{o}{=} \PY{n}{f}\PY{p}{(}\PY{n}{x}\PY{p}{)}
\PY{n}{mask} \PY{o}{=} \PY{n}{where}\PY{p}{(}\PY{n+nb}{abs}\PY{p}{(}\PY{n}{y}\PY{p}{)} \PY{o}{\PYZgt{}} \PY{l+m+mi}{50}\PY{p}{)}
\PY{n}{x}\PY{p}{[}\PY{n}{mask}\PY{p}{]} \PY{o}{=} \PY{n}{y}\PY{p}{[}\PY{n}{mask}\PY{p}{]} \PY{o}{=} \PY{n}{NaN} \PY{c+c1}{\PYZsh{} get rid of vertical line when the function flip sign}
\PY{n}{ax}\PY{o}{.}\PY{n}{plot}\PY{p}{(}\PY{n}{x}\PY{p}{,} \PY{n}{y}\PY{p}{)}
\PY{n}{ax}\PY{o}{.}\PY{n}{plot}\PY{p}{(}\PY{p}{[}\PY{l+m+mi}{0}\PY{p}{,} \PY{l+m+mi}{3}\PY{p}{]}\PY{p}{,} \PY{p}{[}\PY{l+m+mi}{0}\PY{p}{,} \PY{l+m+mi}{0}\PY{p}{]}\PY{p}{,} \PY{l+s+s1}{\PYZsq{}}\PY{l+s+s1}{k}\PY{l+s+s1}{\PYZsq{}}\PY{p}{)}
\PY{n}{ax}\PY{o}{.}\PY{n}{set\PYZus{}ylim}\PY{p}{(}\PY{o}{\PYZhy{}}\PY{l+m+mi}{5}\PY{p}{,}\PY{l+m+mi}{5}\PY{p}{)}\PY{p}{;}
\end{Verbatim}
\end{tcolorbox}

    \begin{Verbatim}[commandchars=\\\{\}]
<ipython-input-74-8fa973be786d>:2: DeprecationWarning: scipy.linspace is
deprecated and will be removed in SciPy 2.0.0, use numpy.linspace instead
  x = linspace(0, 3, 1000)
<ipython-input-73-35ba93b217b0>:4: DeprecationWarning: scipy.tan is deprecated
and will be removed in SciPy 2.0.0, use numpy.tan instead
  return tan(2*pi*omega) - omega\_c/omega
<ipython-input-73-35ba93b217b0>:4: RuntimeWarning: divide by zero encountered in
true\_divide
  return tan(2*pi*omega) - omega\_c/omega
<ipython-input-74-8fa973be786d>:4: DeprecationWarning: scipy.where is deprecated
and will be removed in SciPy 2.0.0, use numpy.where instead
  mask = where(abs(y) > 50)
    \end{Verbatim}

    \begin{center}
    \adjustimage{max size={0.9\linewidth}{0.9\paperheight}}{Lecture-3-Scipy_files/Lecture-3-Scipy_125_1.png}
    \end{center}
    { \hspace*{\fill} \\}
    
    \begin{tcolorbox}[breakable, size=fbox, boxrule=1pt, pad at break*=1mm,colback=cellbackground, colframe=cellborder]
\prompt{In}{incolor}{75}{\boxspacing}
\begin{Verbatim}[commandchars=\\\{\}]
\PY{n}{optimize}\PY{o}{.}\PY{n}{fsolve}\PY{p}{(}\PY{n}{f}\PY{p}{,} \PY{l+m+mf}{0.1}\PY{p}{)}
\end{Verbatim}
\end{tcolorbox}

    \begin{Verbatim}[commandchars=\\\{\}]
<ipython-input-73-35ba93b217b0>:4: DeprecationWarning: scipy.tan is deprecated
and will be removed in SciPy 2.0.0, use numpy.tan instead
  return tan(2*pi*omega) - omega\_c/omega
    \end{Verbatim}

            \begin{tcolorbox}[breakable, size=fbox, boxrule=.5pt, pad at break*=1mm, opacityfill=0]
\prompt{Out}{outcolor}{75}{\boxspacing}
\begin{Verbatim}[commandchars=\\\{\}]
array([0.23743014])
\end{Verbatim}
\end{tcolorbox}
        
    \begin{tcolorbox}[breakable, size=fbox, boxrule=1pt, pad at break*=1mm,colback=cellbackground, colframe=cellborder]
\prompt{In}{incolor}{76}{\boxspacing}
\begin{Verbatim}[commandchars=\\\{\}]
\PY{n}{optimize}\PY{o}{.}\PY{n}{fsolve}\PY{p}{(}\PY{n}{f}\PY{p}{,} \PY{l+m+mf}{0.6}\PY{p}{)}
\end{Verbatim}
\end{tcolorbox}

    \begin{Verbatim}[commandchars=\\\{\}]
<ipython-input-73-35ba93b217b0>:4: DeprecationWarning: scipy.tan is deprecated
and will be removed in SciPy 2.0.0, use numpy.tan instead
  return tan(2*pi*omega) - omega\_c/omega
    \end{Verbatim}

            \begin{tcolorbox}[breakable, size=fbox, boxrule=.5pt, pad at break*=1mm, opacityfill=0]
\prompt{Out}{outcolor}{76}{\boxspacing}
\begin{Verbatim}[commandchars=\\\{\}]
array([0.71286972])
\end{Verbatim}
\end{tcolorbox}
        
    \begin{tcolorbox}[breakable, size=fbox, boxrule=1pt, pad at break*=1mm,colback=cellbackground, colframe=cellborder]
\prompt{In}{incolor}{77}{\boxspacing}
\begin{Verbatim}[commandchars=\\\{\}]
\PY{n}{optimize}\PY{o}{.}\PY{n}{fsolve}\PY{p}{(}\PY{n}{f}\PY{p}{,} \PY{l+m+mf}{1.1}\PY{p}{)}
\end{Verbatim}
\end{tcolorbox}

    \begin{Verbatim}[commandchars=\\\{\}]
<ipython-input-73-35ba93b217b0>:4: DeprecationWarning: scipy.tan is deprecated
and will be removed in SciPy 2.0.0, use numpy.tan instead
  return tan(2*pi*omega) - omega\_c/omega
    \end{Verbatim}

            \begin{tcolorbox}[breakable, size=fbox, boxrule=.5pt, pad at break*=1mm, opacityfill=0]
\prompt{Out}{outcolor}{77}{\boxspacing}
\begin{Verbatim}[commandchars=\\\{\}]
array([1.18990285])
\end{Verbatim}
\end{tcolorbox}
        
    \hypertarget{interpolation}{%
\subsection{Interpolation}\label{interpolation}}

    Interpolation is simple and convenient in scipy: The \texttt{interp1d}
function, when given arrays describing X and Y data, returns and object
that behaves like a function that can be called for an arbitrary value
of x (in the range covered by X), and it returns the corresponding
interpolated y value:

    \begin{tcolorbox}[breakable, size=fbox, boxrule=1pt, pad at break*=1mm,colback=cellbackground, colframe=cellborder]
\prompt{In}{incolor}{78}{\boxspacing}
\begin{Verbatim}[commandchars=\\\{\}]
\PY{k+kn}{from} \PY{n+nn}{scipy}\PY{n+nn}{.}\PY{n+nn}{interpolate} \PY{k+kn}{import} \PY{o}{*}
\end{Verbatim}
\end{tcolorbox}

    \begin{tcolorbox}[breakable, size=fbox, boxrule=1pt, pad at break*=1mm,colback=cellbackground, colframe=cellborder]
\prompt{In}{incolor}{79}{\boxspacing}
\begin{Verbatim}[commandchars=\\\{\}]
\PY{k}{def} \PY{n+nf}{f}\PY{p}{(}\PY{n}{x}\PY{p}{)}\PY{p}{:}
    \PY{k}{return} \PY{n}{sin}\PY{p}{(}\PY{n}{x}\PY{p}{)}
\end{Verbatim}
\end{tcolorbox}

    \begin{tcolorbox}[breakable, size=fbox, boxrule=1pt, pad at break*=1mm,colback=cellbackground, colframe=cellborder]
\prompt{In}{incolor}{80}{\boxspacing}
\begin{Verbatim}[commandchars=\\\{\}]
\PY{n}{n} \PY{o}{=} \PY{n}{arange}\PY{p}{(}\PY{l+m+mi}{0}\PY{p}{,} \PY{l+m+mi}{10}\PY{p}{)}  
\PY{n}{x} \PY{o}{=} \PY{n}{linspace}\PY{p}{(}\PY{l+m+mi}{0}\PY{p}{,} \PY{l+m+mi}{9}\PY{p}{,} \PY{l+m+mi}{100}\PY{p}{)}

\PY{n}{y\PYZus{}meas} \PY{o}{=} \PY{n}{f}\PY{p}{(}\PY{n}{n}\PY{p}{)} \PY{o}{+} \PY{l+m+mf}{0.1} \PY{o}{*} \PY{n}{randn}\PY{p}{(}\PY{n+nb}{len}\PY{p}{(}\PY{n}{n}\PY{p}{)}\PY{p}{)} \PY{c+c1}{\PYZsh{} simulate measurement with noise}
\PY{n}{y\PYZus{}real} \PY{o}{=} \PY{n}{f}\PY{p}{(}\PY{n}{x}\PY{p}{)}

\PY{n}{linear\PYZus{}interpolation} \PY{o}{=} \PY{n}{interp1d}\PY{p}{(}\PY{n}{n}\PY{p}{,} \PY{n}{y\PYZus{}meas}\PY{p}{)}
\PY{n}{y\PYZus{}interp1} \PY{o}{=} \PY{n}{linear\PYZus{}interpolation}\PY{p}{(}\PY{n}{x}\PY{p}{)}

\PY{n}{cubic\PYZus{}interpolation} \PY{o}{=} \PY{n}{interp1d}\PY{p}{(}\PY{n}{n}\PY{p}{,} \PY{n}{y\PYZus{}meas}\PY{p}{,} \PY{n}{kind}\PY{o}{=}\PY{l+s+s1}{\PYZsq{}}\PY{l+s+s1}{cubic}\PY{l+s+s1}{\PYZsq{}}\PY{p}{)}
\PY{n}{y\PYZus{}interp2} \PY{o}{=} \PY{n}{cubic\PYZus{}interpolation}\PY{p}{(}\PY{n}{x}\PY{p}{)}
\end{Verbatim}
\end{tcolorbox}

    \begin{Verbatim}[commandchars=\\\{\}]
<ipython-input-80-b19110fe243b>:1: DeprecationWarning: scipy.arange is
deprecated and will be removed in SciPy 2.0.0, use numpy.arange instead
  n = arange(0, 10)
<ipython-input-80-b19110fe243b>:2: DeprecationWarning: scipy.linspace is
deprecated and will be removed in SciPy 2.0.0, use numpy.linspace instead
  x = linspace(0, 9, 100)
<ipython-input-79-2ef07bb30186>:2: DeprecationWarning: scipy.sin is deprecated
and will be removed in SciPy 2.0.0, use numpy.sin instead
  return sin(x)
<ipython-input-80-b19110fe243b>:4: DeprecationWarning: scipy.randn is deprecated
and will be removed in SciPy 2.0.0, use numpy.random.randn instead
  y\_meas = f(n) + 0.1 * randn(len(n)) \# simulate measurement with noise
    \end{Verbatim}

    \begin{tcolorbox}[breakable, size=fbox, boxrule=1pt, pad at break*=1mm,colback=cellbackground, colframe=cellborder]
\prompt{In}{incolor}{81}{\boxspacing}
\begin{Verbatim}[commandchars=\\\{\}]
\PY{n}{fig}\PY{p}{,} \PY{n}{ax} \PY{o}{=} \PY{n}{plt}\PY{o}{.}\PY{n}{subplots}\PY{p}{(}\PY{n}{figsize}\PY{o}{=}\PY{p}{(}\PY{l+m+mi}{10}\PY{p}{,}\PY{l+m+mi}{4}\PY{p}{)}\PY{p}{)}
\PY{n}{ax}\PY{o}{.}\PY{n}{plot}\PY{p}{(}\PY{n}{n}\PY{p}{,} \PY{n}{y\PYZus{}meas}\PY{p}{,} \PY{l+s+s1}{\PYZsq{}}\PY{l+s+s1}{bs}\PY{l+s+s1}{\PYZsq{}}\PY{p}{,} \PY{n}{label}\PY{o}{=}\PY{l+s+s1}{\PYZsq{}}\PY{l+s+s1}{noisy data}\PY{l+s+s1}{\PYZsq{}}\PY{p}{)}
\PY{n}{ax}\PY{o}{.}\PY{n}{plot}\PY{p}{(}\PY{n}{x}\PY{p}{,} \PY{n}{y\PYZus{}real}\PY{p}{,} \PY{l+s+s1}{\PYZsq{}}\PY{l+s+s1}{k}\PY{l+s+s1}{\PYZsq{}}\PY{p}{,} \PY{n}{lw}\PY{o}{=}\PY{l+m+mi}{2}\PY{p}{,} \PY{n}{label}\PY{o}{=}\PY{l+s+s1}{\PYZsq{}}\PY{l+s+s1}{true function}\PY{l+s+s1}{\PYZsq{}}\PY{p}{)}
\PY{n}{ax}\PY{o}{.}\PY{n}{plot}\PY{p}{(}\PY{n}{x}\PY{p}{,} \PY{n}{y\PYZus{}interp1}\PY{p}{,} \PY{l+s+s1}{\PYZsq{}}\PY{l+s+s1}{r}\PY{l+s+s1}{\PYZsq{}}\PY{p}{,} \PY{n}{label}\PY{o}{=}\PY{l+s+s1}{\PYZsq{}}\PY{l+s+s1}{linear interp}\PY{l+s+s1}{\PYZsq{}}\PY{p}{)}
\PY{n}{ax}\PY{o}{.}\PY{n}{plot}\PY{p}{(}\PY{n}{x}\PY{p}{,} \PY{n}{y\PYZus{}interp2}\PY{p}{,} \PY{l+s+s1}{\PYZsq{}}\PY{l+s+s1}{g}\PY{l+s+s1}{\PYZsq{}}\PY{p}{,} \PY{n}{label}\PY{o}{=}\PY{l+s+s1}{\PYZsq{}}\PY{l+s+s1}{cubic interp}\PY{l+s+s1}{\PYZsq{}}\PY{p}{)}
\PY{n}{ax}\PY{o}{.}\PY{n}{legend}\PY{p}{(}\PY{n}{loc}\PY{o}{=}\PY{l+m+mi}{3}\PY{p}{)}\PY{p}{;}
\end{Verbatim}
\end{tcolorbox}

    \begin{center}
    \adjustimage{max size={0.9\linewidth}{0.9\paperheight}}{Lecture-3-Scipy_files/Lecture-3-Scipy_134_0.png}
    \end{center}
    { \hspace*{\fill} \\}
    
    \hypertarget{statistics}{%
\subsection{Statistics}\label{statistics}}

    The \texttt{scipy.stats} module contains a large number of statistical
distributions, statistical functions and tests. For a complete
documentation of its features, see
http://docs.scipy.org/doc/scipy/reference/stats.html.

There is also a very powerful python package for statistical modelling
called statsmodels. See http://statsmodels.sourceforge.net for more
details.

    \begin{tcolorbox}[breakable, size=fbox, boxrule=1pt, pad at break*=1mm,colback=cellbackground, colframe=cellborder]
\prompt{In}{incolor}{82}{\boxspacing}
\begin{Verbatim}[commandchars=\\\{\}]
\PY{k+kn}{from} \PY{n+nn}{scipy} \PY{k+kn}{import} \PY{n}{stats}
\end{Verbatim}
\end{tcolorbox}

    \begin{tcolorbox}[breakable, size=fbox, boxrule=1pt, pad at break*=1mm,colback=cellbackground, colframe=cellborder]
\prompt{In}{incolor}{83}{\boxspacing}
\begin{Verbatim}[commandchars=\\\{\}]
\PY{c+c1}{\PYZsh{} create a (discreet) random variable with poissionian distribution}

\PY{n}{X} \PY{o}{=} \PY{n}{stats}\PY{o}{.}\PY{n}{poisson}\PY{p}{(}\PY{l+m+mf}{3.5}\PY{p}{)} \PY{c+c1}{\PYZsh{} photon distribution for a coherent state with n=3.5 photons}
\end{Verbatim}
\end{tcolorbox}

    \begin{tcolorbox}[breakable, size=fbox, boxrule=1pt, pad at break*=1mm,colback=cellbackground, colframe=cellborder]
\prompt{In}{incolor}{84}{\boxspacing}
\begin{Verbatim}[commandchars=\\\{\}]
\PY{n}{n} \PY{o}{=} \PY{n}{arange}\PY{p}{(}\PY{l+m+mi}{0}\PY{p}{,}\PY{l+m+mi}{15}\PY{p}{)}

\PY{n}{fig}\PY{p}{,} \PY{n}{axes} \PY{o}{=} \PY{n}{plt}\PY{o}{.}\PY{n}{subplots}\PY{p}{(}\PY{l+m+mi}{3}\PY{p}{,}\PY{l+m+mi}{1}\PY{p}{,} \PY{n}{sharex}\PY{o}{=}\PY{k+kc}{True}\PY{p}{)}

\PY{c+c1}{\PYZsh{} plot the probability mass function (PMF)}
\PY{n}{axes}\PY{p}{[}\PY{l+m+mi}{0}\PY{p}{]}\PY{o}{.}\PY{n}{step}\PY{p}{(}\PY{n}{n}\PY{p}{,} \PY{n}{X}\PY{o}{.}\PY{n}{pmf}\PY{p}{(}\PY{n}{n}\PY{p}{)}\PY{p}{)}

\PY{c+c1}{\PYZsh{} plot the commulative distribution function (CDF)}
\PY{n}{axes}\PY{p}{[}\PY{l+m+mi}{1}\PY{p}{]}\PY{o}{.}\PY{n}{step}\PY{p}{(}\PY{n}{n}\PY{p}{,} \PY{n}{X}\PY{o}{.}\PY{n}{cdf}\PY{p}{(}\PY{n}{n}\PY{p}{)}\PY{p}{)}

\PY{c+c1}{\PYZsh{} plot histogram of 1000 random realizations of the stochastic variable X}
\PY{n}{axes}\PY{p}{[}\PY{l+m+mi}{2}\PY{p}{]}\PY{o}{.}\PY{n}{hist}\PY{p}{(}\PY{n}{X}\PY{o}{.}\PY{n}{rvs}\PY{p}{(}\PY{n}{size}\PY{o}{=}\PY{l+m+mi}{1000}\PY{p}{)}\PY{p}{)}\PY{p}{;}
\end{Verbatim}
\end{tcolorbox}

    \begin{Verbatim}[commandchars=\\\{\}]
<ipython-input-84-078f0313a228>:1: DeprecationWarning: scipy.arange is
deprecated and will be removed in SciPy 2.0.0, use numpy.arange instead
  n = arange(0,15)
    \end{Verbatim}

    \begin{center}
    \adjustimage{max size={0.9\linewidth}{0.9\paperheight}}{Lecture-3-Scipy_files/Lecture-3-Scipy_139_1.png}
    \end{center}
    { \hspace*{\fill} \\}
    
    \begin{tcolorbox}[breakable, size=fbox, boxrule=1pt, pad at break*=1mm,colback=cellbackground, colframe=cellborder]
\prompt{In}{incolor}{85}{\boxspacing}
\begin{Verbatim}[commandchars=\\\{\}]
\PY{c+c1}{\PYZsh{} create a (continous) random variable with normal distribution}
\PY{n}{Y} \PY{o}{=} \PY{n}{stats}\PY{o}{.}\PY{n}{norm}\PY{p}{(}\PY{p}{)}
\end{Verbatim}
\end{tcolorbox}

    \begin{tcolorbox}[breakable, size=fbox, boxrule=1pt, pad at break*=1mm,colback=cellbackground, colframe=cellborder]
\prompt{In}{incolor}{86}{\boxspacing}
\begin{Verbatim}[commandchars=\\\{\}]
\PY{n}{x} \PY{o}{=} \PY{n}{linspace}\PY{p}{(}\PY{o}{\PYZhy{}}\PY{l+m+mi}{5}\PY{p}{,}\PY{l+m+mi}{5}\PY{p}{,}\PY{l+m+mi}{100}\PY{p}{)}

\PY{n}{fig}\PY{p}{,} \PY{n}{axes} \PY{o}{=} \PY{n}{plt}\PY{o}{.}\PY{n}{subplots}\PY{p}{(}\PY{l+m+mi}{3}\PY{p}{,}\PY{l+m+mi}{1}\PY{p}{,} \PY{n}{sharex}\PY{o}{=}\PY{k+kc}{True}\PY{p}{)}

\PY{c+c1}{\PYZsh{} plot the probability distribution function (PDF)}
\PY{n}{axes}\PY{p}{[}\PY{l+m+mi}{0}\PY{p}{]}\PY{o}{.}\PY{n}{plot}\PY{p}{(}\PY{n}{x}\PY{p}{,} \PY{n}{Y}\PY{o}{.}\PY{n}{pdf}\PY{p}{(}\PY{n}{x}\PY{p}{)}\PY{p}{)}

\PY{c+c1}{\PYZsh{} plot the commulative distributin function (CDF)}
\PY{n}{axes}\PY{p}{[}\PY{l+m+mi}{1}\PY{p}{]}\PY{o}{.}\PY{n}{plot}\PY{p}{(}\PY{n}{x}\PY{p}{,} \PY{n}{Y}\PY{o}{.}\PY{n}{cdf}\PY{p}{(}\PY{n}{x}\PY{p}{)}\PY{p}{)}\PY{p}{;}

\PY{c+c1}{\PYZsh{} plot histogram of 1000 random realizations of the stochastic variable Y}
\PY{n}{axes}\PY{p}{[}\PY{l+m+mi}{2}\PY{p}{]}\PY{o}{.}\PY{n}{hist}\PY{p}{(}\PY{n}{Y}\PY{o}{.}\PY{n}{rvs}\PY{p}{(}\PY{n}{size}\PY{o}{=}\PY{l+m+mi}{1000}\PY{p}{)}\PY{p}{,} \PY{n}{bins}\PY{o}{=}\PY{l+m+mi}{50}\PY{p}{)}\PY{p}{;}
\end{Verbatim}
\end{tcolorbox}

    \begin{Verbatim}[commandchars=\\\{\}]
<ipython-input-86-c7b37fdaada3>:1: DeprecationWarning: scipy.linspace is
deprecated and will be removed in SciPy 2.0.0, use numpy.linspace instead
  x = linspace(-5,5,100)
    \end{Verbatim}

    \begin{center}
    \adjustimage{max size={0.9\linewidth}{0.9\paperheight}}{Lecture-3-Scipy_files/Lecture-3-Scipy_141_1.png}
    \end{center}
    { \hspace*{\fill} \\}
    
    Statistics:

    \begin{tcolorbox}[breakable, size=fbox, boxrule=1pt, pad at break*=1mm,colback=cellbackground, colframe=cellborder]
\prompt{In}{incolor}{87}{\boxspacing}
\begin{Verbatim}[commandchars=\\\{\}]
\PY{n}{X}\PY{o}{.}\PY{n}{mean}\PY{p}{(}\PY{p}{)}\PY{p}{,} \PY{n}{X}\PY{o}{.}\PY{n}{std}\PY{p}{(}\PY{p}{)}\PY{p}{,} \PY{n}{X}\PY{o}{.}\PY{n}{var}\PY{p}{(}\PY{p}{)} \PY{c+c1}{\PYZsh{} poission distribution}
\end{Verbatim}
\end{tcolorbox}

            \begin{tcolorbox}[breakable, size=fbox, boxrule=.5pt, pad at break*=1mm, opacityfill=0]
\prompt{Out}{outcolor}{87}{\boxspacing}
\begin{Verbatim}[commandchars=\\\{\}]
(3.5, 1.8708286933869707, 3.5)
\end{Verbatim}
\end{tcolorbox}
        
    \begin{tcolorbox}[breakable, size=fbox, boxrule=1pt, pad at break*=1mm,colback=cellbackground, colframe=cellborder]
\prompt{In}{incolor}{88}{\boxspacing}
\begin{Verbatim}[commandchars=\\\{\}]
\PY{n}{Y}\PY{o}{.}\PY{n}{mean}\PY{p}{(}\PY{p}{)}\PY{p}{,} \PY{n}{Y}\PY{o}{.}\PY{n}{std}\PY{p}{(}\PY{p}{)}\PY{p}{,} \PY{n}{Y}\PY{o}{.}\PY{n}{var}\PY{p}{(}\PY{p}{)} \PY{c+c1}{\PYZsh{} normal distribution}
\end{Verbatim}
\end{tcolorbox}

            \begin{tcolorbox}[breakable, size=fbox, boxrule=.5pt, pad at break*=1mm, opacityfill=0]
\prompt{Out}{outcolor}{88}{\boxspacing}
\begin{Verbatim}[commandchars=\\\{\}]
(0.0, 1.0, 1.0)
\end{Verbatim}
\end{tcolorbox}
        
    \hypertarget{statistical-tests}{%
\subsubsection{Statistical tests}\label{statistical-tests}}

    Test if two sets of (independent) random data comes from the same
distribution:

    \begin{tcolorbox}[breakable, size=fbox, boxrule=1pt, pad at break*=1mm,colback=cellbackground, colframe=cellborder]
\prompt{In}{incolor}{89}{\boxspacing}
\begin{Verbatim}[commandchars=\\\{\}]
\PY{n}{t\PYZus{}statistic}\PY{p}{,} \PY{n}{p\PYZus{}value} \PY{o}{=} \PY{n}{stats}\PY{o}{.}\PY{n}{ttest\PYZus{}ind}\PY{p}{(}\PY{n}{X}\PY{o}{.}\PY{n}{rvs}\PY{p}{(}\PY{n}{size}\PY{o}{=}\PY{l+m+mi}{1000}\PY{p}{)}\PY{p}{,} \PY{n}{X}\PY{o}{.}\PY{n}{rvs}\PY{p}{(}\PY{n}{size}\PY{o}{=}\PY{l+m+mi}{1000}\PY{p}{)}\PY{p}{)}

\PY{n+nb}{print} \PY{l+s+s2}{\PYZdq{}}\PY{l+s+s2}{t\PYZhy{}statistic =}\PY{l+s+s2}{\PYZdq{}}\PY{p}{,} \PY{n}{t\PYZus{}statistic}
\PY{n+nb}{print} \PY{l+s+s2}{\PYZdq{}}\PY{l+s+s2}{p\PYZhy{}value =}\PY{l+s+s2}{\PYZdq{}}\PY{p}{,} \PY{n}{p\PYZus{}value}
\end{Verbatim}
\end{tcolorbox}

    \begin{Verbatim}[commandchars=\\\{\}]

          File "<ipython-input-89-0db034d77d0f>", line 3
        print "t-statistic =", t\_statistic
              \^{}
    SyntaxError: Missing parentheses in call to 'print'. Did you mean print("t-statistic =", t\_statistic)?
    

    \end{Verbatim}

    Since the p value is very large we cannot reject the hypothesis that the
two sets of random data have \emph{different} means.

    To test if the mean of a single sample of data has mean 0.1 (the true
mean is 0.0):

    \begin{tcolorbox}[breakable, size=fbox, boxrule=1pt, pad at break*=1mm,colback=cellbackground, colframe=cellborder]
\prompt{In}{incolor}{90}{\boxspacing}
\begin{Verbatim}[commandchars=\\\{\}]
\PY{n}{stats}\PY{o}{.}\PY{n}{ttest\PYZus{}1samp}\PY{p}{(}\PY{n}{Y}\PY{o}{.}\PY{n}{rvs}\PY{p}{(}\PY{n}{size}\PY{o}{=}\PY{l+m+mi}{1000}\PY{p}{)}\PY{p}{,} \PY{l+m+mf}{0.1}\PY{p}{)}
\end{Verbatim}
\end{tcolorbox}

            \begin{tcolorbox}[breakable, size=fbox, boxrule=.5pt, pad at break*=1mm, opacityfill=0]
\prompt{Out}{outcolor}{90}{\boxspacing}
\begin{Verbatim}[commandchars=\\\{\}]
Ttest\_1sampResult(statistic=-3.6509309802271876, pvalue=0.00027483761569202465)
\end{Verbatim}
\end{tcolorbox}
        
    Low p-value means that we can reject the hypothesis that the mean of Y
is 0.1.

    \begin{tcolorbox}[breakable, size=fbox, boxrule=1pt, pad at break*=1mm,colback=cellbackground, colframe=cellborder]
\prompt{In}{incolor}{91}{\boxspacing}
\begin{Verbatim}[commandchars=\\\{\}]
\PY{n}{Y}\PY{o}{.}\PY{n}{mean}\PY{p}{(}\PY{p}{)}
\end{Verbatim}
\end{tcolorbox}

            \begin{tcolorbox}[breakable, size=fbox, boxrule=.5pt, pad at break*=1mm, opacityfill=0]
\prompt{Out}{outcolor}{91}{\boxspacing}
\begin{Verbatim}[commandchars=\\\{\}]
0.0
\end{Verbatim}
\end{tcolorbox}
        
    \begin{tcolorbox}[breakable, size=fbox, boxrule=1pt, pad at break*=1mm,colback=cellbackground, colframe=cellborder]
\prompt{In}{incolor}{92}{\boxspacing}
\begin{Verbatim}[commandchars=\\\{\}]
\PY{n}{stats}\PY{o}{.}\PY{n}{ttest\PYZus{}1samp}\PY{p}{(}\PY{n}{Y}\PY{o}{.}\PY{n}{rvs}\PY{p}{(}\PY{n}{size}\PY{o}{=}\PY{l+m+mi}{1000}\PY{p}{)}\PY{p}{,} \PY{n}{Y}\PY{o}{.}\PY{n}{mean}\PY{p}{(}\PY{p}{)}\PY{p}{)}
\end{Verbatim}
\end{tcolorbox}

            \begin{tcolorbox}[breakable, size=fbox, boxrule=.5pt, pad at break*=1mm, opacityfill=0]
\prompt{Out}{outcolor}{92}{\boxspacing}
\begin{Verbatim}[commandchars=\\\{\}]
Ttest\_1sampResult(statistic=-1.9098489555750828, pvalue=0.05643882242266383)
\end{Verbatim}
\end{tcolorbox}
        
    \hypertarget{further-reading}{%
\subsection{Further reading}\label{further-reading}}

    \begin{itemize}
\tightlist
\item
  http://www.scipy.org - The official web page for the SciPy project.
\item
  http://docs.scipy.org/doc/scipy/reference/tutorial/index.html - A
  tutorial on how to get started using SciPy.
\item
  https://github.com/scipy/scipy/ - The SciPy source code.
\end{itemize}

    \hypertarget{versions}{%
\subsection{Versions}\label{versions}}

    \begin{tcolorbox}[breakable, size=fbox, boxrule=1pt, pad at break*=1mm,colback=cellbackground, colframe=cellborder]
\prompt{In}{incolor}{93}{\boxspacing}
\begin{Verbatim}[commandchars=\\\{\}]
\PY{o}{\PYZpc{}}\PY{k}{reload\PYZus{}ext} version\PYZus{}information

\PY{o}{\PYZpc{}}\PY{k}{version\PYZus{}information} numpy, matplotlib, scipy
\end{Verbatim}
\end{tcolorbox}
 
            
\prompt{Out}{outcolor}{93}{}
    
    \begin{tabular}{|l|l|}\hline
{\bf Software} & {\bf Version} \\ \hline\hline
Python & 3.8.2 64bit [MSC v.1916 64 bit (AMD64)] \\ \hline
IPython & 7.15.0 \\ \hline
OS & Windows 10 10.0.19041 SP0 \\ \hline
numpy & 1.18.5 \\ \hline
matplotlib & 3.2.1 \\ \hline
scipy & 1.4.1 \\ \hline
\hline \multicolumn{2}{|l|}{Fri Jun 19 11:25:42 2020 India Standard Time} \\ \hline
\end{tabular}


    


    % Add a bibliography block to the postdoc
    
    
    
\end{document}
