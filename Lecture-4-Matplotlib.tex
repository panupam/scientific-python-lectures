\documentclass[11pt]{article}

    \usepackage[breakable]{tcolorbox}
    \usepackage{parskip} % Stop auto-indenting (to mimic markdown behaviour)
    
    \usepackage{iftex}
    \ifPDFTeX
    	\usepackage[T1]{fontenc}
    	\usepackage{mathpazo}
    \else
    	\usepackage{fontspec}
    \fi

    % Basic figure setup, for now with no caption control since it's done
    % automatically by Pandoc (which extracts ![](path) syntax from Markdown).
    \usepackage{graphicx}
    % Maintain compatibility with old templates. Remove in nbconvert 6.0
    \let\Oldincludegraphics\includegraphics
    % Ensure that by default, figures have no caption (until we provide a
    % proper Figure object with a Caption API and a way to capture that
    % in the conversion process - todo).
    \usepackage{caption}
    \DeclareCaptionFormat{nocaption}{}
    \captionsetup{format=nocaption,aboveskip=0pt,belowskip=0pt}

    \usepackage[Export]{adjustbox} % Used to constrain images to a maximum size
    \adjustboxset{max size={0.9\linewidth}{0.9\paperheight}}
    \usepackage{float}
    \floatplacement{figure}{H} % forces figures to be placed at the correct location
    \usepackage{xcolor} % Allow colors to be defined
    \usepackage{enumerate} % Needed for markdown enumerations to work
    \usepackage{geometry} % Used to adjust the document margins
    \usepackage{amsmath} % Equations
    \usepackage{amssymb} % Equations
    \usepackage{textcomp} % defines textquotesingle
    % Hack from http://tex.stackexchange.com/a/47451/13684:
    \AtBeginDocument{%
        \def\PYZsq{\textquotesingle}% Upright quotes in Pygmentized code
    }
    \usepackage{upquote} % Upright quotes for verbatim code
    \usepackage{eurosym} % defines \euro
    \usepackage[mathletters]{ucs} % Extended unicode (utf-8) support
    \usepackage{fancyvrb} % verbatim replacement that allows latex
    \usepackage{grffile} % extends the file name processing of package graphics 
                         % to support a larger range
    \makeatletter % fix for grffile with XeLaTeX
    \def\Gread@@xetex#1{%
      \IfFileExists{"\Gin@base".bb}%
      {\Gread@eps{\Gin@base.bb}}%
      {\Gread@@xetex@aux#1}%
    }
    \makeatother

    % The hyperref package gives us a pdf with properly built
    % internal navigation ('pdf bookmarks' for the table of contents,
    % internal cross-reference links, web links for URLs, etc.)
    \usepackage{hyperref}
    % The default LaTeX title has an obnoxious amount of whitespace. By default,
    % titling removes some of it. It also provides customization options.
    \usepackage{titling}
    \usepackage{longtable} % longtable support required by pandoc >1.10
    \usepackage{booktabs}  % table support for pandoc > 1.12.2
    \usepackage[inline]{enumitem} % IRkernel/repr support (it uses the enumerate* environment)
    \usepackage[normalem]{ulem} % ulem is needed to support strikethroughs (\sout)
                                % normalem makes italics be italics, not underlines
    \usepackage{mathrsfs}
    

    
    % Colors for the hyperref package
    \definecolor{urlcolor}{rgb}{0,.145,.698}
    \definecolor{linkcolor}{rgb}{.71,0.21,0.01}
    \definecolor{citecolor}{rgb}{.12,.54,.11}

    % ANSI colors
    \definecolor{ansi-black}{HTML}{3E424D}
    \definecolor{ansi-black-intense}{HTML}{282C36}
    \definecolor{ansi-red}{HTML}{E75C58}
    \definecolor{ansi-red-intense}{HTML}{B22B31}
    \definecolor{ansi-green}{HTML}{00A250}
    \definecolor{ansi-green-intense}{HTML}{007427}
    \definecolor{ansi-yellow}{HTML}{DDB62B}
    \definecolor{ansi-yellow-intense}{HTML}{B27D12}
    \definecolor{ansi-blue}{HTML}{208FFB}
    \definecolor{ansi-blue-intense}{HTML}{0065CA}
    \definecolor{ansi-magenta}{HTML}{D160C4}
    \definecolor{ansi-magenta-intense}{HTML}{A03196}
    \definecolor{ansi-cyan}{HTML}{60C6C8}
    \definecolor{ansi-cyan-intense}{HTML}{258F8F}
    \definecolor{ansi-white}{HTML}{C5C1B4}
    \definecolor{ansi-white-intense}{HTML}{A1A6B2}
    \definecolor{ansi-default-inverse-fg}{HTML}{FFFFFF}
    \definecolor{ansi-default-inverse-bg}{HTML}{000000}

    % commands and environments needed by pandoc snippets
    % extracted from the output of `pandoc -s`
    \providecommand{\tightlist}{%
      \setlength{\itemsep}{0pt}\setlength{\parskip}{0pt}}
    \DefineVerbatimEnvironment{Highlighting}{Verbatim}{commandchars=\\\{\}}
    % Add ',fontsize=\small' for more characters per line
    \newenvironment{Shaded}{}{}
    \newcommand{\KeywordTok}[1]{\textcolor[rgb]{0.00,0.44,0.13}{\textbf{{#1}}}}
    \newcommand{\DataTypeTok}[1]{\textcolor[rgb]{0.56,0.13,0.00}{{#1}}}
    \newcommand{\DecValTok}[1]{\textcolor[rgb]{0.25,0.63,0.44}{{#1}}}
    \newcommand{\BaseNTok}[1]{\textcolor[rgb]{0.25,0.63,0.44}{{#1}}}
    \newcommand{\FloatTok}[1]{\textcolor[rgb]{0.25,0.63,0.44}{{#1}}}
    \newcommand{\CharTok}[1]{\textcolor[rgb]{0.25,0.44,0.63}{{#1}}}
    \newcommand{\StringTok}[1]{\textcolor[rgb]{0.25,0.44,0.63}{{#1}}}
    \newcommand{\CommentTok}[1]{\textcolor[rgb]{0.38,0.63,0.69}{\textit{{#1}}}}
    \newcommand{\OtherTok}[1]{\textcolor[rgb]{0.00,0.44,0.13}{{#1}}}
    \newcommand{\AlertTok}[1]{\textcolor[rgb]{1.00,0.00,0.00}{\textbf{{#1}}}}
    \newcommand{\FunctionTok}[1]{\textcolor[rgb]{0.02,0.16,0.49}{{#1}}}
    \newcommand{\RegionMarkerTok}[1]{{#1}}
    \newcommand{\ErrorTok}[1]{\textcolor[rgb]{1.00,0.00,0.00}{\textbf{{#1}}}}
    \newcommand{\NormalTok}[1]{{#1}}
    
    % Additional commands for more recent versions of Pandoc
    \newcommand{\ConstantTok}[1]{\textcolor[rgb]{0.53,0.00,0.00}{{#1}}}
    \newcommand{\SpecialCharTok}[1]{\textcolor[rgb]{0.25,0.44,0.63}{{#1}}}
    \newcommand{\VerbatimStringTok}[1]{\textcolor[rgb]{0.25,0.44,0.63}{{#1}}}
    \newcommand{\SpecialStringTok}[1]{\textcolor[rgb]{0.73,0.40,0.53}{{#1}}}
    \newcommand{\ImportTok}[1]{{#1}}
    \newcommand{\DocumentationTok}[1]{\textcolor[rgb]{0.73,0.13,0.13}{\textit{{#1}}}}
    \newcommand{\AnnotationTok}[1]{\textcolor[rgb]{0.38,0.63,0.69}{\textbf{\textit{{#1}}}}}
    \newcommand{\CommentVarTok}[1]{\textcolor[rgb]{0.38,0.63,0.69}{\textbf{\textit{{#1}}}}}
    \newcommand{\VariableTok}[1]{\textcolor[rgb]{0.10,0.09,0.49}{{#1}}}
    \newcommand{\ControlFlowTok}[1]{\textcolor[rgb]{0.00,0.44,0.13}{\textbf{{#1}}}}
    \newcommand{\OperatorTok}[1]{\textcolor[rgb]{0.40,0.40,0.40}{{#1}}}
    \newcommand{\BuiltInTok}[1]{{#1}}
    \newcommand{\ExtensionTok}[1]{{#1}}
    \newcommand{\PreprocessorTok}[1]{\textcolor[rgb]{0.74,0.48,0.00}{{#1}}}
    \newcommand{\AttributeTok}[1]{\textcolor[rgb]{0.49,0.56,0.16}{{#1}}}
    \newcommand{\InformationTok}[1]{\textcolor[rgb]{0.38,0.63,0.69}{\textbf{\textit{{#1}}}}}
    \newcommand{\WarningTok}[1]{\textcolor[rgb]{0.38,0.63,0.69}{\textbf{\textit{{#1}}}}}
    
    
    % Define a nice break command that doesn't care if a line doesn't already
    % exist.
    \def\br{\hspace*{\fill} \\* }
    % Math Jax compatibility definitions
    \def\gt{>}
    \def\lt{<}
    \let\Oldtex\TeX
    \let\Oldlatex\LaTeX
    \renewcommand{\TeX}{\textrm{\Oldtex}}
    \renewcommand{\LaTeX}{\textrm{\Oldlatex}}
    % Document parameters
    % Document title
    \title{Lecture-4-Matplotlib}
    
    
    
    
    
% Pygments definitions
\makeatletter
\def\PY@reset{\let\PY@it=\relax \let\PY@bf=\relax%
    \let\PY@ul=\relax \let\PY@tc=\relax%
    \let\PY@bc=\relax \let\PY@ff=\relax}
\def\PY@tok#1{\csname PY@tok@#1\endcsname}
\def\PY@toks#1+{\ifx\relax#1\empty\else%
    \PY@tok{#1}\expandafter\PY@toks\fi}
\def\PY@do#1{\PY@bc{\PY@tc{\PY@ul{%
    \PY@it{\PY@bf{\PY@ff{#1}}}}}}}
\def\PY#1#2{\PY@reset\PY@toks#1+\relax+\PY@do{#2}}

\expandafter\def\csname PY@tok@w\endcsname{\def\PY@tc##1{\textcolor[rgb]{0.73,0.73,0.73}{##1}}}
\expandafter\def\csname PY@tok@c\endcsname{\let\PY@it=\textit\def\PY@tc##1{\textcolor[rgb]{0.25,0.50,0.50}{##1}}}
\expandafter\def\csname PY@tok@cp\endcsname{\def\PY@tc##1{\textcolor[rgb]{0.74,0.48,0.00}{##1}}}
\expandafter\def\csname PY@tok@k\endcsname{\let\PY@bf=\textbf\def\PY@tc##1{\textcolor[rgb]{0.00,0.50,0.00}{##1}}}
\expandafter\def\csname PY@tok@kp\endcsname{\def\PY@tc##1{\textcolor[rgb]{0.00,0.50,0.00}{##1}}}
\expandafter\def\csname PY@tok@kt\endcsname{\def\PY@tc##1{\textcolor[rgb]{0.69,0.00,0.25}{##1}}}
\expandafter\def\csname PY@tok@o\endcsname{\def\PY@tc##1{\textcolor[rgb]{0.40,0.40,0.40}{##1}}}
\expandafter\def\csname PY@tok@ow\endcsname{\let\PY@bf=\textbf\def\PY@tc##1{\textcolor[rgb]{0.67,0.13,1.00}{##1}}}
\expandafter\def\csname PY@tok@nb\endcsname{\def\PY@tc##1{\textcolor[rgb]{0.00,0.50,0.00}{##1}}}
\expandafter\def\csname PY@tok@nf\endcsname{\def\PY@tc##1{\textcolor[rgb]{0.00,0.00,1.00}{##1}}}
\expandafter\def\csname PY@tok@nc\endcsname{\let\PY@bf=\textbf\def\PY@tc##1{\textcolor[rgb]{0.00,0.00,1.00}{##1}}}
\expandafter\def\csname PY@tok@nn\endcsname{\let\PY@bf=\textbf\def\PY@tc##1{\textcolor[rgb]{0.00,0.00,1.00}{##1}}}
\expandafter\def\csname PY@tok@ne\endcsname{\let\PY@bf=\textbf\def\PY@tc##1{\textcolor[rgb]{0.82,0.25,0.23}{##1}}}
\expandafter\def\csname PY@tok@nv\endcsname{\def\PY@tc##1{\textcolor[rgb]{0.10,0.09,0.49}{##1}}}
\expandafter\def\csname PY@tok@no\endcsname{\def\PY@tc##1{\textcolor[rgb]{0.53,0.00,0.00}{##1}}}
\expandafter\def\csname PY@tok@nl\endcsname{\def\PY@tc##1{\textcolor[rgb]{0.63,0.63,0.00}{##1}}}
\expandafter\def\csname PY@tok@ni\endcsname{\let\PY@bf=\textbf\def\PY@tc##1{\textcolor[rgb]{0.60,0.60,0.60}{##1}}}
\expandafter\def\csname PY@tok@na\endcsname{\def\PY@tc##1{\textcolor[rgb]{0.49,0.56,0.16}{##1}}}
\expandafter\def\csname PY@tok@nt\endcsname{\let\PY@bf=\textbf\def\PY@tc##1{\textcolor[rgb]{0.00,0.50,0.00}{##1}}}
\expandafter\def\csname PY@tok@nd\endcsname{\def\PY@tc##1{\textcolor[rgb]{0.67,0.13,1.00}{##1}}}
\expandafter\def\csname PY@tok@s\endcsname{\def\PY@tc##1{\textcolor[rgb]{0.73,0.13,0.13}{##1}}}
\expandafter\def\csname PY@tok@sd\endcsname{\let\PY@it=\textit\def\PY@tc##1{\textcolor[rgb]{0.73,0.13,0.13}{##1}}}
\expandafter\def\csname PY@tok@si\endcsname{\let\PY@bf=\textbf\def\PY@tc##1{\textcolor[rgb]{0.73,0.40,0.53}{##1}}}
\expandafter\def\csname PY@tok@se\endcsname{\let\PY@bf=\textbf\def\PY@tc##1{\textcolor[rgb]{0.73,0.40,0.13}{##1}}}
\expandafter\def\csname PY@tok@sr\endcsname{\def\PY@tc##1{\textcolor[rgb]{0.73,0.40,0.53}{##1}}}
\expandafter\def\csname PY@tok@ss\endcsname{\def\PY@tc##1{\textcolor[rgb]{0.10,0.09,0.49}{##1}}}
\expandafter\def\csname PY@tok@sx\endcsname{\def\PY@tc##1{\textcolor[rgb]{0.00,0.50,0.00}{##1}}}
\expandafter\def\csname PY@tok@m\endcsname{\def\PY@tc##1{\textcolor[rgb]{0.40,0.40,0.40}{##1}}}
\expandafter\def\csname PY@tok@gh\endcsname{\let\PY@bf=\textbf\def\PY@tc##1{\textcolor[rgb]{0.00,0.00,0.50}{##1}}}
\expandafter\def\csname PY@tok@gu\endcsname{\let\PY@bf=\textbf\def\PY@tc##1{\textcolor[rgb]{0.50,0.00,0.50}{##1}}}
\expandafter\def\csname PY@tok@gd\endcsname{\def\PY@tc##1{\textcolor[rgb]{0.63,0.00,0.00}{##1}}}
\expandafter\def\csname PY@tok@gi\endcsname{\def\PY@tc##1{\textcolor[rgb]{0.00,0.63,0.00}{##1}}}
\expandafter\def\csname PY@tok@gr\endcsname{\def\PY@tc##1{\textcolor[rgb]{1.00,0.00,0.00}{##1}}}
\expandafter\def\csname PY@tok@ge\endcsname{\let\PY@it=\textit}
\expandafter\def\csname PY@tok@gs\endcsname{\let\PY@bf=\textbf}
\expandafter\def\csname PY@tok@gp\endcsname{\let\PY@bf=\textbf\def\PY@tc##1{\textcolor[rgb]{0.00,0.00,0.50}{##1}}}
\expandafter\def\csname PY@tok@go\endcsname{\def\PY@tc##1{\textcolor[rgb]{0.53,0.53,0.53}{##1}}}
\expandafter\def\csname PY@tok@gt\endcsname{\def\PY@tc##1{\textcolor[rgb]{0.00,0.27,0.87}{##1}}}
\expandafter\def\csname PY@tok@err\endcsname{\def\PY@bc##1{\setlength{\fboxsep}{0pt}\fcolorbox[rgb]{1.00,0.00,0.00}{1,1,1}{\strut ##1}}}
\expandafter\def\csname PY@tok@kc\endcsname{\let\PY@bf=\textbf\def\PY@tc##1{\textcolor[rgb]{0.00,0.50,0.00}{##1}}}
\expandafter\def\csname PY@tok@kd\endcsname{\let\PY@bf=\textbf\def\PY@tc##1{\textcolor[rgb]{0.00,0.50,0.00}{##1}}}
\expandafter\def\csname PY@tok@kn\endcsname{\let\PY@bf=\textbf\def\PY@tc##1{\textcolor[rgb]{0.00,0.50,0.00}{##1}}}
\expandafter\def\csname PY@tok@kr\endcsname{\let\PY@bf=\textbf\def\PY@tc##1{\textcolor[rgb]{0.00,0.50,0.00}{##1}}}
\expandafter\def\csname PY@tok@bp\endcsname{\def\PY@tc##1{\textcolor[rgb]{0.00,0.50,0.00}{##1}}}
\expandafter\def\csname PY@tok@fm\endcsname{\def\PY@tc##1{\textcolor[rgb]{0.00,0.00,1.00}{##1}}}
\expandafter\def\csname PY@tok@vc\endcsname{\def\PY@tc##1{\textcolor[rgb]{0.10,0.09,0.49}{##1}}}
\expandafter\def\csname PY@tok@vg\endcsname{\def\PY@tc##1{\textcolor[rgb]{0.10,0.09,0.49}{##1}}}
\expandafter\def\csname PY@tok@vi\endcsname{\def\PY@tc##1{\textcolor[rgb]{0.10,0.09,0.49}{##1}}}
\expandafter\def\csname PY@tok@vm\endcsname{\def\PY@tc##1{\textcolor[rgb]{0.10,0.09,0.49}{##1}}}
\expandafter\def\csname PY@tok@sa\endcsname{\def\PY@tc##1{\textcolor[rgb]{0.73,0.13,0.13}{##1}}}
\expandafter\def\csname PY@tok@sb\endcsname{\def\PY@tc##1{\textcolor[rgb]{0.73,0.13,0.13}{##1}}}
\expandafter\def\csname PY@tok@sc\endcsname{\def\PY@tc##1{\textcolor[rgb]{0.73,0.13,0.13}{##1}}}
\expandafter\def\csname PY@tok@dl\endcsname{\def\PY@tc##1{\textcolor[rgb]{0.73,0.13,0.13}{##1}}}
\expandafter\def\csname PY@tok@s2\endcsname{\def\PY@tc##1{\textcolor[rgb]{0.73,0.13,0.13}{##1}}}
\expandafter\def\csname PY@tok@sh\endcsname{\def\PY@tc##1{\textcolor[rgb]{0.73,0.13,0.13}{##1}}}
\expandafter\def\csname PY@tok@s1\endcsname{\def\PY@tc##1{\textcolor[rgb]{0.73,0.13,0.13}{##1}}}
\expandafter\def\csname PY@tok@mb\endcsname{\def\PY@tc##1{\textcolor[rgb]{0.40,0.40,0.40}{##1}}}
\expandafter\def\csname PY@tok@mf\endcsname{\def\PY@tc##1{\textcolor[rgb]{0.40,0.40,0.40}{##1}}}
\expandafter\def\csname PY@tok@mh\endcsname{\def\PY@tc##1{\textcolor[rgb]{0.40,0.40,0.40}{##1}}}
\expandafter\def\csname PY@tok@mi\endcsname{\def\PY@tc##1{\textcolor[rgb]{0.40,0.40,0.40}{##1}}}
\expandafter\def\csname PY@tok@il\endcsname{\def\PY@tc##1{\textcolor[rgb]{0.40,0.40,0.40}{##1}}}
\expandafter\def\csname PY@tok@mo\endcsname{\def\PY@tc##1{\textcolor[rgb]{0.40,0.40,0.40}{##1}}}
\expandafter\def\csname PY@tok@ch\endcsname{\let\PY@it=\textit\def\PY@tc##1{\textcolor[rgb]{0.25,0.50,0.50}{##1}}}
\expandafter\def\csname PY@tok@cm\endcsname{\let\PY@it=\textit\def\PY@tc##1{\textcolor[rgb]{0.25,0.50,0.50}{##1}}}
\expandafter\def\csname PY@tok@cpf\endcsname{\let\PY@it=\textit\def\PY@tc##1{\textcolor[rgb]{0.25,0.50,0.50}{##1}}}
\expandafter\def\csname PY@tok@c1\endcsname{\let\PY@it=\textit\def\PY@tc##1{\textcolor[rgb]{0.25,0.50,0.50}{##1}}}
\expandafter\def\csname PY@tok@cs\endcsname{\let\PY@it=\textit\def\PY@tc##1{\textcolor[rgb]{0.25,0.50,0.50}{##1}}}

\def\PYZbs{\char`\\}
\def\PYZus{\char`\_}
\def\PYZob{\char`\{}
\def\PYZcb{\char`\}}
\def\PYZca{\char`\^}
\def\PYZam{\char`\&}
\def\PYZlt{\char`\<}
\def\PYZgt{\char`\>}
\def\PYZsh{\char`\#}
\def\PYZpc{\char`\%}
\def\PYZdl{\char`\$}
\def\PYZhy{\char`\-}
\def\PYZsq{\char`\'}
\def\PYZdq{\char`\"}
\def\PYZti{\char`\~}
% for compatibility with earlier versions
\def\PYZat{@}
\def\PYZlb{[}
\def\PYZrb{]}
\makeatother


    % For linebreaks inside Verbatim environment from package fancyvrb. 
    \makeatletter
        \newbox\Wrappedcontinuationbox 
        \newbox\Wrappedvisiblespacebox 
        \newcommand*\Wrappedvisiblespace {\textcolor{red}{\textvisiblespace}} 
        \newcommand*\Wrappedcontinuationsymbol {\textcolor{red}{\llap{\tiny$\m@th\hookrightarrow$}}} 
        \newcommand*\Wrappedcontinuationindent {3ex } 
        \newcommand*\Wrappedafterbreak {\kern\Wrappedcontinuationindent\copy\Wrappedcontinuationbox} 
        % Take advantage of the already applied Pygments mark-up to insert 
        % potential linebreaks for TeX processing. 
        %        {, <, #, %, $, ' and ": go to next line. 
        %        _, }, ^, &, >, - and ~: stay at end of broken line. 
        % Use of \textquotesingle for straight quote. 
        \newcommand*\Wrappedbreaksatspecials {% 
            \def\PYGZus{\discretionary{\char`\_}{\Wrappedafterbreak}{\char`\_}}% 
            \def\PYGZob{\discretionary{}{\Wrappedafterbreak\char`\{}{\char`\{}}% 
            \def\PYGZcb{\discretionary{\char`\}}{\Wrappedafterbreak}{\char`\}}}% 
            \def\PYGZca{\discretionary{\char`\^}{\Wrappedafterbreak}{\char`\^}}% 
            \def\PYGZam{\discretionary{\char`\&}{\Wrappedafterbreak}{\char`\&}}% 
            \def\PYGZlt{\discretionary{}{\Wrappedafterbreak\char`\<}{\char`\<}}% 
            \def\PYGZgt{\discretionary{\char`\>}{\Wrappedafterbreak}{\char`\>}}% 
            \def\PYGZsh{\discretionary{}{\Wrappedafterbreak\char`\#}{\char`\#}}% 
            \def\PYGZpc{\discretionary{}{\Wrappedafterbreak\char`\%}{\char`\%}}% 
            \def\PYGZdl{\discretionary{}{\Wrappedafterbreak\char`\$}{\char`\$}}% 
            \def\PYGZhy{\discretionary{\char`\-}{\Wrappedafterbreak}{\char`\-}}% 
            \def\PYGZsq{\discretionary{}{\Wrappedafterbreak\textquotesingle}{\textquotesingle}}% 
            \def\PYGZdq{\discretionary{}{\Wrappedafterbreak\char`\"}{\char`\"}}% 
            \def\PYGZti{\discretionary{\char`\~}{\Wrappedafterbreak}{\char`\~}}% 
        } 
        % Some characters . , ; ? ! / are not pygmentized. 
        % This macro makes them "active" and they will insert potential linebreaks 
        \newcommand*\Wrappedbreaksatpunct {% 
            \lccode`\~`\.\lowercase{\def~}{\discretionary{\hbox{\char`\.}}{\Wrappedafterbreak}{\hbox{\char`\.}}}% 
            \lccode`\~`\,\lowercase{\def~}{\discretionary{\hbox{\char`\,}}{\Wrappedafterbreak}{\hbox{\char`\,}}}% 
            \lccode`\~`\;\lowercase{\def~}{\discretionary{\hbox{\char`\;}}{\Wrappedafterbreak}{\hbox{\char`\;}}}% 
            \lccode`\~`\:\lowercase{\def~}{\discretionary{\hbox{\char`\:}}{\Wrappedafterbreak}{\hbox{\char`\:}}}% 
            \lccode`\~`\?\lowercase{\def~}{\discretionary{\hbox{\char`\?}}{\Wrappedafterbreak}{\hbox{\char`\?}}}% 
            \lccode`\~`\!\lowercase{\def~}{\discretionary{\hbox{\char`\!}}{\Wrappedafterbreak}{\hbox{\char`\!}}}% 
            \lccode`\~`\/\lowercase{\def~}{\discretionary{\hbox{\char`\/}}{\Wrappedafterbreak}{\hbox{\char`\/}}}% 
            \catcode`\.\active
            \catcode`\,\active 
            \catcode`\;\active
            \catcode`\:\active
            \catcode`\?\active
            \catcode`\!\active
            \catcode`\/\active 
            \lccode`\~`\~ 	
        }
    \makeatother

    \let\OriginalVerbatim=\Verbatim
    \makeatletter
    \renewcommand{\Verbatim}[1][1]{%
        %\parskip\z@skip
        \sbox\Wrappedcontinuationbox {\Wrappedcontinuationsymbol}%
        \sbox\Wrappedvisiblespacebox {\FV@SetupFont\Wrappedvisiblespace}%
        \def\FancyVerbFormatLine ##1{\hsize\linewidth
            \vtop{\raggedright\hyphenpenalty\z@\exhyphenpenalty\z@
                \doublehyphendemerits\z@\finalhyphendemerits\z@
                \strut ##1\strut}%
        }%
        % If the linebreak is at a space, the latter will be displayed as visible
        % space at end of first line, and a continuation symbol starts next line.
        % Stretch/shrink are however usually zero for typewriter font.
        \def\FV@Space {%
            \nobreak\hskip\z@ plus\fontdimen3\font minus\fontdimen4\font
            \discretionary{\copy\Wrappedvisiblespacebox}{\Wrappedafterbreak}
            {\kern\fontdimen2\font}%
        }%
        
        % Allow breaks at special characters using \PYG... macros.
        \Wrappedbreaksatspecials
        % Breaks at punctuation characters . , ; ? ! and / need catcode=\active 	
        \OriginalVerbatim[#1,codes*=\Wrappedbreaksatpunct]%
    }
    \makeatother

    % Exact colors from NB
    \definecolor{incolor}{HTML}{303F9F}
    \definecolor{outcolor}{HTML}{D84315}
    \definecolor{cellborder}{HTML}{CFCFCF}
    \definecolor{cellbackground}{HTML}{F7F7F7}
    
    % prompt
    \makeatletter
    \newcommand{\boxspacing}{\kern\kvtcb@left@rule\kern\kvtcb@boxsep}
    \makeatother
    \newcommand{\prompt}[4]{
        \ttfamily\llap{{\color{#2}[#3]:\hspace{3pt}#4}}\vspace{-\baselineskip}
    }
    

    
    % Prevent overflowing lines due to hard-to-break entities
    \sloppy 
    % Setup hyperref package
    \hypersetup{
      breaklinks=true,  % so long urls are correctly broken across lines
      colorlinks=true,
      urlcolor=urlcolor,
      linkcolor=linkcolor,
      citecolor=citecolor,
      }
    % Slightly bigger margins than the latex defaults
    
    \geometry{verbose,tmargin=1in,bmargin=1in,lmargin=1in,rmargin=1in}
    
    

\begin{document}
    
    \maketitle
    
    

    
    \hypertarget{matplotlib---2d-and-3d-plotting-in-python}{%
\section{matplotlib - 2D and 3D plotting in
Python}\label{matplotlib---2d-and-3d-plotting-in-python}}

    J.R. Johansson (jrjohansson at gmail.com)

The latest version of this
\href{http://ipython.org/notebook.html}{IPython notebook} lecture is
available at
\url{http://github.com/jrjohansson/scientific-python-lectures}.

The other notebooks in this lecture series are indexed at
\url{http://jrjohansson.github.io}.

    \begin{tcolorbox}[breakable, size=fbox, boxrule=1pt, pad at break*=1mm,colback=cellbackground, colframe=cellborder]
\prompt{In}{incolor}{1}{\boxspacing}
\begin{Verbatim}[commandchars=\\\{\}]
\PY{c+c1}{\PYZsh{} This line configures matplotlib to show figures embedded in the notebook, }
\PY{c+c1}{\PYZsh{} instead of opening a new window for each figure. More about that later. }
\PY{c+c1}{\PYZsh{} If you are using an old version of IPython, try using \PYZsq{}\PYZpc{}pylab inline\PYZsq{} instead.}
\PY{o}{\PYZpc{}}\PY{k}{matplotlib} inline
\end{Verbatim}
\end{tcolorbox}

    \hypertarget{introduction}{%
\subsection{Introduction}\label{introduction}}

    Matplotlib is an excellent 2D and 3D graphics library for generating
scientific figures. Some of the many advantages of this library include:

\begin{itemize}
\tightlist
\item
  Easy to get started
\item
  Support for \(\LaTeX\) formatted labels and texts
\item
  Great control of every element in a figure, including figure size and
  DPI.
\item
  High-quality output in many formats, including PNG, PDF, SVG, EPS, and
  PGF.
\item
  GUI for interactively exploring figures \emph{and} support for
  headless generation of figure files (useful for batch jobs).
\end{itemize}

One of the key features of matplotlib that I would like to emphasize,
and that I think makes matplotlib highly suitable for generating figures
for scientific publications is that all aspects of the figure can be
controlled \emph{programmatically}. This is important for
reproducibility and convenient when one needs to regenerate the figure
with updated data or change its appearance.

More information at the Matplotlib web page: http://matplotlib.org/

    To get started using Matplotlib in a Python program, either include the
symbols from the \texttt{pylab} module (the easy way):

    \begin{tcolorbox}[breakable, size=fbox, boxrule=1pt, pad at break*=1mm,colback=cellbackground, colframe=cellborder]
\prompt{In}{incolor}{2}{\boxspacing}
\begin{Verbatim}[commandchars=\\\{\}]
\PY{k+kn}{from} \PY{n+nn}{pylab} \PY{k+kn}{import} \PY{o}{*}
\end{Verbatim}
\end{tcolorbox}

    or import the \texttt{matplotlib.pyplot} module under the name
\texttt{plt} (the tidy way):

    \begin{tcolorbox}[breakable, size=fbox, boxrule=1pt, pad at break*=1mm,colback=cellbackground, colframe=cellborder]
\prompt{In}{incolor}{3}{\boxspacing}
\begin{Verbatim}[commandchars=\\\{\}]
\PY{k+kn}{import} \PY{n+nn}{matplotlib}
\PY{k+kn}{import} \PY{n+nn}{matplotlib}\PY{n+nn}{.}\PY{n+nn}{pyplot} \PY{k}{as} \PY{n+nn}{plt}
\end{Verbatim}
\end{tcolorbox}

    \begin{tcolorbox}[breakable, size=fbox, boxrule=1pt, pad at break*=1mm,colback=cellbackground, colframe=cellborder]
\prompt{In}{incolor}{4}{\boxspacing}
\begin{Verbatim}[commandchars=\\\{\}]
\PY{k+kn}{import} \PY{n+nn}{numpy} \PY{k}{as} \PY{n+nn}{np}
\end{Verbatim}
\end{tcolorbox}

    \hypertarget{matlab-like-api}{%
\subsection{MATLAB-like API}\label{matlab-like-api}}

    The easiest way to get started with plotting using matplotlib is often
to use the MATLAB-like API provided by matplotlib.

It is designed to be compatible with MATLAB's plotting functions, so it
is easy to get started with if you are familiar with MATLAB.

To use this API from matplotlib, we need to include the symbols in the
\texttt{pylab} module:

    \begin{tcolorbox}[breakable, size=fbox, boxrule=1pt, pad at break*=1mm,colback=cellbackground, colframe=cellborder]
\prompt{In}{incolor}{5}{\boxspacing}
\begin{Verbatim}[commandchars=\\\{\}]
\PY{k+kn}{from} \PY{n+nn}{pylab} \PY{k+kn}{import} \PY{o}{*}
\end{Verbatim}
\end{tcolorbox}

    \hypertarget{example}{%
\subsubsection{Example}\label{example}}

    A simple figure with MATLAB-like plotting API:

    \begin{tcolorbox}[breakable, size=fbox, boxrule=1pt, pad at break*=1mm,colback=cellbackground, colframe=cellborder]
\prompt{In}{incolor}{6}{\boxspacing}
\begin{Verbatim}[commandchars=\\\{\}]
\PY{n}{x} \PY{o}{=} \PY{n}{np}\PY{o}{.}\PY{n}{linspace}\PY{p}{(}\PY{l+m+mi}{0}\PY{p}{,} \PY{l+m+mi}{5}\PY{p}{,} \PY{l+m+mi}{10}\PY{p}{)}
\PY{n}{y} \PY{o}{=} \PY{n}{x} \PY{o}{*}\PY{o}{*} \PY{l+m+mi}{2}
\end{Verbatim}
\end{tcolorbox}

    \begin{tcolorbox}[breakable, size=fbox, boxrule=1pt, pad at break*=1mm,colback=cellbackground, colframe=cellborder]
\prompt{In}{incolor}{7}{\boxspacing}
\begin{Verbatim}[commandchars=\\\{\}]
\PY{n}{figure}\PY{p}{(}\PY{p}{)}
\PY{n}{plot}\PY{p}{(}\PY{n}{x}\PY{p}{,} \PY{n}{y}\PY{p}{,} \PY{l+s+s1}{\PYZsq{}}\PY{l+s+s1}{r}\PY{l+s+s1}{\PYZsq{}}\PY{p}{)}
\PY{n}{xlabel}\PY{p}{(}\PY{l+s+s1}{\PYZsq{}}\PY{l+s+s1}{x}\PY{l+s+s1}{\PYZsq{}}\PY{p}{)}
\PY{n}{ylabel}\PY{p}{(}\PY{l+s+s1}{\PYZsq{}}\PY{l+s+s1}{y}\PY{l+s+s1}{\PYZsq{}}\PY{p}{)}
\PY{n}{title}\PY{p}{(}\PY{l+s+s1}{\PYZsq{}}\PY{l+s+s1}{title}\PY{l+s+s1}{\PYZsq{}}\PY{p}{)}
\PY{n}{show}\PY{p}{(}\PY{p}{)}
\end{Verbatim}
\end{tcolorbox}

    \begin{center}
    \adjustimage{max size={0.9\linewidth}{0.9\paperheight}}{Lecture-4-Matplotlib_files/Lecture-4-Matplotlib_16_0.png}
    \end{center}
    { \hspace*{\fill} \\}
    
    Most of the plotting related functions in MATLAB are covered by the
\texttt{pylab} module. For example, subplot and color/symbol selection:

    \begin{tcolorbox}[breakable, size=fbox, boxrule=1pt, pad at break*=1mm,colback=cellbackground, colframe=cellborder]
\prompt{In}{incolor}{8}{\boxspacing}
\begin{Verbatim}[commandchars=\\\{\}]
\PY{n}{subplot}\PY{p}{(}\PY{l+m+mi}{1}\PY{p}{,}\PY{l+m+mi}{2}\PY{p}{,}\PY{l+m+mi}{1}\PY{p}{)}
\PY{n}{plot}\PY{p}{(}\PY{n}{x}\PY{p}{,} \PY{n}{y}\PY{p}{,} \PY{l+s+s1}{\PYZsq{}}\PY{l+s+s1}{r\PYZhy{}\PYZhy{}}\PY{l+s+s1}{\PYZsq{}}\PY{p}{)}
\PY{n}{subplot}\PY{p}{(}\PY{l+m+mi}{1}\PY{p}{,}\PY{l+m+mi}{2}\PY{p}{,}\PY{l+m+mi}{2}\PY{p}{)}
\PY{n}{plot}\PY{p}{(}\PY{n}{y}\PY{p}{,} \PY{n}{x}\PY{p}{,} \PY{l+s+s1}{\PYZsq{}}\PY{l+s+s1}{g*\PYZhy{}}\PY{l+s+s1}{\PYZsq{}}\PY{p}{)}\PY{p}{;}
\end{Verbatim}
\end{tcolorbox}

    \begin{center}
    \adjustimage{max size={0.9\linewidth}{0.9\paperheight}}{Lecture-4-Matplotlib_files/Lecture-4-Matplotlib_18_0.png}
    \end{center}
    { \hspace*{\fill} \\}
    
    The good thing about the pylab MATLAB-style API is that it is easy to
get started with if you are familiar with MATLAB, and it has a minumum
of coding overhead for simple plots.

However, I'd encourrage not using the MATLAB compatible API for anything
but the simplest figures.

Instead, I recommend learning and using matplotlib's object-oriented
plotting API. It is remarkably powerful. For advanced figures with
subplots, insets and other components it is very nice to work with.

    \hypertarget{the-matplotlib-object-oriented-api}{%
\subsection{The matplotlib object-oriented
API}\label{the-matplotlib-object-oriented-api}}

    The main idea with object-oriented programming is to have objects that
one can apply functions and actions on, and no object or program states
should be global (such as the MATLAB-like API). The real advantage of
this approach becomes apparent when more than one figure is created, or
when a figure contains more than one subplot.

To use the object-oriented API we start out very much like in the
previous example, but instead of creating a new global figure instance
we store a reference to the newly created figure instance in the
\texttt{fig} variable, and from it we create a new axis instance
\texttt{axes} using the \texttt{add\_axes} method in the \texttt{Figure}
class instance \texttt{fig}:

    \begin{tcolorbox}[breakable, size=fbox, boxrule=1pt, pad at break*=1mm,colback=cellbackground, colframe=cellborder]
\prompt{In}{incolor}{9}{\boxspacing}
\begin{Verbatim}[commandchars=\\\{\}]
\PY{n}{fig} \PY{o}{=} \PY{n}{plt}\PY{o}{.}\PY{n}{figure}\PY{p}{(}\PY{p}{)}

\PY{n}{axes} \PY{o}{=} \PY{n}{fig}\PY{o}{.}\PY{n}{add\PYZus{}axes}\PY{p}{(}\PY{p}{[}\PY{l+m+mf}{0.1}\PY{p}{,} \PY{l+m+mf}{0.1}\PY{p}{,} \PY{l+m+mf}{0.8}\PY{p}{,} \PY{l+m+mf}{0.8}\PY{p}{]}\PY{p}{)} \PY{c+c1}{\PYZsh{} left, bottom, width, height (range 0 to 1)}

\PY{n}{axes}\PY{o}{.}\PY{n}{plot}\PY{p}{(}\PY{n}{x}\PY{p}{,} \PY{n}{y}\PY{p}{,} \PY{l+s+s1}{\PYZsq{}}\PY{l+s+s1}{r}\PY{l+s+s1}{\PYZsq{}}\PY{p}{)}

\PY{n}{axes}\PY{o}{.}\PY{n}{set\PYZus{}xlabel}\PY{p}{(}\PY{l+s+s1}{\PYZsq{}}\PY{l+s+s1}{x}\PY{l+s+s1}{\PYZsq{}}\PY{p}{)}
\PY{n}{axes}\PY{o}{.}\PY{n}{set\PYZus{}ylabel}\PY{p}{(}\PY{l+s+s1}{\PYZsq{}}\PY{l+s+s1}{y}\PY{l+s+s1}{\PYZsq{}}\PY{p}{)}
\PY{n}{axes}\PY{o}{.}\PY{n}{set\PYZus{}title}\PY{p}{(}\PY{l+s+s1}{\PYZsq{}}\PY{l+s+s1}{title}\PY{l+s+s1}{\PYZsq{}}\PY{p}{)}\PY{p}{;}
\end{Verbatim}
\end{tcolorbox}

    \begin{center}
    \adjustimage{max size={0.9\linewidth}{0.9\paperheight}}{Lecture-4-Matplotlib_files/Lecture-4-Matplotlib_22_0.png}
    \end{center}
    { \hspace*{\fill} \\}
    
    Although a little bit more code is involved, the advantage is that we
now have full control of where the plot axes are placed, and we can
easily add more than one axis to the figure:

    \begin{tcolorbox}[breakable, size=fbox, boxrule=1pt, pad at break*=1mm,colback=cellbackground, colframe=cellborder]
\prompt{In}{incolor}{10}{\boxspacing}
\begin{Verbatim}[commandchars=\\\{\}]
\PY{n}{fig} \PY{o}{=} \PY{n}{plt}\PY{o}{.}\PY{n}{figure}\PY{p}{(}\PY{p}{)}

\PY{n}{axes1} \PY{o}{=} \PY{n}{fig}\PY{o}{.}\PY{n}{add\PYZus{}axes}\PY{p}{(}\PY{p}{[}\PY{l+m+mf}{0.1}\PY{p}{,} \PY{l+m+mf}{0.1}\PY{p}{,} \PY{l+m+mf}{0.8}\PY{p}{,} \PY{l+m+mf}{0.8}\PY{p}{]}\PY{p}{)} \PY{c+c1}{\PYZsh{} main axes}
\PY{n}{axes2} \PY{o}{=} \PY{n}{fig}\PY{o}{.}\PY{n}{add\PYZus{}axes}\PY{p}{(}\PY{p}{[}\PY{l+m+mf}{0.2}\PY{p}{,} \PY{l+m+mf}{0.5}\PY{p}{,} \PY{l+m+mf}{0.4}\PY{p}{,} \PY{l+m+mf}{0.3}\PY{p}{]}\PY{p}{)} \PY{c+c1}{\PYZsh{} inset axes}

\PY{c+c1}{\PYZsh{} main figure}
\PY{n}{axes1}\PY{o}{.}\PY{n}{plot}\PY{p}{(}\PY{n}{x}\PY{p}{,} \PY{n}{y}\PY{p}{,} \PY{l+s+s1}{\PYZsq{}}\PY{l+s+s1}{r}\PY{l+s+s1}{\PYZsq{}}\PY{p}{)}
\PY{n}{axes1}\PY{o}{.}\PY{n}{set\PYZus{}xlabel}\PY{p}{(}\PY{l+s+s1}{\PYZsq{}}\PY{l+s+s1}{x}\PY{l+s+s1}{\PYZsq{}}\PY{p}{)}
\PY{n}{axes1}\PY{o}{.}\PY{n}{set\PYZus{}ylabel}\PY{p}{(}\PY{l+s+s1}{\PYZsq{}}\PY{l+s+s1}{y}\PY{l+s+s1}{\PYZsq{}}\PY{p}{)}
\PY{n}{axes1}\PY{o}{.}\PY{n}{set\PYZus{}title}\PY{p}{(}\PY{l+s+s1}{\PYZsq{}}\PY{l+s+s1}{title}\PY{l+s+s1}{\PYZsq{}}\PY{p}{)}

\PY{c+c1}{\PYZsh{} insert}
\PY{n}{axes2}\PY{o}{.}\PY{n}{plot}\PY{p}{(}\PY{n}{y}\PY{p}{,} \PY{n}{x}\PY{p}{,} \PY{l+s+s1}{\PYZsq{}}\PY{l+s+s1}{g}\PY{l+s+s1}{\PYZsq{}}\PY{p}{)}
\PY{n}{axes2}\PY{o}{.}\PY{n}{set\PYZus{}xlabel}\PY{p}{(}\PY{l+s+s1}{\PYZsq{}}\PY{l+s+s1}{y}\PY{l+s+s1}{\PYZsq{}}\PY{p}{)}
\PY{n}{axes2}\PY{o}{.}\PY{n}{set\PYZus{}ylabel}\PY{p}{(}\PY{l+s+s1}{\PYZsq{}}\PY{l+s+s1}{x}\PY{l+s+s1}{\PYZsq{}}\PY{p}{)}
\PY{n}{axes2}\PY{o}{.}\PY{n}{set\PYZus{}title}\PY{p}{(}\PY{l+s+s1}{\PYZsq{}}\PY{l+s+s1}{insert title}\PY{l+s+s1}{\PYZsq{}}\PY{p}{)}\PY{p}{;}
\end{Verbatim}
\end{tcolorbox}

    \begin{center}
    \adjustimage{max size={0.9\linewidth}{0.9\paperheight}}{Lecture-4-Matplotlib_files/Lecture-4-Matplotlib_24_0.png}
    \end{center}
    { \hspace*{\fill} \\}
    
    If we don't care about being explicit about where our plot axes are
placed in the figure canvas, then we can use one of the many axis layout
managers in matplotlib. My favorite is \texttt{subplots}, which can be
used like this:

    \begin{tcolorbox}[breakable, size=fbox, boxrule=1pt, pad at break*=1mm,colback=cellbackground, colframe=cellborder]
\prompt{In}{incolor}{11}{\boxspacing}
\begin{Verbatim}[commandchars=\\\{\}]
\PY{n}{fig}\PY{p}{,} \PY{n}{axes} \PY{o}{=} \PY{n}{plt}\PY{o}{.}\PY{n}{subplots}\PY{p}{(}\PY{p}{)}

\PY{n}{axes}\PY{o}{.}\PY{n}{plot}\PY{p}{(}\PY{n}{x}\PY{p}{,} \PY{n}{y}\PY{p}{,} \PY{l+s+s1}{\PYZsq{}}\PY{l+s+s1}{r}\PY{l+s+s1}{\PYZsq{}}\PY{p}{)}
\PY{n}{axes}\PY{o}{.}\PY{n}{set\PYZus{}xlabel}\PY{p}{(}\PY{l+s+s1}{\PYZsq{}}\PY{l+s+s1}{x}\PY{l+s+s1}{\PYZsq{}}\PY{p}{)}
\PY{n}{axes}\PY{o}{.}\PY{n}{set\PYZus{}ylabel}\PY{p}{(}\PY{l+s+s1}{\PYZsq{}}\PY{l+s+s1}{y}\PY{l+s+s1}{\PYZsq{}}\PY{p}{)}
\PY{n}{axes}\PY{o}{.}\PY{n}{set\PYZus{}title}\PY{p}{(}\PY{l+s+s1}{\PYZsq{}}\PY{l+s+s1}{title}\PY{l+s+s1}{\PYZsq{}}\PY{p}{)}\PY{p}{;}
\end{Verbatim}
\end{tcolorbox}

    \begin{center}
    \adjustimage{max size={0.9\linewidth}{0.9\paperheight}}{Lecture-4-Matplotlib_files/Lecture-4-Matplotlib_26_0.png}
    \end{center}
    { \hspace*{\fill} \\}
    
    \begin{tcolorbox}[breakable, size=fbox, boxrule=1pt, pad at break*=1mm,colback=cellbackground, colframe=cellborder]
\prompt{In}{incolor}{12}{\boxspacing}
\begin{Verbatim}[commandchars=\\\{\}]
\PY{n}{fig}\PY{p}{,} \PY{n}{axes} \PY{o}{=} \PY{n}{plt}\PY{o}{.}\PY{n}{subplots}\PY{p}{(}\PY{n}{nrows}\PY{o}{=}\PY{l+m+mi}{1}\PY{p}{,} \PY{n}{ncols}\PY{o}{=}\PY{l+m+mi}{2}\PY{p}{)}

\PY{k}{for} \PY{n}{ax} \PY{o+ow}{in} \PY{n}{axes}\PY{p}{:}
    \PY{n}{ax}\PY{o}{.}\PY{n}{plot}\PY{p}{(}\PY{n}{x}\PY{p}{,} \PY{n}{y}\PY{p}{,} \PY{l+s+s1}{\PYZsq{}}\PY{l+s+s1}{r}\PY{l+s+s1}{\PYZsq{}}\PY{p}{)}
    \PY{n}{ax}\PY{o}{.}\PY{n}{set\PYZus{}xlabel}\PY{p}{(}\PY{l+s+s1}{\PYZsq{}}\PY{l+s+s1}{x}\PY{l+s+s1}{\PYZsq{}}\PY{p}{)}
    \PY{n}{ax}\PY{o}{.}\PY{n}{set\PYZus{}ylabel}\PY{p}{(}\PY{l+s+s1}{\PYZsq{}}\PY{l+s+s1}{y}\PY{l+s+s1}{\PYZsq{}}\PY{p}{)}
    \PY{n}{ax}\PY{o}{.}\PY{n}{set\PYZus{}title}\PY{p}{(}\PY{l+s+s1}{\PYZsq{}}\PY{l+s+s1}{title}\PY{l+s+s1}{\PYZsq{}}\PY{p}{)}
\end{Verbatim}
\end{tcolorbox}

    \begin{center}
    \adjustimage{max size={0.9\linewidth}{0.9\paperheight}}{Lecture-4-Matplotlib_files/Lecture-4-Matplotlib_27_0.png}
    \end{center}
    { \hspace*{\fill} \\}
    
    That was easy, but it isn't so pretty with overlapping figure axes and
labels, right?

We can deal with that by using the \texttt{fig.tight\_layout} method,
which automatically adjusts the positions of the axes on the figure
canvas so that there is no overlapping content:

    \begin{tcolorbox}[breakable, size=fbox, boxrule=1pt, pad at break*=1mm,colback=cellbackground, colframe=cellborder]
\prompt{In}{incolor}{13}{\boxspacing}
\begin{Verbatim}[commandchars=\\\{\}]
\PY{n}{fig}\PY{p}{,} \PY{n}{axes} \PY{o}{=} \PY{n}{plt}\PY{o}{.}\PY{n}{subplots}\PY{p}{(}\PY{n}{nrows}\PY{o}{=}\PY{l+m+mi}{1}\PY{p}{,} \PY{n}{ncols}\PY{o}{=}\PY{l+m+mi}{2}\PY{p}{)}

\PY{k}{for} \PY{n}{ax} \PY{o+ow}{in} \PY{n}{axes}\PY{p}{:}
    \PY{n}{ax}\PY{o}{.}\PY{n}{plot}\PY{p}{(}\PY{n}{x}\PY{p}{,} \PY{n}{y}\PY{p}{,} \PY{l+s+s1}{\PYZsq{}}\PY{l+s+s1}{r}\PY{l+s+s1}{\PYZsq{}}\PY{p}{)}
    \PY{n}{ax}\PY{o}{.}\PY{n}{set\PYZus{}xlabel}\PY{p}{(}\PY{l+s+s1}{\PYZsq{}}\PY{l+s+s1}{x}\PY{l+s+s1}{\PYZsq{}}\PY{p}{)}
    \PY{n}{ax}\PY{o}{.}\PY{n}{set\PYZus{}ylabel}\PY{p}{(}\PY{l+s+s1}{\PYZsq{}}\PY{l+s+s1}{y}\PY{l+s+s1}{\PYZsq{}}\PY{p}{)}
    \PY{n}{ax}\PY{o}{.}\PY{n}{set\PYZus{}title}\PY{p}{(}\PY{l+s+s1}{\PYZsq{}}\PY{l+s+s1}{title}\PY{l+s+s1}{\PYZsq{}}\PY{p}{)}
    
\PY{n}{fig}\PY{o}{.}\PY{n}{tight\PYZus{}layout}\PY{p}{(}\PY{p}{)}
\end{Verbatim}
\end{tcolorbox}

    \begin{center}
    \adjustimage{max size={0.9\linewidth}{0.9\paperheight}}{Lecture-4-Matplotlib_files/Lecture-4-Matplotlib_29_0.png}
    \end{center}
    { \hspace*{\fill} \\}
    
    \hypertarget{figure-size-aspect-ratio-and-dpi}{%
\subsubsection{Figure size, aspect ratio and
DPI}\label{figure-size-aspect-ratio-and-dpi}}

    Matplotlib allows the aspect ratio, DPI and figure size to be specified
when the \texttt{Figure} object is created, using the \texttt{figsize}
and \texttt{dpi} keyword arguments. \texttt{figsize} is a tuple of the
width and height of the figure in inches, and \texttt{dpi} is the
dots-per-inch (pixel per inch). To create an 800x400 pixel, 100
dots-per-inch figure, we can do:

    \begin{tcolorbox}[breakable, size=fbox, boxrule=1pt, pad at break*=1mm,colback=cellbackground, colframe=cellborder]
\prompt{In}{incolor}{14}{\boxspacing}
\begin{Verbatim}[commandchars=\\\{\}]
\PY{n}{fig} \PY{o}{=} \PY{n}{plt}\PY{o}{.}\PY{n}{figure}\PY{p}{(}\PY{n}{figsize}\PY{o}{=}\PY{p}{(}\PY{l+m+mi}{8}\PY{p}{,}\PY{l+m+mi}{4}\PY{p}{)}\PY{p}{,} \PY{n}{dpi}\PY{o}{=}\PY{l+m+mi}{100}\PY{p}{)}
\end{Verbatim}
\end{tcolorbox}

    
    \begin{verbatim}
<Figure size 800x400 with 0 Axes>
    \end{verbatim}

    
    The same arguments can also be passed to layout managers, such as the
\texttt{subplots} function:

    \begin{tcolorbox}[breakable, size=fbox, boxrule=1pt, pad at break*=1mm,colback=cellbackground, colframe=cellborder]
\prompt{In}{incolor}{15}{\boxspacing}
\begin{Verbatim}[commandchars=\\\{\}]
\PY{n}{fig}\PY{p}{,} \PY{n}{axes} \PY{o}{=} \PY{n}{plt}\PY{o}{.}\PY{n}{subplots}\PY{p}{(}\PY{n}{figsize}\PY{o}{=}\PY{p}{(}\PY{l+m+mi}{12}\PY{p}{,}\PY{l+m+mi}{3}\PY{p}{)}\PY{p}{)}

\PY{n}{axes}\PY{o}{.}\PY{n}{plot}\PY{p}{(}\PY{n}{x}\PY{p}{,} \PY{n}{y}\PY{p}{,} \PY{l+s+s1}{\PYZsq{}}\PY{l+s+s1}{r}\PY{l+s+s1}{\PYZsq{}}\PY{p}{)}
\PY{n}{axes}\PY{o}{.}\PY{n}{set\PYZus{}xlabel}\PY{p}{(}\PY{l+s+s1}{\PYZsq{}}\PY{l+s+s1}{x}\PY{l+s+s1}{\PYZsq{}}\PY{p}{)}
\PY{n}{axes}\PY{o}{.}\PY{n}{set\PYZus{}ylabel}\PY{p}{(}\PY{l+s+s1}{\PYZsq{}}\PY{l+s+s1}{y}\PY{l+s+s1}{\PYZsq{}}\PY{p}{)}
\PY{n}{axes}\PY{o}{.}\PY{n}{set\PYZus{}title}\PY{p}{(}\PY{l+s+s1}{\PYZsq{}}\PY{l+s+s1}{title}\PY{l+s+s1}{\PYZsq{}}\PY{p}{)}\PY{p}{;}
\end{Verbatim}
\end{tcolorbox}

    \begin{center}
    \adjustimage{max size={0.9\linewidth}{0.9\paperheight}}{Lecture-4-Matplotlib_files/Lecture-4-Matplotlib_34_0.png}
    \end{center}
    { \hspace*{\fill} \\}
    
    \hypertarget{saving-figures}{%
\subsubsection{Saving figures}\label{saving-figures}}

    To save a figure to a file we can use the \texttt{savefig} method in the
\texttt{Figure} class:

    \begin{tcolorbox}[breakable, size=fbox, boxrule=1pt, pad at break*=1mm,colback=cellbackground, colframe=cellborder]
\prompt{In}{incolor}{16}{\boxspacing}
\begin{Verbatim}[commandchars=\\\{\}]
\PY{n}{fig}\PY{o}{.}\PY{n}{savefig}\PY{p}{(}\PY{l+s+s2}{\PYZdq{}}\PY{l+s+s2}{filename.png}\PY{l+s+s2}{\PYZdq{}}\PY{p}{)}
\end{Verbatim}
\end{tcolorbox}

    Here we can also optionally specify the DPI and choose between different
output formats:

    \begin{tcolorbox}[breakable, size=fbox, boxrule=1pt, pad at break*=1mm,colback=cellbackground, colframe=cellborder]
\prompt{In}{incolor}{17}{\boxspacing}
\begin{Verbatim}[commandchars=\\\{\}]
\PY{n}{fig}\PY{o}{.}\PY{n}{savefig}\PY{p}{(}\PY{l+s+s2}{\PYZdq{}}\PY{l+s+s2}{filename.png}\PY{l+s+s2}{\PYZdq{}}\PY{p}{,} \PY{n}{dpi}\PY{o}{=}\PY{l+m+mi}{200}\PY{p}{)}
\end{Verbatim}
\end{tcolorbox}

    \hypertarget{what-formats-are-available-and-which-ones-should-be-used-for-best-quality}{%
\paragraph{What formats are available and which ones should be used for
best
quality?}\label{what-formats-are-available-and-which-ones-should-be-used-for-best-quality}}

    Matplotlib can generate high-quality output in a number formats,
including PNG, JPG, EPS, SVG, PGF and PDF. For scientific papers, I
recommend using PDF whenever possible. (LaTeX documents compiled with
\texttt{pdflatex} can include PDFs using the \texttt{includegraphics}
command). In some cases, PGF can also be good alternative.

    \hypertarget{legends-labels-and-titles}{%
\subsubsection{Legends, labels and
titles}\label{legends-labels-and-titles}}

    Now that we have covered the basics of how to create a figure canvas and
add axes instances to the canvas, let's look at how decorate a figure
with titles, axis labels, and legends.

    \textbf{Figure titles}

A title can be added to each axis instance in a figure. To set the
title, use the \texttt{set\_title} method in the axes instance:

    \begin{tcolorbox}[breakable, size=fbox, boxrule=1pt, pad at break*=1mm,colback=cellbackground, colframe=cellborder]
\prompt{In}{incolor}{18}{\boxspacing}
\begin{Verbatim}[commandchars=\\\{\}]
\PY{n}{ax}\PY{o}{.}\PY{n}{set\PYZus{}title}\PY{p}{(}\PY{l+s+s2}{\PYZdq{}}\PY{l+s+s2}{title}\PY{l+s+s2}{\PYZdq{}}\PY{p}{)}\PY{p}{;}
\end{Verbatim}
\end{tcolorbox}

    \textbf{Axis labels}

Similarly, with the methods \texttt{set\_xlabel} and
\texttt{set\_ylabel}, we can set the labels of the X and Y axes:

    \begin{tcolorbox}[breakable, size=fbox, boxrule=1pt, pad at break*=1mm,colback=cellbackground, colframe=cellborder]
\prompt{In}{incolor}{19}{\boxspacing}
\begin{Verbatim}[commandchars=\\\{\}]
\PY{n}{ax}\PY{o}{.}\PY{n}{set\PYZus{}xlabel}\PY{p}{(}\PY{l+s+s2}{\PYZdq{}}\PY{l+s+s2}{x}\PY{l+s+s2}{\PYZdq{}}\PY{p}{)}
\PY{n}{ax}\PY{o}{.}\PY{n}{set\PYZus{}ylabel}\PY{p}{(}\PY{l+s+s2}{\PYZdq{}}\PY{l+s+s2}{y}\PY{l+s+s2}{\PYZdq{}}\PY{p}{)}\PY{p}{;}
\end{Verbatim}
\end{tcolorbox}

    \textbf{Legends}

Legends for curves in a figure can be added in two ways. One method is
to use the \texttt{legend} method of the axis object and pass a
list/tuple of legend texts for the previously defined curves:

    \begin{tcolorbox}[breakable, size=fbox, boxrule=1pt, pad at break*=1mm,colback=cellbackground, colframe=cellborder]
\prompt{In}{incolor}{20}{\boxspacing}
\begin{Verbatim}[commandchars=\\\{\}]
\PY{n}{ax}\PY{o}{.}\PY{n}{legend}\PY{p}{(}\PY{p}{[}\PY{l+s+s2}{\PYZdq{}}\PY{l+s+s2}{curve1}\PY{l+s+s2}{\PYZdq{}}\PY{p}{,} \PY{l+s+s2}{\PYZdq{}}\PY{l+s+s2}{curve2}\PY{l+s+s2}{\PYZdq{}}\PY{p}{,} \PY{l+s+s2}{\PYZdq{}}\PY{l+s+s2}{curve3}\PY{l+s+s2}{\PYZdq{}}\PY{p}{]}\PY{p}{)}\PY{p}{;}
\end{Verbatim}
\end{tcolorbox}

    The method described above follows the MATLAB API. It is somewhat prone
to errors and unflexible if curves are added to or removed from the
figure (resulting in a wrongly labelled curve).

A better method is to use the \texttt{label="label\ text"} keyword
argument when plots or other objects are added to the figure, and then
using the \texttt{legend} method without arguments to add the legend to
the figure:

    \begin{tcolorbox}[breakable, size=fbox, boxrule=1pt, pad at break*=1mm,colback=cellbackground, colframe=cellborder]
\prompt{In}{incolor}{21}{\boxspacing}
\begin{Verbatim}[commandchars=\\\{\}]
\PY{n}{ax}\PY{o}{.}\PY{n}{plot}\PY{p}{(}\PY{n}{x}\PY{p}{,} \PY{n}{x}\PY{o}{*}\PY{o}{*}\PY{l+m+mi}{2}\PY{p}{,} \PY{n}{label}\PY{o}{=}\PY{l+s+s2}{\PYZdq{}}\PY{l+s+s2}{curve1}\PY{l+s+s2}{\PYZdq{}}\PY{p}{)}
\PY{n}{ax}\PY{o}{.}\PY{n}{plot}\PY{p}{(}\PY{n}{x}\PY{p}{,} \PY{n}{x}\PY{o}{*}\PY{o}{*}\PY{l+m+mi}{3}\PY{p}{,} \PY{n}{label}\PY{o}{=}\PY{l+s+s2}{\PYZdq{}}\PY{l+s+s2}{curve2}\PY{l+s+s2}{\PYZdq{}}\PY{p}{)}
\PY{n}{ax}\PY{o}{.}\PY{n}{legend}\PY{p}{(}\PY{p}{)}\PY{p}{;}
\end{Verbatim}
\end{tcolorbox}

    The advantage with this method is that if curves are added or removed
from the figure, the legend is automatically updated accordingly.

The \texttt{legend} function takes an optional keyword argument
\texttt{loc} that can be used to specify where in the figure the legend
is to be drawn. The allowed values of \texttt{loc} are numerical codes
for the various places the legend can be drawn. See
http://matplotlib.org/users/legend\_guide.html\#legend-location for
details. Some of the most common \texttt{loc} values are:

    \begin{tcolorbox}[breakable, size=fbox, boxrule=1pt, pad at break*=1mm,colback=cellbackground, colframe=cellborder]
\prompt{In}{incolor}{22}{\boxspacing}
\begin{Verbatim}[commandchars=\\\{\}]
\PY{n}{ax}\PY{o}{.}\PY{n}{legend}\PY{p}{(}\PY{n}{loc}\PY{o}{=}\PY{l+m+mi}{0}\PY{p}{)} \PY{c+c1}{\PYZsh{} let matplotlib decide the optimal location}
\PY{n}{ax}\PY{o}{.}\PY{n}{legend}\PY{p}{(}\PY{n}{loc}\PY{o}{=}\PY{l+m+mi}{1}\PY{p}{)} \PY{c+c1}{\PYZsh{} upper right corner}
\PY{n}{ax}\PY{o}{.}\PY{n}{legend}\PY{p}{(}\PY{n}{loc}\PY{o}{=}\PY{l+m+mi}{2}\PY{p}{)} \PY{c+c1}{\PYZsh{} upper left corner}
\PY{n}{ax}\PY{o}{.}\PY{n}{legend}\PY{p}{(}\PY{n}{loc}\PY{o}{=}\PY{l+m+mi}{3}\PY{p}{)} \PY{c+c1}{\PYZsh{} lower left corner}
\PY{n}{ax}\PY{o}{.}\PY{n}{legend}\PY{p}{(}\PY{n}{loc}\PY{o}{=}\PY{l+m+mi}{4}\PY{p}{)} \PY{c+c1}{\PYZsh{} lower right corner}
\PY{c+c1}{\PYZsh{} .. many more options are available}
\end{Verbatim}
\end{tcolorbox}

            \begin{tcolorbox}[breakable, size=fbox, boxrule=.5pt, pad at break*=1mm, opacityfill=0]
\prompt{Out}{outcolor}{22}{\boxspacing}
\begin{Verbatim}[commandchars=\\\{\}]
<matplotlib.legend.Legend at 0x1cadb54c5b0>
\end{Verbatim}
\end{tcolorbox}
        
    The following figure shows how to use the figure title, axis labels and
legends described above:

    \begin{tcolorbox}[breakable, size=fbox, boxrule=1pt, pad at break*=1mm,colback=cellbackground, colframe=cellborder]
\prompt{In}{incolor}{23}{\boxspacing}
\begin{Verbatim}[commandchars=\\\{\}]
\PY{n}{fig}\PY{p}{,} \PY{n}{ax} \PY{o}{=} \PY{n}{plt}\PY{o}{.}\PY{n}{subplots}\PY{p}{(}\PY{p}{)}

\PY{n}{ax}\PY{o}{.}\PY{n}{plot}\PY{p}{(}\PY{n}{x}\PY{p}{,} \PY{n}{x}\PY{o}{*}\PY{o}{*}\PY{l+m+mi}{2}\PY{p}{,} \PY{n}{label}\PY{o}{=}\PY{l+s+s2}{\PYZdq{}}\PY{l+s+s2}{y = x**2}\PY{l+s+s2}{\PYZdq{}}\PY{p}{)}
\PY{n}{ax}\PY{o}{.}\PY{n}{plot}\PY{p}{(}\PY{n}{x}\PY{p}{,} \PY{n}{x}\PY{o}{*}\PY{o}{*}\PY{l+m+mi}{3}\PY{p}{,} \PY{n}{label}\PY{o}{=}\PY{l+s+s2}{\PYZdq{}}\PY{l+s+s2}{y = x**3}\PY{l+s+s2}{\PYZdq{}}\PY{p}{)}
\PY{n}{ax}\PY{o}{.}\PY{n}{legend}\PY{p}{(}\PY{n}{loc}\PY{o}{=}\PY{l+m+mi}{2}\PY{p}{)}\PY{p}{;} \PY{c+c1}{\PYZsh{} upper left corner}
\PY{n}{ax}\PY{o}{.}\PY{n}{set\PYZus{}xlabel}\PY{p}{(}\PY{l+s+s1}{\PYZsq{}}\PY{l+s+s1}{x}\PY{l+s+s1}{\PYZsq{}}\PY{p}{)}
\PY{n}{ax}\PY{o}{.}\PY{n}{set\PYZus{}ylabel}\PY{p}{(}\PY{l+s+s1}{\PYZsq{}}\PY{l+s+s1}{y}\PY{l+s+s1}{\PYZsq{}}\PY{p}{)}
\PY{n}{ax}\PY{o}{.}\PY{n}{set\PYZus{}title}\PY{p}{(}\PY{l+s+s1}{\PYZsq{}}\PY{l+s+s1}{title}\PY{l+s+s1}{\PYZsq{}}\PY{p}{)}\PY{p}{;}
\end{Verbatim}
\end{tcolorbox}

    \begin{center}
    \adjustimage{max size={0.9\linewidth}{0.9\paperheight}}{Lecture-4-Matplotlib_files/Lecture-4-Matplotlib_55_0.png}
    \end{center}
    { \hspace*{\fill} \\}
    
    \hypertarget{formatting-text-latex-fontsize-font-family}{%
\subsubsection{Formatting text: LaTeX, fontsize, font
family}\label{formatting-text-latex-fontsize-font-family}}

    The figure above is functional, but it does not (yet) satisfy the
criteria for a figure used in a publication. First and foremost, we need
to have LaTeX formatted text, and second, we need to be able to adjust
the font size to appear right in a publication.

Matplotlib has great support for LaTeX. All we need to do is to use
dollar signs encapsulate LaTeX in any text (legend, title, label, etc.).
For example, \texttt{"\$y=x\^{}3\$"}.

But here we can run into a slightly subtle problem with LaTeX code and
Python text strings. In LaTeX, we frequently use the backslash in
commands, for example \texttt{\textbackslash{}alpha} to produce the
symbol \(\alpha\). But the backslash already has a meaning in Python
strings (the escape code character). To avoid Python messing up our
latex code, we need to use ``raw'' text strings. Raw text strings are
prepended with an `\texttt{r}', like \texttt{r"\textbackslash{}alpha"}
or \texttt{r\textquotesingle{}\textbackslash{}alpha\textquotesingle{}}
instead of \texttt{"\textbackslash{}alpha"} or
\texttt{\textquotesingle{}\textbackslash{}alpha\textquotesingle{}}:

    \begin{tcolorbox}[breakable, size=fbox, boxrule=1pt, pad at break*=1mm,colback=cellbackground, colframe=cellborder]
\prompt{In}{incolor}{24}{\boxspacing}
\begin{Verbatim}[commandchars=\\\{\}]
\PY{n}{fig}\PY{p}{,} \PY{n}{ax} \PY{o}{=} \PY{n}{plt}\PY{o}{.}\PY{n}{subplots}\PY{p}{(}\PY{p}{)}

\PY{n}{ax}\PY{o}{.}\PY{n}{plot}\PY{p}{(}\PY{n}{x}\PY{p}{,} \PY{n}{x}\PY{o}{*}\PY{o}{*}\PY{l+m+mi}{2}\PY{p}{,} \PY{n}{label}\PY{o}{=}\PY{l+s+sa}{r}\PY{l+s+s2}{\PYZdq{}}\PY{l+s+s2}{\PYZdl{}y = }\PY{l+s+s2}{\PYZbs{}}\PY{l+s+s2}{alpha\PYZca{}2\PYZdl{}}\PY{l+s+s2}{\PYZdq{}}\PY{p}{)}
\PY{n}{ax}\PY{o}{.}\PY{n}{plot}\PY{p}{(}\PY{n}{x}\PY{p}{,} \PY{n}{x}\PY{o}{*}\PY{o}{*}\PY{l+m+mi}{3}\PY{p}{,} \PY{n}{label}\PY{o}{=}\PY{l+s+sa}{r}\PY{l+s+s2}{\PYZdq{}}\PY{l+s+s2}{\PYZdl{}y = }\PY{l+s+s2}{\PYZbs{}}\PY{l+s+s2}{alpha\PYZca{}3\PYZdl{}}\PY{l+s+s2}{\PYZdq{}}\PY{p}{)}
\PY{n}{ax}\PY{o}{.}\PY{n}{legend}\PY{p}{(}\PY{n}{loc}\PY{o}{=}\PY{l+m+mi}{2}\PY{p}{)} \PY{c+c1}{\PYZsh{} upper left corner}
\PY{n}{ax}\PY{o}{.}\PY{n}{set\PYZus{}xlabel}\PY{p}{(}\PY{l+s+sa}{r}\PY{l+s+s1}{\PYZsq{}}\PY{l+s+s1}{\PYZdl{}}\PY{l+s+s1}{\PYZbs{}}\PY{l+s+s1}{alpha\PYZdl{}}\PY{l+s+s1}{\PYZsq{}}\PY{p}{,} \PY{n}{fontsize}\PY{o}{=}\PY{l+m+mi}{18}\PY{p}{)}
\PY{n}{ax}\PY{o}{.}\PY{n}{set\PYZus{}ylabel}\PY{p}{(}\PY{l+s+sa}{r}\PY{l+s+s1}{\PYZsq{}}\PY{l+s+s1}{\PYZdl{}y\PYZdl{}}\PY{l+s+s1}{\PYZsq{}}\PY{p}{,} \PY{n}{fontsize}\PY{o}{=}\PY{l+m+mi}{18}\PY{p}{)}
\PY{n}{ax}\PY{o}{.}\PY{n}{set\PYZus{}title}\PY{p}{(}\PY{l+s+s1}{\PYZsq{}}\PY{l+s+s1}{title}\PY{l+s+s1}{\PYZsq{}}\PY{p}{)}\PY{p}{;}
\end{Verbatim}
\end{tcolorbox}

    \begin{center}
    \adjustimage{max size={0.9\linewidth}{0.9\paperheight}}{Lecture-4-Matplotlib_files/Lecture-4-Matplotlib_58_0.png}
    \end{center}
    { \hspace*{\fill} \\}
    
    We can also change the global font size and font family, which applies
to all text elements in a figure (tick labels, axis labels and titles,
legends, etc.):

    \begin{tcolorbox}[breakable, size=fbox, boxrule=1pt, pad at break*=1mm,colback=cellbackground, colframe=cellborder]
\prompt{In}{incolor}{25}{\boxspacing}
\begin{Verbatim}[commandchars=\\\{\}]
\PY{c+c1}{\PYZsh{} Update the matplotlib configuration parameters:}
\PY{n}{matplotlib}\PY{o}{.}\PY{n}{rcParams}\PY{o}{.}\PY{n}{update}\PY{p}{(}\PY{p}{\PYZob{}}\PY{l+s+s1}{\PYZsq{}}\PY{l+s+s1}{font.size}\PY{l+s+s1}{\PYZsq{}}\PY{p}{:} \PY{l+m+mi}{18}\PY{p}{,} \PY{l+s+s1}{\PYZsq{}}\PY{l+s+s1}{font.family}\PY{l+s+s1}{\PYZsq{}}\PY{p}{:} \PY{l+s+s1}{\PYZsq{}}\PY{l+s+s1}{serif}\PY{l+s+s1}{\PYZsq{}}\PY{p}{\PYZcb{}}\PY{p}{)}
\end{Verbatim}
\end{tcolorbox}

    \begin{tcolorbox}[breakable, size=fbox, boxrule=1pt, pad at break*=1mm,colback=cellbackground, colframe=cellborder]
\prompt{In}{incolor}{26}{\boxspacing}
\begin{Verbatim}[commandchars=\\\{\}]
\PY{n}{fig}\PY{p}{,} \PY{n}{ax} \PY{o}{=} \PY{n}{plt}\PY{o}{.}\PY{n}{subplots}\PY{p}{(}\PY{p}{)}

\PY{n}{ax}\PY{o}{.}\PY{n}{plot}\PY{p}{(}\PY{n}{x}\PY{p}{,} \PY{n}{x}\PY{o}{*}\PY{o}{*}\PY{l+m+mi}{2}\PY{p}{,} \PY{n}{label}\PY{o}{=}\PY{l+s+sa}{r}\PY{l+s+s2}{\PYZdq{}}\PY{l+s+s2}{\PYZdl{}y = }\PY{l+s+s2}{\PYZbs{}}\PY{l+s+s2}{alpha\PYZca{}2\PYZdl{}}\PY{l+s+s2}{\PYZdq{}}\PY{p}{)}
\PY{n}{ax}\PY{o}{.}\PY{n}{plot}\PY{p}{(}\PY{n}{x}\PY{p}{,} \PY{n}{x}\PY{o}{*}\PY{o}{*}\PY{l+m+mi}{3}\PY{p}{,} \PY{n}{label}\PY{o}{=}\PY{l+s+sa}{r}\PY{l+s+s2}{\PYZdq{}}\PY{l+s+s2}{\PYZdl{}y = }\PY{l+s+s2}{\PYZbs{}}\PY{l+s+s2}{alpha\PYZca{}3\PYZdl{}}\PY{l+s+s2}{\PYZdq{}}\PY{p}{)}
\PY{n}{ax}\PY{o}{.}\PY{n}{legend}\PY{p}{(}\PY{n}{loc}\PY{o}{=}\PY{l+m+mi}{2}\PY{p}{)} \PY{c+c1}{\PYZsh{} upper left corner}
\PY{n}{ax}\PY{o}{.}\PY{n}{set\PYZus{}xlabel}\PY{p}{(}\PY{l+s+sa}{r}\PY{l+s+s1}{\PYZsq{}}\PY{l+s+s1}{\PYZdl{}}\PY{l+s+s1}{\PYZbs{}}\PY{l+s+s1}{alpha\PYZdl{}}\PY{l+s+s1}{\PYZsq{}}\PY{p}{)}
\PY{n}{ax}\PY{o}{.}\PY{n}{set\PYZus{}ylabel}\PY{p}{(}\PY{l+s+sa}{r}\PY{l+s+s1}{\PYZsq{}}\PY{l+s+s1}{\PYZdl{}y\PYZdl{}}\PY{l+s+s1}{\PYZsq{}}\PY{p}{)}
\PY{n}{ax}\PY{o}{.}\PY{n}{set\PYZus{}title}\PY{p}{(}\PY{l+s+s1}{\PYZsq{}}\PY{l+s+s1}{title}\PY{l+s+s1}{\PYZsq{}}\PY{p}{)}\PY{p}{;}
\end{Verbatim}
\end{tcolorbox}

    \begin{center}
    \adjustimage{max size={0.9\linewidth}{0.9\paperheight}}{Lecture-4-Matplotlib_files/Lecture-4-Matplotlib_61_0.png}
    \end{center}
    { \hspace*{\fill} \\}
    
    A good choice of global fonts are the STIX fonts:

    \begin{tcolorbox}[breakable, size=fbox, boxrule=1pt, pad at break*=1mm,colback=cellbackground, colframe=cellborder]
\prompt{In}{incolor}{27}{\boxspacing}
\begin{Verbatim}[commandchars=\\\{\}]
\PY{c+c1}{\PYZsh{} Update the matplotlib configuration parameters:}
\PY{n}{matplotlib}\PY{o}{.}\PY{n}{rcParams}\PY{o}{.}\PY{n}{update}\PY{p}{(}\PY{p}{\PYZob{}}\PY{l+s+s1}{\PYZsq{}}\PY{l+s+s1}{font.size}\PY{l+s+s1}{\PYZsq{}}\PY{p}{:} \PY{l+m+mi}{18}\PY{p}{,} \PY{l+s+s1}{\PYZsq{}}\PY{l+s+s1}{font.family}\PY{l+s+s1}{\PYZsq{}}\PY{p}{:} \PY{l+s+s1}{\PYZsq{}}\PY{l+s+s1}{STIXGeneral}\PY{l+s+s1}{\PYZsq{}}\PY{p}{,} \PY{l+s+s1}{\PYZsq{}}\PY{l+s+s1}{mathtext.fontset}\PY{l+s+s1}{\PYZsq{}}\PY{p}{:} \PY{l+s+s1}{\PYZsq{}}\PY{l+s+s1}{stix}\PY{l+s+s1}{\PYZsq{}}\PY{p}{\PYZcb{}}\PY{p}{)}
\end{Verbatim}
\end{tcolorbox}

    \begin{tcolorbox}[breakable, size=fbox, boxrule=1pt, pad at break*=1mm,colback=cellbackground, colframe=cellborder]
\prompt{In}{incolor}{28}{\boxspacing}
\begin{Verbatim}[commandchars=\\\{\}]
\PY{n}{fig}\PY{p}{,} \PY{n}{ax} \PY{o}{=} \PY{n}{plt}\PY{o}{.}\PY{n}{subplots}\PY{p}{(}\PY{p}{)}

\PY{n}{ax}\PY{o}{.}\PY{n}{plot}\PY{p}{(}\PY{n}{x}\PY{p}{,} \PY{n}{x}\PY{o}{*}\PY{o}{*}\PY{l+m+mi}{2}\PY{p}{,} \PY{n}{label}\PY{o}{=}\PY{l+s+sa}{r}\PY{l+s+s2}{\PYZdq{}}\PY{l+s+s2}{\PYZdl{}y = }\PY{l+s+s2}{\PYZbs{}}\PY{l+s+s2}{alpha\PYZca{}2\PYZdl{}}\PY{l+s+s2}{\PYZdq{}}\PY{p}{)}
\PY{n}{ax}\PY{o}{.}\PY{n}{plot}\PY{p}{(}\PY{n}{x}\PY{p}{,} \PY{n}{x}\PY{o}{*}\PY{o}{*}\PY{l+m+mi}{3}\PY{p}{,} \PY{n}{label}\PY{o}{=}\PY{l+s+sa}{r}\PY{l+s+s2}{\PYZdq{}}\PY{l+s+s2}{\PYZdl{}y = }\PY{l+s+s2}{\PYZbs{}}\PY{l+s+s2}{alpha\PYZca{}3\PYZdl{}}\PY{l+s+s2}{\PYZdq{}}\PY{p}{)}
\PY{n}{ax}\PY{o}{.}\PY{n}{legend}\PY{p}{(}\PY{n}{loc}\PY{o}{=}\PY{l+m+mi}{2}\PY{p}{)} \PY{c+c1}{\PYZsh{} upper left corner}
\PY{n}{ax}\PY{o}{.}\PY{n}{set\PYZus{}xlabel}\PY{p}{(}\PY{l+s+sa}{r}\PY{l+s+s1}{\PYZsq{}}\PY{l+s+s1}{\PYZdl{}}\PY{l+s+s1}{\PYZbs{}}\PY{l+s+s1}{alpha\PYZdl{}}\PY{l+s+s1}{\PYZsq{}}\PY{p}{)}
\PY{n}{ax}\PY{o}{.}\PY{n}{set\PYZus{}ylabel}\PY{p}{(}\PY{l+s+sa}{r}\PY{l+s+s1}{\PYZsq{}}\PY{l+s+s1}{\PYZdl{}y\PYZdl{}}\PY{l+s+s1}{\PYZsq{}}\PY{p}{)}
\PY{n}{ax}\PY{o}{.}\PY{n}{set\PYZus{}title}\PY{p}{(}\PY{l+s+s1}{\PYZsq{}}\PY{l+s+s1}{title}\PY{l+s+s1}{\PYZsq{}}\PY{p}{)}\PY{p}{;}
\end{Verbatim}
\end{tcolorbox}

    \begin{center}
    \adjustimage{max size={0.9\linewidth}{0.9\paperheight}}{Lecture-4-Matplotlib_files/Lecture-4-Matplotlib_64_0.png}
    \end{center}
    { \hspace*{\fill} \\}
    
    Or, alternatively, we can request that matplotlib uses LaTeX to render
the text elements in the figure:

    \begin{tcolorbox}[breakable, size=fbox, boxrule=1pt, pad at break*=1mm,colback=cellbackground, colframe=cellborder]
\prompt{In}{incolor}{29}{\boxspacing}
\begin{Verbatim}[commandchars=\\\{\}]
\PY{n}{matplotlib}\PY{o}{.}\PY{n}{rcParams}\PY{o}{.}\PY{n}{update}\PY{p}{(}\PY{p}{\PYZob{}}\PY{l+s+s1}{\PYZsq{}}\PY{l+s+s1}{font.size}\PY{l+s+s1}{\PYZsq{}}\PY{p}{:} \PY{l+m+mi}{18}\PY{p}{,} \PY{l+s+s1}{\PYZsq{}}\PY{l+s+s1}{text.usetex}\PY{l+s+s1}{\PYZsq{}}\PY{p}{:} \PY{k+kc}{True}\PY{p}{\PYZcb{}}\PY{p}{)}
\end{Verbatim}
\end{tcolorbox}

    \begin{tcolorbox}[breakable, size=fbox, boxrule=1pt, pad at break*=1mm,colback=cellbackground, colframe=cellborder]
\prompt{In}{incolor}{30}{\boxspacing}
\begin{Verbatim}[commandchars=\\\{\}]
\PY{n}{fig}\PY{p}{,} \PY{n}{ax} \PY{o}{=} \PY{n}{plt}\PY{o}{.}\PY{n}{subplots}\PY{p}{(}\PY{p}{)}

\PY{n}{ax}\PY{o}{.}\PY{n}{plot}\PY{p}{(}\PY{n}{x}\PY{p}{,} \PY{n}{x}\PY{o}{*}\PY{o}{*}\PY{l+m+mi}{2}\PY{p}{,} \PY{n}{label}\PY{o}{=}\PY{l+s+sa}{r}\PY{l+s+s2}{\PYZdq{}}\PY{l+s+s2}{\PYZdl{}y = }\PY{l+s+s2}{\PYZbs{}}\PY{l+s+s2}{alpha\PYZca{}2\PYZdl{}}\PY{l+s+s2}{\PYZdq{}}\PY{p}{)}
\PY{n}{ax}\PY{o}{.}\PY{n}{plot}\PY{p}{(}\PY{n}{x}\PY{p}{,} \PY{n}{x}\PY{o}{*}\PY{o}{*}\PY{l+m+mi}{3}\PY{p}{,} \PY{n}{label}\PY{o}{=}\PY{l+s+sa}{r}\PY{l+s+s2}{\PYZdq{}}\PY{l+s+s2}{\PYZdl{}y = }\PY{l+s+s2}{\PYZbs{}}\PY{l+s+s2}{alpha\PYZca{}3\PYZdl{}}\PY{l+s+s2}{\PYZdq{}}\PY{p}{)}
\PY{n}{ax}\PY{o}{.}\PY{n}{legend}\PY{p}{(}\PY{n}{loc}\PY{o}{=}\PY{l+m+mi}{2}\PY{p}{)} \PY{c+c1}{\PYZsh{} upper left corner}
\PY{n}{ax}\PY{o}{.}\PY{n}{set\PYZus{}xlabel}\PY{p}{(}\PY{l+s+sa}{r}\PY{l+s+s1}{\PYZsq{}}\PY{l+s+s1}{\PYZdl{}}\PY{l+s+s1}{\PYZbs{}}\PY{l+s+s1}{alpha\PYZdl{}}\PY{l+s+s1}{\PYZsq{}}\PY{p}{)}
\PY{n}{ax}\PY{o}{.}\PY{n}{set\PYZus{}ylabel}\PY{p}{(}\PY{l+s+sa}{r}\PY{l+s+s1}{\PYZsq{}}\PY{l+s+s1}{\PYZdl{}y\PYZdl{}}\PY{l+s+s1}{\PYZsq{}}\PY{p}{)}
\PY{n}{ax}\PY{o}{.}\PY{n}{set\PYZus{}title}\PY{p}{(}\PY{l+s+s1}{\PYZsq{}}\PY{l+s+s1}{title}\PY{l+s+s1}{\PYZsq{}}\PY{p}{)}\PY{p}{;}
\end{Verbatim}
\end{tcolorbox}

    \begin{center}
    \adjustimage{max size={0.9\linewidth}{0.9\paperheight}}{Lecture-4-Matplotlib_files/Lecture-4-Matplotlib_67_0.png}
    \end{center}
    { \hspace*{\fill} \\}
    
    \begin{tcolorbox}[breakable, size=fbox, boxrule=1pt, pad at break*=1mm,colback=cellbackground, colframe=cellborder]
\prompt{In}{incolor}{31}{\boxspacing}
\begin{Verbatim}[commandchars=\\\{\}]
\PY{c+c1}{\PYZsh{} restore}
\PY{n}{matplotlib}\PY{o}{.}\PY{n}{rcParams}\PY{o}{.}\PY{n}{update}\PY{p}{(}\PY{p}{\PYZob{}}\PY{l+s+s1}{\PYZsq{}}\PY{l+s+s1}{font.size}\PY{l+s+s1}{\PYZsq{}}\PY{p}{:} \PY{l+m+mi}{12}\PY{p}{,} \PY{l+s+s1}{\PYZsq{}}\PY{l+s+s1}{font.family}\PY{l+s+s1}{\PYZsq{}}\PY{p}{:} \PY{l+s+s1}{\PYZsq{}}\PY{l+s+s1}{sans}\PY{l+s+s1}{\PYZsq{}}\PY{p}{,} \PY{l+s+s1}{\PYZsq{}}\PY{l+s+s1}{text.usetex}\PY{l+s+s1}{\PYZsq{}}\PY{p}{:} \PY{k+kc}{False}\PY{p}{\PYZcb{}}\PY{p}{)}
\end{Verbatim}
\end{tcolorbox}

    \hypertarget{setting-colors-linewidths-linetypes}{%
\subsubsection{Setting colors, linewidths,
linetypes}\label{setting-colors-linewidths-linetypes}}

    \hypertarget{colors}{%
\paragraph{Colors}\label{colors}}

    With matplotlib, we can define the colors of lines and other graphical
elements in a number of ways. First of all, we can use the MATLAB-like
syntax where \texttt{\textquotesingle{}b\textquotesingle{}} means blue,
\texttt{\textquotesingle{}g\textquotesingle{}} means green, etc. The
MATLAB API for selecting line styles are also supported: where, for
example, `b.-' means a blue line with dots:

    \begin{tcolorbox}[breakable, size=fbox, boxrule=1pt, pad at break*=1mm,colback=cellbackground, colframe=cellborder]
\prompt{In}{incolor}{32}{\boxspacing}
\begin{Verbatim}[commandchars=\\\{\}]
\PY{c+c1}{\PYZsh{} MATLAB style line color and style }
\PY{n}{ax}\PY{o}{.}\PY{n}{plot}\PY{p}{(}\PY{n}{x}\PY{p}{,} \PY{n}{x}\PY{o}{*}\PY{o}{*}\PY{l+m+mi}{2}\PY{p}{,} \PY{l+s+s1}{\PYZsq{}}\PY{l+s+s1}{b.\PYZhy{}}\PY{l+s+s1}{\PYZsq{}}\PY{p}{)} \PY{c+c1}{\PYZsh{} blue line with dots}
\PY{n}{ax}\PY{o}{.}\PY{n}{plot}\PY{p}{(}\PY{n}{x}\PY{p}{,} \PY{n}{x}\PY{o}{*}\PY{o}{*}\PY{l+m+mi}{3}\PY{p}{,} \PY{l+s+s1}{\PYZsq{}}\PY{l+s+s1}{g\PYZhy{}\PYZhy{}}\PY{l+s+s1}{\PYZsq{}}\PY{p}{)} \PY{c+c1}{\PYZsh{} green dashed line}
\end{Verbatim}
\end{tcolorbox}

            \begin{tcolorbox}[breakable, size=fbox, boxrule=.5pt, pad at break*=1mm, opacityfill=0]
\prompt{Out}{outcolor}{32}{\boxspacing}
\begin{Verbatim}[commandchars=\\\{\}]
[<matplotlib.lines.Line2D at 0x1cadb29edc0>]
\end{Verbatim}
\end{tcolorbox}
        
    We can also define colors by their names or RGB hex codes and optionally
provide an alpha value using the \texttt{color} and \texttt{alpha}
keyword arguments:

    \begin{tcolorbox}[breakable, size=fbox, boxrule=1pt, pad at break*=1mm,colback=cellbackground, colframe=cellborder]
\prompt{In}{incolor}{33}{\boxspacing}
\begin{Verbatim}[commandchars=\\\{\}]
\PY{n}{fig}\PY{p}{,} \PY{n}{ax} \PY{o}{=} \PY{n}{plt}\PY{o}{.}\PY{n}{subplots}\PY{p}{(}\PY{p}{)}

\PY{n}{ax}\PY{o}{.}\PY{n}{plot}\PY{p}{(}\PY{n}{x}\PY{p}{,} \PY{n}{x}\PY{o}{+}\PY{l+m+mi}{1}\PY{p}{,} \PY{n}{color}\PY{o}{=}\PY{l+s+s2}{\PYZdq{}}\PY{l+s+s2}{red}\PY{l+s+s2}{\PYZdq{}}\PY{p}{,} \PY{n}{alpha}\PY{o}{=}\PY{l+m+mf}{0.5}\PY{p}{)} \PY{c+c1}{\PYZsh{} half\PYZhy{}transparant red}
\PY{n}{ax}\PY{o}{.}\PY{n}{plot}\PY{p}{(}\PY{n}{x}\PY{p}{,} \PY{n}{x}\PY{o}{+}\PY{l+m+mi}{2}\PY{p}{,} \PY{n}{color}\PY{o}{=}\PY{l+s+s2}{\PYZdq{}}\PY{l+s+s2}{\PYZsh{}1155dd}\PY{l+s+s2}{\PYZdq{}}\PY{p}{)}        \PY{c+c1}{\PYZsh{} RGB hex code for a bluish color}
\PY{n}{ax}\PY{o}{.}\PY{n}{plot}\PY{p}{(}\PY{n}{x}\PY{p}{,} \PY{n}{x}\PY{o}{+}\PY{l+m+mi}{3}\PY{p}{,} \PY{n}{color}\PY{o}{=}\PY{l+s+s2}{\PYZdq{}}\PY{l+s+s2}{\PYZsh{}15cc55}\PY{l+s+s2}{\PYZdq{}}\PY{p}{)}        \PY{c+c1}{\PYZsh{} RGB hex code for a greenish color}
\end{Verbatim}
\end{tcolorbox}

            \begin{tcolorbox}[breakable, size=fbox, boxrule=.5pt, pad at break*=1mm, opacityfill=0]
\prompt{Out}{outcolor}{33}{\boxspacing}
\begin{Verbatim}[commandchars=\\\{\}]
[<matplotlib.lines.Line2D at 0x1cadbb23c10>]
\end{Verbatim}
\end{tcolorbox}
        
    \begin{center}
    \adjustimage{max size={0.9\linewidth}{0.9\paperheight}}{Lecture-4-Matplotlib_files/Lecture-4-Matplotlib_74_1.png}
    \end{center}
    { \hspace*{\fill} \\}
    
    \hypertarget{line-and-marker-styles}{%
\paragraph{Line and marker styles}\label{line-and-marker-styles}}

    To change the line width, we can use the \texttt{linewidth} or
\texttt{lw} keyword argument. The line style can be selected using the
\texttt{linestyle} or \texttt{ls} keyword arguments:

    \begin{tcolorbox}[breakable, size=fbox, boxrule=1pt, pad at break*=1mm,colback=cellbackground, colframe=cellborder]
\prompt{In}{incolor}{34}{\boxspacing}
\begin{Verbatim}[commandchars=\\\{\}]
\PY{n}{fig}\PY{p}{,} \PY{n}{ax} \PY{o}{=} \PY{n}{plt}\PY{o}{.}\PY{n}{subplots}\PY{p}{(}\PY{n}{figsize}\PY{o}{=}\PY{p}{(}\PY{l+m+mi}{12}\PY{p}{,}\PY{l+m+mi}{6}\PY{p}{)}\PY{p}{)}

\PY{n}{ax}\PY{o}{.}\PY{n}{plot}\PY{p}{(}\PY{n}{x}\PY{p}{,} \PY{n}{x}\PY{o}{+}\PY{l+m+mi}{1}\PY{p}{,} \PY{n}{color}\PY{o}{=}\PY{l+s+s2}{\PYZdq{}}\PY{l+s+s2}{blue}\PY{l+s+s2}{\PYZdq{}}\PY{p}{,} \PY{n}{linewidth}\PY{o}{=}\PY{l+m+mf}{0.25}\PY{p}{)}
\PY{n}{ax}\PY{o}{.}\PY{n}{plot}\PY{p}{(}\PY{n}{x}\PY{p}{,} \PY{n}{x}\PY{o}{+}\PY{l+m+mi}{2}\PY{p}{,} \PY{n}{color}\PY{o}{=}\PY{l+s+s2}{\PYZdq{}}\PY{l+s+s2}{blue}\PY{l+s+s2}{\PYZdq{}}\PY{p}{,} \PY{n}{linewidth}\PY{o}{=}\PY{l+m+mf}{0.50}\PY{p}{)}
\PY{n}{ax}\PY{o}{.}\PY{n}{plot}\PY{p}{(}\PY{n}{x}\PY{p}{,} \PY{n}{x}\PY{o}{+}\PY{l+m+mi}{3}\PY{p}{,} \PY{n}{color}\PY{o}{=}\PY{l+s+s2}{\PYZdq{}}\PY{l+s+s2}{blue}\PY{l+s+s2}{\PYZdq{}}\PY{p}{,} \PY{n}{linewidth}\PY{o}{=}\PY{l+m+mf}{1.00}\PY{p}{)}
\PY{n}{ax}\PY{o}{.}\PY{n}{plot}\PY{p}{(}\PY{n}{x}\PY{p}{,} \PY{n}{x}\PY{o}{+}\PY{l+m+mi}{4}\PY{p}{,} \PY{n}{color}\PY{o}{=}\PY{l+s+s2}{\PYZdq{}}\PY{l+s+s2}{blue}\PY{l+s+s2}{\PYZdq{}}\PY{p}{,} \PY{n}{linewidth}\PY{o}{=}\PY{l+m+mf}{2.00}\PY{p}{)}

\PY{c+c1}{\PYZsh{} possible linestype options ‘\PYZhy{}‘, ‘\PYZhy{}\PYZhy{}’, ‘\PYZhy{}.’, ‘:’, ‘steps’}
\PY{n}{ax}\PY{o}{.}\PY{n}{plot}\PY{p}{(}\PY{n}{x}\PY{p}{,} \PY{n}{x}\PY{o}{+}\PY{l+m+mi}{5}\PY{p}{,} \PY{n}{color}\PY{o}{=}\PY{l+s+s2}{\PYZdq{}}\PY{l+s+s2}{red}\PY{l+s+s2}{\PYZdq{}}\PY{p}{,} \PY{n}{lw}\PY{o}{=}\PY{l+m+mi}{2}\PY{p}{,} \PY{n}{linestyle}\PY{o}{=}\PY{l+s+s1}{\PYZsq{}}\PY{l+s+s1}{\PYZhy{}}\PY{l+s+s1}{\PYZsq{}}\PY{p}{)}
\PY{n}{ax}\PY{o}{.}\PY{n}{plot}\PY{p}{(}\PY{n}{x}\PY{p}{,} \PY{n}{x}\PY{o}{+}\PY{l+m+mi}{6}\PY{p}{,} \PY{n}{color}\PY{o}{=}\PY{l+s+s2}{\PYZdq{}}\PY{l+s+s2}{red}\PY{l+s+s2}{\PYZdq{}}\PY{p}{,} \PY{n}{lw}\PY{o}{=}\PY{l+m+mi}{2}\PY{p}{,} \PY{n}{ls}\PY{o}{=}\PY{l+s+s1}{\PYZsq{}}\PY{l+s+s1}{\PYZhy{}.}\PY{l+s+s1}{\PYZsq{}}\PY{p}{)}
\PY{n}{ax}\PY{o}{.}\PY{n}{plot}\PY{p}{(}\PY{n}{x}\PY{p}{,} \PY{n}{x}\PY{o}{+}\PY{l+m+mi}{7}\PY{p}{,} \PY{n}{color}\PY{o}{=}\PY{l+s+s2}{\PYZdq{}}\PY{l+s+s2}{red}\PY{l+s+s2}{\PYZdq{}}\PY{p}{,} \PY{n}{lw}\PY{o}{=}\PY{l+m+mi}{2}\PY{p}{,} \PY{n}{ls}\PY{o}{=}\PY{l+s+s1}{\PYZsq{}}\PY{l+s+s1}{:}\PY{l+s+s1}{\PYZsq{}}\PY{p}{)}

\PY{c+c1}{\PYZsh{} custom dash}
\PY{n}{line}\PY{p}{,} \PY{o}{=} \PY{n}{ax}\PY{o}{.}\PY{n}{plot}\PY{p}{(}\PY{n}{x}\PY{p}{,} \PY{n}{x}\PY{o}{+}\PY{l+m+mi}{8}\PY{p}{,} \PY{n}{color}\PY{o}{=}\PY{l+s+s2}{\PYZdq{}}\PY{l+s+s2}{black}\PY{l+s+s2}{\PYZdq{}}\PY{p}{,} \PY{n}{lw}\PY{o}{=}\PY{l+m+mf}{1.50}\PY{p}{)}
\PY{n}{line}\PY{o}{.}\PY{n}{set\PYZus{}dashes}\PY{p}{(}\PY{p}{[}\PY{l+m+mi}{5}\PY{p}{,} \PY{l+m+mi}{10}\PY{p}{,} \PY{l+m+mi}{15}\PY{p}{,} \PY{l+m+mi}{10}\PY{p}{]}\PY{p}{)} \PY{c+c1}{\PYZsh{} format: line length, space length, ...}

\PY{c+c1}{\PYZsh{} possible marker symbols: marker = \PYZsq{}+\PYZsq{}, \PYZsq{}o\PYZsq{}, \PYZsq{}*\PYZsq{}, \PYZsq{}s\PYZsq{}, \PYZsq{},\PYZsq{}, \PYZsq{}.\PYZsq{}, \PYZsq{}1\PYZsq{}, \PYZsq{}2\PYZsq{}, \PYZsq{}3\PYZsq{}, \PYZsq{}4\PYZsq{}, ...}
\PY{n}{ax}\PY{o}{.}\PY{n}{plot}\PY{p}{(}\PY{n}{x}\PY{p}{,} \PY{n}{x}\PY{o}{+} \PY{l+m+mi}{9}\PY{p}{,} \PY{n}{color}\PY{o}{=}\PY{l+s+s2}{\PYZdq{}}\PY{l+s+s2}{green}\PY{l+s+s2}{\PYZdq{}}\PY{p}{,} \PY{n}{lw}\PY{o}{=}\PY{l+m+mi}{2}\PY{p}{,} \PY{n}{ls}\PY{o}{=}\PY{l+s+s1}{\PYZsq{}}\PY{l+s+s1}{\PYZhy{}\PYZhy{}}\PY{l+s+s1}{\PYZsq{}}\PY{p}{,} \PY{n}{marker}\PY{o}{=}\PY{l+s+s1}{\PYZsq{}}\PY{l+s+s1}{+}\PY{l+s+s1}{\PYZsq{}}\PY{p}{)}
\PY{n}{ax}\PY{o}{.}\PY{n}{plot}\PY{p}{(}\PY{n}{x}\PY{p}{,} \PY{n}{x}\PY{o}{+}\PY{l+m+mi}{10}\PY{p}{,} \PY{n}{color}\PY{o}{=}\PY{l+s+s2}{\PYZdq{}}\PY{l+s+s2}{green}\PY{l+s+s2}{\PYZdq{}}\PY{p}{,} \PY{n}{lw}\PY{o}{=}\PY{l+m+mi}{2}\PY{p}{,} \PY{n}{ls}\PY{o}{=}\PY{l+s+s1}{\PYZsq{}}\PY{l+s+s1}{\PYZhy{}\PYZhy{}}\PY{l+s+s1}{\PYZsq{}}\PY{p}{,} \PY{n}{marker}\PY{o}{=}\PY{l+s+s1}{\PYZsq{}}\PY{l+s+s1}{o}\PY{l+s+s1}{\PYZsq{}}\PY{p}{)}
\PY{n}{ax}\PY{o}{.}\PY{n}{plot}\PY{p}{(}\PY{n}{x}\PY{p}{,} \PY{n}{x}\PY{o}{+}\PY{l+m+mi}{11}\PY{p}{,} \PY{n}{color}\PY{o}{=}\PY{l+s+s2}{\PYZdq{}}\PY{l+s+s2}{green}\PY{l+s+s2}{\PYZdq{}}\PY{p}{,} \PY{n}{lw}\PY{o}{=}\PY{l+m+mi}{2}\PY{p}{,} \PY{n}{ls}\PY{o}{=}\PY{l+s+s1}{\PYZsq{}}\PY{l+s+s1}{\PYZhy{}\PYZhy{}}\PY{l+s+s1}{\PYZsq{}}\PY{p}{,} \PY{n}{marker}\PY{o}{=}\PY{l+s+s1}{\PYZsq{}}\PY{l+s+s1}{s}\PY{l+s+s1}{\PYZsq{}}\PY{p}{)}
\PY{n}{ax}\PY{o}{.}\PY{n}{plot}\PY{p}{(}\PY{n}{x}\PY{p}{,} \PY{n}{x}\PY{o}{+}\PY{l+m+mi}{12}\PY{p}{,} \PY{n}{color}\PY{o}{=}\PY{l+s+s2}{\PYZdq{}}\PY{l+s+s2}{green}\PY{l+s+s2}{\PYZdq{}}\PY{p}{,} \PY{n}{lw}\PY{o}{=}\PY{l+m+mi}{2}\PY{p}{,} \PY{n}{ls}\PY{o}{=}\PY{l+s+s1}{\PYZsq{}}\PY{l+s+s1}{\PYZhy{}\PYZhy{}}\PY{l+s+s1}{\PYZsq{}}\PY{p}{,} \PY{n}{marker}\PY{o}{=}\PY{l+s+s1}{\PYZsq{}}\PY{l+s+s1}{1}\PY{l+s+s1}{\PYZsq{}}\PY{p}{)}

\PY{c+c1}{\PYZsh{} marker size and color}
\PY{n}{ax}\PY{o}{.}\PY{n}{plot}\PY{p}{(}\PY{n}{x}\PY{p}{,} \PY{n}{x}\PY{o}{+}\PY{l+m+mi}{13}\PY{p}{,} \PY{n}{color}\PY{o}{=}\PY{l+s+s2}{\PYZdq{}}\PY{l+s+s2}{purple}\PY{l+s+s2}{\PYZdq{}}\PY{p}{,} \PY{n}{lw}\PY{o}{=}\PY{l+m+mi}{1}\PY{p}{,} \PY{n}{ls}\PY{o}{=}\PY{l+s+s1}{\PYZsq{}}\PY{l+s+s1}{\PYZhy{}}\PY{l+s+s1}{\PYZsq{}}\PY{p}{,} \PY{n}{marker}\PY{o}{=}\PY{l+s+s1}{\PYZsq{}}\PY{l+s+s1}{o}\PY{l+s+s1}{\PYZsq{}}\PY{p}{,} \PY{n}{markersize}\PY{o}{=}\PY{l+m+mi}{2}\PY{p}{)}
\PY{n}{ax}\PY{o}{.}\PY{n}{plot}\PY{p}{(}\PY{n}{x}\PY{p}{,} \PY{n}{x}\PY{o}{+}\PY{l+m+mi}{14}\PY{p}{,} \PY{n}{color}\PY{o}{=}\PY{l+s+s2}{\PYZdq{}}\PY{l+s+s2}{purple}\PY{l+s+s2}{\PYZdq{}}\PY{p}{,} \PY{n}{lw}\PY{o}{=}\PY{l+m+mi}{1}\PY{p}{,} \PY{n}{ls}\PY{o}{=}\PY{l+s+s1}{\PYZsq{}}\PY{l+s+s1}{\PYZhy{}}\PY{l+s+s1}{\PYZsq{}}\PY{p}{,} \PY{n}{marker}\PY{o}{=}\PY{l+s+s1}{\PYZsq{}}\PY{l+s+s1}{o}\PY{l+s+s1}{\PYZsq{}}\PY{p}{,} \PY{n}{markersize}\PY{o}{=}\PY{l+m+mi}{4}\PY{p}{)}
\PY{n}{ax}\PY{o}{.}\PY{n}{plot}\PY{p}{(}\PY{n}{x}\PY{p}{,} \PY{n}{x}\PY{o}{+}\PY{l+m+mi}{15}\PY{p}{,} \PY{n}{color}\PY{o}{=}\PY{l+s+s2}{\PYZdq{}}\PY{l+s+s2}{purple}\PY{l+s+s2}{\PYZdq{}}\PY{p}{,} \PY{n}{lw}\PY{o}{=}\PY{l+m+mi}{1}\PY{p}{,} \PY{n}{ls}\PY{o}{=}\PY{l+s+s1}{\PYZsq{}}\PY{l+s+s1}{\PYZhy{}}\PY{l+s+s1}{\PYZsq{}}\PY{p}{,} \PY{n}{marker}\PY{o}{=}\PY{l+s+s1}{\PYZsq{}}\PY{l+s+s1}{o}\PY{l+s+s1}{\PYZsq{}}\PY{p}{,} \PY{n}{markersize}\PY{o}{=}\PY{l+m+mi}{8}\PY{p}{,} \PY{n}{markerfacecolor}\PY{o}{=}\PY{l+s+s2}{\PYZdq{}}\PY{l+s+s2}{red}\PY{l+s+s2}{\PYZdq{}}\PY{p}{)}
\PY{n}{ax}\PY{o}{.}\PY{n}{plot}\PY{p}{(}\PY{n}{x}\PY{p}{,} \PY{n}{x}\PY{o}{+}\PY{l+m+mi}{16}\PY{p}{,} \PY{n}{color}\PY{o}{=}\PY{l+s+s2}{\PYZdq{}}\PY{l+s+s2}{purple}\PY{l+s+s2}{\PYZdq{}}\PY{p}{,} \PY{n}{lw}\PY{o}{=}\PY{l+m+mi}{1}\PY{p}{,} \PY{n}{ls}\PY{o}{=}\PY{l+s+s1}{\PYZsq{}}\PY{l+s+s1}{\PYZhy{}}\PY{l+s+s1}{\PYZsq{}}\PY{p}{,} \PY{n}{marker}\PY{o}{=}\PY{l+s+s1}{\PYZsq{}}\PY{l+s+s1}{s}\PY{l+s+s1}{\PYZsq{}}\PY{p}{,} \PY{n}{markersize}\PY{o}{=}\PY{l+m+mi}{8}\PY{p}{,} 
        \PY{n}{markerfacecolor}\PY{o}{=}\PY{l+s+s2}{\PYZdq{}}\PY{l+s+s2}{yellow}\PY{l+s+s2}{\PYZdq{}}\PY{p}{,} \PY{n}{markeredgewidth}\PY{o}{=}\PY{l+m+mi}{2}\PY{p}{,} \PY{n}{markeredgecolor}\PY{o}{=}\PY{l+s+s2}{\PYZdq{}}\PY{l+s+s2}{blue}\PY{l+s+s2}{\PYZdq{}}\PY{p}{)}\PY{p}{;}
\end{Verbatim}
\end{tcolorbox}

    \begin{center}
    \adjustimage{max size={0.9\linewidth}{0.9\paperheight}}{Lecture-4-Matplotlib_files/Lecture-4-Matplotlib_77_0.png}
    \end{center}
    { \hspace*{\fill} \\}
    
    \hypertarget{control-over-axis-appearance}{%
\subsubsection{Control over axis
appearance}\label{control-over-axis-appearance}}

    The appearance of the axes is an important aspect of a figure that we
often need to modify to make a publication quality graphics. We need to
be able to control where the ticks and labels are placed, modify the
font size and possibly the labels used on the axes. In this section we
will look at controling those properties in a matplotlib figure.

    \hypertarget{plot-range}{%
\paragraph{Plot range}\label{plot-range}}

    The first thing we might want to configure is the ranges of the axes. We
can do this using the \texttt{set\_ylim} and \texttt{set\_xlim} methods
in the axis object, or
\texttt{axis(\textquotesingle{}tight\textquotesingle{})} for
automatrically getting ``tightly fitted'' axes ranges:

    \begin{tcolorbox}[breakable, size=fbox, boxrule=1pt, pad at break*=1mm,colback=cellbackground, colframe=cellborder]
\prompt{In}{incolor}{35}{\boxspacing}
\begin{Verbatim}[commandchars=\\\{\}]
\PY{n}{fig}\PY{p}{,} \PY{n}{axes} \PY{o}{=} \PY{n}{plt}\PY{o}{.}\PY{n}{subplots}\PY{p}{(}\PY{l+m+mi}{1}\PY{p}{,} \PY{l+m+mi}{3}\PY{p}{,} \PY{n}{figsize}\PY{o}{=}\PY{p}{(}\PY{l+m+mi}{12}\PY{p}{,} \PY{l+m+mi}{4}\PY{p}{)}\PY{p}{)}

\PY{n}{axes}\PY{p}{[}\PY{l+m+mi}{0}\PY{p}{]}\PY{o}{.}\PY{n}{plot}\PY{p}{(}\PY{n}{x}\PY{p}{,} \PY{n}{x}\PY{o}{*}\PY{o}{*}\PY{l+m+mi}{2}\PY{p}{,} \PY{n}{x}\PY{p}{,} \PY{n}{x}\PY{o}{*}\PY{o}{*}\PY{l+m+mi}{3}\PY{p}{)}
\PY{n}{axes}\PY{p}{[}\PY{l+m+mi}{0}\PY{p}{]}\PY{o}{.}\PY{n}{set\PYZus{}title}\PY{p}{(}\PY{l+s+s2}{\PYZdq{}}\PY{l+s+s2}{default axes ranges}\PY{l+s+s2}{\PYZdq{}}\PY{p}{)}

\PY{n}{axes}\PY{p}{[}\PY{l+m+mi}{1}\PY{p}{]}\PY{o}{.}\PY{n}{plot}\PY{p}{(}\PY{n}{x}\PY{p}{,} \PY{n}{x}\PY{o}{*}\PY{o}{*}\PY{l+m+mi}{2}\PY{p}{,} \PY{n}{x}\PY{p}{,} \PY{n}{x}\PY{o}{*}\PY{o}{*}\PY{l+m+mi}{3}\PY{p}{)}
\PY{n}{axes}\PY{p}{[}\PY{l+m+mi}{1}\PY{p}{]}\PY{o}{.}\PY{n}{axis}\PY{p}{(}\PY{l+s+s1}{\PYZsq{}}\PY{l+s+s1}{tight}\PY{l+s+s1}{\PYZsq{}}\PY{p}{)}
\PY{n}{axes}\PY{p}{[}\PY{l+m+mi}{1}\PY{p}{]}\PY{o}{.}\PY{n}{set\PYZus{}title}\PY{p}{(}\PY{l+s+s2}{\PYZdq{}}\PY{l+s+s2}{tight axes}\PY{l+s+s2}{\PYZdq{}}\PY{p}{)}

\PY{n}{axes}\PY{p}{[}\PY{l+m+mi}{2}\PY{p}{]}\PY{o}{.}\PY{n}{plot}\PY{p}{(}\PY{n}{x}\PY{p}{,} \PY{n}{x}\PY{o}{*}\PY{o}{*}\PY{l+m+mi}{2}\PY{p}{,} \PY{n}{x}\PY{p}{,} \PY{n}{x}\PY{o}{*}\PY{o}{*}\PY{l+m+mi}{3}\PY{p}{)}
\PY{n}{axes}\PY{p}{[}\PY{l+m+mi}{2}\PY{p}{]}\PY{o}{.}\PY{n}{set\PYZus{}ylim}\PY{p}{(}\PY{p}{[}\PY{l+m+mi}{0}\PY{p}{,} \PY{l+m+mi}{60}\PY{p}{]}\PY{p}{)}
\PY{n}{axes}\PY{p}{[}\PY{l+m+mi}{2}\PY{p}{]}\PY{o}{.}\PY{n}{set\PYZus{}xlim}\PY{p}{(}\PY{p}{[}\PY{l+m+mi}{2}\PY{p}{,} \PY{l+m+mi}{5}\PY{p}{]}\PY{p}{)}
\PY{n}{axes}\PY{p}{[}\PY{l+m+mi}{2}\PY{p}{]}\PY{o}{.}\PY{n}{set\PYZus{}title}\PY{p}{(}\PY{l+s+s2}{\PYZdq{}}\PY{l+s+s2}{custom axes range}\PY{l+s+s2}{\PYZdq{}}\PY{p}{)}\PY{p}{;}
\end{Verbatim}
\end{tcolorbox}

    \begin{center}
    \adjustimage{max size={0.9\linewidth}{0.9\paperheight}}{Lecture-4-Matplotlib_files/Lecture-4-Matplotlib_82_0.png}
    \end{center}
    { \hspace*{\fill} \\}
    
    \hypertarget{logarithmic-scale}{%
\paragraph{Logarithmic scale}\label{logarithmic-scale}}

    It is also possible to set a logarithmic scale for one or both axes.
This functionality is in fact only one application of a more general
transformation system in Matplotlib. Each of the axes' scales are set
seperately using \texttt{set\_xscale} and \texttt{set\_yscale} methods
which accept one parameter (with the value ``log'' in this case):

    \begin{tcolorbox}[breakable, size=fbox, boxrule=1pt, pad at break*=1mm,colback=cellbackground, colframe=cellborder]
\prompt{In}{incolor}{36}{\boxspacing}
\begin{Verbatim}[commandchars=\\\{\}]
\PY{n}{fig}\PY{p}{,} \PY{n}{axes} \PY{o}{=} \PY{n}{plt}\PY{o}{.}\PY{n}{subplots}\PY{p}{(}\PY{l+m+mi}{1}\PY{p}{,} \PY{l+m+mi}{2}\PY{p}{,} \PY{n}{figsize}\PY{o}{=}\PY{p}{(}\PY{l+m+mi}{10}\PY{p}{,}\PY{l+m+mi}{4}\PY{p}{)}\PY{p}{)}
      
\PY{n}{axes}\PY{p}{[}\PY{l+m+mi}{0}\PY{p}{]}\PY{o}{.}\PY{n}{plot}\PY{p}{(}\PY{n}{x}\PY{p}{,} \PY{n}{x}\PY{o}{*}\PY{o}{*}\PY{l+m+mi}{2}\PY{p}{,} \PY{n}{x}\PY{p}{,} \PY{n}{np}\PY{o}{.}\PY{n}{exp}\PY{p}{(}\PY{n}{x}\PY{p}{)}\PY{p}{)}
\PY{n}{axes}\PY{p}{[}\PY{l+m+mi}{0}\PY{p}{]}\PY{o}{.}\PY{n}{set\PYZus{}title}\PY{p}{(}\PY{l+s+s2}{\PYZdq{}}\PY{l+s+s2}{Normal scale}\PY{l+s+s2}{\PYZdq{}}\PY{p}{)}

\PY{n}{axes}\PY{p}{[}\PY{l+m+mi}{1}\PY{p}{]}\PY{o}{.}\PY{n}{plot}\PY{p}{(}\PY{n}{x}\PY{p}{,} \PY{n}{x}\PY{o}{*}\PY{o}{*}\PY{l+m+mi}{2}\PY{p}{,} \PY{n}{x}\PY{p}{,} \PY{n}{np}\PY{o}{.}\PY{n}{exp}\PY{p}{(}\PY{n}{x}\PY{p}{)}\PY{p}{)}
\PY{n}{axes}\PY{p}{[}\PY{l+m+mi}{1}\PY{p}{]}\PY{o}{.}\PY{n}{set\PYZus{}yscale}\PY{p}{(}\PY{l+s+s2}{\PYZdq{}}\PY{l+s+s2}{log}\PY{l+s+s2}{\PYZdq{}}\PY{p}{)}
\PY{n}{axes}\PY{p}{[}\PY{l+m+mi}{1}\PY{p}{]}\PY{o}{.}\PY{n}{set\PYZus{}title}\PY{p}{(}\PY{l+s+s2}{\PYZdq{}}\PY{l+s+s2}{Logarithmic scale (y)}\PY{l+s+s2}{\PYZdq{}}\PY{p}{)}\PY{p}{;}
\end{Verbatim}
\end{tcolorbox}

    \begin{center}
    \adjustimage{max size={0.9\linewidth}{0.9\paperheight}}{Lecture-4-Matplotlib_files/Lecture-4-Matplotlib_85_0.png}
    \end{center}
    { \hspace*{\fill} \\}
    
    \hypertarget{placement-of-ticks-and-custom-tick-labels}{%
\subsubsection{Placement of ticks and custom tick
labels}\label{placement-of-ticks-and-custom-tick-labels}}

    We can explicitly determine where we want the axis ticks with
\texttt{set\_xticks} and \texttt{set\_yticks}, which both take a list of
values for where on the axis the ticks are to be placed. We can also use
the \texttt{set\_xticklabels} and \texttt{set\_yticklabels} methods to
provide a list of custom text labels for each tick location:

    \begin{tcolorbox}[breakable, size=fbox, boxrule=1pt, pad at break*=1mm,colback=cellbackground, colframe=cellborder]
\prompt{In}{incolor}{37}{\boxspacing}
\begin{Verbatim}[commandchars=\\\{\}]
\PY{n}{fig}\PY{p}{,} \PY{n}{ax} \PY{o}{=} \PY{n}{plt}\PY{o}{.}\PY{n}{subplots}\PY{p}{(}\PY{n}{figsize}\PY{o}{=}\PY{p}{(}\PY{l+m+mi}{10}\PY{p}{,} \PY{l+m+mi}{4}\PY{p}{)}\PY{p}{)}

\PY{n}{ax}\PY{o}{.}\PY{n}{plot}\PY{p}{(}\PY{n}{x}\PY{p}{,} \PY{n}{x}\PY{o}{*}\PY{o}{*}\PY{l+m+mi}{2}\PY{p}{,} \PY{n}{x}\PY{p}{,} \PY{n}{x}\PY{o}{*}\PY{o}{*}\PY{l+m+mi}{3}\PY{p}{,} \PY{n}{lw}\PY{o}{=}\PY{l+m+mi}{2}\PY{p}{)}

\PY{n}{ax}\PY{o}{.}\PY{n}{set\PYZus{}xticks}\PY{p}{(}\PY{p}{[}\PY{l+m+mi}{1}\PY{p}{,} \PY{l+m+mi}{2}\PY{p}{,} \PY{l+m+mi}{3}\PY{p}{,} \PY{l+m+mi}{4}\PY{p}{,} \PY{l+m+mi}{5}\PY{p}{]}\PY{p}{)}
\PY{n}{ax}\PY{o}{.}\PY{n}{set\PYZus{}xticklabels}\PY{p}{(}\PY{p}{[}\PY{l+s+sa}{r}\PY{l+s+s1}{\PYZsq{}}\PY{l+s+s1}{\PYZdl{}}\PY{l+s+s1}{\PYZbs{}}\PY{l+s+s1}{alpha\PYZdl{}}\PY{l+s+s1}{\PYZsq{}}\PY{p}{,} \PY{l+s+sa}{r}\PY{l+s+s1}{\PYZsq{}}\PY{l+s+s1}{\PYZdl{}}\PY{l+s+s1}{\PYZbs{}}\PY{l+s+s1}{beta\PYZdl{}}\PY{l+s+s1}{\PYZsq{}}\PY{p}{,} \PY{l+s+sa}{r}\PY{l+s+s1}{\PYZsq{}}\PY{l+s+s1}{\PYZdl{}}\PY{l+s+s1}{\PYZbs{}}\PY{l+s+s1}{gamma\PYZdl{}}\PY{l+s+s1}{\PYZsq{}}\PY{p}{,} \PY{l+s+sa}{r}\PY{l+s+s1}{\PYZsq{}}\PY{l+s+s1}{\PYZdl{}}\PY{l+s+s1}{\PYZbs{}}\PY{l+s+s1}{delta\PYZdl{}}\PY{l+s+s1}{\PYZsq{}}\PY{p}{,} \PY{l+s+sa}{r}\PY{l+s+s1}{\PYZsq{}}\PY{l+s+s1}{\PYZdl{}}\PY{l+s+s1}{\PYZbs{}}\PY{l+s+s1}{epsilon\PYZdl{}}\PY{l+s+s1}{\PYZsq{}}\PY{p}{]}\PY{p}{,} \PY{n}{fontsize}\PY{o}{=}\PY{l+m+mi}{18}\PY{p}{)}

\PY{n}{yticks} \PY{o}{=} \PY{p}{[}\PY{l+m+mi}{0}\PY{p}{,} \PY{l+m+mi}{50}\PY{p}{,} \PY{l+m+mi}{100}\PY{p}{,} \PY{l+m+mi}{150}\PY{p}{]}
\PY{n}{ax}\PY{o}{.}\PY{n}{set\PYZus{}yticks}\PY{p}{(}\PY{n}{yticks}\PY{p}{)}
\PY{n}{ax}\PY{o}{.}\PY{n}{set\PYZus{}yticklabels}\PY{p}{(}\PY{p}{[}\PY{l+s+s2}{\PYZdq{}}\PY{l+s+s2}{\PYZdl{}}\PY{l+s+si}{\PYZpc{}.1f}\PY{l+s+s2}{\PYZdl{}}\PY{l+s+s2}{\PYZdq{}} \PY{o}{\PYZpc{}} \PY{n}{y} \PY{k}{for} \PY{n}{y} \PY{o+ow}{in} \PY{n}{yticks}\PY{p}{]}\PY{p}{,} \PY{n}{fontsize}\PY{o}{=}\PY{l+m+mi}{18}\PY{p}{)}\PY{p}{;} \PY{c+c1}{\PYZsh{} use LaTeX formatted labels}
\end{Verbatim}
\end{tcolorbox}

    \begin{center}
    \adjustimage{max size={0.9\linewidth}{0.9\paperheight}}{Lecture-4-Matplotlib_files/Lecture-4-Matplotlib_88_0.png}
    \end{center}
    { \hspace*{\fill} \\}
    
    There are a number of more advanced methods for controlling major and
minor tick placement in matplotlib figures, such as automatic placement
according to different policies. See
http://matplotlib.org/api/ticker\_api.html for details.

    \hypertarget{scientific-notation}{%
\paragraph{Scientific notation}\label{scientific-notation}}

    With large numbers on axes, it is often better use scientific notation:

    \begin{tcolorbox}[breakable, size=fbox, boxrule=1pt, pad at break*=1mm,colback=cellbackground, colframe=cellborder]
\prompt{In}{incolor}{38}{\boxspacing}
\begin{Verbatim}[commandchars=\\\{\}]
\PY{n}{fig}\PY{p}{,} \PY{n}{ax} \PY{o}{=} \PY{n}{plt}\PY{o}{.}\PY{n}{subplots}\PY{p}{(}\PY{l+m+mi}{1}\PY{p}{,} \PY{l+m+mi}{1}\PY{p}{)}
      
\PY{n}{ax}\PY{o}{.}\PY{n}{plot}\PY{p}{(}\PY{n}{x}\PY{p}{,} \PY{n}{x}\PY{o}{*}\PY{o}{*}\PY{l+m+mi}{2}\PY{p}{,} \PY{n}{x}\PY{p}{,} \PY{n}{np}\PY{o}{.}\PY{n}{exp}\PY{p}{(}\PY{n}{x}\PY{p}{)}\PY{p}{)}
\PY{n}{ax}\PY{o}{.}\PY{n}{set\PYZus{}title}\PY{p}{(}\PY{l+s+s2}{\PYZdq{}}\PY{l+s+s2}{scientific notation}\PY{l+s+s2}{\PYZdq{}}\PY{p}{)}

\PY{n}{ax}\PY{o}{.}\PY{n}{set\PYZus{}yticks}\PY{p}{(}\PY{p}{[}\PY{l+m+mi}{0}\PY{p}{,} \PY{l+m+mi}{50}\PY{p}{,} \PY{l+m+mi}{100}\PY{p}{,} \PY{l+m+mi}{150}\PY{p}{]}\PY{p}{)}

\PY{k+kn}{from} \PY{n+nn}{matplotlib} \PY{k+kn}{import} \PY{n}{ticker}
\PY{n}{formatter} \PY{o}{=} \PY{n}{ticker}\PY{o}{.}\PY{n}{ScalarFormatter}\PY{p}{(}\PY{n}{useMathText}\PY{o}{=}\PY{k+kc}{True}\PY{p}{)}
\PY{n}{formatter}\PY{o}{.}\PY{n}{set\PYZus{}scientific}\PY{p}{(}\PY{k+kc}{True}\PY{p}{)} 
\PY{n}{formatter}\PY{o}{.}\PY{n}{set\PYZus{}powerlimits}\PY{p}{(}\PY{p}{(}\PY{o}{\PYZhy{}}\PY{l+m+mi}{1}\PY{p}{,}\PY{l+m+mi}{1}\PY{p}{)}\PY{p}{)} 
\PY{n}{ax}\PY{o}{.}\PY{n}{yaxis}\PY{o}{.}\PY{n}{set\PYZus{}major\PYZus{}formatter}\PY{p}{(}\PY{n}{formatter}\PY{p}{)} 
\end{Verbatim}
\end{tcolorbox}

    \begin{center}
    \adjustimage{max size={0.9\linewidth}{0.9\paperheight}}{Lecture-4-Matplotlib_files/Lecture-4-Matplotlib_92_0.png}
    \end{center}
    { \hspace*{\fill} \\}
    
    \hypertarget{axis-number-and-axis-label-spacing}{%
\subsubsection{Axis number and axis label
spacing}\label{axis-number-and-axis-label-spacing}}

    \begin{tcolorbox}[breakable, size=fbox, boxrule=1pt, pad at break*=1mm,colback=cellbackground, colframe=cellborder]
\prompt{In}{incolor}{39}{\boxspacing}
\begin{Verbatim}[commandchars=\\\{\}]
\PY{c+c1}{\PYZsh{} distance between x and y axis and the numbers on the axes}
\PY{n}{matplotlib}\PY{o}{.}\PY{n}{rcParams}\PY{p}{[}\PY{l+s+s1}{\PYZsq{}}\PY{l+s+s1}{xtick.major.pad}\PY{l+s+s1}{\PYZsq{}}\PY{p}{]} \PY{o}{=} \PY{l+m+mi}{5}
\PY{n}{matplotlib}\PY{o}{.}\PY{n}{rcParams}\PY{p}{[}\PY{l+s+s1}{\PYZsq{}}\PY{l+s+s1}{ytick.major.pad}\PY{l+s+s1}{\PYZsq{}}\PY{p}{]} \PY{o}{=} \PY{l+m+mi}{5}

\PY{n}{fig}\PY{p}{,} \PY{n}{ax} \PY{o}{=} \PY{n}{plt}\PY{o}{.}\PY{n}{subplots}\PY{p}{(}\PY{l+m+mi}{1}\PY{p}{,} \PY{l+m+mi}{1}\PY{p}{)}
      
\PY{n}{ax}\PY{o}{.}\PY{n}{plot}\PY{p}{(}\PY{n}{x}\PY{p}{,} \PY{n}{x}\PY{o}{*}\PY{o}{*}\PY{l+m+mi}{2}\PY{p}{,} \PY{n}{x}\PY{p}{,} \PY{n}{np}\PY{o}{.}\PY{n}{exp}\PY{p}{(}\PY{n}{x}\PY{p}{)}\PY{p}{)}
\PY{n}{ax}\PY{o}{.}\PY{n}{set\PYZus{}yticks}\PY{p}{(}\PY{p}{[}\PY{l+m+mi}{0}\PY{p}{,} \PY{l+m+mi}{50}\PY{p}{,} \PY{l+m+mi}{100}\PY{p}{,} \PY{l+m+mi}{150}\PY{p}{]}\PY{p}{)}

\PY{n}{ax}\PY{o}{.}\PY{n}{set\PYZus{}title}\PY{p}{(}\PY{l+s+s2}{\PYZdq{}}\PY{l+s+s2}{label and axis spacing}\PY{l+s+s2}{\PYZdq{}}\PY{p}{)}

\PY{c+c1}{\PYZsh{} padding between axis label and axis numbers}
\PY{n}{ax}\PY{o}{.}\PY{n}{xaxis}\PY{o}{.}\PY{n}{labelpad} \PY{o}{=} \PY{l+m+mi}{5}
\PY{n}{ax}\PY{o}{.}\PY{n}{yaxis}\PY{o}{.}\PY{n}{labelpad} \PY{o}{=} \PY{l+m+mi}{5}

\PY{n}{ax}\PY{o}{.}\PY{n}{set\PYZus{}xlabel}\PY{p}{(}\PY{l+s+s2}{\PYZdq{}}\PY{l+s+s2}{x}\PY{l+s+s2}{\PYZdq{}}\PY{p}{)}
\PY{n}{ax}\PY{o}{.}\PY{n}{set\PYZus{}ylabel}\PY{p}{(}\PY{l+s+s2}{\PYZdq{}}\PY{l+s+s2}{y}\PY{l+s+s2}{\PYZdq{}}\PY{p}{)}\PY{p}{;}
\end{Verbatim}
\end{tcolorbox}

    \begin{center}
    \adjustimage{max size={0.9\linewidth}{0.9\paperheight}}{Lecture-4-Matplotlib_files/Lecture-4-Matplotlib_94_0.png}
    \end{center}
    { \hspace*{\fill} \\}
    
    \begin{tcolorbox}[breakable, size=fbox, boxrule=1pt, pad at break*=1mm,colback=cellbackground, colframe=cellborder]
\prompt{In}{incolor}{40}{\boxspacing}
\begin{Verbatim}[commandchars=\\\{\}]
\PY{c+c1}{\PYZsh{} restore defaults}
\PY{n}{matplotlib}\PY{o}{.}\PY{n}{rcParams}\PY{p}{[}\PY{l+s+s1}{\PYZsq{}}\PY{l+s+s1}{xtick.major.pad}\PY{l+s+s1}{\PYZsq{}}\PY{p}{]} \PY{o}{=} \PY{l+m+mi}{3}
\PY{n}{matplotlib}\PY{o}{.}\PY{n}{rcParams}\PY{p}{[}\PY{l+s+s1}{\PYZsq{}}\PY{l+s+s1}{ytick.major.pad}\PY{l+s+s1}{\PYZsq{}}\PY{p}{]} \PY{o}{=} \PY{l+m+mi}{3}
\end{Verbatim}
\end{tcolorbox}

    \hypertarget{axis-position-adjustments}{%
\paragraph{Axis position adjustments}\label{axis-position-adjustments}}

    Unfortunately, when saving figures the labels are sometimes clipped, and
it can be necessary to adjust the positions of axes a little bit. This
can be done using \texttt{subplots\_adjust}:

    \begin{tcolorbox}[breakable, size=fbox, boxrule=1pt, pad at break*=1mm,colback=cellbackground, colframe=cellborder]
\prompt{In}{incolor}{41}{\boxspacing}
\begin{Verbatim}[commandchars=\\\{\}]
\PY{n}{fig}\PY{p}{,} \PY{n}{ax} \PY{o}{=} \PY{n}{plt}\PY{o}{.}\PY{n}{subplots}\PY{p}{(}\PY{l+m+mi}{1}\PY{p}{,} \PY{l+m+mi}{1}\PY{p}{)}
      
\PY{n}{ax}\PY{o}{.}\PY{n}{plot}\PY{p}{(}\PY{n}{x}\PY{p}{,} \PY{n}{x}\PY{o}{*}\PY{o}{*}\PY{l+m+mi}{2}\PY{p}{,} \PY{n}{x}\PY{p}{,} \PY{n}{np}\PY{o}{.}\PY{n}{exp}\PY{p}{(}\PY{n}{x}\PY{p}{)}\PY{p}{)}
\PY{n}{ax}\PY{o}{.}\PY{n}{set\PYZus{}yticks}\PY{p}{(}\PY{p}{[}\PY{l+m+mi}{0}\PY{p}{,} \PY{l+m+mi}{50}\PY{p}{,} \PY{l+m+mi}{100}\PY{p}{,} \PY{l+m+mi}{150}\PY{p}{]}\PY{p}{)}

\PY{n}{ax}\PY{o}{.}\PY{n}{set\PYZus{}title}\PY{p}{(}\PY{l+s+s2}{\PYZdq{}}\PY{l+s+s2}{title}\PY{l+s+s2}{\PYZdq{}}\PY{p}{)}
\PY{n}{ax}\PY{o}{.}\PY{n}{set\PYZus{}xlabel}\PY{p}{(}\PY{l+s+s2}{\PYZdq{}}\PY{l+s+s2}{x}\PY{l+s+s2}{\PYZdq{}}\PY{p}{)}
\PY{n}{ax}\PY{o}{.}\PY{n}{set\PYZus{}ylabel}\PY{p}{(}\PY{l+s+s2}{\PYZdq{}}\PY{l+s+s2}{y}\PY{l+s+s2}{\PYZdq{}}\PY{p}{)}

\PY{n}{fig}\PY{o}{.}\PY{n}{subplots\PYZus{}adjust}\PY{p}{(}\PY{n}{left}\PY{o}{=}\PY{l+m+mf}{0.15}\PY{p}{,} \PY{n}{right}\PY{o}{=}\PY{o}{.}\PY{l+m+mi}{9}\PY{p}{,} \PY{n}{bottom}\PY{o}{=}\PY{l+m+mf}{0.1}\PY{p}{,} \PY{n}{top}\PY{o}{=}\PY{l+m+mf}{0.9}\PY{p}{)}\PY{p}{;}
\end{Verbatim}
\end{tcolorbox}

    \begin{center}
    \adjustimage{max size={0.9\linewidth}{0.9\paperheight}}{Lecture-4-Matplotlib_files/Lecture-4-Matplotlib_98_0.png}
    \end{center}
    { \hspace*{\fill} \\}
    
    \hypertarget{axis-grid}{%
\subsubsection{Axis grid}\label{axis-grid}}

    With the \texttt{grid} method in the axis object, we can turn on and off
grid lines. We can also customize the appearance of the grid lines using
the same keyword arguments as the \texttt{plot} function:

    \begin{tcolorbox}[breakable, size=fbox, boxrule=1pt, pad at break*=1mm,colback=cellbackground, colframe=cellborder]
\prompt{In}{incolor}{42}{\boxspacing}
\begin{Verbatim}[commandchars=\\\{\}]
\PY{n}{fig}\PY{p}{,} \PY{n}{axes} \PY{o}{=} \PY{n}{plt}\PY{o}{.}\PY{n}{subplots}\PY{p}{(}\PY{l+m+mi}{1}\PY{p}{,} \PY{l+m+mi}{2}\PY{p}{,} \PY{n}{figsize}\PY{o}{=}\PY{p}{(}\PY{l+m+mi}{10}\PY{p}{,}\PY{l+m+mi}{3}\PY{p}{)}\PY{p}{)}

\PY{c+c1}{\PYZsh{} default grid appearance}
\PY{n}{axes}\PY{p}{[}\PY{l+m+mi}{0}\PY{p}{]}\PY{o}{.}\PY{n}{plot}\PY{p}{(}\PY{n}{x}\PY{p}{,} \PY{n}{x}\PY{o}{*}\PY{o}{*}\PY{l+m+mi}{2}\PY{p}{,} \PY{n}{x}\PY{p}{,} \PY{n}{x}\PY{o}{*}\PY{o}{*}\PY{l+m+mi}{3}\PY{p}{,} \PY{n}{lw}\PY{o}{=}\PY{l+m+mi}{2}\PY{p}{)}
\PY{n}{axes}\PY{p}{[}\PY{l+m+mi}{0}\PY{p}{]}\PY{o}{.}\PY{n}{grid}\PY{p}{(}\PY{k+kc}{True}\PY{p}{)}

\PY{c+c1}{\PYZsh{} custom grid appearance}
\PY{n}{axes}\PY{p}{[}\PY{l+m+mi}{1}\PY{p}{]}\PY{o}{.}\PY{n}{plot}\PY{p}{(}\PY{n}{x}\PY{p}{,} \PY{n}{x}\PY{o}{*}\PY{o}{*}\PY{l+m+mi}{2}\PY{p}{,} \PY{n}{x}\PY{p}{,} \PY{n}{x}\PY{o}{*}\PY{o}{*}\PY{l+m+mi}{3}\PY{p}{,} \PY{n}{lw}\PY{o}{=}\PY{l+m+mi}{2}\PY{p}{)}
\PY{n}{axes}\PY{p}{[}\PY{l+m+mi}{1}\PY{p}{]}\PY{o}{.}\PY{n}{grid}\PY{p}{(}\PY{n}{color}\PY{o}{=}\PY{l+s+s1}{\PYZsq{}}\PY{l+s+s1}{b}\PY{l+s+s1}{\PYZsq{}}\PY{p}{,} \PY{n}{alpha}\PY{o}{=}\PY{l+m+mf}{0.5}\PY{p}{,} \PY{n}{linestyle}\PY{o}{=}\PY{l+s+s1}{\PYZsq{}}\PY{l+s+s1}{dashed}\PY{l+s+s1}{\PYZsq{}}\PY{p}{,} \PY{n}{linewidth}\PY{o}{=}\PY{l+m+mf}{0.5}\PY{p}{)}
\end{Verbatim}
\end{tcolorbox}

    \begin{center}
    \adjustimage{max size={0.9\linewidth}{0.9\paperheight}}{Lecture-4-Matplotlib_files/Lecture-4-Matplotlib_101_0.png}
    \end{center}
    { \hspace*{\fill} \\}
    
    \hypertarget{axis-spines}{%
\subsubsection{Axis spines}\label{axis-spines}}

    We can also change the properties of axis spines:

    \begin{tcolorbox}[breakable, size=fbox, boxrule=1pt, pad at break*=1mm,colback=cellbackground, colframe=cellborder]
\prompt{In}{incolor}{43}{\boxspacing}
\begin{Verbatim}[commandchars=\\\{\}]
\PY{n}{fig}\PY{p}{,} \PY{n}{ax} \PY{o}{=} \PY{n}{plt}\PY{o}{.}\PY{n}{subplots}\PY{p}{(}\PY{n}{figsize}\PY{o}{=}\PY{p}{(}\PY{l+m+mi}{6}\PY{p}{,}\PY{l+m+mi}{2}\PY{p}{)}\PY{p}{)}

\PY{n}{ax}\PY{o}{.}\PY{n}{spines}\PY{p}{[}\PY{l+s+s1}{\PYZsq{}}\PY{l+s+s1}{bottom}\PY{l+s+s1}{\PYZsq{}}\PY{p}{]}\PY{o}{.}\PY{n}{set\PYZus{}color}\PY{p}{(}\PY{l+s+s1}{\PYZsq{}}\PY{l+s+s1}{blue}\PY{l+s+s1}{\PYZsq{}}\PY{p}{)}
\PY{n}{ax}\PY{o}{.}\PY{n}{spines}\PY{p}{[}\PY{l+s+s1}{\PYZsq{}}\PY{l+s+s1}{top}\PY{l+s+s1}{\PYZsq{}}\PY{p}{]}\PY{o}{.}\PY{n}{set\PYZus{}color}\PY{p}{(}\PY{l+s+s1}{\PYZsq{}}\PY{l+s+s1}{blue}\PY{l+s+s1}{\PYZsq{}}\PY{p}{)}

\PY{n}{ax}\PY{o}{.}\PY{n}{spines}\PY{p}{[}\PY{l+s+s1}{\PYZsq{}}\PY{l+s+s1}{left}\PY{l+s+s1}{\PYZsq{}}\PY{p}{]}\PY{o}{.}\PY{n}{set\PYZus{}color}\PY{p}{(}\PY{l+s+s1}{\PYZsq{}}\PY{l+s+s1}{red}\PY{l+s+s1}{\PYZsq{}}\PY{p}{)}
\PY{n}{ax}\PY{o}{.}\PY{n}{spines}\PY{p}{[}\PY{l+s+s1}{\PYZsq{}}\PY{l+s+s1}{left}\PY{l+s+s1}{\PYZsq{}}\PY{p}{]}\PY{o}{.}\PY{n}{set\PYZus{}linewidth}\PY{p}{(}\PY{l+m+mi}{2}\PY{p}{)}

\PY{c+c1}{\PYZsh{} turn off axis spine to the right}
\PY{n}{ax}\PY{o}{.}\PY{n}{spines}\PY{p}{[}\PY{l+s+s1}{\PYZsq{}}\PY{l+s+s1}{right}\PY{l+s+s1}{\PYZsq{}}\PY{p}{]}\PY{o}{.}\PY{n}{set\PYZus{}color}\PY{p}{(}\PY{l+s+s2}{\PYZdq{}}\PY{l+s+s2}{none}\PY{l+s+s2}{\PYZdq{}}\PY{p}{)}
\PY{n}{ax}\PY{o}{.}\PY{n}{yaxis}\PY{o}{.}\PY{n}{tick\PYZus{}left}\PY{p}{(}\PY{p}{)} \PY{c+c1}{\PYZsh{} only ticks on the left side}
\end{Verbatim}
\end{tcolorbox}

    \begin{center}
    \adjustimage{max size={0.9\linewidth}{0.9\paperheight}}{Lecture-4-Matplotlib_files/Lecture-4-Matplotlib_104_0.png}
    \end{center}
    { \hspace*{\fill} \\}
    
    \hypertarget{twin-axes}{%
\subsubsection{Twin axes}\label{twin-axes}}

    Sometimes it is useful to have dual x or y axes in a figure; for
example, when plotting curves with different units together. Matplotlib
supports this with the \texttt{twinx} and \texttt{twiny} functions:

    \begin{tcolorbox}[breakable, size=fbox, boxrule=1pt, pad at break*=1mm,colback=cellbackground, colframe=cellborder]
\prompt{In}{incolor}{44}{\boxspacing}
\begin{Verbatim}[commandchars=\\\{\}]
\PY{n}{fig}\PY{p}{,} \PY{n}{ax1} \PY{o}{=} \PY{n}{plt}\PY{o}{.}\PY{n}{subplots}\PY{p}{(}\PY{p}{)}

\PY{n}{ax1}\PY{o}{.}\PY{n}{plot}\PY{p}{(}\PY{n}{x}\PY{p}{,} \PY{n}{x}\PY{o}{*}\PY{o}{*}\PY{l+m+mi}{2}\PY{p}{,} \PY{n}{lw}\PY{o}{=}\PY{l+m+mi}{2}\PY{p}{,} \PY{n}{color}\PY{o}{=}\PY{l+s+s2}{\PYZdq{}}\PY{l+s+s2}{blue}\PY{l+s+s2}{\PYZdq{}}\PY{p}{)}
\PY{n}{ax1}\PY{o}{.}\PY{n}{set\PYZus{}ylabel}\PY{p}{(}\PY{l+s+sa}{r}\PY{l+s+s2}{\PYZdq{}}\PY{l+s+s2}{area \PYZdl{}(m\PYZca{}2)\PYZdl{}}\PY{l+s+s2}{\PYZdq{}}\PY{p}{,} \PY{n}{fontsize}\PY{o}{=}\PY{l+m+mi}{18}\PY{p}{,} \PY{n}{color}\PY{o}{=}\PY{l+s+s2}{\PYZdq{}}\PY{l+s+s2}{blue}\PY{l+s+s2}{\PYZdq{}}\PY{p}{)}
\PY{k}{for} \PY{n}{label} \PY{o+ow}{in} \PY{n}{ax1}\PY{o}{.}\PY{n}{get\PYZus{}yticklabels}\PY{p}{(}\PY{p}{)}\PY{p}{:}
    \PY{n}{label}\PY{o}{.}\PY{n}{set\PYZus{}color}\PY{p}{(}\PY{l+s+s2}{\PYZdq{}}\PY{l+s+s2}{blue}\PY{l+s+s2}{\PYZdq{}}\PY{p}{)}
    
\PY{n}{ax2} \PY{o}{=} \PY{n}{ax1}\PY{o}{.}\PY{n}{twinx}\PY{p}{(}\PY{p}{)}
\PY{n}{ax2}\PY{o}{.}\PY{n}{plot}\PY{p}{(}\PY{n}{x}\PY{p}{,} \PY{n}{x}\PY{o}{*}\PY{o}{*}\PY{l+m+mi}{3}\PY{p}{,} \PY{n}{lw}\PY{o}{=}\PY{l+m+mi}{2}\PY{p}{,} \PY{n}{color}\PY{o}{=}\PY{l+s+s2}{\PYZdq{}}\PY{l+s+s2}{red}\PY{l+s+s2}{\PYZdq{}}\PY{p}{)}
\PY{n}{ax2}\PY{o}{.}\PY{n}{set\PYZus{}ylabel}\PY{p}{(}\PY{l+s+sa}{r}\PY{l+s+s2}{\PYZdq{}}\PY{l+s+s2}{volume \PYZdl{}(m\PYZca{}3)\PYZdl{}}\PY{l+s+s2}{\PYZdq{}}\PY{p}{,} \PY{n}{fontsize}\PY{o}{=}\PY{l+m+mi}{18}\PY{p}{,} \PY{n}{color}\PY{o}{=}\PY{l+s+s2}{\PYZdq{}}\PY{l+s+s2}{red}\PY{l+s+s2}{\PYZdq{}}\PY{p}{)}
\PY{k}{for} \PY{n}{label} \PY{o+ow}{in} \PY{n}{ax2}\PY{o}{.}\PY{n}{get\PYZus{}yticklabels}\PY{p}{(}\PY{p}{)}\PY{p}{:}
    \PY{n}{label}\PY{o}{.}\PY{n}{set\PYZus{}color}\PY{p}{(}\PY{l+s+s2}{\PYZdq{}}\PY{l+s+s2}{red}\PY{l+s+s2}{\PYZdq{}}\PY{p}{)}
\end{Verbatim}
\end{tcolorbox}

    \begin{center}
    \adjustimage{max size={0.9\linewidth}{0.9\paperheight}}{Lecture-4-Matplotlib_files/Lecture-4-Matplotlib_107_0.png}
    \end{center}
    { \hspace*{\fill} \\}
    
    \hypertarget{axes-where-x-and-y-is-zero}{%
\subsubsection{Axes where x and y is
zero}\label{axes-where-x-and-y-is-zero}}

    \begin{tcolorbox}[breakable, size=fbox, boxrule=1pt, pad at break*=1mm,colback=cellbackground, colframe=cellborder]
\prompt{In}{incolor}{45}{\boxspacing}
\begin{Verbatim}[commandchars=\\\{\}]
\PY{n}{fig}\PY{p}{,} \PY{n}{ax} \PY{o}{=} \PY{n}{plt}\PY{o}{.}\PY{n}{subplots}\PY{p}{(}\PY{p}{)}

\PY{n}{ax}\PY{o}{.}\PY{n}{spines}\PY{p}{[}\PY{l+s+s1}{\PYZsq{}}\PY{l+s+s1}{right}\PY{l+s+s1}{\PYZsq{}}\PY{p}{]}\PY{o}{.}\PY{n}{set\PYZus{}color}\PY{p}{(}\PY{l+s+s1}{\PYZsq{}}\PY{l+s+s1}{none}\PY{l+s+s1}{\PYZsq{}}\PY{p}{)}
\PY{n}{ax}\PY{o}{.}\PY{n}{spines}\PY{p}{[}\PY{l+s+s1}{\PYZsq{}}\PY{l+s+s1}{top}\PY{l+s+s1}{\PYZsq{}}\PY{p}{]}\PY{o}{.}\PY{n}{set\PYZus{}color}\PY{p}{(}\PY{l+s+s1}{\PYZsq{}}\PY{l+s+s1}{none}\PY{l+s+s1}{\PYZsq{}}\PY{p}{)}

\PY{n}{ax}\PY{o}{.}\PY{n}{xaxis}\PY{o}{.}\PY{n}{set\PYZus{}ticks\PYZus{}position}\PY{p}{(}\PY{l+s+s1}{\PYZsq{}}\PY{l+s+s1}{bottom}\PY{l+s+s1}{\PYZsq{}}\PY{p}{)}
\PY{n}{ax}\PY{o}{.}\PY{n}{spines}\PY{p}{[}\PY{l+s+s1}{\PYZsq{}}\PY{l+s+s1}{bottom}\PY{l+s+s1}{\PYZsq{}}\PY{p}{]}\PY{o}{.}\PY{n}{set\PYZus{}position}\PY{p}{(}\PY{p}{(}\PY{l+s+s1}{\PYZsq{}}\PY{l+s+s1}{data}\PY{l+s+s1}{\PYZsq{}}\PY{p}{,}\PY{l+m+mi}{0}\PY{p}{)}\PY{p}{)} \PY{c+c1}{\PYZsh{} set position of x spine to x=0}

\PY{n}{ax}\PY{o}{.}\PY{n}{yaxis}\PY{o}{.}\PY{n}{set\PYZus{}ticks\PYZus{}position}\PY{p}{(}\PY{l+s+s1}{\PYZsq{}}\PY{l+s+s1}{left}\PY{l+s+s1}{\PYZsq{}}\PY{p}{)}
\PY{n}{ax}\PY{o}{.}\PY{n}{spines}\PY{p}{[}\PY{l+s+s1}{\PYZsq{}}\PY{l+s+s1}{left}\PY{l+s+s1}{\PYZsq{}}\PY{p}{]}\PY{o}{.}\PY{n}{set\PYZus{}position}\PY{p}{(}\PY{p}{(}\PY{l+s+s1}{\PYZsq{}}\PY{l+s+s1}{data}\PY{l+s+s1}{\PYZsq{}}\PY{p}{,}\PY{l+m+mi}{0}\PY{p}{)}\PY{p}{)}   \PY{c+c1}{\PYZsh{} set position of y spine to y=0}

\PY{n}{xx} \PY{o}{=} \PY{n}{np}\PY{o}{.}\PY{n}{linspace}\PY{p}{(}\PY{o}{\PYZhy{}}\PY{l+m+mf}{0.75}\PY{p}{,} \PY{l+m+mf}{1.}\PY{p}{,} \PY{l+m+mi}{100}\PY{p}{)}
\PY{n}{ax}\PY{o}{.}\PY{n}{plot}\PY{p}{(}\PY{n}{xx}\PY{p}{,} \PY{n}{xx}\PY{o}{*}\PY{o}{*}\PY{l+m+mi}{3}\PY{p}{)}\PY{p}{;}
\end{Verbatim}
\end{tcolorbox}

    \begin{center}
    \adjustimage{max size={0.9\linewidth}{0.9\paperheight}}{Lecture-4-Matplotlib_files/Lecture-4-Matplotlib_109_0.png}
    \end{center}
    { \hspace*{\fill} \\}
    
    \hypertarget{other-2d-plot-styles}{%
\subsubsection{Other 2D plot styles}\label{other-2d-plot-styles}}

    In addition to the regular \texttt{plot} method, there are a number of
other functions for generating different kind of plots. See the
matplotlib plot gallery for a complete list of available plot types:
http://matplotlib.org/gallery.html. Some of the more useful ones are
show below:

    \begin{tcolorbox}[breakable, size=fbox, boxrule=1pt, pad at break*=1mm,colback=cellbackground, colframe=cellborder]
\prompt{In}{incolor}{46}{\boxspacing}
\begin{Verbatim}[commandchars=\\\{\}]
\PY{n}{n} \PY{o}{=} \PY{n}{np}\PY{o}{.}\PY{n}{array}\PY{p}{(}\PY{p}{[}\PY{l+m+mi}{0}\PY{p}{,}\PY{l+m+mi}{1}\PY{p}{,}\PY{l+m+mi}{2}\PY{p}{,}\PY{l+m+mi}{3}\PY{p}{,}\PY{l+m+mi}{4}\PY{p}{,}\PY{l+m+mi}{5}\PY{p}{]}\PY{p}{)}
\end{Verbatim}
\end{tcolorbox}

    \begin{tcolorbox}[breakable, size=fbox, boxrule=1pt, pad at break*=1mm,colback=cellbackground, colframe=cellborder]
\prompt{In}{incolor}{47}{\boxspacing}
\begin{Verbatim}[commandchars=\\\{\}]
\PY{n}{fig}\PY{p}{,} \PY{n}{axes} \PY{o}{=} \PY{n}{plt}\PY{o}{.}\PY{n}{subplots}\PY{p}{(}\PY{l+m+mi}{1}\PY{p}{,} \PY{l+m+mi}{4}\PY{p}{,} \PY{n}{figsize}\PY{o}{=}\PY{p}{(}\PY{l+m+mi}{12}\PY{p}{,}\PY{l+m+mi}{3}\PY{p}{)}\PY{p}{)}

\PY{n}{axes}\PY{p}{[}\PY{l+m+mi}{0}\PY{p}{]}\PY{o}{.}\PY{n}{scatter}\PY{p}{(}\PY{n}{xx}\PY{p}{,} \PY{n}{xx} \PY{o}{+} \PY{l+m+mf}{0.25}\PY{o}{*}\PY{n}{np}\PY{o}{.}\PY{n}{random}\PY{o}{.}\PY{n}{randn}\PY{p}{(}\PY{n+nb}{len}\PY{p}{(}\PY{n}{xx}\PY{p}{)}\PY{p}{)}\PY{p}{)}
\PY{n}{axes}\PY{p}{[}\PY{l+m+mi}{0}\PY{p}{]}\PY{o}{.}\PY{n}{set\PYZus{}title}\PY{p}{(}\PY{l+s+s2}{\PYZdq{}}\PY{l+s+s2}{scatter}\PY{l+s+s2}{\PYZdq{}}\PY{p}{)}

\PY{n}{axes}\PY{p}{[}\PY{l+m+mi}{1}\PY{p}{]}\PY{o}{.}\PY{n}{step}\PY{p}{(}\PY{n}{n}\PY{p}{,} \PY{n}{n}\PY{o}{*}\PY{o}{*}\PY{l+m+mi}{2}\PY{p}{,} \PY{n}{lw}\PY{o}{=}\PY{l+m+mi}{2}\PY{p}{)}
\PY{n}{axes}\PY{p}{[}\PY{l+m+mi}{1}\PY{p}{]}\PY{o}{.}\PY{n}{set\PYZus{}title}\PY{p}{(}\PY{l+s+s2}{\PYZdq{}}\PY{l+s+s2}{step}\PY{l+s+s2}{\PYZdq{}}\PY{p}{)}

\PY{n}{axes}\PY{p}{[}\PY{l+m+mi}{2}\PY{p}{]}\PY{o}{.}\PY{n}{bar}\PY{p}{(}\PY{n}{n}\PY{p}{,} \PY{n}{n}\PY{o}{*}\PY{o}{*}\PY{l+m+mi}{2}\PY{p}{,} \PY{n}{align}\PY{o}{=}\PY{l+s+s2}{\PYZdq{}}\PY{l+s+s2}{center}\PY{l+s+s2}{\PYZdq{}}\PY{p}{,} \PY{n}{width}\PY{o}{=}\PY{l+m+mf}{0.5}\PY{p}{,} \PY{n}{alpha}\PY{o}{=}\PY{l+m+mf}{0.5}\PY{p}{)}
\PY{n}{axes}\PY{p}{[}\PY{l+m+mi}{2}\PY{p}{]}\PY{o}{.}\PY{n}{set\PYZus{}title}\PY{p}{(}\PY{l+s+s2}{\PYZdq{}}\PY{l+s+s2}{bar}\PY{l+s+s2}{\PYZdq{}}\PY{p}{)}

\PY{n}{axes}\PY{p}{[}\PY{l+m+mi}{3}\PY{p}{]}\PY{o}{.}\PY{n}{fill\PYZus{}between}\PY{p}{(}\PY{n}{x}\PY{p}{,} \PY{n}{x}\PY{o}{*}\PY{o}{*}\PY{l+m+mi}{2}\PY{p}{,} \PY{n}{x}\PY{o}{*}\PY{o}{*}\PY{l+m+mi}{3}\PY{p}{,} \PY{n}{color}\PY{o}{=}\PY{l+s+s2}{\PYZdq{}}\PY{l+s+s2}{green}\PY{l+s+s2}{\PYZdq{}}\PY{p}{,} \PY{n}{alpha}\PY{o}{=}\PY{l+m+mf}{0.5}\PY{p}{)}\PY{p}{;}
\PY{n}{axes}\PY{p}{[}\PY{l+m+mi}{3}\PY{p}{]}\PY{o}{.}\PY{n}{set\PYZus{}title}\PY{p}{(}\PY{l+s+s2}{\PYZdq{}}\PY{l+s+s2}{fill\PYZus{}between}\PY{l+s+s2}{\PYZdq{}}\PY{p}{)}\PY{p}{;}
\end{Verbatim}
\end{tcolorbox}

    \begin{center}
    \adjustimage{max size={0.9\linewidth}{0.9\paperheight}}{Lecture-4-Matplotlib_files/Lecture-4-Matplotlib_113_0.png}
    \end{center}
    { \hspace*{\fill} \\}
    
    \begin{tcolorbox}[breakable, size=fbox, boxrule=1pt, pad at break*=1mm,colback=cellbackground, colframe=cellborder]
\prompt{In}{incolor}{48}{\boxspacing}
\begin{Verbatim}[commandchars=\\\{\}]
\PY{c+c1}{\PYZsh{} polar plot using add\PYZus{}axes and polar projection}
\PY{n}{fig} \PY{o}{=} \PY{n}{plt}\PY{o}{.}\PY{n}{figure}\PY{p}{(}\PY{p}{)}
\PY{n}{ax} \PY{o}{=} \PY{n}{fig}\PY{o}{.}\PY{n}{add\PYZus{}axes}\PY{p}{(}\PY{p}{[}\PY{l+m+mf}{0.0}\PY{p}{,} \PY{l+m+mf}{0.0}\PY{p}{,} \PY{o}{.}\PY{l+m+mi}{6}\PY{p}{,} \PY{o}{.}\PY{l+m+mi}{6}\PY{p}{]}\PY{p}{,} \PY{n}{polar}\PY{o}{=}\PY{k+kc}{True}\PY{p}{)}
\PY{n}{t} \PY{o}{=} \PY{n}{np}\PY{o}{.}\PY{n}{linspace}\PY{p}{(}\PY{l+m+mi}{0}\PY{p}{,} \PY{l+m+mi}{2} \PY{o}{*} \PY{n}{np}\PY{o}{.}\PY{n}{pi}\PY{p}{,} \PY{l+m+mi}{100}\PY{p}{)}
\PY{n}{ax}\PY{o}{.}\PY{n}{plot}\PY{p}{(}\PY{n}{t}\PY{p}{,} \PY{n}{t}\PY{p}{,} \PY{n}{color}\PY{o}{=}\PY{l+s+s1}{\PYZsq{}}\PY{l+s+s1}{blue}\PY{l+s+s1}{\PYZsq{}}\PY{p}{,} \PY{n}{lw}\PY{o}{=}\PY{l+m+mi}{3}\PY{p}{)}\PY{p}{;}
\end{Verbatim}
\end{tcolorbox}

    \begin{center}
    \adjustimage{max size={0.9\linewidth}{0.9\paperheight}}{Lecture-4-Matplotlib_files/Lecture-4-Matplotlib_114_0.png}
    \end{center}
    { \hspace*{\fill} \\}
    
    \begin{tcolorbox}[breakable, size=fbox, boxrule=1pt, pad at break*=1mm,colback=cellbackground, colframe=cellborder]
\prompt{In}{incolor}{49}{\boxspacing}
\begin{Verbatim}[commandchars=\\\{\}]
\PY{c+c1}{\PYZsh{} A histogram}
\PY{n}{n} \PY{o}{=} \PY{n}{np}\PY{o}{.}\PY{n}{random}\PY{o}{.}\PY{n}{randn}\PY{p}{(}\PY{l+m+mi}{100000}\PY{p}{)}
\PY{n}{fig}\PY{p}{,} \PY{n}{axes} \PY{o}{=} \PY{n}{plt}\PY{o}{.}\PY{n}{subplots}\PY{p}{(}\PY{l+m+mi}{1}\PY{p}{,} \PY{l+m+mi}{2}\PY{p}{,} \PY{n}{figsize}\PY{o}{=}\PY{p}{(}\PY{l+m+mi}{12}\PY{p}{,}\PY{l+m+mi}{4}\PY{p}{)}\PY{p}{)}

\PY{n}{axes}\PY{p}{[}\PY{l+m+mi}{0}\PY{p}{]}\PY{o}{.}\PY{n}{hist}\PY{p}{(}\PY{n}{n}\PY{p}{)}
\PY{n}{axes}\PY{p}{[}\PY{l+m+mi}{0}\PY{p}{]}\PY{o}{.}\PY{n}{set\PYZus{}title}\PY{p}{(}\PY{l+s+s2}{\PYZdq{}}\PY{l+s+s2}{Default histogram}\PY{l+s+s2}{\PYZdq{}}\PY{p}{)}
\PY{n}{axes}\PY{p}{[}\PY{l+m+mi}{0}\PY{p}{]}\PY{o}{.}\PY{n}{set\PYZus{}xlim}\PY{p}{(}\PY{p}{(}\PY{n+nb}{min}\PY{p}{(}\PY{n}{n}\PY{p}{)}\PY{p}{,} \PY{n+nb}{max}\PY{p}{(}\PY{n}{n}\PY{p}{)}\PY{p}{)}\PY{p}{)}

\PY{n}{axes}\PY{p}{[}\PY{l+m+mi}{1}\PY{p}{]}\PY{o}{.}\PY{n}{hist}\PY{p}{(}\PY{n}{n}\PY{p}{,} \PY{n}{cumulative}\PY{o}{=}\PY{k+kc}{True}\PY{p}{,} \PY{n}{bins}\PY{o}{=}\PY{l+m+mi}{50}\PY{p}{)}
\PY{n}{axes}\PY{p}{[}\PY{l+m+mi}{1}\PY{p}{]}\PY{o}{.}\PY{n}{set\PYZus{}title}\PY{p}{(}\PY{l+s+s2}{\PYZdq{}}\PY{l+s+s2}{Cumulative detailed histogram}\PY{l+s+s2}{\PYZdq{}}\PY{p}{)}
\PY{n}{axes}\PY{p}{[}\PY{l+m+mi}{1}\PY{p}{]}\PY{o}{.}\PY{n}{set\PYZus{}xlim}\PY{p}{(}\PY{p}{(}\PY{n+nb}{min}\PY{p}{(}\PY{n}{n}\PY{p}{)}\PY{p}{,} \PY{n+nb}{max}\PY{p}{(}\PY{n}{n}\PY{p}{)}\PY{p}{)}\PY{p}{)}\PY{p}{;}
\end{Verbatim}
\end{tcolorbox}

    \begin{center}
    \adjustimage{max size={0.9\linewidth}{0.9\paperheight}}{Lecture-4-Matplotlib_files/Lecture-4-Matplotlib_115_0.png}
    \end{center}
    { \hspace*{\fill} \\}
    
    \hypertarget{text-annotation}{%
\subsubsection{Text annotation}\label{text-annotation}}

    Annotating text in matplotlib figures can be done using the
\texttt{text} function. It supports LaTeX formatting just like axis
label texts and titles:

    \begin{tcolorbox}[breakable, size=fbox, boxrule=1pt, pad at break*=1mm,colback=cellbackground, colframe=cellborder]
\prompt{In}{incolor}{50}{\boxspacing}
\begin{Verbatim}[commandchars=\\\{\}]
\PY{n}{fig}\PY{p}{,} \PY{n}{ax} \PY{o}{=} \PY{n}{plt}\PY{o}{.}\PY{n}{subplots}\PY{p}{(}\PY{p}{)}

\PY{n}{ax}\PY{o}{.}\PY{n}{plot}\PY{p}{(}\PY{n}{xx}\PY{p}{,} \PY{n}{xx}\PY{o}{*}\PY{o}{*}\PY{l+m+mi}{2}\PY{p}{,} \PY{n}{xx}\PY{p}{,} \PY{n}{xx}\PY{o}{*}\PY{o}{*}\PY{l+m+mi}{3}\PY{p}{)}

\PY{n}{ax}\PY{o}{.}\PY{n}{text}\PY{p}{(}\PY{l+m+mf}{0.15}\PY{p}{,} \PY{l+m+mf}{0.2}\PY{p}{,} \PY{l+s+sa}{r}\PY{l+s+s2}{\PYZdq{}}\PY{l+s+s2}{\PYZdl{}y=x\PYZca{}2\PYZdl{}}\PY{l+s+s2}{\PYZdq{}}\PY{p}{,} \PY{n}{fontsize}\PY{o}{=}\PY{l+m+mi}{20}\PY{p}{,} \PY{n}{color}\PY{o}{=}\PY{l+s+s2}{\PYZdq{}}\PY{l+s+s2}{blue}\PY{l+s+s2}{\PYZdq{}}\PY{p}{)}
\PY{n}{ax}\PY{o}{.}\PY{n}{text}\PY{p}{(}\PY{l+m+mf}{0.65}\PY{p}{,} \PY{l+m+mf}{0.1}\PY{p}{,} \PY{l+s+sa}{r}\PY{l+s+s2}{\PYZdq{}}\PY{l+s+s2}{\PYZdl{}y=x\PYZca{}3\PYZdl{}}\PY{l+s+s2}{\PYZdq{}}\PY{p}{,} \PY{n}{fontsize}\PY{o}{=}\PY{l+m+mi}{20}\PY{p}{,} \PY{n}{color}\PY{o}{=}\PY{l+s+s2}{\PYZdq{}}\PY{l+s+s2}{green}\PY{l+s+s2}{\PYZdq{}}\PY{p}{)}\PY{p}{;}
\end{Verbatim}
\end{tcolorbox}

    \begin{center}
    \adjustimage{max size={0.9\linewidth}{0.9\paperheight}}{Lecture-4-Matplotlib_files/Lecture-4-Matplotlib_118_0.png}
    \end{center}
    { \hspace*{\fill} \\}
    
    \hypertarget{figures-with-multiple-subplots-and-insets}{%
\subsubsection{Figures with multiple subplots and
insets}\label{figures-with-multiple-subplots-and-insets}}

    Axes can be added to a matplotlib Figure canvas manually using
\texttt{fig.add\_axes} or using a sub-figure layout manager such as
\texttt{subplots}, \texttt{subplot2grid}, or \texttt{gridspec}:

    \hypertarget{subplots}{%
\paragraph{subplots}\label{subplots}}

    \begin{tcolorbox}[breakable, size=fbox, boxrule=1pt, pad at break*=1mm,colback=cellbackground, colframe=cellborder]
\prompt{In}{incolor}{51}{\boxspacing}
\begin{Verbatim}[commandchars=\\\{\}]
\PY{n}{fig}\PY{p}{,} \PY{n}{ax} \PY{o}{=} \PY{n}{plt}\PY{o}{.}\PY{n}{subplots}\PY{p}{(}\PY{l+m+mi}{2}\PY{p}{,} \PY{l+m+mi}{3}\PY{p}{)}
\PY{n}{fig}\PY{o}{.}\PY{n}{tight\PYZus{}layout}\PY{p}{(}\PY{p}{)}
\end{Verbatim}
\end{tcolorbox}

    \begin{center}
    \adjustimage{max size={0.9\linewidth}{0.9\paperheight}}{Lecture-4-Matplotlib_files/Lecture-4-Matplotlib_122_0.png}
    \end{center}
    { \hspace*{\fill} \\}
    
    \hypertarget{subplot2grid}{%
\paragraph{subplot2grid}\label{subplot2grid}}

    \begin{tcolorbox}[breakable, size=fbox, boxrule=1pt, pad at break*=1mm,colback=cellbackground, colframe=cellborder]
\prompt{In}{incolor}{52}{\boxspacing}
\begin{Verbatim}[commandchars=\\\{\}]
\PY{n}{fig} \PY{o}{=} \PY{n}{plt}\PY{o}{.}\PY{n}{figure}\PY{p}{(}\PY{p}{)}
\PY{n}{ax1} \PY{o}{=} \PY{n}{plt}\PY{o}{.}\PY{n}{subplot2grid}\PY{p}{(}\PY{p}{(}\PY{l+m+mi}{3}\PY{p}{,}\PY{l+m+mi}{3}\PY{p}{)}\PY{p}{,} \PY{p}{(}\PY{l+m+mi}{0}\PY{p}{,}\PY{l+m+mi}{0}\PY{p}{)}\PY{p}{,} \PY{n}{colspan}\PY{o}{=}\PY{l+m+mi}{3}\PY{p}{)}
\PY{n}{ax2} \PY{o}{=} \PY{n}{plt}\PY{o}{.}\PY{n}{subplot2grid}\PY{p}{(}\PY{p}{(}\PY{l+m+mi}{3}\PY{p}{,}\PY{l+m+mi}{3}\PY{p}{)}\PY{p}{,} \PY{p}{(}\PY{l+m+mi}{1}\PY{p}{,}\PY{l+m+mi}{0}\PY{p}{)}\PY{p}{,} \PY{n}{colspan}\PY{o}{=}\PY{l+m+mi}{2}\PY{p}{)}
\PY{n}{ax3} \PY{o}{=} \PY{n}{plt}\PY{o}{.}\PY{n}{subplot2grid}\PY{p}{(}\PY{p}{(}\PY{l+m+mi}{3}\PY{p}{,}\PY{l+m+mi}{3}\PY{p}{)}\PY{p}{,} \PY{p}{(}\PY{l+m+mi}{1}\PY{p}{,}\PY{l+m+mi}{2}\PY{p}{)}\PY{p}{,} \PY{n}{rowspan}\PY{o}{=}\PY{l+m+mi}{2}\PY{p}{)}
\PY{n}{ax4} \PY{o}{=} \PY{n}{plt}\PY{o}{.}\PY{n}{subplot2grid}\PY{p}{(}\PY{p}{(}\PY{l+m+mi}{3}\PY{p}{,}\PY{l+m+mi}{3}\PY{p}{)}\PY{p}{,} \PY{p}{(}\PY{l+m+mi}{2}\PY{p}{,}\PY{l+m+mi}{0}\PY{p}{)}\PY{p}{)}
\PY{n}{ax5} \PY{o}{=} \PY{n}{plt}\PY{o}{.}\PY{n}{subplot2grid}\PY{p}{(}\PY{p}{(}\PY{l+m+mi}{3}\PY{p}{,}\PY{l+m+mi}{3}\PY{p}{)}\PY{p}{,} \PY{p}{(}\PY{l+m+mi}{2}\PY{p}{,}\PY{l+m+mi}{1}\PY{p}{)}\PY{p}{)}
\PY{n}{fig}\PY{o}{.}\PY{n}{tight\PYZus{}layout}\PY{p}{(}\PY{p}{)}
\end{Verbatim}
\end{tcolorbox}

    \begin{center}
    \adjustimage{max size={0.9\linewidth}{0.9\paperheight}}{Lecture-4-Matplotlib_files/Lecture-4-Matplotlib_124_0.png}
    \end{center}
    { \hspace*{\fill} \\}
    
    \hypertarget{gridspec}{%
\paragraph{gridspec}\label{gridspec}}

    \begin{tcolorbox}[breakable, size=fbox, boxrule=1pt, pad at break*=1mm,colback=cellbackground, colframe=cellborder]
\prompt{In}{incolor}{53}{\boxspacing}
\begin{Verbatim}[commandchars=\\\{\}]
\PY{k+kn}{import} \PY{n+nn}{matplotlib}\PY{n+nn}{.}\PY{n+nn}{gridspec} \PY{k}{as} \PY{n+nn}{gridspec}
\end{Verbatim}
\end{tcolorbox}

    \begin{tcolorbox}[breakable, size=fbox, boxrule=1pt, pad at break*=1mm,colback=cellbackground, colframe=cellborder]
\prompt{In}{incolor}{54}{\boxspacing}
\begin{Verbatim}[commandchars=\\\{\}]
\PY{n}{fig} \PY{o}{=} \PY{n}{plt}\PY{o}{.}\PY{n}{figure}\PY{p}{(}\PY{p}{)}

\PY{n}{gs} \PY{o}{=} \PY{n}{gridspec}\PY{o}{.}\PY{n}{GridSpec}\PY{p}{(}\PY{l+m+mi}{2}\PY{p}{,} \PY{l+m+mi}{3}\PY{p}{,} \PY{n}{height\PYZus{}ratios}\PY{o}{=}\PY{p}{[}\PY{l+m+mi}{2}\PY{p}{,}\PY{l+m+mi}{1}\PY{p}{]}\PY{p}{,} \PY{n}{width\PYZus{}ratios}\PY{o}{=}\PY{p}{[}\PY{l+m+mi}{1}\PY{p}{,}\PY{l+m+mi}{2}\PY{p}{,}\PY{l+m+mi}{1}\PY{p}{]}\PY{p}{)}
\PY{k}{for} \PY{n}{g} \PY{o+ow}{in} \PY{n}{gs}\PY{p}{:}
    \PY{n}{ax} \PY{o}{=} \PY{n}{fig}\PY{o}{.}\PY{n}{add\PYZus{}subplot}\PY{p}{(}\PY{n}{g}\PY{p}{)}
    
\PY{n}{fig}\PY{o}{.}\PY{n}{tight\PYZus{}layout}\PY{p}{(}\PY{p}{)}
\end{Verbatim}
\end{tcolorbox}

    \begin{center}
    \adjustimage{max size={0.9\linewidth}{0.9\paperheight}}{Lecture-4-Matplotlib_files/Lecture-4-Matplotlib_127_0.png}
    \end{center}
    { \hspace*{\fill} \\}
    
    \hypertarget{add_axes}{%
\paragraph{add\_axes}\label{add_axes}}

    Manually adding axes with \texttt{add\_axes} is useful for adding insets
to figures:

    \begin{tcolorbox}[breakable, size=fbox, boxrule=1pt, pad at break*=1mm,colback=cellbackground, colframe=cellborder]
\prompt{In}{incolor}{55}{\boxspacing}
\begin{Verbatim}[commandchars=\\\{\}]
\PY{n}{fig}\PY{p}{,} \PY{n}{ax} \PY{o}{=} \PY{n}{plt}\PY{o}{.}\PY{n}{subplots}\PY{p}{(}\PY{p}{)}

\PY{n}{ax}\PY{o}{.}\PY{n}{plot}\PY{p}{(}\PY{n}{xx}\PY{p}{,} \PY{n}{xx}\PY{o}{*}\PY{o}{*}\PY{l+m+mi}{2}\PY{p}{,} \PY{n}{xx}\PY{p}{,} \PY{n}{xx}\PY{o}{*}\PY{o}{*}\PY{l+m+mi}{3}\PY{p}{)}
\PY{n}{fig}\PY{o}{.}\PY{n}{tight\PYZus{}layout}\PY{p}{(}\PY{p}{)}

\PY{c+c1}{\PYZsh{} inset}
\PY{n}{inset\PYZus{}ax} \PY{o}{=} \PY{n}{fig}\PY{o}{.}\PY{n}{add\PYZus{}axes}\PY{p}{(}\PY{p}{[}\PY{l+m+mf}{0.2}\PY{p}{,} \PY{l+m+mf}{0.55}\PY{p}{,} \PY{l+m+mf}{0.35}\PY{p}{,} \PY{l+m+mf}{0.35}\PY{p}{]}\PY{p}{)} \PY{c+c1}{\PYZsh{} X, Y, width, height}

\PY{n}{inset\PYZus{}ax}\PY{o}{.}\PY{n}{plot}\PY{p}{(}\PY{n}{xx}\PY{p}{,} \PY{n}{xx}\PY{o}{*}\PY{o}{*}\PY{l+m+mi}{2}\PY{p}{,} \PY{n}{xx}\PY{p}{,} \PY{n}{xx}\PY{o}{*}\PY{o}{*}\PY{l+m+mi}{3}\PY{p}{)}
\PY{n}{inset\PYZus{}ax}\PY{o}{.}\PY{n}{set\PYZus{}title}\PY{p}{(}\PY{l+s+s1}{\PYZsq{}}\PY{l+s+s1}{zoom near origin}\PY{l+s+s1}{\PYZsq{}}\PY{p}{)}

\PY{c+c1}{\PYZsh{} set axis range}
\PY{n}{inset\PYZus{}ax}\PY{o}{.}\PY{n}{set\PYZus{}xlim}\PY{p}{(}\PY{o}{\PYZhy{}}\PY{o}{.}\PY{l+m+mi}{2}\PY{p}{,} \PY{o}{.}\PY{l+m+mi}{2}\PY{p}{)}
\PY{n}{inset\PYZus{}ax}\PY{o}{.}\PY{n}{set\PYZus{}ylim}\PY{p}{(}\PY{o}{\PYZhy{}}\PY{o}{.}\PY{l+m+mi}{005}\PY{p}{,} \PY{o}{.}\PY{l+m+mi}{01}\PY{p}{)}

\PY{c+c1}{\PYZsh{} set axis tick locations}
\PY{n}{inset\PYZus{}ax}\PY{o}{.}\PY{n}{set\PYZus{}yticks}\PY{p}{(}\PY{p}{[}\PY{l+m+mi}{0}\PY{p}{,} \PY{l+m+mf}{0.005}\PY{p}{,} \PY{l+m+mf}{0.01}\PY{p}{]}\PY{p}{)}
\PY{n}{inset\PYZus{}ax}\PY{o}{.}\PY{n}{set\PYZus{}xticks}\PY{p}{(}\PY{p}{[}\PY{o}{\PYZhy{}}\PY{l+m+mf}{0.1}\PY{p}{,}\PY{l+m+mi}{0}\PY{p}{,}\PY{o}{.}\PY{l+m+mi}{1}\PY{p}{]}\PY{p}{)}\PY{p}{;}
\end{Verbatim}
\end{tcolorbox}

    \begin{center}
    \adjustimage{max size={0.9\linewidth}{0.9\paperheight}}{Lecture-4-Matplotlib_files/Lecture-4-Matplotlib_130_0.png}
    \end{center}
    { \hspace*{\fill} \\}
    
    \hypertarget{colormap-and-contour-figures}{%
\subsubsection{Colormap and contour
figures}\label{colormap-and-contour-figures}}

    Colormaps and contour figures are useful for plotting functions of two
variables. In most of these functions we will use a colormap to encode
one dimension of the data. There are a number of predefined colormaps.
It is relatively straightforward to define custom colormaps. For a list
of pre-defined colormaps, see:
http://www.scipy.org/Cookbook/Matplotlib/Show\_colormaps

    \begin{tcolorbox}[breakable, size=fbox, boxrule=1pt, pad at break*=1mm,colback=cellbackground, colframe=cellborder]
\prompt{In}{incolor}{56}{\boxspacing}
\begin{Verbatim}[commandchars=\\\{\}]
\PY{n}{alpha} \PY{o}{=} \PY{l+m+mf}{0.7}
\PY{n}{phi\PYZus{}ext} \PY{o}{=} \PY{l+m+mi}{2} \PY{o}{*} \PY{n}{np}\PY{o}{.}\PY{n}{pi} \PY{o}{*} \PY{l+m+mf}{0.5}

\PY{k}{def} \PY{n+nf}{flux\PYZus{}qubit\PYZus{}potential}\PY{p}{(}\PY{n}{phi\PYZus{}m}\PY{p}{,} \PY{n}{phi\PYZus{}p}\PY{p}{)}\PY{p}{:}
    \PY{k}{return} \PY{l+m+mi}{2} \PY{o}{+} \PY{n}{alpha} \PY{o}{\PYZhy{}} \PY{l+m+mi}{2} \PY{o}{*} \PY{n}{np}\PY{o}{.}\PY{n}{cos}\PY{p}{(}\PY{n}{phi\PYZus{}p}\PY{p}{)} \PY{o}{*} \PY{n}{np}\PY{o}{.}\PY{n}{cos}\PY{p}{(}\PY{n}{phi\PYZus{}m}\PY{p}{)} \PY{o}{\PYZhy{}} \PY{n}{alpha} \PY{o}{*} \PY{n}{np}\PY{o}{.}\PY{n}{cos}\PY{p}{(}\PY{n}{phi\PYZus{}ext} \PY{o}{\PYZhy{}} \PY{l+m+mi}{2}\PY{o}{*}\PY{n}{phi\PYZus{}p}\PY{p}{)}
\end{Verbatim}
\end{tcolorbox}

    \begin{tcolorbox}[breakable, size=fbox, boxrule=1pt, pad at break*=1mm,colback=cellbackground, colframe=cellborder]
\prompt{In}{incolor}{57}{\boxspacing}
\begin{Verbatim}[commandchars=\\\{\}]
\PY{n}{phi\PYZus{}m} \PY{o}{=} \PY{n}{np}\PY{o}{.}\PY{n}{linspace}\PY{p}{(}\PY{l+m+mi}{0}\PY{p}{,} \PY{l+m+mi}{2}\PY{o}{*}\PY{n}{np}\PY{o}{.}\PY{n}{pi}\PY{p}{,} \PY{l+m+mi}{100}\PY{p}{)}
\PY{n}{phi\PYZus{}p} \PY{o}{=} \PY{n}{np}\PY{o}{.}\PY{n}{linspace}\PY{p}{(}\PY{l+m+mi}{0}\PY{p}{,} \PY{l+m+mi}{2}\PY{o}{*}\PY{n}{np}\PY{o}{.}\PY{n}{pi}\PY{p}{,} \PY{l+m+mi}{100}\PY{p}{)}
\PY{n}{X}\PY{p}{,}\PY{n}{Y} \PY{o}{=} \PY{n}{np}\PY{o}{.}\PY{n}{meshgrid}\PY{p}{(}\PY{n}{phi\PYZus{}p}\PY{p}{,} \PY{n}{phi\PYZus{}m}\PY{p}{)}
\PY{n}{Z} \PY{o}{=} \PY{n}{flux\PYZus{}qubit\PYZus{}potential}\PY{p}{(}\PY{n}{X}\PY{p}{,} \PY{n}{Y}\PY{p}{)}\PY{o}{.}\PY{n}{T}
\end{Verbatim}
\end{tcolorbox}

    \hypertarget{pcolor}{%
\paragraph{pcolor}\label{pcolor}}

    \begin{tcolorbox}[breakable, size=fbox, boxrule=1pt, pad at break*=1mm,colback=cellbackground, colframe=cellborder]
\prompt{In}{incolor}{58}{\boxspacing}
\begin{Verbatim}[commandchars=\\\{\}]
\PY{n}{fig}\PY{p}{,} \PY{n}{ax} \PY{o}{=} \PY{n}{plt}\PY{o}{.}\PY{n}{subplots}\PY{p}{(}\PY{p}{)}

\PY{n}{p} \PY{o}{=} \PY{n}{ax}\PY{o}{.}\PY{n}{pcolor}\PY{p}{(}\PY{n}{X}\PY{o}{/}\PY{p}{(}\PY{l+m+mi}{2}\PY{o}{*}\PY{n}{np}\PY{o}{.}\PY{n}{pi}\PY{p}{)}\PY{p}{,} \PY{n}{Y}\PY{o}{/}\PY{p}{(}\PY{l+m+mi}{2}\PY{o}{*}\PY{n}{np}\PY{o}{.}\PY{n}{pi}\PY{p}{)}\PY{p}{,} \PY{n}{Z}\PY{p}{,} \PY{n}{cmap}\PY{o}{=}\PY{n}{matplotlib}\PY{o}{.}\PY{n}{cm}\PY{o}{.}\PY{n}{RdBu}\PY{p}{,} \PY{n}{vmin}\PY{o}{=}\PY{n+nb}{abs}\PY{p}{(}\PY{n}{Z}\PY{p}{)}\PY{o}{.}\PY{n}{min}\PY{p}{(}\PY{p}{)}\PY{p}{,} \PY{n}{vmax}\PY{o}{=}\PY{n+nb}{abs}\PY{p}{(}\PY{n}{Z}\PY{p}{)}\PY{o}{.}\PY{n}{max}\PY{p}{(}\PY{p}{)}\PY{p}{)}
\PY{n}{cb} \PY{o}{=} \PY{n}{fig}\PY{o}{.}\PY{n}{colorbar}\PY{p}{(}\PY{n}{p}\PY{p}{,} \PY{n}{ax}\PY{o}{=}\PY{n}{ax}\PY{p}{)}
\end{Verbatim}
\end{tcolorbox}

    \begin{center}
    \adjustimage{max size={0.9\linewidth}{0.9\paperheight}}{Lecture-4-Matplotlib_files/Lecture-4-Matplotlib_136_0.png}
    \end{center}
    { \hspace*{\fill} \\}
    
    \hypertarget{imshow}{%
\paragraph{imshow}\label{imshow}}

    \begin{tcolorbox}[breakable, size=fbox, boxrule=1pt, pad at break*=1mm,colback=cellbackground, colframe=cellborder]
\prompt{In}{incolor}{59}{\boxspacing}
\begin{Verbatim}[commandchars=\\\{\}]
\PY{n}{fig}\PY{p}{,} \PY{n}{ax} \PY{o}{=} \PY{n}{plt}\PY{o}{.}\PY{n}{subplots}\PY{p}{(}\PY{p}{)}

\PY{n}{im} \PY{o}{=} \PY{n}{ax}\PY{o}{.}\PY{n}{imshow}\PY{p}{(}\PY{n}{Z}\PY{p}{,} \PY{n}{cmap}\PY{o}{=}\PY{n}{matplotlib}\PY{o}{.}\PY{n}{cm}\PY{o}{.}\PY{n}{RdBu}\PY{p}{,} \PY{n}{vmin}\PY{o}{=}\PY{n+nb}{abs}\PY{p}{(}\PY{n}{Z}\PY{p}{)}\PY{o}{.}\PY{n}{min}\PY{p}{(}\PY{p}{)}\PY{p}{,} \PY{n}{vmax}\PY{o}{=}\PY{n+nb}{abs}\PY{p}{(}\PY{n}{Z}\PY{p}{)}\PY{o}{.}\PY{n}{max}\PY{p}{(}\PY{p}{)}\PY{p}{,} \PY{n}{extent}\PY{o}{=}\PY{p}{[}\PY{l+m+mi}{0}\PY{p}{,} \PY{l+m+mi}{1}\PY{p}{,} \PY{l+m+mi}{0}\PY{p}{,} \PY{l+m+mi}{1}\PY{p}{]}\PY{p}{)}
\PY{n}{im}\PY{o}{.}\PY{n}{set\PYZus{}interpolation}\PY{p}{(}\PY{l+s+s1}{\PYZsq{}}\PY{l+s+s1}{bilinear}\PY{l+s+s1}{\PYZsq{}}\PY{p}{)}

\PY{n}{cb} \PY{o}{=} \PY{n}{fig}\PY{o}{.}\PY{n}{colorbar}\PY{p}{(}\PY{n}{im}\PY{p}{,} \PY{n}{ax}\PY{o}{=}\PY{n}{ax}\PY{p}{)}
\end{Verbatim}
\end{tcolorbox}

    \begin{center}
    \adjustimage{max size={0.9\linewidth}{0.9\paperheight}}{Lecture-4-Matplotlib_files/Lecture-4-Matplotlib_138_0.png}
    \end{center}
    { \hspace*{\fill} \\}
    
    \hypertarget{contour}{%
\paragraph{contour}\label{contour}}

    \begin{tcolorbox}[breakable, size=fbox, boxrule=1pt, pad at break*=1mm,colback=cellbackground, colframe=cellborder]
\prompt{In}{incolor}{60}{\boxspacing}
\begin{Verbatim}[commandchars=\\\{\}]
\PY{n}{fig}\PY{p}{,} \PY{n}{ax} \PY{o}{=} \PY{n}{plt}\PY{o}{.}\PY{n}{subplots}\PY{p}{(}\PY{p}{)}

\PY{n}{cnt} \PY{o}{=} \PY{n}{ax}\PY{o}{.}\PY{n}{contour}\PY{p}{(}\PY{n}{Z}\PY{p}{,} \PY{n}{cmap}\PY{o}{=}\PY{n}{matplotlib}\PY{o}{.}\PY{n}{cm}\PY{o}{.}\PY{n}{RdBu}\PY{p}{,} \PY{n}{vmin}\PY{o}{=}\PY{n+nb}{abs}\PY{p}{(}\PY{n}{Z}\PY{p}{)}\PY{o}{.}\PY{n}{min}\PY{p}{(}\PY{p}{)}\PY{p}{,} \PY{n}{vmax}\PY{o}{=}\PY{n+nb}{abs}\PY{p}{(}\PY{n}{Z}\PY{p}{)}\PY{o}{.}\PY{n}{max}\PY{p}{(}\PY{p}{)}\PY{p}{,} \PY{n}{extent}\PY{o}{=}\PY{p}{[}\PY{l+m+mi}{0}\PY{p}{,} \PY{l+m+mi}{1}\PY{p}{,} \PY{l+m+mi}{0}\PY{p}{,} \PY{l+m+mi}{1}\PY{p}{]}\PY{p}{)}
\end{Verbatim}
\end{tcolorbox}

    \begin{center}
    \adjustimage{max size={0.9\linewidth}{0.9\paperheight}}{Lecture-4-Matplotlib_files/Lecture-4-Matplotlib_140_0.png}
    \end{center}
    { \hspace*{\fill} \\}
    
    \hypertarget{d-figures}{%
\subsection{3D figures}\label{d-figures}}

    To use 3D graphics in matplotlib, we first need to create an instance of
the \texttt{Axes3D} class. 3D axes can be added to a matplotlib figure
canvas in exactly the same way as 2D axes; or, more conveniently, by
passing a \texttt{projection=\textquotesingle{}3d\textquotesingle{}}
keyword argument to the \texttt{add\_axes} or \texttt{add\_subplot}
methods.

    \begin{tcolorbox}[breakable, size=fbox, boxrule=1pt, pad at break*=1mm,colback=cellbackground, colframe=cellborder]
\prompt{In}{incolor}{61}{\boxspacing}
\begin{Verbatim}[commandchars=\\\{\}]
\PY{k+kn}{from} \PY{n+nn}{mpl\PYZus{}toolkits}\PY{n+nn}{.}\PY{n+nn}{mplot3d}\PY{n+nn}{.}\PY{n+nn}{axes3d} \PY{k+kn}{import} \PY{n}{Axes3D}
\end{Verbatim}
\end{tcolorbox}

    \hypertarget{surface-plots}{%
\paragraph{Surface plots}\label{surface-plots}}

    \begin{tcolorbox}[breakable, size=fbox, boxrule=1pt, pad at break*=1mm,colback=cellbackground, colframe=cellborder]
\prompt{In}{incolor}{62}{\boxspacing}
\begin{Verbatim}[commandchars=\\\{\}]
\PY{n}{fig} \PY{o}{=} \PY{n}{plt}\PY{o}{.}\PY{n}{figure}\PY{p}{(}\PY{n}{figsize}\PY{o}{=}\PY{p}{(}\PY{l+m+mi}{14}\PY{p}{,}\PY{l+m+mi}{6}\PY{p}{)}\PY{p}{)}

\PY{c+c1}{\PYZsh{} `ax` is a 3D\PYZhy{}aware axis instance because of the projection=\PYZsq{}3d\PYZsq{} keyword argument to add\PYZus{}subplot}
\PY{n}{ax} \PY{o}{=} \PY{n}{fig}\PY{o}{.}\PY{n}{add\PYZus{}subplot}\PY{p}{(}\PY{l+m+mi}{1}\PY{p}{,} \PY{l+m+mi}{2}\PY{p}{,} \PY{l+m+mi}{1}\PY{p}{,} \PY{n}{projection}\PY{o}{=}\PY{l+s+s1}{\PYZsq{}}\PY{l+s+s1}{3d}\PY{l+s+s1}{\PYZsq{}}\PY{p}{)}

\PY{n}{p} \PY{o}{=} \PY{n}{ax}\PY{o}{.}\PY{n}{plot\PYZus{}surface}\PY{p}{(}\PY{n}{X}\PY{p}{,} \PY{n}{Y}\PY{p}{,} \PY{n}{Z}\PY{p}{,} \PY{n}{rstride}\PY{o}{=}\PY{l+m+mi}{4}\PY{p}{,} \PY{n}{cstride}\PY{o}{=}\PY{l+m+mi}{4}\PY{p}{,} \PY{n}{linewidth}\PY{o}{=}\PY{l+m+mi}{0}\PY{p}{)}

\PY{c+c1}{\PYZsh{} surface\PYZus{}plot with color grading and color bar}
\PY{n}{ax} \PY{o}{=} \PY{n}{fig}\PY{o}{.}\PY{n}{add\PYZus{}subplot}\PY{p}{(}\PY{l+m+mi}{1}\PY{p}{,} \PY{l+m+mi}{2}\PY{p}{,} \PY{l+m+mi}{2}\PY{p}{,} \PY{n}{projection}\PY{o}{=}\PY{l+s+s1}{\PYZsq{}}\PY{l+s+s1}{3d}\PY{l+s+s1}{\PYZsq{}}\PY{p}{)}
\PY{n}{p} \PY{o}{=} \PY{n}{ax}\PY{o}{.}\PY{n}{plot\PYZus{}surface}\PY{p}{(}\PY{n}{X}\PY{p}{,} \PY{n}{Y}\PY{p}{,} \PY{n}{Z}\PY{p}{,} \PY{n}{rstride}\PY{o}{=}\PY{l+m+mi}{1}\PY{p}{,} \PY{n}{cstride}\PY{o}{=}\PY{l+m+mi}{1}\PY{p}{,} \PY{n}{cmap}\PY{o}{=}\PY{n}{matplotlib}\PY{o}{.}\PY{n}{cm}\PY{o}{.}\PY{n}{coolwarm}\PY{p}{,} \PY{n}{linewidth}\PY{o}{=}\PY{l+m+mi}{0}\PY{p}{,} \PY{n}{antialiased}\PY{o}{=}\PY{k+kc}{False}\PY{p}{)}
\PY{n}{cb} \PY{o}{=} \PY{n}{fig}\PY{o}{.}\PY{n}{colorbar}\PY{p}{(}\PY{n}{p}\PY{p}{,} \PY{n}{shrink}\PY{o}{=}\PY{l+m+mf}{0.5}\PY{p}{)}
\end{Verbatim}
\end{tcolorbox}

    \begin{center}
    \adjustimage{max size={0.9\linewidth}{0.9\paperheight}}{Lecture-4-Matplotlib_files/Lecture-4-Matplotlib_145_0.png}
    \end{center}
    { \hspace*{\fill} \\}
    
    \hypertarget{wire-frame-plot}{%
\paragraph{Wire-frame plot}\label{wire-frame-plot}}

    \begin{tcolorbox}[breakable, size=fbox, boxrule=1pt, pad at break*=1mm,colback=cellbackground, colframe=cellborder]
\prompt{In}{incolor}{63}{\boxspacing}
\begin{Verbatim}[commandchars=\\\{\}]
\PY{n}{fig} \PY{o}{=} \PY{n}{plt}\PY{o}{.}\PY{n}{figure}\PY{p}{(}\PY{n}{figsize}\PY{o}{=}\PY{p}{(}\PY{l+m+mi}{8}\PY{p}{,}\PY{l+m+mi}{6}\PY{p}{)}\PY{p}{)}

\PY{n}{ax} \PY{o}{=} \PY{n}{fig}\PY{o}{.}\PY{n}{add\PYZus{}subplot}\PY{p}{(}\PY{l+m+mi}{1}\PY{p}{,} \PY{l+m+mi}{1}\PY{p}{,} \PY{l+m+mi}{1}\PY{p}{,} \PY{n}{projection}\PY{o}{=}\PY{l+s+s1}{\PYZsq{}}\PY{l+s+s1}{3d}\PY{l+s+s1}{\PYZsq{}}\PY{p}{)}

\PY{n}{p} \PY{o}{=} \PY{n}{ax}\PY{o}{.}\PY{n}{plot\PYZus{}wireframe}\PY{p}{(}\PY{n}{X}\PY{p}{,} \PY{n}{Y}\PY{p}{,} \PY{n}{Z}\PY{p}{,} \PY{n}{rstride}\PY{o}{=}\PY{l+m+mi}{4}\PY{p}{,} \PY{n}{cstride}\PY{o}{=}\PY{l+m+mi}{4}\PY{p}{)}
\end{Verbatim}
\end{tcolorbox}

    \begin{center}
    \adjustimage{max size={0.9\linewidth}{0.9\paperheight}}{Lecture-4-Matplotlib_files/Lecture-4-Matplotlib_147_0.png}
    \end{center}
    { \hspace*{\fill} \\}
    
    \hypertarget{coutour-plots-with-projections}{%
\paragraph{Coutour plots with
projections}\label{coutour-plots-with-projections}}

    \begin{tcolorbox}[breakable, size=fbox, boxrule=1pt, pad at break*=1mm,colback=cellbackground, colframe=cellborder]
\prompt{In}{incolor}{64}{\boxspacing}
\begin{Verbatim}[commandchars=\\\{\}]
\PY{n}{fig} \PY{o}{=} \PY{n}{plt}\PY{o}{.}\PY{n}{figure}\PY{p}{(}\PY{n}{figsize}\PY{o}{=}\PY{p}{(}\PY{l+m+mi}{8}\PY{p}{,}\PY{l+m+mi}{6}\PY{p}{)}\PY{p}{)}

\PY{n}{ax} \PY{o}{=} \PY{n}{fig}\PY{o}{.}\PY{n}{add\PYZus{}subplot}\PY{p}{(}\PY{l+m+mi}{1}\PY{p}{,}\PY{l+m+mi}{1}\PY{p}{,}\PY{l+m+mi}{1}\PY{p}{,} \PY{n}{projection}\PY{o}{=}\PY{l+s+s1}{\PYZsq{}}\PY{l+s+s1}{3d}\PY{l+s+s1}{\PYZsq{}}\PY{p}{)}

\PY{n}{ax}\PY{o}{.}\PY{n}{plot\PYZus{}surface}\PY{p}{(}\PY{n}{X}\PY{p}{,} \PY{n}{Y}\PY{p}{,} \PY{n}{Z}\PY{p}{,} \PY{n}{rstride}\PY{o}{=}\PY{l+m+mi}{4}\PY{p}{,} \PY{n}{cstride}\PY{o}{=}\PY{l+m+mi}{4}\PY{p}{,} \PY{n}{alpha}\PY{o}{=}\PY{l+m+mf}{0.25}\PY{p}{)}
\PY{n}{cset} \PY{o}{=} \PY{n}{ax}\PY{o}{.}\PY{n}{contour}\PY{p}{(}\PY{n}{X}\PY{p}{,} \PY{n}{Y}\PY{p}{,} \PY{n}{Z}\PY{p}{,} \PY{n}{zdir}\PY{o}{=}\PY{l+s+s1}{\PYZsq{}}\PY{l+s+s1}{z}\PY{l+s+s1}{\PYZsq{}}\PY{p}{,} \PY{n}{offset}\PY{o}{=}\PY{o}{\PYZhy{}}\PY{n}{np}\PY{o}{.}\PY{n}{pi}\PY{p}{,} \PY{n}{cmap}\PY{o}{=}\PY{n}{matplotlib}\PY{o}{.}\PY{n}{cm}\PY{o}{.}\PY{n}{coolwarm}\PY{p}{)}
\PY{n}{cset} \PY{o}{=} \PY{n}{ax}\PY{o}{.}\PY{n}{contour}\PY{p}{(}\PY{n}{X}\PY{p}{,} \PY{n}{Y}\PY{p}{,} \PY{n}{Z}\PY{p}{,} \PY{n}{zdir}\PY{o}{=}\PY{l+s+s1}{\PYZsq{}}\PY{l+s+s1}{x}\PY{l+s+s1}{\PYZsq{}}\PY{p}{,} \PY{n}{offset}\PY{o}{=}\PY{o}{\PYZhy{}}\PY{n}{np}\PY{o}{.}\PY{n}{pi}\PY{p}{,} \PY{n}{cmap}\PY{o}{=}\PY{n}{matplotlib}\PY{o}{.}\PY{n}{cm}\PY{o}{.}\PY{n}{coolwarm}\PY{p}{)}
\PY{n}{cset} \PY{o}{=} \PY{n}{ax}\PY{o}{.}\PY{n}{contour}\PY{p}{(}\PY{n}{X}\PY{p}{,} \PY{n}{Y}\PY{p}{,} \PY{n}{Z}\PY{p}{,} \PY{n}{zdir}\PY{o}{=}\PY{l+s+s1}{\PYZsq{}}\PY{l+s+s1}{y}\PY{l+s+s1}{\PYZsq{}}\PY{p}{,} \PY{n}{offset}\PY{o}{=}\PY{l+m+mi}{3}\PY{o}{*}\PY{n}{np}\PY{o}{.}\PY{n}{pi}\PY{p}{,} \PY{n}{cmap}\PY{o}{=}\PY{n}{matplotlib}\PY{o}{.}\PY{n}{cm}\PY{o}{.}\PY{n}{coolwarm}\PY{p}{)}

\PY{n}{ax}\PY{o}{.}\PY{n}{set\PYZus{}xlim3d}\PY{p}{(}\PY{o}{\PYZhy{}}\PY{n}{np}\PY{o}{.}\PY{n}{pi}\PY{p}{,} \PY{l+m+mi}{2}\PY{o}{*}\PY{n}{np}\PY{o}{.}\PY{n}{pi}\PY{p}{)}\PY{p}{;}
\PY{n}{ax}\PY{o}{.}\PY{n}{set\PYZus{}ylim3d}\PY{p}{(}\PY{l+m+mi}{0}\PY{p}{,} \PY{l+m+mi}{3}\PY{o}{*}\PY{n}{np}\PY{o}{.}\PY{n}{pi}\PY{p}{)}\PY{p}{;}
\PY{n}{ax}\PY{o}{.}\PY{n}{set\PYZus{}zlim3d}\PY{p}{(}\PY{o}{\PYZhy{}}\PY{n}{np}\PY{o}{.}\PY{n}{pi}\PY{p}{,} \PY{l+m+mi}{2}\PY{o}{*}\PY{n}{np}\PY{o}{.}\PY{n}{pi}\PY{p}{)}\PY{p}{;}
\end{Verbatim}
\end{tcolorbox}

    \begin{center}
    \adjustimage{max size={0.9\linewidth}{0.9\paperheight}}{Lecture-4-Matplotlib_files/Lecture-4-Matplotlib_149_0.png}
    \end{center}
    { \hspace*{\fill} \\}
    
    \hypertarget{change-the-view-angle}{%
\paragraph{Change the view angle}\label{change-the-view-angle}}

    We can change the perspective of a 3D plot using the \texttt{view\_init}
method, which takes two arguments: \texttt{elevation} and
\texttt{azimuth} angle (in degrees):

    \begin{tcolorbox}[breakable, size=fbox, boxrule=1pt, pad at break*=1mm,colback=cellbackground, colframe=cellborder]
\prompt{In}{incolor}{65}{\boxspacing}
\begin{Verbatim}[commandchars=\\\{\}]
\PY{n}{fig} \PY{o}{=} \PY{n}{plt}\PY{o}{.}\PY{n}{figure}\PY{p}{(}\PY{n}{figsize}\PY{o}{=}\PY{p}{(}\PY{l+m+mi}{12}\PY{p}{,}\PY{l+m+mi}{6}\PY{p}{)}\PY{p}{)}

\PY{n}{ax} \PY{o}{=} \PY{n}{fig}\PY{o}{.}\PY{n}{add\PYZus{}subplot}\PY{p}{(}\PY{l+m+mi}{1}\PY{p}{,}\PY{l+m+mi}{2}\PY{p}{,}\PY{l+m+mi}{1}\PY{p}{,} \PY{n}{projection}\PY{o}{=}\PY{l+s+s1}{\PYZsq{}}\PY{l+s+s1}{3d}\PY{l+s+s1}{\PYZsq{}}\PY{p}{)}
\PY{n}{ax}\PY{o}{.}\PY{n}{plot\PYZus{}surface}\PY{p}{(}\PY{n}{X}\PY{p}{,} \PY{n}{Y}\PY{p}{,} \PY{n}{Z}\PY{p}{,} \PY{n}{rstride}\PY{o}{=}\PY{l+m+mi}{4}\PY{p}{,} \PY{n}{cstride}\PY{o}{=}\PY{l+m+mi}{4}\PY{p}{,} \PY{n}{alpha}\PY{o}{=}\PY{l+m+mf}{0.25}\PY{p}{)}
\PY{n}{ax}\PY{o}{.}\PY{n}{view\PYZus{}init}\PY{p}{(}\PY{l+m+mi}{30}\PY{p}{,} \PY{l+m+mi}{45}\PY{p}{)}

\PY{n}{ax} \PY{o}{=} \PY{n}{fig}\PY{o}{.}\PY{n}{add\PYZus{}subplot}\PY{p}{(}\PY{l+m+mi}{1}\PY{p}{,}\PY{l+m+mi}{2}\PY{p}{,}\PY{l+m+mi}{2}\PY{p}{,} \PY{n}{projection}\PY{o}{=}\PY{l+s+s1}{\PYZsq{}}\PY{l+s+s1}{3d}\PY{l+s+s1}{\PYZsq{}}\PY{p}{)}
\PY{n}{ax}\PY{o}{.}\PY{n}{plot\PYZus{}surface}\PY{p}{(}\PY{n}{X}\PY{p}{,} \PY{n}{Y}\PY{p}{,} \PY{n}{Z}\PY{p}{,} \PY{n}{rstride}\PY{o}{=}\PY{l+m+mi}{4}\PY{p}{,} \PY{n}{cstride}\PY{o}{=}\PY{l+m+mi}{4}\PY{p}{,} \PY{n}{alpha}\PY{o}{=}\PY{l+m+mf}{0.25}\PY{p}{)}
\PY{n}{ax}\PY{o}{.}\PY{n}{view\PYZus{}init}\PY{p}{(}\PY{l+m+mi}{70}\PY{p}{,} \PY{l+m+mi}{30}\PY{p}{)}

\PY{n}{fig}\PY{o}{.}\PY{n}{tight\PYZus{}layout}\PY{p}{(}\PY{p}{)}
\end{Verbatim}
\end{tcolorbox}

    \begin{center}
    \adjustimage{max size={0.9\linewidth}{0.9\paperheight}}{Lecture-4-Matplotlib_files/Lecture-4-Matplotlib_152_0.png}
    \end{center}
    { \hspace*{\fill} \\}
    
    \hypertarget{animations}{%
\subsubsection{Animations}\label{animations}}

    Matplotlib also includes a simple API for generating animations for
sequences of figures. With the \texttt{FuncAnimation} function we can
generate a movie file from sequences of figures. The function takes the
following arguments: \texttt{fig}, a figure canvas, \texttt{func}, a
function that we provide which updates the figure, \texttt{init\_func},
a function we provide to setup the figure, \texttt{frame}, the number of
frames to generate, and \texttt{blit}, which tells the animation
function to only update parts of the frame which have changed (for
smoother animations):

\begin{verbatim}
def init():
    # setup figure

def update(frame_counter):
    # update figure for new frame

anim = animation.FuncAnimation(fig, update, init_func=init, frames=200, blit=True)

anim.save('animation.mp4', fps=30) # fps = frames per second
\end{verbatim}

To use the animation features in matplotlib we first need to import the
module \texttt{matplotlib.animation}:

    \begin{tcolorbox}[breakable, size=fbox, boxrule=1pt, pad at break*=1mm,colback=cellbackground, colframe=cellborder]
\prompt{In}{incolor}{66}{\boxspacing}
\begin{Verbatim}[commandchars=\\\{\}]
\PY{k+kn}{from} \PY{n+nn}{matplotlib} \PY{k+kn}{import} \PY{n}{animation}
\end{Verbatim}
\end{tcolorbox}

    \begin{tcolorbox}[breakable, size=fbox, boxrule=1pt, pad at break*=1mm,colback=cellbackground, colframe=cellborder]
\prompt{In}{incolor}{67}{\boxspacing}
\begin{Verbatim}[commandchars=\\\{\}]
\PY{c+c1}{\PYZsh{} solve the ode problem of the double compound pendulum again}

\PY{k+kn}{from} \PY{n+nn}{scipy}\PY{n+nn}{.}\PY{n+nn}{integrate} \PY{k+kn}{import} \PY{n}{odeint}
\PY{k+kn}{from} \PY{n+nn}{numpy} \PY{k+kn}{import} \PY{n}{cos}\PY{p}{,} \PY{n}{sin}

\PY{n}{g} \PY{o}{=} \PY{l+m+mf}{9.82}\PY{p}{;} \PY{n}{L} \PY{o}{=} \PY{l+m+mf}{0.5}\PY{p}{;} \PY{n}{m} \PY{o}{=} \PY{l+m+mf}{0.1}

\PY{k}{def} \PY{n+nf}{dx}\PY{p}{(}\PY{n}{x}\PY{p}{,} \PY{n}{t}\PY{p}{)}\PY{p}{:}
    \PY{n}{x1}\PY{p}{,} \PY{n}{x2}\PY{p}{,} \PY{n}{x3}\PY{p}{,} \PY{n}{x4} \PY{o}{=} \PY{n}{x}\PY{p}{[}\PY{l+m+mi}{0}\PY{p}{]}\PY{p}{,} \PY{n}{x}\PY{p}{[}\PY{l+m+mi}{1}\PY{p}{]}\PY{p}{,} \PY{n}{x}\PY{p}{[}\PY{l+m+mi}{2}\PY{p}{]}\PY{p}{,} \PY{n}{x}\PY{p}{[}\PY{l+m+mi}{3}\PY{p}{]}
    
    \PY{n}{dx1} \PY{o}{=} \PY{l+m+mf}{6.0}\PY{o}{/}\PY{p}{(}\PY{n}{m}\PY{o}{*}\PY{n}{L}\PY{o}{*}\PY{o}{*}\PY{l+m+mi}{2}\PY{p}{)} \PY{o}{*} \PY{p}{(}\PY{l+m+mi}{2} \PY{o}{*} \PY{n}{x3} \PY{o}{\PYZhy{}} \PY{l+m+mi}{3} \PY{o}{*} \PY{n}{cos}\PY{p}{(}\PY{n}{x1}\PY{o}{\PYZhy{}}\PY{n}{x2}\PY{p}{)} \PY{o}{*} \PY{n}{x4}\PY{p}{)}\PY{o}{/}\PY{p}{(}\PY{l+m+mi}{16} \PY{o}{\PYZhy{}} \PY{l+m+mi}{9} \PY{o}{*} \PY{n}{cos}\PY{p}{(}\PY{n}{x1}\PY{o}{\PYZhy{}}\PY{n}{x2}\PY{p}{)}\PY{o}{*}\PY{o}{*}\PY{l+m+mi}{2}\PY{p}{)}
    \PY{n}{dx2} \PY{o}{=} \PY{l+m+mf}{6.0}\PY{o}{/}\PY{p}{(}\PY{n}{m}\PY{o}{*}\PY{n}{L}\PY{o}{*}\PY{o}{*}\PY{l+m+mi}{2}\PY{p}{)} \PY{o}{*} \PY{p}{(}\PY{l+m+mi}{8} \PY{o}{*} \PY{n}{x4} \PY{o}{\PYZhy{}} \PY{l+m+mi}{3} \PY{o}{*} \PY{n}{cos}\PY{p}{(}\PY{n}{x1}\PY{o}{\PYZhy{}}\PY{n}{x2}\PY{p}{)} \PY{o}{*} \PY{n}{x3}\PY{p}{)}\PY{o}{/}\PY{p}{(}\PY{l+m+mi}{16} \PY{o}{\PYZhy{}} \PY{l+m+mi}{9} \PY{o}{*} \PY{n}{cos}\PY{p}{(}\PY{n}{x1}\PY{o}{\PYZhy{}}\PY{n}{x2}\PY{p}{)}\PY{o}{*}\PY{o}{*}\PY{l+m+mi}{2}\PY{p}{)}
    \PY{n}{dx3} \PY{o}{=} \PY{o}{\PYZhy{}}\PY{l+m+mf}{0.5} \PY{o}{*} \PY{n}{m} \PY{o}{*} \PY{n}{L}\PY{o}{*}\PY{o}{*}\PY{l+m+mi}{2} \PY{o}{*} \PY{p}{(} \PY{n}{dx1} \PY{o}{*} \PY{n}{dx2} \PY{o}{*} \PY{n}{sin}\PY{p}{(}\PY{n}{x1}\PY{o}{\PYZhy{}}\PY{n}{x2}\PY{p}{)} \PY{o}{+} \PY{l+m+mi}{3} \PY{o}{*} \PY{p}{(}\PY{n}{g}\PY{o}{/}\PY{n}{L}\PY{p}{)} \PY{o}{*} \PY{n}{sin}\PY{p}{(}\PY{n}{x1}\PY{p}{)}\PY{p}{)}
    \PY{n}{dx4} \PY{o}{=} \PY{o}{\PYZhy{}}\PY{l+m+mf}{0.5} \PY{o}{*} \PY{n}{m} \PY{o}{*} \PY{n}{L}\PY{o}{*}\PY{o}{*}\PY{l+m+mi}{2} \PY{o}{*} \PY{p}{(}\PY{o}{\PYZhy{}}\PY{n}{dx1} \PY{o}{*} \PY{n}{dx2} \PY{o}{*} \PY{n}{sin}\PY{p}{(}\PY{n}{x1}\PY{o}{\PYZhy{}}\PY{n}{x2}\PY{p}{)} \PY{o}{+} \PY{p}{(}\PY{n}{g}\PY{o}{/}\PY{n}{L}\PY{p}{)} \PY{o}{*} \PY{n}{sin}\PY{p}{(}\PY{n}{x2}\PY{p}{)}\PY{p}{)}
    \PY{k}{return} \PY{p}{[}\PY{n}{dx1}\PY{p}{,} \PY{n}{dx2}\PY{p}{,} \PY{n}{dx3}\PY{p}{,} \PY{n}{dx4}\PY{p}{]}

\PY{n}{x0} \PY{o}{=} \PY{p}{[}\PY{n}{np}\PY{o}{.}\PY{n}{pi}\PY{o}{/}\PY{l+m+mi}{2}\PY{p}{,} \PY{n}{np}\PY{o}{.}\PY{n}{pi}\PY{o}{/}\PY{l+m+mi}{2}\PY{p}{,} \PY{l+m+mi}{0}\PY{p}{,} \PY{l+m+mi}{0}\PY{p}{]}  \PY{c+c1}{\PYZsh{} initial state}
\PY{n}{t} \PY{o}{=} \PY{n}{np}\PY{o}{.}\PY{n}{linspace}\PY{p}{(}\PY{l+m+mi}{0}\PY{p}{,} \PY{l+m+mi}{10}\PY{p}{,} \PY{l+m+mi}{250}\PY{p}{)} \PY{c+c1}{\PYZsh{} time coordinates}
\PY{n}{x} \PY{o}{=} \PY{n}{odeint}\PY{p}{(}\PY{n}{dx}\PY{p}{,} \PY{n}{x0}\PY{p}{,} \PY{n}{t}\PY{p}{)}    \PY{c+c1}{\PYZsh{} solve the ODE problem}
\end{Verbatim}
\end{tcolorbox}

    Generate an animation that shows the positions of the pendulums as a
function of time:

    \begin{tcolorbox}[breakable, size=fbox, boxrule=1pt, pad at break*=1mm,colback=cellbackground, colframe=cellborder]
\prompt{In}{incolor}{68}{\boxspacing}
\begin{Verbatim}[commandchars=\\\{\}]
\PY{n}{fig}\PY{p}{,} \PY{n}{ax} \PY{o}{=} \PY{n}{plt}\PY{o}{.}\PY{n}{subplots}\PY{p}{(}\PY{n}{figsize}\PY{o}{=}\PY{p}{(}\PY{l+m+mi}{5}\PY{p}{,}\PY{l+m+mi}{5}\PY{p}{)}\PY{p}{)}

\PY{n}{ax}\PY{o}{.}\PY{n}{set\PYZus{}ylim}\PY{p}{(}\PY{p}{[}\PY{o}{\PYZhy{}}\PY{l+m+mf}{1.5}\PY{p}{,} \PY{l+m+mf}{0.5}\PY{p}{]}\PY{p}{)}
\PY{n}{ax}\PY{o}{.}\PY{n}{set\PYZus{}xlim}\PY{p}{(}\PY{p}{[}\PY{l+m+mi}{1}\PY{p}{,} \PY{o}{\PYZhy{}}\PY{l+m+mi}{1}\PY{p}{]}\PY{p}{)}

\PY{n}{pendulum1}\PY{p}{,} \PY{o}{=} \PY{n}{ax}\PY{o}{.}\PY{n}{plot}\PY{p}{(}\PY{p}{[}\PY{p}{]}\PY{p}{,} \PY{p}{[}\PY{p}{]}\PY{p}{,} \PY{n}{color}\PY{o}{=}\PY{l+s+s2}{\PYZdq{}}\PY{l+s+s2}{red}\PY{l+s+s2}{\PYZdq{}}\PY{p}{,} \PY{n}{lw}\PY{o}{=}\PY{l+m+mi}{2}\PY{p}{)}
\PY{n}{pendulum2}\PY{p}{,} \PY{o}{=} \PY{n}{ax}\PY{o}{.}\PY{n}{plot}\PY{p}{(}\PY{p}{[}\PY{p}{]}\PY{p}{,} \PY{p}{[}\PY{p}{]}\PY{p}{,} \PY{n}{color}\PY{o}{=}\PY{l+s+s2}{\PYZdq{}}\PY{l+s+s2}{blue}\PY{l+s+s2}{\PYZdq{}}\PY{p}{,} \PY{n}{lw}\PY{o}{=}\PY{l+m+mi}{2}\PY{p}{)}

\PY{k}{def} \PY{n+nf}{init}\PY{p}{(}\PY{p}{)}\PY{p}{:}
    \PY{n}{pendulum1}\PY{o}{.}\PY{n}{set\PYZus{}data}\PY{p}{(}\PY{p}{[}\PY{p}{]}\PY{p}{,} \PY{p}{[}\PY{p}{]}\PY{p}{)}
    \PY{n}{pendulum2}\PY{o}{.}\PY{n}{set\PYZus{}data}\PY{p}{(}\PY{p}{[}\PY{p}{]}\PY{p}{,} \PY{p}{[}\PY{p}{]}\PY{p}{)}

\PY{k}{def} \PY{n+nf}{update}\PY{p}{(}\PY{n}{n}\PY{p}{)}\PY{p}{:} 
    \PY{c+c1}{\PYZsh{} n = frame counter}
    \PY{c+c1}{\PYZsh{} calculate the positions of the pendulums}
    \PY{n}{x1} \PY{o}{=} \PY{o}{+} \PY{n}{L} \PY{o}{*} \PY{n}{sin}\PY{p}{(}\PY{n}{x}\PY{p}{[}\PY{n}{n}\PY{p}{,} \PY{l+m+mi}{0}\PY{p}{]}\PY{p}{)}
    \PY{n}{y1} \PY{o}{=} \PY{o}{\PYZhy{}} \PY{n}{L} \PY{o}{*} \PY{n}{cos}\PY{p}{(}\PY{n}{x}\PY{p}{[}\PY{n}{n}\PY{p}{,} \PY{l+m+mi}{0}\PY{p}{]}\PY{p}{)}
    \PY{n}{x2} \PY{o}{=} \PY{n}{x1} \PY{o}{+} \PY{n}{L} \PY{o}{*} \PY{n}{sin}\PY{p}{(}\PY{n}{x}\PY{p}{[}\PY{n}{n}\PY{p}{,} \PY{l+m+mi}{1}\PY{p}{]}\PY{p}{)}
    \PY{n}{y2} \PY{o}{=} \PY{n}{y1} \PY{o}{\PYZhy{}} \PY{n}{L} \PY{o}{*} \PY{n}{cos}\PY{p}{(}\PY{n}{x}\PY{p}{[}\PY{n}{n}\PY{p}{,} \PY{l+m+mi}{1}\PY{p}{]}\PY{p}{)}
    
    \PY{c+c1}{\PYZsh{} update the line data}
    \PY{n}{pendulum1}\PY{o}{.}\PY{n}{set\PYZus{}data}\PY{p}{(}\PY{p}{[}\PY{l+m+mi}{0} \PY{p}{,}\PY{n}{x1}\PY{p}{]}\PY{p}{,} \PY{p}{[}\PY{l+m+mi}{0} \PY{p}{,}\PY{n}{y1}\PY{p}{]}\PY{p}{)}
    \PY{n}{pendulum2}\PY{o}{.}\PY{n}{set\PYZus{}data}\PY{p}{(}\PY{p}{[}\PY{n}{x1}\PY{p}{,}\PY{n}{x2}\PY{p}{]}\PY{p}{,} \PY{p}{[}\PY{n}{y1}\PY{p}{,}\PY{n}{y2}\PY{p}{]}\PY{p}{)}

\PY{n}{anim} \PY{o}{=} \PY{n}{animation}\PY{o}{.}\PY{n}{FuncAnimation}\PY{p}{(}\PY{n}{fig}\PY{p}{,} \PY{n}{update}\PY{p}{,} \PY{n}{init\PYZus{}func}\PY{o}{=}\PY{n}{init}\PY{p}{,} \PY{n}{frames}\PY{o}{=}\PY{n+nb}{len}\PY{p}{(}\PY{n}{t}\PY{p}{)}\PY{p}{,} \PY{n}{blit}\PY{o}{=}\PY{k+kc}{True}\PY{p}{)}

\PY{c+c1}{\PYZsh{} anim.save can be called in a few different ways, some which might or might not work}
\PY{c+c1}{\PYZsh{} on different platforms and with different versions of matplotlib and video encoders}
\PY{c+c1}{\PYZsh{}anim.save(\PYZsq{}animation.mp4\PYZsq{}, fps=20, extra\PYZus{}args=[\PYZsq{}\PYZhy{}vcodec\PYZsq{}, \PYZsq{}libx264\PYZsq{}], writer=animation.FFMpegWriter())}
\PY{c+c1}{\PYZsh{}anim.save(\PYZsq{}animation.mp4\PYZsq{}, fps=20, extra\PYZus{}args=[\PYZsq{}\PYZhy{}vcodec\PYZsq{}, \PYZsq{}libx264\PYZsq{}])}
\PY{c+c1}{\PYZsh{}anim.save(\PYZsq{}animation.mp4\PYZsq{}, fps=20, writer=\PYZdq{}ffmpeg\PYZdq{}, codec=\PYZdq{}libx264\PYZdq{})}
\PY{n}{anim}\PY{o}{.}\PY{n}{save}\PY{p}{(}\PY{l+s+s1}{\PYZsq{}}\PY{l+s+s1}{animation.mp4}\PY{l+s+s1}{\PYZsq{}}\PY{p}{,} \PY{n}{fps}\PY{o}{=}\PY{l+m+mi}{20}\PY{p}{,} \PY{n}{writer}\PY{o}{=}\PY{l+s+s2}{\PYZdq{}}\PY{l+s+s2}{avconv}\PY{l+s+s2}{\PYZdq{}}\PY{p}{,} \PY{n}{codec}\PY{o}{=}\PY{l+s+s2}{\PYZdq{}}\PY{l+s+s2}{libx264}\PY{l+s+s2}{\PYZdq{}}\PY{p}{)}

\PY{n}{plt}\PY{o}{.}\PY{n}{close}\PY{p}{(}\PY{n}{fig}\PY{p}{)}
\end{Verbatim}
\end{tcolorbox}

    \begin{Verbatim}[commandchars=\\\{\}]
Traceback (most recent call last):
  File "c:\textbackslash{}program files\textbackslash{}python38\textbackslash{}lib\textbackslash{}site-
packages\textbackslash{}matplotlib\textbackslash{}cbook\textbackslash{}\_\_init\_\_.py", line 196, in process
    func(*args, **kwargs)
  File "c:\textbackslash{}program files\textbackslash{}python38\textbackslash{}lib\textbackslash{}site-packages\textbackslash{}matplotlib\textbackslash{}animation.py",
line 951, in \_start
    self.\_init\_draw()
  File "c:\textbackslash{}program files\textbackslash{}python38\textbackslash{}lib\textbackslash{}site-packages\textbackslash{}matplotlib\textbackslash{}animation.py",
line 1749, in \_init\_draw
    raise RuntimeError('The init\_func must return a '
RuntimeError: The init\_func must return a sequence of Artist objects.
    \end{Verbatim}

    \begin{Verbatim}[commandchars=\\\{\}]

        ---------------------------------------------------------------------------

        TypeError                                 Traceback (most recent call last)

        <ipython-input-68-d981632d7f5f> in <module>
         30 \#anim.save('animation.mp4', fps=20, extra\_args=['-vcodec', 'libx264'])
         31 \#anim.save('animation.mp4', fps=20, writer="ffmpeg", codec="libx264")
    ---> 32 anim.save('animation.mp4', fps=20, writer="avconv", codec="libx264")
         33 
         34 plt.close(fig)
    

        c:\textbackslash{}program files\textbackslash{}python38\textbackslash{}lib\textbackslash{}site-packages\textbackslash{}matplotlib\textbackslash{}animation.py in save(self, filename, writer, fps, dpi, codec, bitrate, extra\_args, metadata, extra\_anim, savefig\_kwargs, progress\_callback)
       1100                                          metadata=metadata)
       1101             else:
    -> 1102                 alt\_writer = next(writers, None)
       1103                 if alt\_writer is None:
       1104                     raise ValueError("Cannot save animation: no writers are "
    

        TypeError: 'MovieWriterRegistry' object is not an iterator

    \end{Verbatim}

    \begin{center}
    \adjustimage{max size={0.9\linewidth}{0.9\paperheight}}{Lecture-4-Matplotlib_files/Lecture-4-Matplotlib_158_2.png}
    \end{center}
    { \hspace*{\fill} \\}
    
    Note: To generate the movie file we need to have either \texttt{ffmpeg}
or \texttt{avconv} installed. Install it on Ubuntu using:

\begin{verbatim}
$ sudo apt-get install ffmpeg
\end{verbatim}

or (newer versions)

\begin{verbatim}
$ sudo apt-get install libav-tools
\end{verbatim}

On MacOSX, try:

\begin{verbatim}
$ sudo port install ffmpeg
\end{verbatim}

    \begin{tcolorbox}[breakable, size=fbox, boxrule=1pt, pad at break*=1mm,colback=cellbackground, colframe=cellborder]
\prompt{In}{incolor}{69}{\boxspacing}
\begin{Verbatim}[commandchars=\\\{\}]
\PY{k+kn}{from} \PY{n+nn}{IPython}\PY{n+nn}{.}\PY{n+nn}{display} \PY{k+kn}{import} \PY{n}{HTML}
\PY{n}{video} \PY{o}{=} \PY{n+nb}{open}\PY{p}{(}\PY{l+s+s2}{\PYZdq{}}\PY{l+s+s2}{animation.mp4}\PY{l+s+s2}{\PYZdq{}}\PY{p}{,} \PY{l+s+s2}{\PYZdq{}}\PY{l+s+s2}{rb}\PY{l+s+s2}{\PYZdq{}}\PY{p}{)}\PY{o}{.}\PY{n}{read}\PY{p}{(}\PY{p}{)}
\PY{n}{video\PYZus{}encoded} \PY{o}{=} \PY{n}{video}\PY{o}{.}\PY{n}{encode}\PY{p}{(}\PY{l+s+s2}{\PYZdq{}}\PY{l+s+s2}{base64}\PY{l+s+s2}{\PYZdq{}}\PY{p}{)}
\PY{n}{video\PYZus{}tag} \PY{o}{=} \PY{l+s+s1}{\PYZsq{}}\PY{l+s+s1}{\PYZlt{}video controls alt=}\PY{l+s+s1}{\PYZdq{}}\PY{l+s+s1}{test}\PY{l+s+s1}{\PYZdq{}}\PY{l+s+s1}{ src=}\PY{l+s+s1}{\PYZdq{}}\PY{l+s+s1}{data:video/x\PYZhy{}m4v;base64,}\PY{l+s+si}{\PYZob{}0\PYZcb{}}\PY{l+s+s1}{\PYZdq{}}\PY{l+s+s1}{\PYZgt{}}\PY{l+s+s1}{\PYZsq{}}\PY{o}{.}\PY{n}{format}\PY{p}{(}\PY{n}{video\PYZus{}encoded}\PY{p}{)}
\PY{n}{HTML}\PY{p}{(}\PY{n}{video\PYZus{}tag}\PY{p}{)}
\end{Verbatim}
\end{tcolorbox}

    \begin{Verbatim}[commandchars=\\\{\}]

        ---------------------------------------------------------------------------

        FileNotFoundError                         Traceback (most recent call last)

        <ipython-input-69-353c13a5d53a> in <module>
          1 from IPython.display import HTML
    ----> 2 video = open("animation.mp4", "rb").read()
          3 video\_encoded = video.encode("base64")
          4 video\_tag = '<video controls alt="test" src="data:video/x-m4v;base64,\{0\}">'.format(video\_encoded)
          5 HTML(video\_tag)
    

        FileNotFoundError: [Errno 2] No such file or directory: 'animation.mp4'

    \end{Verbatim}

    \hypertarget{backends}{%
\subsubsection{Backends}\label{backends}}

    Matplotlib has a number of ``backends'' which are responsible for
rendering graphs. The different backends are able to generate graphics
with different formats and display/event loops. There is a distinction
between noninteractive backends (such as `agg', `svg', `pdf', etc.) that
are only used to generate image files (e.g.~with the \texttt{savefig}
function), and interactive backends (such as Qt4Agg, GTK, MaxOSX) that
can display a GUI window for interactively exploring figures.

A list of available backends are:

    \begin{tcolorbox}[breakable, size=fbox, boxrule=1pt, pad at break*=1mm,colback=cellbackground, colframe=cellborder]
\prompt{In}{incolor}{70}{\boxspacing}
\begin{Verbatim}[commandchars=\\\{\}]
\PY{n+nb}{print}\PY{p}{(}\PY{n}{matplotlib}\PY{o}{.}\PY{n}{rcsetup}\PY{o}{.}\PY{n}{all\PYZus{}backends}\PY{p}{)}
\end{Verbatim}
\end{tcolorbox}

    \begin{Verbatim}[commandchars=\\\{\}]
['GTK3Agg', 'GTK3Cairo', 'MacOSX', 'nbAgg', 'Qt4Agg', 'Qt4Cairo', 'Qt5Agg',
'Qt5Cairo', 'TkAgg', 'TkCairo', 'WebAgg', 'WX', 'WXAgg', 'WXCairo', 'agg',
'cairo', 'pdf', 'pgf', 'ps', 'svg', 'template']
    \end{Verbatim}

    The default backend, called \texttt{agg}, is based on a library for
raster graphics which is great for generating raster formats like PNG.

Normally we don't need to bother with changing the default backend; but
sometimes it can be useful to switch to, for example, PDF or GTKCairo
(if you are using Linux) to produce high-quality vector graphics instead
of raster based graphics.

    \hypertarget{generating-svg-with-the-svg-backend}{%
\paragraph{Generating SVG with the svg
backend}\label{generating-svg-with-the-svg-backend}}

    \begin{tcolorbox}[breakable, size=fbox, boxrule=1pt, pad at break*=1mm,colback=cellbackground, colframe=cellborder]
\prompt{In}{incolor}{71}{\boxspacing}
\begin{Verbatim}[commandchars=\\\{\}]
\PY{c+c1}{\PYZsh{}}
\PY{c+c1}{\PYZsh{} RESTART THE NOTEBOOK: the matplotlib backend can only be selected before pylab is imported!}
\PY{c+c1}{\PYZsh{} (e.g. Kernel \PYZgt{} Restart)}
\PY{c+c1}{\PYZsh{} }
\PY{k+kn}{import} \PY{n+nn}{matplotlib}
\PY{n}{matplotlib}\PY{o}{.}\PY{n}{use}\PY{p}{(}\PY{l+s+s1}{\PYZsq{}}\PY{l+s+s1}{svg}\PY{l+s+s1}{\PYZsq{}}\PY{p}{)}
\PY{k+kn}{import} \PY{n+nn}{matplotlib}\PY{n+nn}{.}\PY{n+nn}{pylab} \PY{k}{as} \PY{n+nn}{plt}
\PY{k+kn}{import} \PY{n+nn}{numpy}
\PY{k+kn}{from} \PY{n+nn}{IPython}\PY{n+nn}{.}\PY{n+nn}{display} \PY{k+kn}{import} \PY{n}{Image}\PY{p}{,} \PY{n}{SVG}
\end{Verbatim}
\end{tcolorbox}

    \begin{tcolorbox}[breakable, size=fbox, boxrule=1pt, pad at break*=1mm,colback=cellbackground, colframe=cellborder]
\prompt{In}{incolor}{72}{\boxspacing}
\begin{Verbatim}[commandchars=\\\{\}]
\PY{c+c1}{\PYZsh{}}
\PY{c+c1}{\PYZsh{} Now we are using the svg backend to produce SVG vector graphics}
\PY{c+c1}{\PYZsh{}}
\PY{n}{fig}\PY{p}{,} \PY{n}{ax} \PY{o}{=} \PY{n}{plt}\PY{o}{.}\PY{n}{subplots}\PY{p}{(}\PY{p}{)}
\PY{n}{t} \PY{o}{=} \PY{n}{numpy}\PY{o}{.}\PY{n}{linspace}\PY{p}{(}\PY{l+m+mi}{0}\PY{p}{,} \PY{l+m+mi}{10}\PY{p}{,} \PY{l+m+mi}{100}\PY{p}{)}
\PY{n}{ax}\PY{o}{.}\PY{n}{plot}\PY{p}{(}\PY{n}{t}\PY{p}{,} \PY{n}{numpy}\PY{o}{.}\PY{n}{cos}\PY{p}{(}\PY{n}{t}\PY{p}{)}\PY{o}{*}\PY{n}{numpy}\PY{o}{.}\PY{n}{sin}\PY{p}{(}\PY{n}{t}\PY{p}{)}\PY{p}{)}
\PY{n}{plt}\PY{o}{.}\PY{n}{savefig}\PY{p}{(}\PY{l+s+s2}{\PYZdq{}}\PY{l+s+s2}{test.svg}\PY{l+s+s2}{\PYZdq{}}\PY{p}{)}
\end{Verbatim}
\end{tcolorbox}

    \begin{center}
    \adjustimage{max size={0.9\linewidth}{0.9\paperheight}}{Lecture-4-Matplotlib_files/Lecture-4-Matplotlib_167_0.png}
    \end{center}
    { \hspace*{\fill} \\}
    
    \begin{tcolorbox}[breakable, size=fbox, boxrule=1pt, pad at break*=1mm,colback=cellbackground, colframe=cellborder]
\prompt{In}{incolor}{73}{\boxspacing}
\begin{Verbatim}[commandchars=\\\{\}]
\PY{c+c1}{\PYZsh{}}
\PY{c+c1}{\PYZsh{} Show the produced SVG file. }
\PY{c+c1}{\PYZsh{}}
\PY{n}{SVG}\PY{p}{(}\PY{n}{filename}\PY{o}{=}\PY{l+s+s2}{\PYZdq{}}\PY{l+s+s2}{test.svg}\PY{l+s+s2}{\PYZdq{}}\PY{p}{)}
\end{Verbatim}
\end{tcolorbox}
 
            
\prompt{Out}{outcolor}{73}{}
    
    \begin{center}
    \adjustimage{max size={0.9\linewidth}{0.9\paperheight}}{Lecture-4-Matplotlib_files/Lecture-4-Matplotlib_168_0.pdf}
    \end{center}
    { \hspace*{\fill} \\}
    

    \hypertarget{the-ipython-notebook-inline-backend}{%
\paragraph{The IPython notebook inline
backend}\label{the-ipython-notebook-inline-backend}}

    When we use IPython notebook it is convenient to use a matplotlib
backend that outputs the graphics embedded in the notebook file. To
activate this backend, somewhere in the beginning on the notebook, we
add:

\begin{verbatim}
%matplotlib inline
\end{verbatim}

It is also possible to activate inline matplotlib plotting with:

\begin{verbatim}
%pylab inline
\end{verbatim}

The difference is that \texttt{\%pylab\ inline} imports a number of
packages into the global address space (scipy, numpy), while
\texttt{\%matplotlib\ inline} only sets up inline plotting. In new
notebooks created for IPython 1.0+, I would recommend using
\texttt{\%matplotlib\ inline}, since it is tidier and you have more
control over which packages are imported and how. Commonly, scipy and
numpy are imported separately with:

\begin{verbatim}
import numpy as np
import scipy as sp
import matplotlib.pyplot as plt
\end{verbatim}

    The inline backend has a number of configuration options that can be set
by using the IPython magic command \texttt{\%config} to update settings
in \texttt{InlineBackend}. For example, we can switch to SVG figures or
higher resolution figures with either:

\begin{verbatim}
%config InlineBackend.figure_format='svg'
 
\end{verbatim}

or:

\begin{verbatim}
%config InlineBackend.figure_format='retina'
\end{verbatim}

For more information, type:

\begin{verbatim}
%config InlineBackend
\end{verbatim}

    \begin{tcolorbox}[breakable, size=fbox, boxrule=1pt, pad at break*=1mm,colback=cellbackground, colframe=cellborder]
\prompt{In}{incolor}{74}{\boxspacing}
\begin{Verbatim}[commandchars=\\\{\}]
\PY{o}{\PYZpc{}}\PY{k}{matplotlib} inline
\PY{o}{\PYZpc{}}\PY{k}{config} InlineBackend.figure\PYZus{}format=\PYZsq{}svg\PYZsq{}

\PY{k+kn}{import} \PY{n+nn}{matplotlib}\PY{n+nn}{.}\PY{n+nn}{pylab} \PY{k}{as} \PY{n+nn}{plt}
\PY{k+kn}{import} \PY{n+nn}{numpy}
\end{Verbatim}
\end{tcolorbox}

    \begin{tcolorbox}[breakable, size=fbox, boxrule=1pt, pad at break*=1mm,colback=cellbackground, colframe=cellborder]
\prompt{In}{incolor}{75}{\boxspacing}
\begin{Verbatim}[commandchars=\\\{\}]
\PY{c+c1}{\PYZsh{}}
\PY{c+c1}{\PYZsh{} Now we are using the SVG vector graphics displaced inline in the notebook}
\PY{c+c1}{\PYZsh{}}
\PY{n}{fig}\PY{p}{,} \PY{n}{ax} \PY{o}{=} \PY{n}{plt}\PY{o}{.}\PY{n}{subplots}\PY{p}{(}\PY{p}{)}
\PY{n}{t} \PY{o}{=} \PY{n}{numpy}\PY{o}{.}\PY{n}{linspace}\PY{p}{(}\PY{l+m+mi}{0}\PY{p}{,} \PY{l+m+mi}{10}\PY{p}{,} \PY{l+m+mi}{100}\PY{p}{)}
\PY{n}{ax}\PY{o}{.}\PY{n}{plot}\PY{p}{(}\PY{n}{t}\PY{p}{,} \PY{n}{numpy}\PY{o}{.}\PY{n}{cos}\PY{p}{(}\PY{n}{t}\PY{p}{)}\PY{o}{*}\PY{n}{numpy}\PY{o}{.}\PY{n}{sin}\PY{p}{(}\PY{n}{t}\PY{p}{)}\PY{p}{)}
\PY{n}{plt}\PY{o}{.}\PY{n}{savefig}\PY{p}{(}\PY{l+s+s2}{\PYZdq{}}\PY{l+s+s2}{test.svg}\PY{l+s+s2}{\PYZdq{}}\PY{p}{)}
\end{Verbatim}
\end{tcolorbox}

    \begin{center}
    \adjustimage{max size={0.9\linewidth}{0.9\paperheight}}{Lecture-4-Matplotlib_files/Lecture-4-Matplotlib_173_0.pdf}
    \end{center}
    { \hspace*{\fill} \\}
    
    \hypertarget{interactive-backend-this-makes-more-sense-in-a-python-script-file}{%
\paragraph{Interactive backend (this makes more sense in a python script
file)}\label{interactive-backend-this-makes-more-sense-in-a-python-script-file}}

    \begin{tcolorbox}[breakable, size=fbox, boxrule=1pt, pad at break*=1mm,colback=cellbackground, colframe=cellborder]
\prompt{In}{incolor}{76}{\boxspacing}
\begin{Verbatim}[commandchars=\\\{\}]
\PY{c+c1}{\PYZsh{}}
\PY{c+c1}{\PYZsh{} RESTART THE NOTEBOOK: the matplotlib backend can only be selected before pylab is imported!}
\PY{c+c1}{\PYZsh{} (e.g. Kernel \PYZgt{} Restart)}
\PY{c+c1}{\PYZsh{} }
\PY{k+kn}{import} \PY{n+nn}{matplotlib}
\PY{n}{matplotlib}\PY{o}{.}\PY{n}{use}\PY{p}{(}\PY{l+s+s1}{\PYZsq{}}\PY{l+s+s1}{Qt4Agg}\PY{l+s+s1}{\PYZsq{}}\PY{p}{)} \PY{c+c1}{\PYZsh{} or for example MacOSX}
\PY{k+kn}{import} \PY{n+nn}{matplotlib}\PY{n+nn}{.}\PY{n+nn}{pylab} \PY{k}{as} \PY{n+nn}{plt}
\PY{k+kn}{import} \PY{n+nn}{numpy} \PY{k}{as} \PY{n+nn}{np}
\end{Verbatim}
\end{tcolorbox}

    \begin{Verbatim}[commandchars=\\\{\}]

        ---------------------------------------------------------------------------

        ImportError                               Traceback (most recent call last)

        <ipython-input-76-faa3e9d72229> in <module>
          4 \#
          5 import matplotlib
    ----> 6 matplotlib.use('Qt4Agg') \# or for example MacOSX
          7 import matplotlib.pylab as plt
          8 import numpy as np
    

        c:\textbackslash{}program files\textbackslash{}python38\textbackslash{}lib\textbackslash{}site-packages\textbackslash{}matplotlib\textbackslash{}cbook\textbackslash{}deprecation.py in wrapper(*args, **kwargs)
        294                 f"for the old name will be dropped \%(removal)s.")
        295             kwargs[new] = kwargs.pop(old)
    --> 296         return func(*args, **kwargs)
        297 
        298     \# wrapper() must keep the same documented signature as func(): if we
    

        c:\textbackslash{}program files\textbackslash{}python38\textbackslash{}lib\textbackslash{}site-packages\textbackslash{}matplotlib\textbackslash{}cbook\textbackslash{}deprecation.py in wrapper(*args, **kwargs)
        356                 f"\%(removal)s.  If any parameter follows \{name!r\}, they "
        357                 f"should be pass as keyword, not positionally.")
    --> 358         return func(*args, **kwargs)
        359 
        360     return wrapper
    

        c:\textbackslash{}program files\textbackslash{}python38\textbackslash{}lib\textbackslash{}site-packages\textbackslash{}matplotlib\textbackslash{}\_\_init\_\_.py in use(backend, warn, force)
       1279         try:
       1280             from matplotlib import pyplot as plt
    -> 1281             plt.switch\_backend(name)
       1282         except ImportError as exc:
       1283             if force:
    

        c:\textbackslash{}program files\textbackslash{}python38\textbackslash{}lib\textbackslash{}site-packages\textbackslash{}matplotlib\textbackslash{}pyplot.py in switch\_backend(newbackend)
        219         else "matplotlib.backends.backend\_\{\}".format(newbackend.lower()))
        220 
    --> 221     backend\_mod = importlib.import\_module(backend\_name)
        222     Backend = type(
        223         "Backend", (matplotlib.backends.\_Backend,), vars(backend\_mod))
    

        c:\textbackslash{}program files\textbackslash{}python38\textbackslash{}lib\textbackslash{}importlib\textbackslash{}\_\_init\_\_.py in import\_module(name, package)
        125                 break
        126             level += 1
    --> 127     return \_bootstrap.\_gcd\_import(name[level:], package, level)
        128 
        129 
    

        c:\textbackslash{}program files\textbackslash{}python38\textbackslash{}lib\textbackslash{}importlib\textbackslash{}\_bootstrap.py in \_gcd\_import(name, package, level)
    

        c:\textbackslash{}program files\textbackslash{}python38\textbackslash{}lib\textbackslash{}importlib\textbackslash{}\_bootstrap.py in \_find\_and\_load(name, import\_)
    

        c:\textbackslash{}program files\textbackslash{}python38\textbackslash{}lib\textbackslash{}importlib\textbackslash{}\_bootstrap.py in \_find\_and\_load\_unlocked(name, import\_)
    

        c:\textbackslash{}program files\textbackslash{}python38\textbackslash{}lib\textbackslash{}importlib\textbackslash{}\_bootstrap.py in \_load\_unlocked(spec)
    

        c:\textbackslash{}program files\textbackslash{}python38\textbackslash{}lib\textbackslash{}importlib\textbackslash{}\_bootstrap\_external.py in exec\_module(self, module)
    

        c:\textbackslash{}program files\textbackslash{}python38\textbackslash{}lib\textbackslash{}importlib\textbackslash{}\_bootstrap.py in \_call\_with\_frames\_removed(f, *args, **kwds)
    

        c:\textbackslash{}program files\textbackslash{}python38\textbackslash{}lib\textbackslash{}site-packages\textbackslash{}matplotlib\textbackslash{}backends\textbackslash{}backend\_qt4agg.py in <module>
          3 """
          4 
    ----> 5 from .backend\_qt5agg import (
          6     \_BackendQT5Agg, FigureCanvasQTAgg, FigureManagerQT, NavigationToolbar2QT)
          7 
    

        c:\textbackslash{}program files\textbackslash{}python38\textbackslash{}lib\textbackslash{}site-packages\textbackslash{}matplotlib\textbackslash{}backends\textbackslash{}backend\_qt5agg.py in <module>
          9 from .. import cbook
         10 from .backend\_agg import FigureCanvasAgg
    ---> 11 from .backend\_qt5 import (
         12     QtCore, QtGui, QtWidgets, \_BackendQT5, FigureCanvasQT, FigureManagerQT,
         13     NavigationToolbar2QT, backend\_version)
    

        c:\textbackslash{}program files\textbackslash{}python38\textbackslash{}lib\textbackslash{}site-packages\textbackslash{}matplotlib\textbackslash{}backends\textbackslash{}backend\_qt5.py in <module>
         13     \_Backend, FigureCanvasBase, FigureManagerBase, NavigationToolbar2,
         14     TimerBase, cursors, ToolContainerBase, StatusbarBase, MouseButton)
    ---> 15 import matplotlib.backends.qt\_editor.figureoptions as figureoptions
         16 from matplotlib.backends.qt\_editor.formsubplottool import UiSubplotTool
         17 from matplotlib.backend\_managers import ToolManager
    

        c:\textbackslash{}program files\textbackslash{}python38\textbackslash{}lib\textbackslash{}site-packages\textbackslash{}matplotlib\textbackslash{}backends\textbackslash{}qt\_editor\textbackslash{}figureoptions.py in <module>
         10 import matplotlib
         11 from matplotlib import cbook, cm, colors as mcolors, markers, image as mimage
    ---> 12 from matplotlib.backends.qt\_compat import QtGui
         13 from matplotlib.backends.qt\_editor import \_formlayout
         14 
    

        c:\textbackslash{}program files\textbackslash{}python38\textbackslash{}lib\textbackslash{}site-packages\textbackslash{}matplotlib\textbackslash{}backends\textbackslash{}qt\_compat.py in <module>
        166         break
        167     else:
    --> 168         raise ImportError("Failed to import any qt binding")
        169 else:  \# We should not get there.
        170     raise AssertionError("Unexpected QT\_API: \{\}".format(QT\_API))
    

        ImportError: Failed to import any qt binding

    \end{Verbatim}

    \begin{tcolorbox}[breakable, size=fbox, boxrule=1pt, pad at break*=1mm,colback=cellbackground, colframe=cellborder]
\prompt{In}{incolor}{77}{\boxspacing}
\begin{Verbatim}[commandchars=\\\{\}]
\PY{c+c1}{\PYZsh{} Now, open an interactive plot window with the Qt4Agg backend}
\PY{n}{fig}\PY{p}{,} \PY{n}{ax} \PY{o}{=} \PY{n}{plt}\PY{o}{.}\PY{n}{subplots}\PY{p}{(}\PY{p}{)}
\PY{n}{t} \PY{o}{=} \PY{n}{np}\PY{o}{.}\PY{n}{linspace}\PY{p}{(}\PY{l+m+mi}{0}\PY{p}{,} \PY{l+m+mi}{10}\PY{p}{,} \PY{l+m+mi}{100}\PY{p}{)}
\PY{n}{ax}\PY{o}{.}\PY{n}{plot}\PY{p}{(}\PY{n}{t}\PY{p}{,} \PY{n}{np}\PY{o}{.}\PY{n}{cos}\PY{p}{(}\PY{n}{t}\PY{p}{)} \PY{o}{*} \PY{n}{np}\PY{o}{.}\PY{n}{sin}\PY{p}{(}\PY{n}{t}\PY{p}{)}\PY{p}{)}
\PY{n}{plt}\PY{o}{.}\PY{n}{show}\PY{p}{(}\PY{p}{)}
\end{Verbatim}
\end{tcolorbox}

    \begin{center}
    \adjustimage{max size={0.9\linewidth}{0.9\paperheight}}{Lecture-4-Matplotlib_files/Lecture-4-Matplotlib_176_0.pdf}
    \end{center}
    { \hspace*{\fill} \\}
    
    Note that when we use an interactive backend, we must call
\texttt{plt.show()} to make the figure appear on the screen.

    \hypertarget{further-reading}{%
\subsection{Further reading}\label{further-reading}}

    \begin{itemize}
\tightlist
\item
  http://www.matplotlib.org - The project web page for matplotlib.
\item
  https://github.com/matplotlib/matplotlib - The source code for
  matplotlib.
\item
  http://matplotlib.org/gallery.html - A large gallery showcaseing
  various types of plots matplotlib can create. Highly recommended!
\item
  http://www.loria.fr/\textasciitilde rougier/teaching/matplotlib - A
  good matplotlib tutorial.
\item
  http://scipy-lectures.github.io/matplotlib/matplotlib.html - Another
  good matplotlib reference.
\end{itemize}

    \hypertarget{versions}{%
\subsection{Versions}\label{versions}}

    \begin{tcolorbox}[breakable, size=fbox, boxrule=1pt, pad at break*=1mm,colback=cellbackground, colframe=cellborder]
\prompt{In}{incolor}{78}{\boxspacing}
\begin{Verbatim}[commandchars=\\\{\}]
\PY{o}{\PYZpc{}}\PY{k}{reload\PYZus{}ext} version\PYZus{}information
\PY{o}{\PYZpc{}}\PY{k}{version\PYZus{}information} numpy, scipy, matplotlib
\end{Verbatim}
\end{tcolorbox}
 
            
\prompt{Out}{outcolor}{78}{}
    
    \begin{tabular}{|l|l|}\hline
{\bf Software} & {\bf Version} \\ \hline\hline
Python & 3.8.2 64bit [MSC v.1916 64 bit (AMD64)] \\ \hline
IPython & 7.15.0 \\ \hline
OS & Windows 10 10.0.19041 SP0 \\ \hline
numpy & 1.18.5 \\ \hline
scipy & 1.4.1 \\ \hline
matplotlib & 3.2.1 \\ \hline
\hline \multicolumn{2}{|l|}{Fri Jun 19 11:29:04 2020 India Standard Time} \\ \hline
\end{tabular}


    


    % Add a bibliography block to the postdoc
    
    
    
\end{document}
